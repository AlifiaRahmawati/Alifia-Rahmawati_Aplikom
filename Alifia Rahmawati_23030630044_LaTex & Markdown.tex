\documentclass[a4paper,10pt]{article}
\usepackage{eumat}

\begin{document}
\begin{eulernotebook}
\eulersubheading{}
\begin{eulercomment}
TUGAS APLIKASI KOMPUTER - AL JABAR

Nama  : Alifia Rahmawati\\
NIM   : 23030630044\\
Kelas : Matematika E 2023

\begin{eulercomment}
\eulerheading{EMT untuk Perhitungan Aljabar}
\begin{eulercomment}
Pada notebook ini Anda belajar menggunakan EMT untuk melakukan
berbagai perhitungan terkait dengan materi atau topik dalam Aljabar.
Kegiatan yang harus Anda lakukan adalah sebagai berikut:

- Membaca secara cermat dan teliti notebook ini;\\
- Menerjemahkan teks bahasa Inggris ke bahasa Indonesia;\\
- Mencoba contoh-contoh perhitungan (perintah EMT) dengan cara
meng-ENTER setiap perintah EMT yang ada (pindahkan kursor ke baris
perintah)\\
- Jika perlu Anda dapat memodifikasi perintah yang ada dan memberikan
keterangan/penjelasan tambahan terkait hasilnya.\\
- Menyisipkan baris-baris perintah baru untuk mengerjakan soal-soal
Aljabar dari file PDF yang saya berikan;\\
- Memberi catatan hasilnya.\\
- Jika perlu tuliskan soalnya pada teks notebook (menggunakan format
LaTeX).\\
- Gunakan tampilan hasil semua perhitungan yang eksak atau simbolik
dengan format LaTeX. (Seperti contoh-contoh pada notebook ini.)

\end{eulercomment}
\eulersubheading{Contoh pertama}
\begin{eulercomment}
Menyederhanakan bentuk aljabar:

\end{eulercomment}
\begin{eulerformula}
\[
6x^{-3}y^5\times -7x^2y^{-9}
\]
\end{eulerformula}
\begin{eulercomment}
\end{eulercomment}
\begin{eulerprompt}
> $&6*x^(-3)*y^5*-7*x^2*y^(-9)
\end{eulerprompt}
\begin{eulerformula}
\[
-\frac{42}{x\,y^4}
\]
\end{eulerformula}
\begin{eulercomment}
Menjabarkan:

\end{eulercomment}
\begin{eulerformula}
\[
(6x^{-3}+y^5)(-7x^2-y^{-9})
\]
\end{eulerformula}
\begin{eulerprompt}
>$&showev('expand((6*x^(-3)+y^5)*(-7*x^2-y^(-9))))
\end{eulerprompt}
\begin{eulerformula}
\[
{\it expand}\left(\left(-\frac{1}{y^9}-7\,x^2\right)\,\left(y^5+  \frac{6}{x^3}\right)\right)=-7\,x^2\,y^5-\frac{1}{y^4}-\frac{6}{x^3  \,y^9}-\frac{42}{x}
\]
\end{eulerformula}
\eulersubheading{Baris Perintah}
\begin{eulercomment}
Baris perintah Euler terdiri dari satu atau beberapa perintah Euler
yang diikuti oleh titik koma ";" atau koma ",". Titik koma mencegah
pencetakan hasil. Koma setelah perintah terakhir dapat dihilangkan.

Baris perintah berikut hanya akan mencetak hasil ekspresi, bukan
perintah penugasan atau format.
\end{eulercomment}
\begin{eulerprompt}
>r:=2; h:=4; pi*r^2*h/3
\end{eulerprompt}
\begin{euleroutput}
  16.7551608191
\end{euleroutput}
\begin{eulercomment}
Perintah harus dipisahkan dengan spasi. Baris perintah berikut
mencetak dua hasilnya.
\end{eulercomment}
\begin{eulerprompt}
>pi*2*r*h, %+2*pi*r*h // Ingat tanda % menyatakan hasil perhitungan terakhir sebelumnya
\end{eulerprompt}
\begin{euleroutput}
  50.2654824574
  100.530964915
\end{euleroutput}
\begin{eulercomment}
Baris perintah dieksekusi sesuai urutan pengguna menekan tombol enter.
Jadi Anda akan mendapatkan nilai baru setiap kali Anda mengeksekusi
baris kedua.
\end{eulercomment}
\begin{eulerprompt}
>x := 1;
>x := cos(x) // nilai cosinus (x dalam radian)
\end{eulerprompt}
\begin{euleroutput}
  0.540302305868
\end{euleroutput}
\begin{eulerprompt}
>x := cos(x)
\end{eulerprompt}
\begin{euleroutput}
  0.857553215846
\end{euleroutput}
\begin{eulercomment}
Jika dua baris dihubungkan dengan "..." kedua baris akan selalu
dieksekusi secara bersamaan.
\end{eulercomment}
\begin{eulerprompt}
>x := 1.5; ...
>x := (x+2/x)/2, x := (x+2/x)/2, x := (x+2/x)/2, 
\end{eulerprompt}
\begin{euleroutput}
  1.41666666667
  1.41421568627
  1.41421356237
\end{euleroutput}
\begin{eulercomment}
Ini juga merupakan cara yang baik untuk menyebarkan perintah yang
panjang ke dua atau lebih baris. Anda dapat menekan Ctrl+Return untuk
membagi baris menjadi dua pada posisi kursor saat ini, atau Ctlr+Back
untuk menggabungkan baris-baris tersebut.

Untuk melipat semua baris yang terdiri dari beberapa baris, tekan
Ctrl+L. Kemudian baris-baris berikutnya hanya akan terlihat, jika
salah satunya menjadi fokus. Untuk melipat satu baris yang terdiri
dari beberapa baris, mulailah baris pertama dengan "\%+ ".
\end{eulercomment}
\begin{eulerprompt}
>%+ x=4+5; ...
\end{eulerprompt}
\begin{eulercomment}
Baris yang dimulai dengan \%\% tidak akan terlihat sama sekali.
\end{eulercomment}
\begin{euleroutput}
  81
\end{euleroutput}
\begin{eulercomment}
Euler mendukung perulangan dalam baris perintah, asalkan dapat
dimasukkan ke dalam satu baris atau beberapa baris. Dalam program,
pembatasan ini tentu saja tidak berlaku. Untuk informasi lebih lanjut,
lihat pengantar berikut.
\end{eulercomment}
\begin{eulerprompt}
>x=1; for i=1 to 5; x := (x+2/x)/2, end; // menghitung akar 2
\end{eulerprompt}
\begin{euleroutput}
  1.5
  1.41666666667
  1.41421568627
  1.41421356237
  1.41421356237
\end{euleroutput}
\begin{eulercomment}
Tidak apa-apa menggunakan beberapa baris. Pastikan baris diakhiri
dengan "...".
\end{eulercomment}
\begin{eulerprompt}
>x := 1.5; // comments go here before the ...
>repeat xnew:=(x+2/x)/2; until xnew~=x; ...
>   x := xnew; ...
>end; ...
>x,
\end{eulerprompt}
\begin{euleroutput}
  1.41421356237
\end{euleroutput}
\begin{eulercomment}
Struktur kondisional juga berfungsi.
\end{eulercomment}
\begin{eulerprompt}
>if E^pi>pi^E; then "Thought so!", endif;
\end{eulerprompt}
\begin{euleroutput}
  Thought so!
\end{euleroutput}
\begin{eulercomment}
Saat Anda menjalankan perintah, kursor dapat berada di posisi mana pun
di baris perintah. Anda dapat kembali ke perintah sebelumnya atau
melompat ke perintah berikutnya dengan tombol panah. Atau Anda dapat
mengklik bagian komentar di atas perintah untuk membuka perintah
tersebut.

Saat Anda menggerakkan kursor di sepanjang baris, pasangan tanda
kurung buka dan tutup akan disorot. Perhatikan juga baris status.
Setelah tanda kurung buka fungsi sqrt(), baris status akan menampilkan
teks bantuan untuk fungsi tersebut. Jalankan perintah dengan tombol
return.
\end{eulercomment}
\begin{eulerprompt}
>sqrt(sin(10°)/cos(20°))
\end{eulerprompt}
\begin{euleroutput}
  0.429875017772
\end{euleroutput}
\begin{eulercomment}
Untuk melihat bantuan untuk perintah terbaru, buka jendela bantuan
dengan F1. Di sana, Anda dapat memasukkan teks untuk dicari. Pada
baris kosong, bantuan untuk jendela bantuan akan ditampilkan. Anda
dapat menekan escape untuk menghapus baris, atau untuk menutup jendela
bantuan.

Anda dapat mengklik dua kali pada perintah apa pun untuk membuka
bantuan untuk perintah ini. Coba klik dua kali perintah exp di bawah
ini pada baris perintah.
\end{eulercomment}
\begin{eulerprompt}
>exp(log(2.5))
\end{eulerprompt}
\begin{euleroutput}
  2.5
\end{euleroutput}
\begin{eulercomment}
Anda juga dapat menyalin dan menempel di Euler. Gunakan Ctrl-C dan
Ctrl-V untuk ini. Untuk menandai teks, seret tetikus atau gunakan
shift bersamaan dengan tombol kursor apa pun. Selain itu, Anda dapat
menyalin tanda kurung yang disorot.
\end{eulercomment}
\eulersubheading{Sintaksis Dasar}
\begin{eulercomment}
Euler mengetahui fungsi matematika yang umum. Seperti yang telah Anda
lihat di atas, fungsi trigonometri bekerja dalam radian atau derajat.
Untuk mengonversi ke derajat, tambahkan simbol derajat (dengan tombol
F7) ke nilai, atau gunakan fungsi rad(x). Fungsi akar kuadrat disebut
sqrt di Euler. Tentu saja, x\textasciicircum{}(1/2) juga memungkinkan.

Untuk mengatur variabel, gunakan "=" atau ":=". Demi kejelasan,
pengantar ini menggunakan bentuk yang terakhir. Spasi tidak menjadi
masalah. Namun, spasi di antara perintah diharapkan.

Beberapa perintah dalam satu baris dipisahkan dengan "," atau ";".
Titik koma menghilangkan keluaran perintah. Di akhir baris perintah,
"," diasumsikan, jika ";" tidak ada.
\end{eulercomment}
\begin{eulerprompt}
>g:=9.81; t:=2.5; 1/2*g*t^2
\end{eulerprompt}
\begin{euleroutput}
  30.65625
\end{euleroutput}
\begin{eulercomment}
EMT menggunakan sintaks pemrograman untuk ekspresi. Untuk memasukkan

\end{eulercomment}
\begin{eulerformula}
\[
e^2 \cdot \left( \frac{1}{3+4 \log(0.6)}+\frac{1}{7} \right)
\]
\end{eulerformula}
\begin{eulercomment}
Anda harus menetapkan tanda kurung yang benar dan menggunakan / untuk
pecahan. Perhatikan tanda kurung yang disorot untuk bantuan.
Perhatikan bahwa konstanta Euler e diberi nama E dalam EMT.
\end{eulercomment}
\begin{eulerprompt}
>E^2*(1/(3+4*log(0.6))+1/7)
\end{eulerprompt}
\begin{euleroutput}
  8.77908249441
\end{euleroutput}
\begin{eulercomment}
Untuk menghitung ekspresi rumit seperti

\end{eulercomment}
\begin{eulerformula}
\[
\left(\frac{\frac17 + \frac18 + 2}{\frac13 + \frac12}\right)^2 \pi
\]
\end{eulerformula}
\begin{eulercomment}
Anda perlu memasukkannya dalam bentuk baris.
\end{eulercomment}
\begin{eulerprompt}
>((1/7 + 1/8 + 2) / (1/3 + 1/2))^2 * pi
\end{eulerprompt}
\begin{euleroutput}
  23.2671801626
\end{euleroutput}
\begin{eulercomment}
Letakkan tanda kurung di sekitar sub-ekspresi yang perlu dihitung
terlebih dahulu dengan hati-hati. EMT membantu Anda dengan menyorot
ekspresi yang diakhiri tanda kurung tutup. Anda juga harus memasukkan
nama "pi" untuk huruf Yunani pi.

Hasil perhitungan ini adalah angka floating point. Secara default,
angka ini dicetak dengan akurasi sekitar 12 digit. Pada baris perintah
berikut, kita juga mempelajari cara merujuk ke hasil sebelumnya dalam
baris yang sama.
\end{eulercomment}
\begin{eulerprompt}
>1/3+1/7, fraction %
\end{eulerprompt}
\begin{euleroutput}
  0.47619047619
  10/21
\end{euleroutput}
\begin{eulercomment}
Perintah Euler dapat berupa ekspresi atau perintah primitif. Ekspresi
terdiri dari operator dan fungsi. Jika perlu, ekspresi harus berisi
tanda kurung untuk memaksakan urutan eksekusi yang benar. Jika ragu,
sebaiknya gunakan tanda kurung. Perhatikan bahwa EMT menampilkan tanda
kurung buka dan tutup saat mengedit baris perintah.
\end{eulercomment}
\begin{eulerprompt}
>(cos(pi/4)+1)^3*(sin(pi/4)+1)^2
\end{eulerprompt}
\begin{euleroutput}
  14.4978445072
\end{euleroutput}
\begin{eulercomment}
Operator numerik Euler meliputi

+ unary atau operator plus\\
- unary atau operator minus\\
*,/\\
. perkalian matriks\\
a\textasciicircum{}b pangkat untuk a positif atau integer b (a**b juga berfungsi)\\
n! operator faktorial

dan masih banyak lagi.

Berikut ini beberapa fungsi yang mungkin Anda perlukan. Masih banyak
lagi.

sin,cos,tan,atan,asin,acos,rad,deg\\
log,exp,log10,sqrt,logbase\\
sin,logbin,logfac,mod,floor,ceil,round,abs,sign\\
conj,re,im,arg,conj,real,complex\\
beta,betai,gamma,complexgamma,ellrf,ellf,ellrd,elle\\
bitand,bitor,bitxor,bitnot

Beberapa perintah memiliki alias, misalnya ln untuk log.
\end{eulercomment}
\begin{eulerprompt}
>ln(E^2), arctan(tan(0.5))
\end{eulerprompt}
\begin{euleroutput}
  2
  0.5
\end{euleroutput}
\begin{eulerprompt}
>sin(30°)
\end{eulerprompt}
\begin{euleroutput}
  0.5
\end{euleroutput}
\begin{eulercomment}
Pastikan untuk menggunakan tanda kurung (kurung bundar), jika ada
keraguan tentang urutan eksekusi! Berikut ini tidak sama dengan
(2\textasciicircum{}3)\textasciicircum{}4, yang merupakan default untuk 2\textasciicircum{}3\textasciicircum{}4 dalam EMT (beberapa sistem
numerik melakukannya dengan cara lain).
\end{eulercomment}
\begin{eulerprompt}
>2^3^4, (2^3)^4, 2^(3^4) ...
\end{eulerprompt}
\eulersubheading{Bilangan Riil}
\begin{eulercomment}
Tipe data utama dalam Euler adalah bilangan riil. Bilangan riil
direpresentasikan dalam format IEEE dengan akurasi sekitar 16 digit
desimal.
\end{eulercomment}
\begin{eulerprompt}
>longest 1/3
\end{eulerprompt}
\begin{euleroutput}
       0.3333333333333333 
\end{euleroutput}
\begin{eulercomment}
Representasi ganda internal membutuhkan 8 byte.
\end{eulercomment}
\begin{eulerprompt}
>printdual(1/3)
\end{eulerprompt}
\begin{euleroutput}
  1.0101010101010101010101010101010101010101010101010101*2^-2
\end{euleroutput}
\begin{eulerprompt}
>printhex(1/3)
\end{eulerprompt}
\begin{euleroutput}
  5.5555555555554*16^-1
\end{euleroutput}
\eulersubheading{String}
\begin{eulercomment}
String dalam Euler didefinisikan dengan "...".
\end{eulercomment}
\begin{eulerprompt}
>"A string can contain anything."
\end{eulerprompt}
\begin{euleroutput}
  A string can contain anything.
\end{euleroutput}
\begin{eulercomment}
String dapat dirangkai dengan \textbar{} atau dengan +. Ini juga berlaku untuk
angka, yang dalam kasus tersebut diubah menjadi string.
\end{eulercomment}
\begin{eulerprompt}
>"The area of the circle with radius " + 2 + " cm is " + pi*4 + " cm^2."
\end{eulerprompt}
\begin{euleroutput}
  The area of the circle with radius 2 cm is 12.5663706144 cm^2.
\end{euleroutput}
\begin{eulercomment}
Fungsi cetak juga mengonversi angka menjadi string. Fungsi ini dapat
mengambil sejumlah digit dan sejumlah tempat (0 untuk keluaran padat),
dan optimalnya satu unit.
\end{eulercomment}
\begin{eulerprompt}
>"Golden Ratio : " + print((1+sqrt(5))/2,5,0)
\end{eulerprompt}
\begin{euleroutput}
  Golden Ratio : 1.61803
\end{euleroutput}
\begin{eulercomment}
Ada string khusus none, yang tidak dicetak. String ini dikembalikan
oleh beberapa fungsi, ketika hasilnya tidak penting. (Dikembalikan
secara otomatis, jika fungsi tersebut tidak memiliki pernyataan
return.)
\end{eulercomment}
\begin{eulerprompt}
>none
\end{eulerprompt}
\begin{eulercomment}
Untuk mengubah string menjadi angka, cukup evaluasi string tersebut.
Ini juga berlaku untuk ekspresi (lihat di bawah).
\end{eulercomment}
\begin{eulerprompt}
>"1234.5"()
\end{eulerprompt}
\begin{euleroutput}
  1234.5
\end{euleroutput}
\begin{eulercomment}
Untuk mendefinisikan vektor string, gunakan notasi vektor [...]
\end{eulercomment}
\begin{eulerprompt}
>v:=["affe","charlie","bravo"]
\end{eulerprompt}
\begin{euleroutput}
  affe
  charlie
  bravo
\end{euleroutput}
\begin{eulercomment}
Vektor string kosong dilambangkan dengan [none]. Vektor string dapat
dirangkai.
\end{eulercomment}
\begin{eulerprompt}
>w:=[none]; w|v|v
\end{eulerprompt}
\begin{euleroutput}
  affe
  charlie
  bravo
  affe
  charlie
  bravo
\end{euleroutput}
\begin{eulercomment}
String dapat berisi karakter Unicode. Secara internal, string ini
berisi kode UTF-8. Untuk membuat string seperti itu, gunakan u"..."
dan salah satu entitas HTML.

String Unicode dapat dirangkai seperti string lainnya.
\end{eulercomment}
\begin{eulerprompt}
>u"&alpha; = " + 45 + u"&deg;" // pdfLaTeX mungkin gagal menampilkan secara benar
\end{eulerprompt}
\begin{euleroutput}
  α = 45°
\end{euleroutput}
\begin{eulercomment}
I
\end{eulercomment}
\begin{eulercomment}
Dalam komentar, entitas yang sama seperti alpha;, beta; dll. dapat
digunakan. Ini mungkin merupakan alternatif cepat untuk Latex.
(Rincian lebih lanjut pada komentar di bawah).
\end{eulercomment}
\begin{eulercomment}
Ada beberapa fungsi untuk membuat atau menganalisis string unicode.
Fungsi strtochar() akan mengenali string Unicode dan menerjemahkannya
dengan benar.
\end{eulercomment}
\begin{eulerprompt}
>v=strtochar(u"&Auml; is a German letter")
\end{eulerprompt}
\begin{euleroutput}
  [196,  32,  105,  115,  32,  97,  32,  71,  101,  114,  109,  97,  110,
  32,  108,  101,  116,  116,  101,  114]
\end{euleroutput}
\begin{eulercomment}
Hasilnya adalah vektor angka Unicode. Fungsi kebalikannya adalah
chartoutf().
\end{eulercomment}
\begin{eulerprompt}
>v[1]=strtochar(u"&Uuml;")[1]; chartoutf(v)
\end{eulerprompt}
\begin{euleroutput}
  Ü is a German letter
\end{euleroutput}
\begin{eulercomment}
Fungsi utf() dapat menerjemahkan string dengan entitas dalam variabel
menjadi string Unicode.
\end{eulercomment}
\begin{eulerprompt}
>s="We have &alpha;=&beta;."; utf(s) // pdfLaTeX mungkin gagal menampilkan secara benar
\end{eulerprompt}
\begin{euleroutput}
  We have α=β.
\end{euleroutput}
\begin{eulercomment}
Dimungkinkan juga untuk menggunakan entitas numerik.
\end{eulercomment}
\begin{eulerprompt}
>u"&#196;hnliches"
\end{eulerprompt}
\begin{euleroutput}
  Ähnliches
\end{euleroutput}
\eulersubheading{Nilai Boolean}
\begin{eulercomment}
Nilai Boolean direpresentasikan dengan 1=benar atau 0=salah dalam
Euler. String dapat dibandingkan, seperti halnya angka.
\end{eulercomment}
\begin{eulerprompt}
>2<1, "apel"<"banana"
\end{eulerprompt}
\begin{euleroutput}
  0
  1
\end{euleroutput}
\begin{eulercomment}
"dan" adalah operator "\&\&" dan "atau" adalah operator "\textbar{}\textbar{}", seperti
dalam bahasa C. (Kata "dan" dan "atau" hanya dapat digunakan dalam
kondisi "jika".)
\end{eulercomment}
\begin{eulerprompt}
>2<E && E<3
\end{eulerprompt}
\begin{euleroutput}
  1
\end{euleroutput}
\begin{eulercomment}
Operator Boolean mematuhi aturan bahasa matriks.
\end{eulercomment}
\begin{eulerprompt}
>(1:10)>5, nonzeros(%)
\end{eulerprompt}
\begin{euleroutput}
  [0,  0,  0,  0,  0,  1,  1,  1,  1,  1]
  [6,  7,  8,  9,  10]
\end{euleroutput}
\begin{eulercomment}
Anda dapat menggunakan fungsi nonzeros() untuk mengekstrak elemen
tertentu dari sebuah vektor. Dalam contoh ini, kami menggunakan
kondisional isprime(n).
\end{eulercomment}
\begin{eulerprompt}
>N=2|3:2:99 // N berisi elemen 2 dan bilangan2 ganjil dari 3 s.d. 99
\end{eulerprompt}
\begin{euleroutput}
  [2,  3,  5,  7,  9,  11,  13,  15,  17,  19,  21,  23,  25,  27,  29,
  31,  33,  35,  37,  39,  41,  43,  45,  47,  49,  51,  53,  55,  57,
  59,  61,  63,  65,  67,  69,  71,  73,  75,  77,  79,  81,  83,  85,
  87,  89,  91,  93,  95,  97,  99]
\end{euleroutput}
\begin{eulerprompt}
>N[nonzeros(isprime(N))] //pilih anggota2 N yang prima
\end{eulerprompt}
\begin{euleroutput}
  [2,  3,  5,  7,  11,  13,  17,  19,  23,  29,  31,  37,  41,  43,  47,
  53,  59,  61,  67,  71,  73,  79,  83,  89,  97]
\end{euleroutput}
\eulersubheading{Format Keluaran}
\begin{eulercomment}
Format keluaran default EMT mencetak 12 digit. Untuk memastikan bahwa
kita melihat default, kita mengatur ulang formatnya.
\end{eulercomment}
\begin{eulerprompt}
>defformat; pi
\end{eulerprompt}
\begin{euleroutput}
  3.14159265359
\end{euleroutput}
\begin{eulercomment}
Secara internal, EMT menggunakan standar IEEE untuk angka ganda dengan
sekitar 16 digit desimal. Untuk melihat jumlah digit lengkap, gunakan
perintah "longestformat", atau kami menggunakan operator "longest"
untuk menampilkan hasil dalam format terpanjang.
\end{eulercomment}
\begin{eulerprompt}
>longest pi
\end{eulerprompt}
\begin{euleroutput}
        3.141592653589793 
\end{euleroutput}
\begin{eulercomment}
Berikut adalah representasi heksadesimal internal dari angka ganda.
\end{eulercomment}
\begin{eulerprompt}
>printhex(pi)
\end{eulerprompt}
\begin{euleroutput}
  3.243F6A8885A30*16^0
\end{euleroutput}
\begin{eulercomment}
Format keluaran dapat diubah secara permanen dengan perintah format.
\end{eulercomment}
\begin{eulerprompt}
>format(12,5); 1/3, pi, sin(1)
\end{eulerprompt}
\begin{euleroutput}
      0.33333 
      3.14159 
      0.84147 
\end{euleroutput}
\begin{eulercomment}
Format defaultnya adalah(12).
\end{eulercomment}
\begin{eulerprompt}
>format(12); 1/3
\end{eulerprompt}
\begin{euleroutput}
  0.333333333333
\end{euleroutput}
\begin{eulercomment}
Fungsi seperti "shortestformat", "shortformat", "longformat" bekerja
untuk vektor dengan cara berikut.
\end{eulercomment}
\begin{eulerprompt}
>shortestformat; random(3,8)
\end{eulerprompt}
\begin{euleroutput}
    0.66    0.2   0.89   0.28   0.53   0.31   0.44    0.3 
    0.28   0.88   0.27    0.7   0.22   0.45   0.31   0.91 
    0.19   0.46  0.095    0.6   0.43   0.73   0.47   0.32 
\end{euleroutput}
\begin{eulercomment}
Format default untuk skalar adalah format(12). Namun, ini dapat
diubah.
\end{eulercomment}
\begin{eulerprompt}
>setscalarformat(5); pi
\end{eulerprompt}
\begin{euleroutput}
  3.1416
\end{euleroutput}
\begin{eulercomment}
Fungsi "longestformat" juga mengatur format skalar.
\end{eulercomment}
\begin{eulerprompt}
>longestformat; pi
\end{eulerprompt}
\begin{euleroutput}
  3.141592653589793
\end{euleroutput}
\begin{eulercomment}
Sebagai referensi, berikut adalah daftar format output yang paling
penting.

shortestformat shortformat longformat, longestformat\\
format(length,digits) goodformat(length)\\
fracformat(length)\\
defformat

Keakuratan internal EMT adalah sekitar 16 tempat desimal, yang
merupakan standar IEEE. Angka disimpan dalam format internal ini.

Namun, format output EMT dapat diatur dengan cara yang fleksibel.
\end{eulercomment}
\begin{eulerprompt}
>longestformat; pi,
\end{eulerprompt}
\begin{euleroutput}
  3.141592653589793
\end{euleroutput}
\begin{eulerprompt}
>format(10,5); pi
\end{eulerprompt}
\begin{euleroutput}
    3.14159 
\end{euleroutput}
\begin{eulercomment}
Standarnya adalah defformat().
\end{eulercomment}
\begin{eulerprompt}
>defformat; // default
\end{eulerprompt}
\begin{eulercomment}
Ada operator pendek yang hanya mencetak satu nilai. Operator
"terpanjang" akan mencetak semua digit angka yang valid.
\end{eulercomment}
\begin{eulerprompt}
>longest pi^2/2
\end{eulerprompt}
\begin{euleroutput}
        4.934802200544679 
\end{euleroutput}
\begin{eulercomment}
Ada juga operator pendek untuk mencetak hasil dalam format pecahan.
Kami telah menggunakannya di atas.
\end{eulercomment}
\begin{eulerprompt}
>fraction 1+1/2+1/3+1/4
\end{eulerprompt}
\begin{euleroutput}
  25/12
\end{euleroutput}
\begin{eulercomment}
Karena format internal menggunakan cara biner untuk menyimpan angka,
nilai 0,1 tidak akan terwakili secara tepat. Kesalahannya bertambah
sedikit, seperti yang Anda lihat dalam perhitungan berikut.
\end{eulercomment}
\begin{eulerprompt}
>longest 0.1+0.1+0.1+0.1+0.1+0.1+0.1+0.1+0.1+0.1-1
\end{eulerprompt}
\begin{euleroutput}
   -1.110223024625157e-16 
\end{euleroutput}
\begin{eulercomment}
Namun dengan "longformat" default, Anda tidak akan melihat hal ini.
Demi kenyamanan, output angka yang sangat kecil adalah 0.
\end{eulercomment}
\begin{eulerprompt}
>0.1+0.1+0.1+0.1+0.1+0.1+0.1+0.1+0.1+0.1-1
\end{eulerprompt}
\begin{euleroutput}
  0
\end{euleroutput}
\eulerheading{Ekspresi}
\begin{eulercomment}
String atau nama dapat digunakan untuk menyimpan ekspresi matematika,
yang dapat dievaluasi oleh EMT. Untuk ini, gunakan tanda kurung
setelah ekspresi. Jika Anda ingin menggunakan string sebagai ekspresi,
gunakan konvensi untuk menamainya "fx" atau "fxy", dst. Ekspresi lebih
diutamakan daripada fungsi.

Variabel global dapat digunakan dalam evaluasi.
\end{eulercomment}
\begin{eulerprompt}
>r:=2; fx:="pi*r^2"; longest fx()
\end{eulerprompt}
\begin{euleroutput}
        12.56637061435917 
\end{euleroutput}
\begin{eulercomment}
Parameter ditetapkan ke x, y, dan z dalam urutan tersebut. Parameter
tambahan dapat ditambahkan menggunakan parameter yang ditetapkan.
\end{eulercomment}
\begin{eulerprompt}
>fx:="a*sin(x)^2"; fx(5,a=-1)
\end{eulerprompt}
\begin{euleroutput}
  -0.919535764538
\end{euleroutput}
\begin{eulercomment}
Perhatikan bahwa ekspresi akan selalu menggunakan variabel global,
bahkan jika ada variabel dalam suatu fungsi dengan nama yang sama.
(Jika tidak, evaluasi ekspresi dalam fungsi dapat memberikan hasil
yang sangat membingungkan bagi pengguna yang memanggil fungsi
tersebut.)
\end{eulercomment}
\begin{eulerprompt}
>at:=4; function f(expr,x,at) := expr(x); ...
>f("at*x^2",3,5) // computes 4*3^2 not 5*3^2
\end{eulerprompt}
\begin{euleroutput}
  36
\end{euleroutput}
\begin{eulercomment}
Jika Anda ingin menggunakan nilai lain untuk "at" selain nilai global,
Anda perlu menambahkan "at=value".
\end{eulercomment}
\begin{eulerprompt}
>at:=4; function f(expr,x,a) := expr(x,at=a); ...
>f("at*x^2",3,5)
\end{eulerprompt}
\begin{euleroutput}
  45
\end{euleroutput}
\begin{eulercomment}
Sebagai referensi, kami mencatat bahwa koleksi panggilan (dibahas di
tempat lain) dapat berisi ekspresi. Jadi, kita dapat membuat contoh di
atas sebagai berikut.
\end{eulercomment}
\begin{eulerprompt}
>at:=4; function f(expr,x) := expr(x); ...
>f(\{\{"at*x^2",at=5\}\},3)
\end{eulerprompt}
\begin{euleroutput}
  45
\end{euleroutput}
\begin{eulercomment}
Ekspresi dalam x sering digunakan seperti fungsi.\\
erlu dicatat bahwa mendefinisikan fungsi dengan nama yang sama seperti
ekspresi simbolik global akan menghapus variabel ini untuk menghindari
kebingungan antara ekspresi simbolik dan fungsi.
\end{eulercomment}
\begin{eulerprompt}
>f &= 5*x;
>function f(x) := 6*x;
>f(2)
\end{eulerprompt}
\begin{euleroutput}
  12
\end{euleroutput}
\begin{eulercomment}
Berdasarkan konvensi, ekspresi simbolik atau numerik harus diberi nama
fx, fxy, dst. Skema penamaan ini tidak boleh digunakan untuk fungsi.
\end{eulercomment}
\begin{eulerprompt}
>fx &= diff(x^x,x); $&fx
\end{eulerprompt}
\begin{eulerformula}
\[
x^{x}\,\left(\log x+1\right)
\]
\end{eulerformula}
\begin{eulercomment}
Bentuk khusus dari suatu ekspresi memperbolehkan variabel apa pun
sebagai parameter tanpa nama untuk evaluasi ekspresi, bukan hanya "x",
"y", dst. Untuk ini, awali ekspresi dengan "@(variabel) ...".
\end{eulercomment}
\begin{eulerprompt}
>"@(a,b) a^2+b^2", %(4,5)
\end{eulerprompt}
\begin{euleroutput}
  @(a,b) a^2+b^2
  41
\end{euleroutput}
\begin{eulercomment}
Hal ini memungkinkan untuk memanipulasi ekspresi dalam variabel lain
untuk fungsi EMT yang memerlukan ekspresi dalam "x".

Cara paling dasar untuk mendefinisikan fungsi sederhana adalah dengan
menyimpan rumusnya dalam ekspresi simbolik atau numerik. Jika variabel
utamanya adalah x, ekspresi tersebut dapat dievaluasi seperti halnya
fungsi.

Seperti yang Anda lihat dalam contoh berikut, variabel global terlihat
selama evaluasi.
\end{eulercomment}
\begin{eulerprompt}
>fx &= x^3-a*x;  ...
>a=1.2; fx(0.5)
\end{eulerprompt}
\begin{euleroutput}
  -0.475
\end{euleroutput}
\begin{eulercomment}
Semua variabel lain dalam ekspresi dapat ditentukan dalam evaluasi
menggunakan parameter yang ditetapkan.
\end{eulercomment}
\begin{eulerprompt}
>fx(0.5,a=1.1)
\end{eulerprompt}
\begin{euleroutput}
  -0.425
\end{euleroutput}
\begin{eulercomment}
Suatu ekspresi tidak harus simbolis. Hal ini diperlukan jika ekspresi
tersebut mengandung fungsi yang hanya diketahui dalam kernel numerik,
bukan dalam Maxima.

\begin{eulercomment}
\eulerheading{Matematika Simbolis}
\begin{eulercomment}
EMT mengerjakan matematika simbolis dengan bantuan Maxima. Untuk
detailnya, mulailah dengan tutorial berikut, atau telusuri referensi
untuk Maxima. Para ahli di Maxima harus memperhatikan bahwa terdapat
perbedaan sintaksis antara sintaksis asli Maxima dan sintaksis default
ekspresi simbolis dalam EMT.

Matematika simbolis terintegrasi dengan mulus ke dalam Euler dengan \&.
Setiap ekspresi yang dimulai dengan \& adalah ekspresi simbolis.
Ekspresi tersebut dievaluasi dan dicetak oleh Maxima.

Pertama-tama, Maxima memiliki aritmatika "tak terbatas" yang dapat
menangani angka yang sangat besar.
\end{eulercomment}
\begin{eulerprompt}
>$&44!
\end{eulerprompt}
\begin{eulerformula}
\[
2658271574788448768043625811014615890319638528000000000
\]
\end{eulerformula}
\begin{eulercomment}
Dengan cara ini, Anda dapat menghitung hasil yang besar secara tepat.
Mari kita hitung

\end{eulercomment}
\begin{eulerformula}
\[
C(44,10) = \frac{44!}{34! \cdot 10!}
\]
\end{eulerformula}
\begin{eulerprompt}
>$& 44!/(34!*10!) // nilai C(44,10)
\end{eulerprompt}
\begin{eulerformula}
\[
2481256778
\]
\end{eulerformula}
\begin{eulercomment}
Tentu saja, Maxima memiliki fungsi yang lebih efisien untuk ini
(seperti halnya bagian numerik EMT).
\end{eulercomment}
\begin{eulerprompt}
>$binomial(44,10) //menghitung C(44,10) menggunakan fungsi binomial()
\end{eulerprompt}
\begin{eulerformula}
\[
2481256778
\]
\end{eulerformula}
\begin{eulercomment}
Untuk mempelajari lebih lanjut tentang fungsi tertentu, klik dua kali
pada fungsi tersebut. Misalnya, coba klik dua kali pada "\&binomial" di
baris perintah sebelumnya. Ini akan membuka dokumentasi Maxima
sebagaimana disediakan oleh penulis program tersebut.

Anda akan mempelajari bahwa hal berikut juga berfungsi.

\end{eulercomment}
\begin{eulerformula}
\[
C(x,3)=\frac{x!}{(x-3)!3!}=\frac{(x-2)(x-1)x}{6}
\]
\end{eulerformula}
\begin{eulerprompt}
>$binomial(x,3) // C(x,3)
\end{eulerprompt}
\begin{eulerformula}
\[
\frac{\left(x-2\right)\,\left(x-1\right)\,x}{6}
\]
\end{eulerformula}
\begin{eulercomment}
Jika Anda ingin mengganti x dengan nilai tertentu, gunakan "with".
\end{eulercomment}
\begin{eulerprompt}
>$&binomial(x,3) with x=10 // substitusi x=10 ke C(x,3)
\end{eulerprompt}
\begin{eulerformula}
\[
120
\]
\end{eulerformula}
\begin{eulercomment}
Dengan cara itu Anda dapat menggunakan solusi persamaan dalam
persamaan lain.

Ekspresi simbolik dicetak oleh Maxima dalam bentuk 2D. Alasannya
adalah adanya tanda simbolik khusus dalam string.

Seperti yang telah Anda lihat pada contoh sebelumnya dan berikutnya,
jika Anda telah menginstal LaTeX, Anda dapat mencetak ekspresi
simbolik dengan Latex. Jika tidak, perintah berikut akan mengeluarkan
pesan kesalahan.

Untuk mencetak ekspresi simbolik dengan LaTeX, gunakan \textdollar{} di depan \&
(atau Anda dapat menghilangkan \&) sebelum perintah. Jangan jalankan
perintah Maxima dengan \textdollar{}, jika Anda tidak menginstal LaTeX.
\end{eulercomment}
\begin{eulerprompt}
>$(3+x)/(x^2+1)
\end{eulerprompt}
\begin{eulerformula}
\[
\frac{x+3}{x^2+1}
\]
\end{eulerformula}
\begin{eulercomment}
Ekspresi simbolik diurai oleh Euler. Jika Anda memerlukan sintaksis
yang kompleks dalam satu ekspresi, Anda dapat melampirkan ekspresi
tersebut dalam "...". Menggunakan lebih dari satu ekspresi sederhana
dimungkinkan, tetapi sangat tidak disarankan.
\end{eulercomment}
\begin{eulerprompt}
>&"v := 5; v^2"
\end{eulerprompt}
\begin{euleroutput}
  
                                    25
  
\end{euleroutput}
\begin{eulercomment}
Untuk kelengkapan, kami mencatat bahwa ekspresi simbolik dapat
digunakan dalam program, tetapi harus disertakan dalam tanda kutip.
Selain itu, akan jauh lebih efektif untuk memanggil Maxima pada waktu
kompilasi jika memungkinkan.
\end{eulercomment}
\begin{eulerprompt}
>$&expand((1+x)^4), $&factor(diff(%,x)) // diff: turunan, factor: faktor
\end{eulerprompt}
\begin{eulerformula}
\[
4\,\left(x+1\right)^3
\]
\end{eulerformula}
\eulerimg{0}{images/Alifia Rahmawati_23030630044_LaTex & Markdown-011-large.png}
\begin{eulercomment}
Sekali lagi, \% merujuk pada hasil sebelumnya.

Untuk mempermudah, kami menyimpan solusi ke variabel simbolik.
Variabel simbolik didefinisikan dengan "\&=".
\end{eulercomment}
\begin{eulerprompt}
>fx &= (x+1)/(x^4+1); $&fx
\end{eulerprompt}
\begin{eulerformula}
\[
\frac{x+1}{x^4+1}
\]
\end{eulerformula}
\begin{eulercomment}
Ekspresi simbolik dapat digunakan dalam ekspresi simbolik lainnya.
\end{eulercomment}
\begin{eulerprompt}
>$&factor(diff(fx,x))
\end{eulerprompt}
\begin{eulerformula}
\[
\frac{-3\,x^4-4\,x^3+1}{\left(x^4+1\right)^2}
\]
\end{eulerformula}
\begin{eulercomment}
Input langsung perintah Maxima juga tersedia. Awali baris perintah
dengan "::". Sintaks Maxima disesuaikan dengan sintaks EMT (disebut
"mode kompatibilitas").
\end{eulercomment}
\begin{eulerprompt}
>&factor(20!)
\end{eulerprompt}
\begin{euleroutput}
  
                           2432902008176640000
  
\end{euleroutput}
\begin{eulerprompt}
>::: factor(10!)
\end{eulerprompt}
\begin{euleroutput}
  
                                 8  4  2
                                2  3  5  7
  
\end{euleroutput}
\begin{eulerprompt}
>:: factor(20!)
\end{eulerprompt}
\begin{euleroutput}
  
                          18  8  4  2
                         2   3  5  7  11 13 17 19
  
\end{euleroutput}
\begin{eulercomment}
Jika Anda ahli dalam Maxima, Anda mungkin ingin menggunakan sintaksis
asli Maxima. Anda dapat melakukannya dengan ":::".
\end{eulercomment}
\begin{eulerprompt}
>::: av:g$ av^2;
\end{eulerprompt}
\begin{euleroutput}
  
                                     2
                                    g
  
\end{euleroutput}
\begin{eulerprompt}
>fx &= x^3*exp(x), $fx
\end{eulerprompt}
\begin{euleroutput}
  
                                   3  x
                                  x  E
  
\end{euleroutput}
\begin{eulerformula}
\[
x^3\,e^{x}
\]
\end{eulerformula}
\begin{eulercomment}
Variabel tersebut dapat digunakan dalam ekspresi simbolik lainnya.
Perhatikan bahwa dalam perintah berikut sisi kanan \&= dievaluasi
sebelum penugasan ke Fx.
\end{eulercomment}
\begin{eulerprompt}
>&(fx with x=5), $%, &float(%)
\end{eulerprompt}
\begin{euleroutput}
  
                                       5
                                  125 E
  
\end{euleroutput}
\begin{eulerformula}
\[
125\,e^5
\]
\end{eulerformula}
\begin{euleroutput}
  
                            18551.64488782208
  
\end{euleroutput}
\begin{eulerprompt}
>fx(5)
\end{eulerprompt}
\begin{euleroutput}
  18551.6448878
\end{euleroutput}
\begin{eulercomment}
Untuk mengevaluasi ekspresi dengan nilai variabel tertentu, Anda dapat
menggunakan operator "with".

Baris perintah berikut juga menunjukkan bahwa Maxima dapat
mengevaluasi ekspresi secara numerik dengan float().
\end{eulercomment}
\begin{eulerprompt}
>&(fx with x=10)-(fx with x=5), &float(%)
\end{eulerprompt}
\begin{euleroutput}
  
                                  10        5
                            1000 E   - 125 E
  
  
                           2.20079141499189e+7
  
\end{euleroutput}
\begin{eulerprompt}
>$factor(diff(fx,x,2))
\end{eulerprompt}
\begin{eulerformula}
\[
x\,\left(x^2+6\,x+6\right)\,e^{x}
\]
\end{eulerformula}
\begin{eulercomment}
Untuk mendapatkan kode Latex untuk suatu ekspresi, Anda dapat
menggunakan perintah tex.
\end{eulercomment}
\begin{eulerprompt}
>tex(fx)
\end{eulerprompt}
\begin{euleroutput}
  x^3\(\backslash\),e^\{x\}
\end{euleroutput}
\begin{eulercomment}
Ekspresi simbolik dapat dievaluasi seperti halnya ekspresi numerik.
\end{eulercomment}
\begin{eulerprompt}
>fx(0.5)
\end{eulerprompt}
\begin{euleroutput}
  0.206090158838
\end{euleroutput}
\begin{eulercomment}
Dalam ekspresi simbolik, ini tidak berfungsi, karena Maxima tidak
mendukungnya. Sebagai gantinya, gunakan sintaks "with" (bentuk yang
lebih baik dari perintah at(...) dari Maxima).
\end{eulercomment}
\begin{eulerprompt}
>$&fx with x=1/2
\end{eulerprompt}
\begin{eulerformula}
\[
\frac{\sqrt{e}}{8}
\]
\end{eulerformula}
\begin{eulercomment}
Penugasan tersebut juga dapat bersifat simbolis.
\end{eulercomment}
\begin{eulerprompt}
>$&fx with x=1+t
\end{eulerprompt}
\begin{eulerformula}
\[
\left(t+1\right)^3\,e^{t+1}
\]
\end{eulerformula}
\begin{eulercomment}
Perintah solve memecahkan ekspresi simbolik untuk variabel dalam
Maxima. Hasilnya adalah vektor solusi.
\end{eulercomment}
\begin{eulerprompt}
>$&solve(x^2+x=4,x)
\end{eulerprompt}
\begin{eulerformula}
\[
\left[ x=\frac{-\sqrt{17}-1}{2} , x=\frac{\sqrt{17}-1}{2} \right] 
\]
\end{eulerformula}
\begin{eulercomment}
Bandingkan dengan perintah numerik "solve" di Euler, yang memerlukan
nilai awal, dan secara opsional nilai target.
\end{eulercomment}
\begin{eulerprompt}
>solve("x^2+x",1,y=4)
\end{eulerprompt}
\begin{euleroutput}
  1.56155281281
\end{euleroutput}
\begin{eulercomment}
Nilai numerik dari solusi simbolik dapat dihitung dengan mengevaluasi
hasil simbolik. Euler akan membaca ulang penugasan x= dst. Jika Anda
tidak memerlukan hasil numerik untuk perhitungan lebih lanjut, Anda
juga dapat membiarkan Maxima menemukan nilai numeriknya.
\end{eulercomment}
\begin{eulerprompt}
>sol &= solve(x^2+2*x=4,x); $&sol, sol(), $&float(sol)
\end{eulerprompt}
\begin{eulerformula}
\[
\left[ x=-\sqrt{5}-1 , x=\sqrt{5}-1 \right] 
\]
\end{eulerformula}
\begin{euleroutput}
  [-3.23607,  1.23607]
\end{euleroutput}
\begin{eulerformula}
\[
\left[ x=-3.23606797749979 , x=1.23606797749979 \right] 
\]
\end{eulerformula}
\begin{eulercomment}
Untuk mendapatkan solusi simbolis yang spesifik, seseorang dapat
menggunakan "with" dan indeks.
\end{eulercomment}
\begin{eulerprompt}
>$&solve(x^2+x=1,x), x2 &= x with %[2]; $&x2
\end{eulerprompt}
\begin{eulerformula}
\[
\frac{\sqrt{5}-1}{2}
\]
\end{eulerformula}
\eulerimg{1}{images/Alifia Rahmawati_23030630044_LaTex & Markdown-023-large.png}
\begin{eulercomment}
Untuk menyelesaikan sistem persamaan, gunakan vektor persamaan.
Hasilnya adalah vektor solusi.
\end{eulercomment}
\begin{eulerprompt}
>sol &= solve([x+y=3,x^2+y^2=5],[x,y]); $&sol, $&x*y with sol[1]
\end{eulerprompt}
\begin{eulerformula}
\[
2
\]
\end{eulerformula}
\eulerimg{0}{images/Alifia Rahmawati_23030630044_LaTex & Markdown-025-large.png}
\begin{eulercomment}
Ekspresi simbolik dapat memiliki tanda, yang menunjukkan perlakuan
khusus di Maxima. Beberapa tanda dapat digunakan sebagai perintah
juga, yang lainnya tidak. Tanda ditambahkan dengan "\textbar{}" (bentuk yang
lebih baik dari "ev(...,flags)")
\end{eulercomment}
\begin{eulerprompt}
>$& diff((x^3-1)/(x+1),x) //turunan bentuk pecahan
\end{eulerprompt}
\begin{eulerformula}
\[
\frac{3\,x^2}{x+1}-\frac{x^3-1}{\left(x+1\right)^2}
\]
\end{eulerformula}
\begin{eulerprompt}
>$& diff((x^3-1)/(x+1),x) | ratsimp //menyederhanakan pecahan
\end{eulerprompt}
\begin{eulerformula}
\[
\frac{2\,x^3+3\,x^2+1}{x^2+2\,x+1}
\]
\end{eulerformula}
\begin{eulerprompt}
>$&factor(%)
\end{eulerprompt}
\begin{eulerformula}
\[
\frac{2\,x^3+3\,x^2+1}{\left(x+1\right)^2}
\]
\end{eulerformula}
\eulerheading{Fungsi}
\begin{eulercomment}
Dalam EMT, fungsi adalah program yang didefinisikan dengan perintah
"function". Fungsi ini dapat berupa fungsi satu baris atau fungsi
multibaris.\\
ungsi satu baris dapat berupa numerik atau simbolik. Fungsi satu baris
numerik didefinisikan oleh ":=".
\end{eulercomment}
\begin{eulerprompt}
>function f(x) := x*sqrt(x^2+1)
\end{eulerprompt}
\begin{eulercomment}
Sebagai gambaran umum, kami tampilkan semua definisi yang mungkin
untuk fungsi satu baris. Suatu fungsi dapat dievaluasi seperti fungsi
Euler bawaan lainnya.
\end{eulercomment}
\begin{eulerprompt}
>f(2)
\end{eulerprompt}
\begin{euleroutput}
  4.472135955
\end{euleroutput}
\begin{eulercomment}
Fungsi ini juga akan bekerja untuk vektor, mematuhi bahasa matriks
Euler, karena ekspresi yang digunakan dalam fungsi tersebut
divektorkan.
\end{eulercomment}
\begin{eulerprompt}
>f(0:0.1:1)
\end{eulerprompt}
\begin{euleroutput}
  [0,  0.100499,  0.203961,  0.313209,  0.430813,  0.559017,  0.699714,
  0.854459,  1.0245,  1.21083,  1.41421]
\end{euleroutput}
\begin{eulercomment}
Fungsi dapat diplot. Alih-alih ekspresi, kita hanya perlu memberikan
nama fungsi.

Berbeda dengan ekspresi simbolik atau numerik, nama fungsi harus
diberikan dalam bentuk string.
\end{eulercomment}
\begin{eulerprompt}
>solve("f",1,y=1)
\end{eulerprompt}
\begin{euleroutput}
  0.786151377757
\end{euleroutput}
\begin{eulercomment}
Secara default, jika Anda perlu menimpa fungsi bawaan, Anda harus
menambahkan kata kunci "overwrite". Menimpakan fungsi bawaan berbahaya
dan dapat menyebabkan masalah bagi fungsi lain yang bergantung
padanya.

Anda masih dapat memanggil fungsi bawaan sebagai "\_...", jika itu
adalah fungsi di inti Euler.
\end{eulercomment}
\begin{eulerprompt}
>function overwrite sin (x) := _sin(x°) // redine sine in degrees
>sin(45)
\end{eulerprompt}
\begin{euleroutput}
  0.707106781187
\end{euleroutput}
\begin{eulercomment}
Sebaiknya kita hilangkan pendefinisian ulang dosa ini.
\end{eulercomment}
\begin{eulerprompt}
>forget sin; sin(pi/4)
\end{eulerprompt}
\begin{euleroutput}
  0.707106781187
\end{euleroutput}
\eulersubheading{Parameter Default}
\begin{eulercomment}
Fungsi numerik dapat memiliki parameter default.
\end{eulercomment}
\begin{eulerprompt}
>function f(x,a=1) := a*x^2
\end{eulerprompt}
\begin{eulercomment}
Mengabaikan parameter ini akan menggunakan nilai default.
\end{eulercomment}
\begin{eulerprompt}
>f(4)
\end{eulerprompt}
\begin{euleroutput}
  16
\end{euleroutput}
\begin{eulercomment}
Mengaturnya akan menimpa nilai default.
\end{eulercomment}
\begin{eulerprompt}
>f(4,5)
\end{eulerprompt}
\begin{euleroutput}
  80
\end{euleroutput}
\begin{eulercomment}
Parameter yang ditetapkan juga akan menimpanya. Ini digunakan oleh
banyak fungsi Euler seperti plot2d, plot3d.
\end{eulercomment}
\begin{eulerprompt}
>f(4,a=1)
\end{eulerprompt}
\begin{euleroutput}
  16
\end{euleroutput}
\begin{eulercomment}
Jika suatu variabel bukan parameter, maka variabel tersebut harus
bersifat global. Fungsi satu baris dapat melihat variabel global.
\end{eulercomment}
\begin{eulerprompt}
>function f(x) := a*x^2
>a=6; f(2)
\end{eulerprompt}
\begin{euleroutput}
  24
\end{euleroutput}
\begin{eulercomment}
Namun, parameter yang ditetapkan akan menggantikan nilai global.

Jika argumen tidak ada dalam daftar parameter yang telah ditetapkan
sebelumnya, argumen tersebut harus dideklarasikan dengan ":="!
\end{eulercomment}
\begin{eulerprompt}
>f(2,a:=5)
\end{eulerprompt}
\begin{euleroutput}
  20
\end{euleroutput}
\begin{eulercomment}
Fungsi simbolik didefinisikan dengan "\&=". Fungsi ini didefinisikan
dalam Euler dan Maxima, dan berfungsi di kedua dunia. Ekspresi yang
mendefinisikan dijalankan melalui Maxima sebelum definisi.
\end{eulercomment}
\begin{eulerprompt}
>function g(x) &= x^3-x*exp(-x); $&g(x)
\end{eulerprompt}
\begin{eulerformula}
\[
x^3-x\,e^ {- x }
\]
\end{eulerformula}
\begin{eulercomment}
Fungsi simbolik dapat digunakan dalam ekspresi simbolik.
\end{eulercomment}
\begin{eulerprompt}
>$&diff(g(x),x), $&% with x=4/3
\end{eulerprompt}
\begin{eulerformula}
\[
\frac{e^ {- \frac{4}{3} }}{3}+\frac{16}{3}
\]
\end{eulerformula}
\eulerimg{1}{images/Alifia Rahmawati_23030630044_LaTex & Markdown-031-large.png}
\begin{eulercomment}
Mereka juga dapat digunakan dalam ekspresi numerik. Tentu saja, ini
hanya akan berfungsi jika EMT dapat menginterpretasikan semua hal di
dalam fungsi tersebut.
\end{eulercomment}
\begin{eulerprompt}
>g(5+g(1))
\end{eulerprompt}
\begin{euleroutput}
  178.635099908
\end{euleroutput}
\begin{eulercomment}
Mereka dapat digunakan untuk mendefinisikan fungsi atau ekspresi
simbolis lainnya.
\end{eulercomment}
\begin{eulerprompt}
>function G(x) &= factor(integrate(g(x),x)); $&G(c) // integrate: mengintegralkan
\end{eulerprompt}
\begin{eulerformula}
\[
\frac{e^ {- c }\,\left(c^4\,e^{c}+4\,c+4\right)}{4}
\]
\end{eulerformula}
\begin{eulerprompt}
>solve(&g(x),0.5)
\end{eulerprompt}
\begin{euleroutput}
  0.703467422498
\end{euleroutput}
\begin{eulercomment}
Berikut ini juga berfungsi, karena Euler menggunakan ekspresi simbolik
dalam fungsi g, jika tidak menemukan variabel simbolik g, dan jika ada
fungsi simbolik g.
\end{eulercomment}
\begin{eulerprompt}
>solve(&g,0.5)
\end{eulerprompt}
\begin{euleroutput}
  0.703467422498
\end{euleroutput}
\begin{eulerprompt}
>function P(x,n) &= (2*x-1)^n; $&P(x,n)
\end{eulerprompt}
\begin{eulerformula}
\[
\left(2\,x-1\right)^{n}
\]
\end{eulerformula}
\begin{eulerprompt}
>function Q(x,n) &= (x+2)^n; $&Q(x,n)
\end{eulerprompt}
\begin{eulerformula}
\[
\left(x+2\right)^{n}
\]
\end{eulerformula}
\begin{eulerprompt}
>$&P(x,4), $&expand(%)
\end{eulerprompt}
\begin{eulerformula}
\[
16\,x^4-32\,x^3+24\,x^2-8\,x+1
\]
\end{eulerformula}
\eulerimg{0}{images/Alifia Rahmawati_23030630044_LaTex & Markdown-036-large.png}
\begin{eulerprompt}
>P(3,4)
\end{eulerprompt}
\begin{euleroutput}
  625
\end{euleroutput}
\begin{eulerprompt}
>$&P(x,4)+ Q(x,3), $&expand(%)
\end{eulerprompt}
\begin{eulerformula}
\[
16\,x^4-31\,x^3+30\,x^2+4\,x+9
\]
\end{eulerformula}
\eulerimg{0}{images/Alifia Rahmawati_23030630044_LaTex & Markdown-038-large.png}
\begin{eulerprompt}
>$&P(x,4)-Q(x,3), $&expand(%), $&factor(%)
\end{eulerprompt}
\begin{eulerformula}
\[
16\,x^4-33\,x^3+18\,x^2-20\,x-7
\]
\end{eulerformula}
\eulerimg{0}{images/Alifia Rahmawati_23030630044_LaTex & Markdown-040-large.png}
\eulerimg{0}{images/Alifia Rahmawati_23030630044_LaTex & Markdown-041-large.png}
\begin{eulerprompt}
>$&P(x,4)*Q(x,3), $&expand(%), $&factor(%)
\end{eulerprompt}
\begin{eulerformula}
\[
\left(x+2\right)^3\,\left(2\,x-1\right)^4
\]
\end{eulerformula}
\eulerimg{0}{images/Alifia Rahmawati_23030630044_LaTex & Markdown-043-large.png}
\eulerimg{0}{images/Alifia Rahmawati_23030630044_LaTex & Markdown-044-large.png}
\begin{eulerprompt}
>$&P(x,4)/Q(x,1), $&expand(%), $&factor(%)
\end{eulerprompt}
\begin{eulerformula}
\[
\frac{\left(2\,x-1\right)^4}{x+2}
\]
\end{eulerformula}
\eulerimg{1}{images/Alifia Rahmawati_23030630044_LaTex & Markdown-046-large.png}
\eulerimg{1}{images/Alifia Rahmawati_23030630044_LaTex & Markdown-047-large.png}
\begin{eulerprompt}
>function f(x) &= x^3-x; $&f(x)
\end{eulerprompt}
\begin{eulerformula}
\[
x^3-x
\]
\end{eulerformula}
\begin{eulercomment}
Dengan \&= fungsinya bersifat simbolis, dan dapat digunakan dalam
ekspresi simbolis lainnya.
\end{eulercomment}
\begin{eulerprompt}
>$&integrate(f(x),x)
\end{eulerprompt}
\begin{eulerformula}
\[
\frac{x^4}{4}-\frac{x^2}{2}
\]
\end{eulerformula}
\begin{eulercomment}
Dengan := fungsinya bersifat numerik. Contoh yang bagus adalah
integral tentu seperti

\end{eulercomment}
\begin{eulerformula}
\[
f(x) = \int_1^x t^t \, dt,
\]
\end{eulerformula}
\begin{eulercomment}
yang tidak dapat dievaluasi secara simbolis.

Jika kita mendefinisikan ulang fungsi tersebut dengan kata kunci
"map", fungsi tersebut dapat digunakan untuk vektor x. Secara
internal, fungsi tersebut dipanggil untuk semua nilai x satu kali, dan
hasilnya disimpan dalam vektor.
\end{eulercomment}
\begin{eulerprompt}
>function map f(x) := integrate("x^x",1,x)
>f(0:0.5:2)
\end{eulerprompt}
\begin{euleroutput}
  [-0.783431,  -0.410816,  0,  0.676863,  2.05045]
\end{euleroutput}
\begin{eulercomment}
Fungsi dapat memiliki nilai default untuk parameter.
\end{eulercomment}
\begin{eulerprompt}
>function mylog (x,base=10) := ln(x)/ln(base);
\end{eulerprompt}
\begin{eulercomment}
Sekarang fungsi tersebut dapat dipanggil dengan atau tanpa parameter
"dasar".
\end{eulercomment}
\begin{eulerprompt}
>mylog(100), mylog(2^6.7,2)
\end{eulerprompt}
\begin{euleroutput}
  2
  6.7
\end{euleroutput}
\begin{eulercomment}
Selain itu, dimungkinkan untuk menggunakan parameter yang ditetapkan.
\end{eulercomment}
\begin{eulerprompt}
>mylog(E^2,base=E)
\end{eulerprompt}
\begin{euleroutput}
  2
\end{euleroutput}
\begin{eulercomment}
Sering kali, kita ingin menggunakan fungsi untuk vektor di satu
tempat, dan untuk elemen individual di tempat lain. Hal ini
dimungkinkan dengan parameter vektor.
\end{eulercomment}
\begin{eulerprompt}
>function f([a,b]) &= a^2+b^2-a*b+b; $&f(a,b), $&f(x,y)
\end{eulerprompt}
\begin{eulerformula}
\[
y^2-x\,y+y+x^2
\]
\end{eulerformula}
\eulerimg{0}{images/Alifia Rahmawati_23030630044_LaTex & Markdown-051-large.png}
\begin{eulercomment}
Such a symbolic function can be used for symbolic variables.

But the function can also be used for a numerical vector.
\end{eulercomment}
\begin{eulerprompt}
>v=[3,4]; f(v)
\end{eulerprompt}
\begin{euleroutput}
  17
\end{euleroutput}
\begin{eulercomment}
There are also purely symbolic functions, which cannot be used numerically.
\end{eulercomment}
\begin{eulerprompt}
>function lapl(expr,x,y) &&= diff(expr,x,2)+diff(expr,y,2)//turunan parsial kedua
\end{eulerprompt}
\begin{euleroutput}
  
                   diff(expr, y, 2) + diff(expr, x, 2)
  
\end{euleroutput}
\begin{eulerprompt}
>$&realpart((x+I*y)^4), $&lapl(%,x,y)
\end{eulerprompt}
\begin{eulerformula}
\[
0
\]
\end{eulerformula}
\eulerimg{0}{images/Alifia Rahmawati_23030630044_LaTex & Markdown-053-large.png}
\begin{eulercomment}
Namun tentu saja, mereka dapat digunakan dalam ekspresi simbolik atau
dalam definisi fungsi simbolik.
\end{eulercomment}
\begin{eulerprompt}
>function f(x,y) &= factor(lapl((x+y^2)^5,x,y)); $&f(x,y)
\end{eulerprompt}
\begin{eulerformula}
\[
10\,\left(y^2+x\right)^3\,\left(9\,y^2+x+2\right)
\]
\end{eulerformula}
\begin{eulercomment}
Singkatnya

- \&= mendefinisikan fungsi simbolik,\\
- := mendefinisikan fungsi numerik,\\
- \&\&= mendefinisikan fungsi simbolik murni.

\begin{eulercomment}
\eulerheading{Menyelesaikan Ekspresi}
\begin{eulercomment}
Ekspresi dapat diselesaikan secara numerik dan simbolik.

Untuk menyelesaikan ekspresi sederhana dari satu variabel, kita dapat
menggunakan fungsi solve(). Fungsi ini memerlukan nilai awal untuk
memulai pencarian. Secara internal, solve() menggunakan metode secant.
\end{eulercomment}
\begin{eulerprompt}
>solve("x^2-2",1)
\end{eulerprompt}
\begin{euleroutput}
  1.41421356237
\end{euleroutput}
\begin{eulercomment}
Ini juga berlaku untuk ekspresi simbolik. Ambil fungsi berikut.
\end{eulercomment}
\begin{eulerprompt}
>$&solve(x^2=2,x)
\end{eulerprompt}
\begin{eulerformula}
\[
\left[ x=-\sqrt{2} , x=\sqrt{2} \right] 
\]
\end{eulerformula}
\begin{eulerprompt}
>$&solve(x^2-2,x)
\end{eulerprompt}
\begin{eulerformula}
\[
\left[ x=-\sqrt{2} , x=\sqrt{2} \right] 
\]
\end{eulerformula}
\begin{eulerprompt}
>$&solve(a*x^2+b*x+c=0,x)
\end{eulerprompt}
\begin{eulerformula}
\[
\left[ x=\frac{-\sqrt{b^2-4\,a\,c}-b}{2\,a} , x=\frac{\sqrt{b^2-4\,  a\,c}-b}{2\,a} \right] 
\]
\end{eulerformula}
\begin{eulerprompt}
>$&solve([a*x+b*y=c,d*x+e*y=f],[x,y])
\end{eulerprompt}
\begin{eulerformula}
\[
\left[ \left[ x=-\frac{c\,e}{b\,\left(d-5\right)-a\,e} , y=\frac{c  \,\left(d-5\right)}{b\,\left(d-5\right)-a\,e} \right]  \right] 
\]
\end{eulerformula}
\begin{eulerprompt}
>px &= 4*x^8+x^7-x^4-x; $&px
\end{eulerprompt}
\begin{eulerformula}
\[
4\,x^8+x^7-x^4-x
\]
\end{eulerformula}
\begin{eulercomment}
Sekarang kita cari titik, di mana polinomialnya adalah 2. Dalam
solve(), nilai target default y=0 dapat diubah dengan variabel yang
ditetapkan.\\
ita gunakan y=2 dan periksa dengan mengevaluasi polinomial pada hasil
sebelumnya.
\end{eulercomment}
\begin{eulerprompt}
>solve(px,1,y=2), px(%)
\end{eulerprompt}
\begin{euleroutput}
  0.966715594851
  2
\end{euleroutput}
\begin{eulercomment}
Memecahkan ekspresi simbolik dalam bentuk simbolik akan menghasilkan
daftar solusi. Kami menggunakan pemecah simbolik solve() yang
disediakan oleh Maxima.
\end{eulercomment}
\begin{eulerprompt}
>sol &= solve(x^2-x-1,x); $&sol
\end{eulerprompt}
\begin{eulerformula}
\[
\left[ x=\frac{1-\sqrt{5}}{2} , x=\frac{\sqrt{5}+1}{2} \right] 
\]
\end{eulerformula}
\begin{eulercomment}
Cara termudah untuk mendapatkan nilai numerik adalah dengan
mengevaluasi solusi secara numerik seperti sebuah ekspresi.
\end{eulercomment}
\begin{eulerprompt}
>longest sol()
\end{eulerprompt}
\begin{euleroutput}
      -0.6180339887498949       1.618033988749895 
\end{euleroutput}
\begin{eulercomment}
Untuk menggunakan solusi secara simbolis dalam ekspresi lain, cara
termudah adalah "dengan".
\end{eulercomment}
\begin{eulerprompt}
>$&x^2 with sol[1], $&expand(x^2-x-1 with sol[2])
\end{eulerprompt}
\begin{eulerformula}
\[
0
\]
\end{eulerformula}
\eulerimg{0}{images/Alifia Rahmawati_23030630044_LaTex & Markdown-062-large.png}
\begin{eulercomment}
Memecahkan sistem persamaan secara simbolis dapat dilakukan dengan
vektor persamaan dan penyelesai simbolis solve(). Jawabannya adalah
daftar persamaan.
\end{eulercomment}
\begin{eulerprompt}
>$&solve([x+y=2,x^3+2*y+x=4],[x,y])
\end{eulerprompt}
\begin{eulerformula}
\[
\left[ \left[ x=-1 , y=3 \right]  , \left[ x=1 , y=1 \right]  ,   \left[ x=0 , y=2 \right]  \right] 
\]
\end{eulerformula}
\begin{eulercomment}
Fungsi f() dapat melihat variabel global. Namun, sering kali kita
ingin menggunakan parameter lokal.

\end{eulercomment}
\begin{eulerformula}
\[
a^x-x^a = 0,1
\]
\end{eulerformula}
\begin{eulercomment}
dengan a=3.
\end{eulercomment}
\begin{eulerprompt}
>function f(x,a) := x^a-a^x;
\end{eulerprompt}
\begin{eulercomment}
Salah satu cara untuk meneruskan parameter tambahan ke f() adalah
dengan menggunakan daftar dengan nama fungsi dan parameter (cara
lainnya adalah parameter titik koma).
\end{eulercomment}
\begin{eulerprompt}
>solve(\{\{"f",3\}\},2,y=0.1)
\end{eulerprompt}
\begin{euleroutput}
  2.54116291558
\end{euleroutput}
\begin{eulercomment}
Ini juga berlaku untuk ekspresi. Namun, elemen daftar bernama harus
digunakan. (Informasi lebih lanjut tentang daftar ada di tutorial
tentang sintaks EMT).
\end{eulercomment}
\begin{eulerprompt}
>solve(\{\{"x^a-a^x",a=3\}\},2,y=0.1)
\end{eulerprompt}
\begin{euleroutput}
  2.54116291558
\end{euleroutput}
\eulerheading{Menyelesaikan Pertidaksamaan}
\begin{eulercomment}
Untuk menyelesaikan pertidaksamaan, EMT tidak akan dapat melakukannya,
melainkan dengan bantuan Maxima, artinya secara eksak (simbolik).
Perintah Maxima yang digunakan adalah fourier\_elim(), yang harus
dipanggil dengan perintah "load(fourier\_elim)" terlebih dahulu.
\end{eulercomment}
\begin{eulerprompt}
>&load(fourier_elim)
\end{eulerprompt}
\begin{euleroutput}
  
          D:/New folder/Euler x64/maxima/share/maxima/5.35.1/share/four\(\backslash\)
  ier_elim/fourier_elim.lisp
  
\end{euleroutput}
\begin{eulerprompt}
>$&fourier_elim([x^2 - 1>0],[x]) // x^2-1 > 0
\end{eulerprompt}
\begin{eulerformula}
\[
\left[ 1<x \right] \lor \left[ x<-1 \right] 
\]
\end{eulerformula}
\begin{eulerprompt}
>$&fourier_elim([x^2 - 1<0],[x]) // x^2-1 < 0
\end{eulerprompt}
\begin{eulerformula}
\[
\left[ -1<x , x<1 \right] 
\]
\end{eulerformula}
\begin{eulerprompt}
>$&fourier_elim([x^2 - 1 # 0],[x]) // x^-1 <> 0
\end{eulerprompt}
\begin{eulerformula}
\[
\left[ -1<x , x<1 \right] \lor \left[ 1<x \right] \lor \left[ x<-1   \right] 
\]
\end{eulerformula}
\begin{eulerprompt}
>$&fourier_elim([x # 6],[x])
\end{eulerprompt}
\begin{eulerformula}
\[
\left[ x<6 \right] \lor \left[ 6<x \right] 
\]
\end{eulerformula}
\begin{eulerprompt}
>$&fourier_elim([x < 1, x > 1],[x]) // tidak memiliki penyelesaian
\end{eulerprompt}
\begin{eulerformula}
\[
{\it emptyset}
\]
\end{eulerformula}
\begin{eulerprompt}
>$&fourier_elim([minf < x, x < inf],[x]) // solusinya R
\end{eulerprompt}
\begin{eulerformula}
\[
{\it universalset}
\]
\end{eulerformula}
\begin{eulerprompt}
>$&fourier_elim([x^3 - 1 > 0],[x])
\end{eulerprompt}
\begin{eulerformula}
\[
\left[ 1<x , x^2+x+1>0 \right] \lor \left[ x<1 , -x^2-x-1>0   \right] 
\]
\end{eulerformula}
\begin{eulerprompt}
>$&fourier_elim([cos(x) < 1/2],[x]) // ??? gagal
\end{eulerprompt}
\begin{eulerformula}
\[
\left[ 1-2\,\cos x>0 \right] 
\]
\end{eulerformula}
\begin{eulerprompt}
>$&fourier_elim([y-x < 5, x - y < 7, 10 < y],[x,y]) // sistem pertidaksamaan
\end{eulerprompt}
\begin{eulerformula}
\[
\left[ y-5<x , x<y+7 , 10<y \right] 
\]
\end{eulerformula}
\begin{eulerprompt}
>$&fourier_elim([y-x < 5, x - y < 7, 10 < y],[y,x])
\end{eulerprompt}
\begin{eulerformula}
\[
\left[ {\it max}\left(10 , x-7\right)<y , y<x+5 , 5<x \right] 
\]
\end{eulerformula}
\begin{eulerprompt}
>$&fourier_elim((x + y < 5) and (x - y >8),[x,y])
\end{eulerprompt}
\begin{eulerformula}
\[
\left[ y+8<x , x<5-y , y<-\frac{3}{2} \right] 
\]
\end{eulerformula}
\begin{eulerprompt}
>$&fourier_elim(((x + y < 5) and x < 1) or  (x - y >8),[x,y])
\end{eulerprompt}
\begin{eulerformula}
\[
\left[ y+8<x \right] \lor \left[ x<{\it min}\left(1 , 5-y\right)   \right] 
\]
\end{eulerformula}
\begin{eulerprompt}
>&fourier_elim([max(x,y) > 6, x # 8, abs(y-1) > 12],[x,y])
\end{eulerprompt}
\begin{euleroutput}
  
          [6 < x, x < 8, y < - 11] or [8 < x, y < - 11]
   or [x < 8, 13 < y] or [x = y, 13 < y] or [8 < x, x < y, 13 < y]
   or [y < x, 13 < y]
  
\end{euleroutput}
\begin{eulerprompt}
>$&fourier_elim([(x+6)/(x-9) <= 6],[x])
\end{eulerprompt}
\begin{eulerformula}
\[
\left[ x=12 \right] \lor \left[ 12<x \right] \lor \left[ x<9   \right] 
\]
\end{eulerformula}
\eulerheading{Bahasa Matriks}
\begin{eulercomment}
Dokumentasi inti EMT berisi pembahasan terperinci tentang bahasa
matriks Euler.

Vektor dan matriks dimasukkan dengan tanda kurung siku, elemen
dipisahkan dengan koma, baris dipisahkan dengan titik koma.
\end{eulercomment}
\begin{eulerprompt}
>A=[1,2;3,4]
\end{eulerprompt}
\begin{euleroutput}
              1             2 
              3             4 
\end{euleroutput}
\begin{eulercomment}
Produk matriks dilambangkan dengan sebuah titik.
\end{eulercomment}
\begin{eulerprompt}
>b=[3;4]
\end{eulerprompt}
\begin{euleroutput}
              3 
              4 
\end{euleroutput}
\begin{eulerprompt}
>b' // transpose b
\end{eulerprompt}
\begin{euleroutput}
  [3,  4]
\end{euleroutput}
\begin{eulerprompt}
>inv(A) //inverse A
\end{eulerprompt}
\begin{euleroutput}
             -2             1 
            1.5          -0.5 
\end{euleroutput}
\begin{eulerprompt}
>A.b //perkalian matriks
\end{eulerprompt}
\begin{euleroutput}
             11 
             25 
\end{euleroutput}
\begin{eulerprompt}
>A.inv(A)
\end{eulerprompt}
\begin{euleroutput}
              1             0 
              0             1 
\end{euleroutput}
\begin{eulercomment}
Poin utama dari bahasa matriks adalah bahwa semua fungsi dan operator
bekerja elemen demi elemen.
\end{eulercomment}
\begin{eulerprompt}
>A.A
\end{eulerprompt}
\begin{euleroutput}
              7            10 
             15            22 
\end{euleroutput}
\begin{eulerprompt}
>A^2 //perpangkatan elemen2 A
\end{eulerprompt}
\begin{euleroutput}
              1             4 
              9            16 
\end{euleroutput}
\begin{eulerprompt}
>A.A.A
\end{eulerprompt}
\begin{euleroutput}
             37            54 
             81           118 
\end{euleroutput}
\begin{eulerprompt}
>power(A,3) //perpangkatan matriks
\end{eulerprompt}
\begin{euleroutput}
             37            54 
             81           118 
\end{euleroutput}
\begin{eulerprompt}
>A/A //pembagian elemen-elemen matriks yang seletak
\end{eulerprompt}
\begin{euleroutput}
              1             1 
              1             1 
\end{euleroutput}
\begin{eulerprompt}
>A/b //pembagian elemen2 A oleh elemen2 b kolom demi kolom (karena b vektor kolom)
\end{eulerprompt}
\begin{euleroutput}
       0.333333      0.666667 
           0.75             1 
\end{euleroutput}
\begin{eulerprompt}
>A\(\backslash\)b // hasilkali invers A dan b, A^(-1)b 
\end{eulerprompt}
\begin{euleroutput}
             -2 
            2.5 
\end{euleroutput}
\begin{eulerprompt}
>inv(A).b
\end{eulerprompt}
\begin{euleroutput}
             -2 
            2.5 
\end{euleroutput}
\begin{eulerprompt}
>A\(\backslash\)A   //A^(-1)A
\end{eulerprompt}
\begin{euleroutput}
              1             0 
              0             1 
\end{euleroutput}
\begin{eulerprompt}
>inv(A).A
\end{eulerprompt}
\begin{euleroutput}
              1             0 
              0             1 
\end{euleroutput}
\begin{eulerprompt}
>A*A //perkalin elemen-elemen matriks seletak
\end{eulerprompt}
\begin{euleroutput}
              1             4 
              9            16 
\end{euleroutput}
\begin{eulercomment}
Ini bukan hasil perkalian matriks, tetapi perkalian elemen demi
elemen. Hal yang sama berlaku untuk vektor.
\end{eulercomment}
\begin{eulerprompt}
>b^2 // perpangkatan elemen-elemen matriks/vektor
\end{eulerprompt}
\begin{euleroutput}
              9 
             16 
\end{euleroutput}
\begin{eulercomment}
Jika salah satu operan merupakan vektor atau skalar, ia diekspansi
dengan cara alami.
\end{eulercomment}
\begin{eulerprompt}
>2*A
\end{eulerprompt}
\begin{euleroutput}
              2             4 
              6             8 
\end{euleroutput}
\begin{eulercomment}
Jika salah satu operan merupakan vektor atau skalar, ia diekspansi
dengan cara alami. Misalnya, jika operan merupakan vektor kolom,
elemen-elemennya diterapkan ke semua baris A.
\end{eulercomment}
\begin{eulerprompt}
>[1,2]*A
\end{eulerprompt}
\begin{euleroutput}
              1             4 
              3             8 
\end{euleroutput}
\begin{eulercomment}
Jika itu adalah vektor baris maka diterapkan ke semua kolom A.
\end{eulercomment}
\begin{eulerprompt}
>A*[2,3]
\end{eulerprompt}
\begin{euleroutput}
              2             6 
              6            12 
\end{euleroutput}
\begin{eulercomment}
Seseorang dapat membayangkan perkalian ini seolah-olah vektor baris v
telah diduplikasi untuk membentuk matriks berukuran sama dengan A.
\end{eulercomment}
\begin{eulerprompt}
>dup([1,2],2) // dup: menduplikasi/menggandakan vektor [1,2] sebanyak 2 kali (baris)
\end{eulerprompt}
\begin{euleroutput}
              1             2 
              1             2 
\end{euleroutput}
\begin{eulerprompt}
>A*dup([1,2],2) 
\end{eulerprompt}
\begin{euleroutput}
              1             4 
              3             8 
\end{euleroutput}
\begin{eulercomment}
Hal ini juga berlaku untuk dua vektor, yang satu merupakan vektor
baris dan yang lainnya merupakan vektor kolom. Kita menghitung i*j
untuk i,j dari 1 hingga 5. Caranya adalah dengan mengalikan 1:5 dengan
transposenya. Bahasa matriks Euler secara otomatis menghasilkan tabel
nilai.
\end{eulercomment}
\begin{eulerprompt}
>(1:5)*(1:5)' // hasilkali elemen-elemen vektor baris dan vektor kolom
\end{eulerprompt}
\begin{euleroutput}
              1             2             3             4             5 
              2             4             6             8            10 
              3             6             9            12            15 
              4             8            12            16            20 
              5            10            15            20            25 
\end{euleroutput}
\begin{eulercomment}
Sekali lagi, ingatlah bahwa ini bukan produk matriks!
\end{eulercomment}
\begin{eulerprompt}
>(1:5).(1:5)' // hasilkali vektor baris dan vektor kolom
\end{eulerprompt}
\begin{euleroutput}
  55
\end{euleroutput}
\begin{eulerprompt}
>sum((1:5)*(1:5)) // sama hasilnya
\end{eulerprompt}
\begin{euleroutput}
  55
\end{euleroutput}
\begin{eulercomment}
Bahkan operator seperti \textless{} atau == bekerja dengan cara yang sama.
\end{eulercomment}
\begin{eulerprompt}
>(1:10)<6 // menguji elemen-elemen yang kurang dari 6
\end{eulerprompt}
\begin{euleroutput}
  [1,  1,  1,  1,  1,  0,  0,  0,  0,  0]
\end{euleroutput}
\begin{eulercomment}
Misalnya, kita dapat menghitung jumlah elemen yang memenuhi kondisi
tertentu dengan fungsi sum().
\end{eulercomment}
\begin{eulerprompt}
>sum((1:10)<6) // banyak elemen yang kurang dari 6
\end{eulerprompt}
\begin{euleroutput}
  5
\end{euleroutput}
\begin{eulercomment}
Euler memiliki operator perbandingan, seperti "==", yang memeriksa
kesetaraan.\\
Kita memperoleh vektor 0 dan 1, di mana 1 berarti benar.
\end{eulercomment}
\begin{eulerprompt}
>t=(1:10)^2; t==25 //menguji elemen2 t yang sama dengan 25 (hanya ada 1)
\end{eulerprompt}
\begin{euleroutput}
  [0,  0,  0,  0,  1,  0,  0,  0,  0,  0]
\end{euleroutput}
\begin{eulercomment}
Dari vektor tersebut, "nonzeros" memilih elemen yang bukan nol.\\
Dalam kasus ini, kita memperoleh indeks semua elemen yang lebih besar
dari 50.
\end{eulercomment}
\begin{eulerprompt}
>nonzeros(t>50) //indeks elemen2 t yang lebih besar daripada 50
\end{eulerprompt}
\begin{euleroutput}
  [8,  9,  10]
\end{euleroutput}
\begin{eulercomment}
Tentu saja, kita dapat menggunakan vektor indeks ini untuk mendapatkan
nilai yang sesuai dalam t.
\end{eulercomment}
\begin{eulerprompt}
>t[nonzeros(t>50)] //elemen2 t yang lebih besar daripada 50
\end{eulerprompt}
\begin{euleroutput}
  [64,  81,  100]
\end{euleroutput}
\begin{eulercomment}
Sebagai contoh, mari kita cari semua kuadrat angka 1 hingga 1000,
yaitu 5 modulo 11 dan 3 modulo 13.
\end{eulercomment}
\begin{eulerprompt}
>t=1:1000; nonzeros(mod(t^2,11)==5 && mod(t^2,13)==3)
\end{eulerprompt}
\begin{euleroutput}
  [4,  48,  95,  139,  147,  191,  238,  282,  290,  334,  381,  425,
  433,  477,  524,  568,  576,  620,  667,  711,  719,  763,  810,  854,
  862,  906,  953,  997]
\end{euleroutput}
\begin{eulercomment}
EMT tidak sepenuhnya efektif untuk komputasi integer. Ia menggunakan
floating point presisi ganda secara internal. Namun, ia sering kali
sangat berguna.

Kita dapat memeriksa keutamaan. Mari kita cari tahu, berapa banyak
kuadrat ditambah 1 yang merupakan bilangan prima.
\end{eulercomment}
\begin{eulerprompt}
>t=1:1000; length(nonzeros(isprime(t^2+1)))
\end{eulerprompt}
\begin{euleroutput}
  112
\end{euleroutput}
\begin{eulercomment}
Fungsi nonzeros() hanya berfungsi untuk vektor. Untuk matriks, ada
mnonzeros().
\end{eulercomment}
\begin{eulerprompt}
>seed(2); A=random(3,4)
\end{eulerprompt}
\begin{euleroutput}
       0.765761      0.401188      0.406347      0.267829 
        0.13673      0.390567      0.495975      0.952814 
       0.548138      0.006085      0.444255      0.539246 
\end{euleroutput}
\begin{eulercomment}
Mengembalikan indeks elemen, yang bukan nol.
\end{eulercomment}
\begin{eulerprompt}
>k=mnonzeros(A<0.4) //indeks elemen2 A yang kurang dari 0,4
\end{eulerprompt}
\begin{euleroutput}
              1             4 
              2             1 
              2             2 
              3             2 
\end{euleroutput}
\begin{eulercomment}
Indeks ini dapat digunakan untuk menetapkan elemen pada nilai
tertentu.
\end{eulercomment}
\begin{eulerprompt}
>mset(A,k,0) //mengganti elemen2 suatu matriks pada indeks tertentu
\end{eulerprompt}
\begin{euleroutput}
       0.765761      0.401188      0.406347             0 
              0             0      0.495975      0.952814 
       0.548138             0      0.444255      0.539246 
\end{euleroutput}
\begin{eulercomment}
Fungsi mset() juga dapat mengatur elemen pada indeks ke entri matriks
lainnya.
\end{eulercomment}
\begin{eulerprompt}
>mset(A,k,-random(size(A)))
\end{eulerprompt}
\begin{euleroutput}
       0.765761      0.401188      0.406347     -0.126917 
      -0.122404     -0.691673      0.495975      0.952814 
       0.548138     -0.483902      0.444255      0.539246 
\end{euleroutput}
\begin{eulercomment}
Dan adalah mungkin untuk mendapatkan unsur-unsur dalam sebuah vektor.
\end{eulercomment}
\begin{eulerprompt}
>mget(A,k)
\end{eulerprompt}
\begin{euleroutput}
  [0.267829,  0.13673,  0.390567,  0.006085]
\end{euleroutput}
\begin{eulercomment}
Fungsi lain yang berguna adalah extrema, yang mengembalikan nilai
minimal dan maksimal di setiap baris matriks dan posisinya.
\end{eulercomment}
\begin{eulerprompt}
>ex=extrema(A)
\end{eulerprompt}
\begin{euleroutput}
       0.267829             4      0.765761             1 
        0.13673             1      0.952814             4 
       0.006085             2      0.548138             1 
\end{euleroutput}
\begin{eulercomment}
Kita dapat menggunakan ini untuk mengekstrak nilai maksimal pada
setiap baris.
\end{eulercomment}
\begin{eulerprompt}
>ex[,3]'
\end{eulerprompt}
\begin{euleroutput}
  [0.765761,  0.952814,  0.548138]
\end{euleroutput}
\begin{eulercomment}
Ini tentu saja sama dengan fungsi max().
\end{eulercomment}
\begin{eulerprompt}
>max(A)'
\end{eulerprompt}
\begin{euleroutput}
  [0.765761,  0.952814,  0.548138]
\end{euleroutput}
\begin{eulercomment}
Tetapi dengan mget(), kita dapat mengekstrak indeks dan menggunakan
informasi ini untuk mengekstrak elemen pada posisi yang sama dari
matriks lain.
\end{eulercomment}
\begin{eulerprompt}
>j=(1:rows(A))'|ex[,4], mget(-A,j)
\end{eulerprompt}
\begin{euleroutput}
              1             1 
              2             4 
              3             1 
  [-0.765761,  -0.952814,  -0.548138]
\end{euleroutput}
\eulerheading{Fungsi Matriks Lainnya (Membangun Matriks)}
\begin{eulercomment}
Untuk membangun sebuah matriks, kita dapat menumpuk satu matriks di
atas matriks lainnya. Jika keduanya tidak memiliki jumlah kolom yang
sama, matriks yang lebih pendek akan diisi dengan 0.
\end{eulercomment}
\begin{eulerprompt}
>v=1:3; v_v
\end{eulerprompt}
\begin{euleroutput}
              1             2             3 
              1             2             3 
\end{euleroutput}
\begin{eulercomment}
Dengan cara yang sama, kita dapat menempelkan suatu matriks ke sisi
lain yang berdampingan, jika keduanya memiliki jumlah baris yang sama.
\end{eulercomment}
\begin{eulerprompt}
>A=random(3,4); A|v'
\end{eulerprompt}
\begin{euleroutput}
       0.032444     0.0534171      0.595713      0.564454             1 
        0.83916      0.175552      0.396988       0.83514             2 
      0.0257573      0.658585      0.629832      0.770895             3 
\end{euleroutput}
\begin{eulercomment}
Jika tidak memiliki jumlah baris yang sama, matriks yang lebih pendek
diisi dengan 0.

Ada pengecualian untuk aturan ini. Bilangan riil yang dilampirkan ke
matriks akan digunakan sebagai kolom yang diisi dengan bilangan riil
tersebut.
\end{eulercomment}
\begin{eulerprompt}
>A|1
\end{eulerprompt}
\begin{euleroutput}
       0.032444     0.0534171      0.595713      0.564454             1 
        0.83916      0.175552      0.396988       0.83514             1 
      0.0257573      0.658585      0.629832      0.770895             1 
\end{euleroutput}
\begin{eulercomment}
Dimungkinkan untuk membuat matriks dari vektor baris dan kolom.
\end{eulercomment}
\begin{eulerprompt}
>[v;v]
\end{eulerprompt}
\begin{euleroutput}
              1             2             3 
              1             2             3 
\end{euleroutput}
\begin{eulerprompt}
>[v',v']
\end{eulerprompt}
\begin{euleroutput}
              1             1 
              2             2 
              3             3 
\end{euleroutput}
\begin{eulercomment}
Tujuan utama dari ini adalah untuk menafsirkan vektor ekspresi untuk
vektor kolom.
\end{eulercomment}
\begin{eulerprompt}
>"[x,x^2]"(v')
\end{eulerprompt}
\begin{euleroutput}
              1             1 
              2             4 
              3             9 
\end{euleroutput}
\begin{eulercomment}
Untuk mendapatkan ukuran A, kita dapat menggunakan fungsi berikut.
\end{eulercomment}
\begin{eulerprompt}
>C=zeros(2,4); rows(C), cols(C), size(C), length(C)
\end{eulerprompt}
\begin{euleroutput}
  2
  4
  [2,  4]
  4
\end{euleroutput}
\begin{eulercomment}
Untuk vektor, ada length().
\end{eulercomment}
\begin{eulerprompt}
>length(2:10)
\end{eulerprompt}
\begin{euleroutput}
  9
\end{euleroutput}
\begin{eulercomment}
Ada banyak fungsi lain yang menghasilkan matriks.
\end{eulercomment}
\begin{eulerprompt}
>ones(2,2)
\end{eulerprompt}
\begin{euleroutput}
              1             1 
              1             1 
\end{euleroutput}
\begin{eulercomment}
Ini juga dapat digunakan dengan satu parameter. Untuk mendapatkan
vektor dengan angka selain 1, gunakan yang berikut ini.
\end{eulercomment}
\begin{eulerprompt}
>ones(5)*6
\end{eulerprompt}
\begin{euleroutput}
  [6,  6,  6,  6,  6]
\end{euleroutput}
\begin{eulercomment}
Matriks bilangan acak juga dapat dihasilkan dengan acak (distribusi
seragam) atau normal (distribusi Gauß).
\end{eulercomment}
\begin{eulerprompt}
>random(2,2)
\end{eulerprompt}
\begin{euleroutput}
        0.66566      0.831835 
          0.977      0.544258 
\end{euleroutput}
\begin{eulercomment}
Berikut adalah fungsi berguna lainnya, yang merestrukturisasi
elemen-elemen suatu matriks menjadi matriks lain.
\end{eulercomment}
\begin{eulerprompt}
>redim(1:9,3,3) // menyusun elemen2 1, 2, 3, ..., 9 ke bentuk matriks 3x3
\end{eulerprompt}
\begin{euleroutput}
              1             2             3 
              4             5             6 
              7             8             9 
\end{euleroutput}
\begin{eulercomment}
Dengan fungsi berikut, kita dapat menggunakan ini dan fungsi dup untuk
menulis fungsi rep(), yang mengulang vektor n kali.
\end{eulercomment}
\begin{eulerprompt}
>function rep(v,n) := redim(dup(v,n),1,n*cols(v))
\end{eulerprompt}
\begin{eulercomment}
Mari kita menguji.
\end{eulercomment}
\begin{eulerprompt}
>rep(1:3,5)
\end{eulerprompt}
\begin{euleroutput}
  [1,  2,  3,  1,  2,  3,  1,  2,  3,  1,  2,  3,  1,  2,  3]
\end{euleroutput}
\begin{eulercomment}
Fungsi multdup() menduplikasi elemen suatu vektor.
\end{eulercomment}
\begin{eulerprompt}
>multdup(1:3,5), multdup(1:3,[2,3,2])
\end{eulerprompt}
\begin{euleroutput}
  [1,  1,  1,  1,  1,  2,  2,  2,  2,  2,  3,  3,  3,  3,  3]
  [1,  1,  2,  2,  2,  3,  3]
\end{euleroutput}
\begin{eulercomment}
Fungsi flipx() dan flipy() membalikkan urutan baris atau kolom
matriks. Yaitu, fungsi flipx() membalik secara horizontal.
\end{eulercomment}
\begin{eulerprompt}
>flipx(1:5) //membalik elemen2 vektor baris
\end{eulerprompt}
\begin{euleroutput}
  [5,  4,  3,  2,  1]
\end{euleroutput}
\begin{eulercomment}
Untuk rotasi, Euler memiliki rotleft() dan rotright().
\end{eulercomment}
\begin{eulerprompt}
>rotleft(1:5) // memutar elemen2 vektor baris
\end{eulerprompt}
\begin{euleroutput}
  [2,  3,  4,  5,  1]
\end{euleroutput}
\begin{eulercomment}
Fungsi khusus adalah drop(v,i), yang menghapus elemen dengan indeks di
i dari vektor v.
\end{eulercomment}
\begin{eulerprompt}
>drop(10:20,3)
\end{eulerprompt}
\begin{euleroutput}
  [10,  11,  13,  14,  15,  16,  17,  18,  19,  20]
\end{euleroutput}
\begin{eulercomment}
Perhatikan bahwa vektor i dalam drop(v,i) merujuk pada indeks elemen
dalam v, bukan nilai elemen. Jika Anda ingin menghapus elemen, Anda
perlu menemukan elemen terlebih dahulu. Fungsi indexof(v,x) dapat
digunakan untuk menemukan elemen x dalam vektor v yang diurutkan.
\end{eulercomment}
\begin{eulerprompt}
>v=primes(50), i=indexof(v,10:20), drop(v,i)
\end{eulerprompt}
\begin{euleroutput}
  [2,  3,  5,  7,  11,  13,  17,  19,  23,  29,  31,  37,  41,  43,  47]
  [0,  5,  0,  6,  0,  0,  0,  7,  0,  8,  0]
  [2,  3,  5,  7,  23,  29,  31,  37,  41,  43,  47]
\end{euleroutput}
\begin{eulercomment}
Seperti yang Anda lihat, tidak ada salahnya menyertakan indeks di luar
rentang (seperti 0), indeks ganda, atau indeks yang tidak diurutkan.
\end{eulercomment}
\begin{eulerprompt}
>drop(1:10,shuffle([0,0,5,5,7,12,12]))
\end{eulerprompt}
\begin{euleroutput}
  [1,  2,  3,  4,  6,  8,  9,  10]
\end{euleroutput}
\begin{eulercomment}
Ada beberapa fungsi khusus untuk mengatur diagonal atau membuat
matriks diagonal.

Kita mulai dengan matriks identitas.
\end{eulercomment}
\begin{eulerprompt}
>A=id(5) // matriks identitas 5x5
\end{eulerprompt}
\begin{euleroutput}
              1             0             0             0             0 
              0             1             0             0             0 
              0             0             1             0             0 
              0             0             0             1             0 
              0             0             0             0             1 
\end{euleroutput}
\begin{eulercomment}
Kemudian kita atur diagonal bawah (-1) menjadi 1:4.
\end{eulercomment}
\begin{eulerprompt}
>setdiag(A,-1,1:4) //mengganti diagonal di bawah diagonal utama
\end{eulerprompt}
\begin{euleroutput}
              1             0             0             0             0 
              1             1             0             0             0 
              0             2             1             0             0 
              0             0             3             1             0 
              0             0             0             4             1 
\end{euleroutput}
\begin{eulercomment}
Perhatikan bahwa kita tidak mengubah matriks A. Kita mendapatkan
matriks baru sebagai hasil dari setdiag().

Berikut ini adalah fungsi yang mengembalikan matriks tri-diagonal.
\end{eulercomment}
\begin{eulerprompt}
>function tridiag (n,a,b,c) := setdiag(setdiag(b*id(n),1,c),-1,a); ...
>tridiag(5,1,2,3)
\end{eulerprompt}
\begin{euleroutput}
              2             3             0             0             0 
              1             2             3             0             0 
              0             1             2             3             0 
              0             0             1             2             3 
              0             0             0             1             2 
\end{euleroutput}
\begin{eulercomment}
Diagonal matriks juga dapat diekstraksi dari matriks. Untuk
menunjukkan hal ini, kami merestrukturisasi vektor 1:9 menjadi matriks
3x3.
\end{eulercomment}
\begin{eulerprompt}
>A=redim(1:9,3,3)
\end{eulerprompt}
\begin{euleroutput}
              1             2             3 
              4             5             6 
              7             8             9 
\end{euleroutput}
\begin{eulercomment}
Sekarang kita dapat mengekstrak diagonalnya.
\end{eulercomment}
\begin{eulerprompt}
>d=getdiag(A,0)
\end{eulerprompt}
\begin{euleroutput}
  [1,  5,  9]
\end{euleroutput}
\begin{eulercomment}
Misalnya, kita dapat membagi matriks berdasarkan diagonalnya. Bahasa
matriks memastikan bahwa vektor kolom d diterapkan ke matriks baris
demi baris.
\end{eulercomment}
\begin{eulerprompt}
>fraction A/d'
\end{eulerprompt}
\begin{euleroutput}
          1         2         3 
        4/5         1       6/5 
        7/9       8/9         1 
\end{euleroutput}
\eulerheading{Vektorisasi}
\begin{eulercomment}
Hampir semua fungsi di Euler juga berfungsi untuk masukan matriks dan
vektor, jika ini masuk akal.

Misalnya, fungsi sqrt() menghitung akar kuadrat dari semua elemen
vektor atau matriks.
\end{eulercomment}
\begin{eulerprompt}
>sqrt(1:3)
\end{eulerprompt}
\begin{euleroutput}
  [1,  1.41421,  1.73205]
\end{euleroutput}
\begin{eulercomment}
Jadi Anda dapat dengan mudah membuat tabel nilai. Ini adalah salah
satu cara untuk memplot fungsi (alternatifnya menggunakan ekspresi).
\end{eulercomment}
\begin{eulerprompt}
>x=1:0.01:5; y=log(x)/x^2; // terlalu panjang untuk ditampikan
\end{eulerprompt}
\begin{eulercomment}
Dengan ini dan operator titik dua a:delta:b, vektor nilai fungsi dapat
dibuat dengan mudah.

Dalam contoh berikut, kita buat vektor nilai t[i] dengan spasi 0,1
dari -1 hingga 1. Kemudian kita buat vektor nilai fungsi

\end{eulercomment}
\begin{eulerformula}
\[
s = t^3-t
\]
\end{eulerformula}
\begin{eulerprompt}
>t=-1:0.1:1; s=t^3-t
\end{eulerprompt}
\begin{euleroutput}
  [0,  0.171,  0.288,  0.357,  0.384,  0.375,  0.336,  0.273,  0.192,
  0.099,  0,  -0.099,  -0.192,  -0.273,  -0.336,  -0.375,  -0.384,
  -0.357,  -0.288,  -0.171,  0]
\end{euleroutput}
\begin{eulercomment}
EMT mengembangkan operator untuk skalar, vektor, dan matriks dengan
cara yang jelas.

Misalnya, vektor kolom dikalikan vektor baris akan mengembang menjadi
matriks, jika operator diterapkan. Berikut ini, v' adalah vektor yang
ditransposisikan (vektor kolom).
\end{eulercomment}
\begin{eulerprompt}
>shortest (1:5)*(1:5)'
\end{eulerprompt}
\begin{euleroutput}
       1      2      3      4      5 
       2      4      6      8     10 
       3      6      9     12     15 
       4      8     12     16     20 
       5     10     15     20     25 
\end{euleroutput}
\begin{eulercomment}
Perhatikan bahwa ini sangat berbeda dari perkalian matriks. Perkalian
matriks dilambangkan dengan titik "." dalam EMT.
\end{eulercomment}
\begin{eulerprompt}
>(1:5).(1:5)'
\end{eulerprompt}
\begin{euleroutput}
  55
\end{euleroutput}
\begin{eulercomment}
Secara default, vektor baris dicetak dalam format ringkas.
\end{eulercomment}
\begin{eulerprompt}
>[1,2,3,4]
\end{eulerprompt}
\begin{euleroutput}
  [1,  2,  3,  4]
\end{euleroutput}
\begin{eulercomment}
Untuk matriks, operator khusus . menunjukkan perkalian matriks, dan A'
menunjukkan transposisi. Matriks 1x1 dapat digunakan seperti bilangan
riil.
\end{eulercomment}
\begin{eulerprompt}
>v:=[1,2]; v.v', %^2
\end{eulerprompt}
\begin{euleroutput}
  5
  25
\end{euleroutput}
\begin{eulercomment}
Untuk mentranspos suatu matriks, kita menggunakan tanda apostrof.
\end{eulercomment}
\begin{eulerprompt}
>v=1:4; v'
\end{eulerprompt}
\begin{euleroutput}
              1 
              2 
              3 
              4 
\end{euleroutput}
\begin{eulercomment}
Jadi kita dapat menghitung matriks A dikali vektor b.
\end{eulercomment}
\begin{eulerprompt}
>A=[1,2,3,4;5,6,7,8]; A.v'
\end{eulerprompt}
\begin{euleroutput}
             30 
             70 
\end{euleroutput}
\begin{eulercomment}
Perhatikan bahwa v masih merupakan vektor baris. Jadi v'.v berbeda
dari v.v'.
\end{eulercomment}
\begin{eulerprompt}
>v'.v
\end{eulerprompt}
\begin{euleroutput}
              1             2             3             4 
              2             4             6             8 
              3             6             9            12 
              4             8            12            16 
\end{euleroutput}
\begin{eulercomment}
v.v' menghitung norma v kuadrat untuk vektor baris v. Hasilnya adalah
vektor 1x1, yang bekerja seperti bilangan riil.
\end{eulercomment}
\begin{eulerprompt}
>v.v'
\end{eulerprompt}
\begin{euleroutput}
  30
\end{euleroutput}
\begin{eulercomment}
Ada juga norma fungsi (bersama dengan banyak fungsi Aljabar Linear
lainnya).
\end{eulercomment}
\begin{eulerprompt}
>norm(v)^2
\end{eulerprompt}
\begin{euleroutput}
  30
\end{euleroutput}
\begin{eulercomment}
Operator dan fungsi mematuhi bahasa matriks Euler.

Berikut ini ringkasan aturannya.

- Fungsi yang diterapkan pada vektor atau matriks diterapkan pada
setiap elemen.

- Operator yang beroperasi pada dua matriks dengan ukuran yang sama
diterapkan secara berpasangan pada elemen-elemen matriks.

- Jika kedua matriks memiliki dimensi yang berbeda, keduanya
diekspansi dengan cara yang masuk akal, sehingga memiliki ukuran yang
sama.

Misalnya, nilai skalar dikalikan vektor mengalikan nilai dengan setiap
elemen vektor. Atau matriks dikalikan vektor (dengan *, bukan .)
mengekspansi vektor ke ukuran matriks dengan menduplikasinya.

Berikut ini adalah kasus sederhana dengan operator \textasciicircum{}.
\end{eulercomment}
\begin{eulerprompt}
>[1,2,3]^2
\end{eulerprompt}
\begin{euleroutput}
  [1,  4,  9]
\end{euleroutput}
\begin{eulercomment}
Berikut ini adalah kasus yang lebih rumit. Vektor baris dikalikan
vektor kolom, keduanya diekspansi dengan cara menduplikasi.
\end{eulercomment}
\begin{eulerprompt}
>v:=[1,2,3]; v*v'
\end{eulerprompt}
\begin{euleroutput}
              1             2             3 
              2             4             6 
              3             6             9 
\end{euleroutput}
\begin{eulercomment}
Perhatikan bahwa produk skalar menggunakan produk matriks, bukan *!
\end{eulercomment}
\begin{eulerprompt}
>v.v'
\end{eulerprompt}
\begin{euleroutput}
  14
\end{euleroutput}
\begin{eulercomment}
Ada banyak fungsi untuk matriks. Kami memberikan daftar singkatnya.
Anda harus merujuk ke dokumentasi untuk informasi lebih lanjut tentang
perintah-perintah ini.

sum,prod menghitung jumlah dan hasil perkalian baris-baris\\
cumsum,cumprod melakukan hal yang sama secara kumulatif\\
menghitung nilai ekstrem dari setiap baris\\
extrema mengembalikan vektor dengan informasi ekstrem\\
diag(A,i) mengembalikan diagonal ke-i\\
setdiag(A,i,v) menetapkan diagonal ke-i\\
id(n) matriks identitas\\
det(A) determinan\\
charpoly(A) polinomial karakteristik\\
eigenvalues(A) nilai eigen
\end{eulercomment}
\begin{eulerprompt}
>v*v, sum(v*v), cumsum(v*v)
\end{eulerprompt}
\begin{euleroutput}
  [1,  4,  9]
  14
  [1,  5,  14]
\end{euleroutput}
\begin{eulercomment}
Operator : menghasilkan vektor baris dengan spasi yang sama, secara
opsional dengan ukuran langkah.
\end{eulercomment}
\begin{eulerprompt}
>1:4, 1:2:10
\end{eulerprompt}
\begin{euleroutput}
  [1,  2,  3,  4]
  [1,  3,  5,  7,  9]
\end{euleroutput}
\begin{eulercomment}
Untuk menggabungkan matriks dan vektor ada operator "\textbar{}" dan "\_".
\end{eulercomment}
\begin{eulerprompt}
>[1,2,3]|[4,5], [1,2,3]_1
\end{eulerprompt}
\begin{euleroutput}
  [1,  2,  3,  4,  5]
              1             2             3 
              1             1             1 
\end{euleroutput}
\begin{eulercomment}
Elemen-elemen suatu matriks disebut dengan "A[i,j]".
\end{eulercomment}
\begin{eulerprompt}
>A:=[1,2,3;4,5,6;7,8,9]; A[2,3]
\end{eulerprompt}
\begin{euleroutput}
  6
\end{euleroutput}
\begin{eulercomment}
Untuk vektor baris atau kolom, v[i] adalah elemen ke-i dari vektor.
Untuk matriks, ini mengembalikan baris ke-i lengkap dari matriks.
\end{eulercomment}
\begin{eulerprompt}
>v:=[2,4,6,8]; v[3], A[3]
\end{eulerprompt}
\begin{euleroutput}
  6
  [7,  8,  9]
\end{euleroutput}
\begin{eulercomment}
Indeks juga dapat berupa vektor baris indeks. : menunjukkan semua
indeks.
\end{eulercomment}
\begin{eulerprompt}
>v[1:2], A[:,2]
\end{eulerprompt}
\begin{euleroutput}
  [2,  4]
              2 
              5 
              8 
\end{euleroutput}
\begin{eulercomment}
Bentuk singkat dari : adalah menghilangkan indeks sepenuhnya.
\end{eulercomment}
\begin{eulerprompt}
>A[,2:3]
\end{eulerprompt}
\begin{euleroutput}
              2             3 
              5             6 
              8             9 
\end{euleroutput}
\begin{eulercomment}
Untuk tujuan vektorisasi, elemen-elemen matriks dapat diakses
seolah-olah mereka adalah vektor.
\end{eulercomment}
\begin{eulerprompt}
>A\{4\}
\end{eulerprompt}
\begin{euleroutput}
  4
\end{euleroutput}
\begin{eulercomment}
Matriks juga dapat diratakan, menggunakan fungsi redim(). Hal ini
diimplementasikan dalam fungsi flatten().
\end{eulercomment}
\begin{eulerprompt}
>redim(A,1,prod(size(A))), flatten(A)
\end{eulerprompt}
\begin{euleroutput}
  [1,  2,  3,  4,  5,  6,  7,  8,  9]
  [1,  2,  3,  4,  5,  6,  7,  8,  9]
\end{euleroutput}
\begin{eulercomment}
Untuk menggunakan matriks pada tabel, mari kita atur ulang ke format
default, dan hitung tabel nilai sinus dan kosinus. Perhatikan bahwa
sudut dalam radian secara default.
\end{eulercomment}
\begin{eulerprompt}
>defformat; w=0°:45°:360°; w=w'; deg(w)
\end{eulerprompt}
\begin{euleroutput}
              0 
             45 
             90 
            135 
            180 
            225 
            270 
            315 
            360 
\end{euleroutput}
\begin{eulercomment}
Sekarang kita tambahkan kolom ke matriks.
\end{eulercomment}
\begin{eulerprompt}
>M = deg(w)|w|cos(w)|sin(w)
\end{eulerprompt}
\begin{euleroutput}
              0             0             1             0 
             45      0.785398      0.707107      0.707107 
             90        1.5708             0             1 
            135       2.35619     -0.707107      0.707107 
            180       3.14159            -1             0 
            225       3.92699     -0.707107     -0.707107 
            270       4.71239             0            -1 
            315       5.49779      0.707107     -0.707107 
            360       6.28319             1             0 
\end{euleroutput}
\begin{eulercomment}
Dengan menggunakan bahasa matriks, kita dapat membuat beberapa tabel
dari beberapa fungsi sekaligus.

Dalam contoh berikut, kita menghitung t[j]\textasciicircum{}i untuk i dari 1 hingga n.
Kita memperoleh matriks, yang setiap barisnya merupakan tabel t\textasciicircum{}i
untuk satu i. Yaitu, matriks tersebut memiliki elemen lateks: a\_\{i,j\}
= t\_j\textasciicircum{}i, \textbackslash{}quad 1 \textbackslash{}le j \textbackslash{}le 101, \textbackslash{}quad 1 \textbackslash{}le i \textbackslash{}le n

Fungsi yang tidak berfungsi untuk input vektor harus "divektorkan".
Ini dapat dicapai dengan kata kunci "map" dalam definisi fungsi.
Kemudian fungsi tersebut akan dievaluasi untuk setiap elemen parameter
vektor.

Integrasi numerik integr() hanya berfungsi untuk batas interval
skalar. Jadi, kita perlu memvektorkannya.
\end{eulercomment}
\begin{eulerprompt}
>function map f(x) := integrate("x^x",1,x)
\end{eulerprompt}
\begin{eulercomment}
The "map" keyword vectorizes the function. The function will now work\\
for vectors of numbers.
\end{eulercomment}
\begin{eulerprompt}
>f([1:5])
\end{eulerprompt}
\begin{euleroutput}
  [0,  2.05045,  13.7251,  113.336,  1241.03]
\end{euleroutput}
\eulerheading{Sub-Matriks dan Elemen Matriks}
\begin{eulercomment}
Untuk mengakses elemen matriks, gunakan notasi tanda kurung.
\end{eulercomment}
\begin{eulerprompt}
>A=[1,2,3;4,5,6;7,8,9], A[2,2]
\end{eulerprompt}
\begin{euleroutput}
              1             2             3 
              4             5             6 
              7             8             9 
  5
\end{euleroutput}
\begin{eulercomment}
Kita dapat mengakses baris matriks yang lengkap.
\end{eulercomment}
\begin{eulerprompt}
>A[2]
\end{eulerprompt}
\begin{euleroutput}
  [4,  5,  6]
\end{euleroutput}
\begin{eulercomment}
Dalam kasus vektor baris atau kolom, ini mengembalikan elemen vektor.
\end{eulercomment}
\begin{eulerprompt}
>v=1:3; v[2]
\end{eulerprompt}
\begin{euleroutput}
  2
\end{euleroutput}
\begin{eulercomment}
Untuk memastikan, Anda mendapatkan baris pertama untuk matriks 1xn dan
mxn, tentukan semua kolom menggunakan indeks kedua yang kosong.
\end{eulercomment}
\begin{eulerprompt}
>A[2,]
\end{eulerprompt}
\begin{euleroutput}
  [4,  5,  6]
\end{euleroutput}
\begin{eulercomment}
Jika indeks adalah vektor indeks, Euler akan mengembalikan baris
matriks yang sesuai.

Di sini kita menginginkan baris pertama dan kedua dari A.
\end{eulercomment}
\begin{eulerprompt}
>A[[1,2]]
\end{eulerprompt}
\begin{euleroutput}
              1             2             3 
              4             5             6 
\end{euleroutput}
\begin{eulercomment}
Kita bahkan dapat menyusun ulang A menggunakan vektor indeks. Untuk
lebih tepatnya, kita tidak mengubah A di sini, tetapi menghitung versi
A yang telah disusun ulang.
\end{eulercomment}
\begin{eulerprompt}
>A[[3,2,1]]
\end{eulerprompt}
\begin{euleroutput}
              7             8             9 
              4             5             6 
              1             2             3 
\end{euleroutput}
\begin{eulercomment}
Trik indeks juga berfungsi dengan kolom.

Contoh ini memilih semua baris A dan kolom kedua dan ketiga.
\end{eulercomment}
\begin{eulerprompt}
>A[1:3,2:3]
\end{eulerprompt}
\begin{euleroutput}
              2             3 
              5             6 
              8             9 
\end{euleroutput}
\begin{eulercomment}
Untuk singkatan ":" menunjukkan semua indeks baris atau kolom.
\end{eulercomment}
\begin{eulerprompt}
>A[:,3]
\end{eulerprompt}
\begin{euleroutput}
              3 
              6 
              9 
\end{euleroutput}
\begin{eulercomment}
Atau, biarkan indeks pertama kosong.
\end{eulercomment}
\begin{eulerprompt}
>A[,2:3]
\end{eulerprompt}
\begin{euleroutput}
              2             3 
              5             6 
              8             9 
\end{euleroutput}
\begin{eulercomment}
Kita juga bisa mendapatkan baris terakhir A.
\end{eulercomment}
\begin{eulerprompt}
>A[-1]
\end{eulerprompt}
\begin{euleroutput}
  [7,  8,  9]
\end{euleroutput}
\begin{eulercomment}
Sekarang mari kita ubah elemen A dengan menetapkan submatriks A ke
suatu nilai. Hal ini pada kenyataannya mengubah matriks A yang
tersimpan.
\end{eulercomment}
\begin{eulerprompt}
>A[1,1]=4
\end{eulerprompt}
\begin{euleroutput}
              4             2             3 
              4             5             6 
              7             8             9 
\end{euleroutput}
\begin{eulercomment}
Kita juga dapat menetapkan nilai ke baris A.
\end{eulercomment}
\begin{eulerprompt}
>A[1]=[-1,-1,-1]
\end{eulerprompt}
\begin{euleroutput}
             -1            -1            -1 
              4             5             6 
              7             8             9 
\end{euleroutput}
\begin{eulercomment}
Kita bahkan dapat menetapkannya ke submatriks jika ukurannya tepat.
\end{eulercomment}
\begin{eulerprompt}
>A[1:2,1:2]=[5,6;7,8]
\end{eulerprompt}
\begin{euleroutput}
              5             6            -1 
              7             8             6 
              7             8             9 
\end{euleroutput}
\begin{eulercomment}
Selain itu, beberapa jalan pintas diperbolehkan.
\end{eulercomment}
\begin{eulerprompt}
>A[1:2,1:2]=0
\end{eulerprompt}
\begin{euleroutput}
              0             0            -1 
              0             0             6 
              7             8             9 
\end{euleroutput}
\begin{eulercomment}
Peringatan: Indeks yang tidak sesuai batas akan mengembalikan matriks
kosong, atau pesan kesalahan, tergantung pada pengaturan sistem. Pesan
kesalahan adalah standar. Namun, perlu diingat bahwa indeks negatif
dapat digunakan untuk mengakses elemen matriks yang dihitung dari
akhir.
\end{eulercomment}
\begin{eulerprompt}
>A[4]
\end{eulerprompt}
\begin{euleroutput}
  Row index 4 out of bounds!
  Error in:
  A[4] ...
      ^
\end{euleroutput}
\eulerheading{Sortir dan Acak}
\begin{eulercomment}
Fungsi sort() mengurutkan vektor baris.
\end{eulercomment}
\begin{eulerprompt}
>sort([5,6,4,8,1,9])
\end{eulerprompt}
\begin{euleroutput}
  [1,  4,  5,  6,  8,  9]
\end{euleroutput}
\begin{eulercomment}
Seringkali perlu untuk mengetahui indeks vektor yang diurutkan dalam
vektor asli. Ini dapat digunakan untuk menyusun ulang vektor lain
dengan cara yang sama.

Mari kita acak sebuah vektor.
\end{eulercomment}
\begin{eulerprompt}
>v=shuffle(1:10)
\end{eulerprompt}
\begin{euleroutput}
  [4,  5,  10,  6,  8,  9,  1,  7,  2,  3]
\end{euleroutput}
\begin{eulercomment}
Indeks berisi urutan v yang tepat.
\end{eulercomment}
\begin{eulerprompt}
>\{vs,ind\}=sort(v); v[ind]
\end{eulerprompt}
\begin{euleroutput}
  [1,  2,  3,  4,  5,  6,  7,  8,  9,  10]
\end{euleroutput}
\begin{eulercomment}
Ini juga berlaku untuk vektor string.
\end{eulercomment}
\begin{eulerprompt}
>s=["a","d","e","a","aa","e"]
\end{eulerprompt}
\begin{euleroutput}
  a
  d
  e
  a
  aa
  e
\end{euleroutput}
\begin{eulerprompt}
>\{ss,ind\}=sort(s); ss
\end{eulerprompt}
\begin{euleroutput}
  a
  a
  aa
  d
  e
  e
\end{euleroutput}
\begin{eulercomment}
Seperti yang Anda lihat, posisi entri ganda agak acak.
\end{eulercomment}
\begin{eulerprompt}
>ind
\end{eulerprompt}
\begin{euleroutput}
  [4,  1,  5,  2,  6,  3]
\end{euleroutput}
\begin{eulercomment}
Fungsi unik mengembalikan daftar yang diurutkan dari elemen unik suatu
vektor.
\end{eulercomment}
\begin{eulerprompt}
>intrandom(1,10,10), unique(%)
\end{eulerprompt}
\begin{euleroutput}
  [4,  4,  9,  2,  6,  5,  10,  6,  5,  1]
  [1,  2,  4,  5,  6,  9,  10]
\end{euleroutput}
\begin{eulercomment}
Ini juga berlaku untuk vektor string.
\end{eulercomment}
\begin{eulerprompt}
>unique(s)
\end{eulerprompt}
\begin{euleroutput}
  a
  aa
  d
  e
\end{euleroutput}
\eulerheading{Aljabar Linier}
\begin{eulercomment}
EMT memiliki banyak fungsi untuk memecahkan sistem linier, sistem
renggang, atau masalah regresi.

Untuk sistem linier Ax=b, Anda dapat menggunakan algoritma Gauss,
matriks invers, atau kecocokan linier. Operator A\textbackslash{}b menggunakan versi
algoritma Gauss.
\end{eulercomment}
\begin{eulerprompt}
>A=[1,2;3,4]; b=[5;6]; A\(\backslash\)b
\end{eulerprompt}
\begin{euleroutput}
             -4 
            4.5 
\end{euleroutput}
\begin{eulercomment}
Untuk contoh lain, kita buat matriks 200x200 dan jumlah barisnya.
Kemudian kita selesaikan Ax=b menggunakan matriks invers. Kita ukur
kesalahan sebagai deviasi maksimal semua elemen dari 1, yang tentu
saja merupakan solusi yang benar.
\end{eulercomment}
\begin{eulerprompt}
>A=normal(200,200); b=sum(A); longest totalmax(abs(inv(A).b-1))
\end{eulerprompt}
\begin{euleroutput}
    8.790745908981989e-13 
\end{euleroutput}
\begin{eulercomment}
Jika sistem tidak mempunyai solusi, penyesuaian linier meminimalkan
norma kesalahan Ax-b.
\end{eulercomment}
\begin{eulerprompt}
>A=[1,2,3;4,5,6;7,8,9]
\end{eulerprompt}
\begin{euleroutput}
              1             2             3 
              4             5             6 
              7             8             9 
\end{euleroutput}
\begin{eulercomment}
Determinan matriks ini adalah 0.
\end{eulercomment}
\begin{eulerprompt}
>det(A)
\end{eulerprompt}
\begin{euleroutput}
  0
\end{euleroutput}
\eulerheading{Matriks Simbolik}
\begin{eulercomment}
Maxima memiliki matriks simbolik. Tentu saja, Maxima dapat digunakan
untuk masalah aljabar linear sederhana tersebut. Kita dapat
mendefinisikan matriks untuk Euler dan Maxima dengan \&:=, lalu
menggunakannya dalam ekspresi simbolik. Bentuk [...] yang biasa
digunakan untuk mendefinisikan matriks dapat digunakan dalam Euler
untuk mendefinisikan matriks simbolik.
\end{eulercomment}
\begin{eulerprompt}
>A &= [a,1,1;1,a,1;1,1,a]; $A
\end{eulerprompt}
\begin{eulerformula}
\[
\begin{pmatrix}a & 1 & 1 \\ 1 & a & 1 \\ 1 & 1 & a \\ \end{pmatrix}
\]
\end{eulerformula}
\begin{eulerprompt}
>$&det(A), $&factor(%)
\end{eulerprompt}
\begin{eulerformula}
\[
\left(a-1\right)^2\,\left(a+2\right)
\]
\end{eulerformula}
\eulerimg{0}{images/Alifia Rahmawati_23030630044_LaTex & Markdown-079-large.png}
\begin{eulerprompt}
>$&invert(A) with a=0
\end{eulerprompt}
\begin{eulerformula}
\[
\begin{pmatrix}-\frac{1}{2} & \frac{1}{2} & \frac{1}{2} \\ \frac{1  }{2} & -\frac{1}{2} & \frac{1}{2} \\ \frac{1}{2} & \frac{1}{2} & -  \frac{1}{2} \\ \end{pmatrix}
\]
\end{eulerformula}
\begin{eulerprompt}
>A &= [1,a;b,2]; $A
\end{eulerprompt}
\begin{eulerformula}
\[
\begin{pmatrix}1 & a \\ b & 2 \\ \end{pmatrix}
\]
\end{eulerformula}
\begin{eulercomment}
Seperti semua variabel simbolik, matriks ini dapat digunakan dalam
ekspresi simbolik lainnya.
\end{eulercomment}
\begin{eulerprompt}
>$&det(A-x*ident(2)), $&solve(%,x)
\end{eulerprompt}
\begin{eulerformula}
\[
\left[ x=\frac{3-\sqrt{4\,a\,b+1}}{2} , x=\frac{\sqrt{4\,a\,b+1}+3  }{2} \right] 
\]
\end{eulerformula}
\eulerimg{1}{images/Alifia Rahmawati_23030630044_LaTex & Markdown-083-large.png}
\begin{eulercomment}
Nilai eigen juga dapat dihitung secara otomatis. Hasilnya adalah
vektor dengan dua vektor nilai eigen dan multiplisitas.
\end{eulercomment}
\begin{eulerprompt}
>$&eigenvalues([a,1;1,a])
\end{eulerprompt}
\begin{eulerformula}
\[
\left[ \left[ a-1 , a+1 \right]  , \left[ 1 , 1 \right]  \right] 
\]
\end{eulerformula}
\begin{eulercomment}
Untuk mengekstrak vektor eigen tertentu dibutuhkan pengindeksan yang
cermat.
\end{eulercomment}
\begin{eulerprompt}
>$&eigenvectors([a,1;1,a]), &%[2][1][1]
\end{eulerprompt}
\begin{eulerformula}
\[
\left[ \left[ \left[ a-1 , a+1 \right]  , \left[ 1 , 1 \right]    \right]  , \left[ \left[ \left[ 1 , -1 \right]  \right]  , \left[   \left[ 1 , 1 \right]  \right]  \right]  \right] 
\]
\end{eulerformula}
\begin{euleroutput}
  
                                 [1, - 1]
  
\end{euleroutput}
\begin{eulercomment}
Matriks simbolik dapat dievaluasi dalam Euler secara numerik seperti
ekspresi simbolik lainnya.
\end{eulercomment}
\begin{eulerprompt}
>A(a=4,b=5)
\end{eulerprompt}
\begin{euleroutput}
              1             4 
              5             2 
\end{euleroutput}
\begin{eulercomment}
Dalam ekspresi simbolik, gunakan dengan.
\end{eulercomment}
\begin{eulerprompt}
>$&A with [a=4,b=5]
\end{eulerprompt}
\begin{eulerformula}
\[
\begin{pmatrix}1 & 4 \\ 5 & 2 \\ \end{pmatrix}
\]
\end{eulerformula}
\begin{eulercomment}
Akses terhadap baris matriks simbolik bekerja seperti halnya matriks
numerik.
\end{eulercomment}
\begin{eulerprompt}
>$&A[1]
\end{eulerprompt}
\begin{eulerformula}
\[
\left[ 1 , a \right] 
\]
\end{eulerformula}
\begin{eulercomment}
Ekspresi simbolik dapat berisi sebuah penugasan. Dan itu mengubah
matriks A.
\end{eulercomment}
\begin{eulerprompt}
>&A[1,1]:=t+1; $&A
\end{eulerprompt}
\begin{eulerformula}
\[
\begin{pmatrix}t+1 & a \\ b & 2 \\ \end{pmatrix}
\]
\end{eulerformula}
\begin{eulercomment}
Terdapat fungsi simbolik di Maxima untuk membuat vektor dan matriks.
Untuk ini, rujuk dokumentasi Maxima atau tutorial tentang Maxima di
EMT.
\end{eulercomment}
\begin{eulerprompt}
>v &= makelist(1/(i+j),i,1,3); $v
\end{eulerprompt}
\begin{eulerformula}
\[
\left[ \frac{1}{j+1} , \frac{1}{j+2} , \frac{1}{j+3} \right] 
\]
\end{eulerformula}
\begin{eulerttcomment}
 
\end{eulerttcomment}
\begin{eulerprompt}
>B &:= [1,2;3,4]; $B, $&invert(B)
\end{eulerprompt}
\begin{eulerformula}
\[
\begin{pmatrix}-2 & 1 \\ \frac{3}{2} & -\frac{1}{2} \\   \end{pmatrix}
\]
\end{eulerformula}
\eulerimg{1}{images/Alifia Rahmawati_23030630044_LaTex & Markdown-091-large.png}
\begin{eulercomment}
The result can be evaluated numerically in Euler. For more information
about Maxima, see the introduction to Maxima.
\end{eulercomment}
\begin{eulerprompt}
>$&invert(B)()
\end{eulerprompt}
\begin{euleroutput}
             -2             1 
            1.5          -0.5 
\end{euleroutput}
\begin{eulercomment}
Euler juga memiliki fungsi xinv() yang hebat, yang melakukan upaya
lebih besar dan mendapatkan hasil yang lebih tepat.

Perlu dicatat, bahwa dengan \&:= matriks B telah didefinisikan sebagai
simbolik dalam ekspresi simbolik dan sebagai numerik dalam ekspresi
numerik. Jadi kita dapat menggunakannya di sini.
\end{eulercomment}
\begin{eulerprompt}
>longest B.xinv(B)
\end{eulerprompt}
\begin{euleroutput}
                        1                       0 
                        0                       1 
\end{euleroutput}
\begin{eulercomment}
Misalnya nilai eigen A dapat dihitung secara numerik.
\end{eulercomment}
\begin{eulerprompt}
>A=[1,2,3;4,5,6;7,8,9]; real(eigenvalues(A))
\end{eulerprompt}
\begin{euleroutput}
  [16.1168,  -1.11684,  0]
\end{euleroutput}
\begin{eulercomment}
Atau secara simbolis. Lihat tutorial tentang Maxima untuk detailnya.
\end{eulercomment}
\begin{eulerprompt}
>$&eigenvalues(@A)
\end{eulerprompt}
\begin{eulerformula}
\[
\left[ \left[ \frac{15-3\,\sqrt{33}}{2} , \frac{3\,\sqrt{33}+15}{2}   , 0 \right]  , \left[ 1 , 1 , 1 \right]  \right] 
\]
\end{eulerformula}
\eulerheading{Nilai Numerik dalam Ekspresi Simbolik}
\begin{eulercomment}
Ekspresi simbolik hanyalah string yang berisi ekspresi. Jika kita
ingin menentukan nilai untuk ekspresi simbolik dan ekspresi numerik,
kita harus menggunakan "\&:=".
\end{eulercomment}
\begin{eulerprompt}
>A &:= [1,pi;4,5]
\end{eulerprompt}
\begin{euleroutput}
              1       3.14159 
              4             5 
\end{euleroutput}
\begin{eulercomment}
Masih terdapat perbedaan antara bentuk numerik dan bentuk simbolik.
Saat mengubah matriks ke bentuk simbolik, pendekatan pecahan untuk
bilangan riil akan digunakan.
\end{eulercomment}
\begin{eulerprompt}
>$&A
\end{eulerprompt}
\begin{eulerformula}
\[
\begin{pmatrix}1 & \frac{1146408}{364913} \\ 4 & 5 \\ \end{pmatrix}
\]
\end{eulerformula}
\begin{eulercomment}
Untuk menghindari hal ini, ada fungsi "mxmset(variabel)".
\end{eulercomment}
\begin{eulerprompt}
>mxmset(A); $&A
\end{eulerprompt}
\begin{eulerformula}
\[
\begin{pmatrix}1 & 3.141592653589793 \\ 4 & 5 \\ \end{pmatrix}
\]
\end{eulerformula}
\begin{eulercomment}
Maxima juga dapat melakukan komputasi dengan angka floating point, dan
bahkan dengan angka floating point besar dengan 32 digit. Namun,
evaluasinya jauh lebih lambat.
\end{eulercomment}
\begin{eulerprompt}
>$&bfloat(sqrt(2)), $&float(sqrt(2))
\end{eulerprompt}
\begin{eulerformula}
\[
1.414213562373095
\]
\end{eulerformula}
\eulerimg{0}{images/Alifia Rahmawati_23030630044_LaTex & Markdown-096-large.png}
\begin{eulercomment}
Ketepatan angka floating point besar dapat diubah.
\end{eulercomment}
\begin{eulerprompt}
>&fpprec:=100; &bfloat(pi)
\end{eulerprompt}
\begin{euleroutput}
  
          3.14159265358979323846264338327950288419716939937510582097494\(\backslash\)
  4592307816406286208998628034825342117068b0
  
\end{euleroutput}
\begin{eulercomment}
Variabel numerik dapat digunakan dalam ekspresi simbolik apa pun
menggunakan "@var".

Perlu dicatat bahwa ini hanya diperlukan jika variabel telah
didefinisikan dengan ":=" atau "=" sebagai variabel numerik.
\end{eulercomment}
\begin{eulerprompt}
>B:=[1,pi;3,4]; $&det(@B)
\end{eulerprompt}
\begin{eulerformula}
\[
-5.424777960769379
\]
\end{eulerformula}
\eulerheading{Demo - Suku Bunga}
\begin{eulercomment}
Di bawah ini, kami menggunakan Euler Math Toolbox (EMT) untuk
menghitung suku bunga. Kami melakukannya secara numerik dan simbolis
untuk menunjukkan kepada Anda bagaimana Euler dapat digunakan untuk
memecahkan masalah kehidupan nyata.

Asumsikan Anda memiliki modal awal sebesar 5000 (misalnya dalam
dolar).
\end{eulercomment}
\begin{eulerprompt}
>K=5000
\end{eulerprompt}
\begin{euleroutput}
  5000
\end{euleroutput}
\begin{eulercomment}
Sekarang kita asumsikan suku bunga 3\% per tahun. Mari kita tambahkan
satu suku bunga sederhana dan hitung hasilnya.
\end{eulercomment}
\begin{eulerprompt}
>K*1.03
\end{eulerprompt}
\begin{euleroutput}
  5150
\end{euleroutput}
\begin{eulercomment}
Euler juga akan memahami sintaksis berikut.
\end{eulercomment}
\begin{eulerprompt}
>K+K*3%
\end{eulerprompt}
\begin{euleroutput}
  5150
\end{euleroutput}
\begin{eulercomment}
Namun lebih mudah menggunakan faktor
\end{eulercomment}
\begin{eulerprompt}
>q=1+3%, K*q
\end{eulerprompt}
\begin{euleroutput}
  1.03
  5150
\end{euleroutput}
\begin{eulercomment}
Selama 10 tahun, kita cukup mengalikan faktor-faktornya dan
mendapatkan nilai akhir dengan suku bunga majemuk.
\end{eulercomment}
\begin{eulerprompt}
>K*q^10
\end{eulerprompt}
\begin{euleroutput}
  6719.58189672
\end{euleroutput}
\begin{eulercomment}
Untuk keperluan kita, kita dapat mengatur format menjadi 2 digit
setelah titik desimal.
\end{eulercomment}
\begin{eulerprompt}
>format(12,2); K*q^10
\end{eulerprompt}
\begin{euleroutput}
      6719.58 
\end{euleroutput}
\begin{eulercomment}
Mari kita cetak angka tersebut dibulatkan menjadi 2 digit dalam
kalimat lengkap.
\end{eulercomment}
\begin{eulerprompt}
>"Starting from " + K + "$ you get " + round(K*q^10,2) + "$."
\end{eulerprompt}
\begin{euleroutput}
  Starting from 5000$ you get 6719.58$.
\end{euleroutput}
\begin{eulercomment}
Bagaimana jika kita ingin mengetahui hasil antara dari tahun 1 hingga
tahun 9? Untuk ini, bahasa matriks Euler sangat membantu. Anda tidak
perlu menulis loop, tetapi cukup masukkan
\end{eulercomment}
\begin{eulerprompt}
>K*q^(0:10)
\end{eulerprompt}
\begin{euleroutput}
  Real 1 x 11 matrix
  
      5000.00     5150.00     5304.50     5463.64     ...
\end{euleroutput}
\begin{eulercomment}
Bagaimana keajaiban ini bekerja? Pertama, ekspresi 0:10 menghasilkan
vektor bilangan bulat.
\end{eulercomment}
\begin{eulerprompt}
>short 0:10
\end{eulerprompt}
\begin{euleroutput}
  [0,  1,  2,  3,  4,  5,  6,  7,  8,  9,  10]
\end{euleroutput}
\begin{eulercomment}
Maka semua operator dan fungsi di Euler dapat diaplikasikan ke vektor
elemen demi elemen. Jadi
\end{eulercomment}
\begin{eulerprompt}
>short q^(0:10)
\end{eulerprompt}
\begin{euleroutput}
  [1,  1.03,  1.0609,  1.0927,  1.1255,  1.1593,  1.1941,  1.2299,
  1.2668,  1.3048,  1.3439]
\end{euleroutput}
\begin{eulercomment}
adalah vektor faktor q\textasciicircum{}0 hingga q\textasciicircum{}10. Ini dikalikan dengan K, dan kita
memperoleh vektor nilai.
\end{eulercomment}
\begin{eulerprompt}
>VK=K*q^(0:10);
\end{eulerprompt}
\begin{eulercomment}
Tentu saja, cara realistis untuk menghitung suku bunga ini adalah
dengan membulatkannya ke sen terdekat setelah setiap tahun. Mari kita
tambahkan fungsi untuk ini.
\end{eulercomment}
\begin{eulerprompt}
>function oneyear (K) := round(K*q,2)
\end{eulerprompt}
\begin{eulercomment}
Mari kita bandingkan kedua hasil, dengan dan tanpa pembulatan.
\end{eulercomment}
\begin{eulerprompt}
>longest oneyear(1234.57), longest 1234.57*q
\end{eulerprompt}
\begin{euleroutput}
                  1271.61 
                1271.6071 
\end{euleroutput}
\begin{eulercomment}
Sekarang tidak ada rumus sederhana untuk tahun ke-n, dan kita harus
mengulangnya selama bertahun-tahun. Euler menyediakan banyak solusi
untuk ini.

Cara termudah adalah fungsi iterate, yang mengulang fungsi yang
diberikan beberapa kali.
\end{eulercomment}
\begin{eulerprompt}
>VKr=iterate("oneyear",5000,10)
\end{eulerprompt}
\begin{euleroutput}
  Real 1 x 11 matrix
  
      5000.00     5150.00     5304.50     5463.64     ...
\end{euleroutput}
\begin{eulercomment}
Kita dapat mencetaknya dengan cara yang ramah, menggunakan format kami
dengan tempat desimal tetap.
\end{eulercomment}
\begin{eulerprompt}
>VKr'
\end{eulerprompt}
\begin{euleroutput}
      5000.00 
      5150.00 
      5304.50 
      5463.64 
      5627.55 
      5796.38 
      5970.27 
      6149.38 
      6333.86 
      6523.88 
      6719.60 
\end{euleroutput}
\begin{eulercomment}
Untuk mendapatkan elemen vektor tertentu, kita menggunakan indeks
dalam tanda kurung siku.
\end{eulercomment}
\begin{eulerprompt}
>VKr[2], VKr[1:3]
\end{eulerprompt}
\begin{euleroutput}
      5150.00 
      5000.00     5150.00     5304.50 
\end{euleroutput}
\begin{eulercomment}
Anehnya, kita juga dapat menggunakan vektor indeks. Ingat bahwa 1:3
menghasilkan vektor [1,2,3].

Mari kita bandingkan elemen terakhir dari nilai yang dibulatkan dengan
nilai penuh.
\end{eulercomment}
\begin{eulerprompt}
>VKr[-1], VK[-1]
\end{eulerprompt}
\begin{euleroutput}
      6719.60 
      6719.58 
\end{euleroutput}
\begin{eulercomment}
Perbedaannya sangat kecil.

\begin{eulercomment}
\eulerheading{Menyelesaikan Persamaan}
\begin{eulercomment}
Sekarang kita ambil fungsi yang lebih maju, yang menambahkan nilai
uang tertentu setiap tahun.
\end{eulercomment}
\begin{eulerprompt}
>function onepay (K) := K*q+R
\end{eulerprompt}
\begin{eulercomment}
Kita tidak perlu menentukan q atau R untuk definisi fungsi. Hanya jika
kita menjalankan perintah, kita harus menentukan nilai-nilai ini. Kita
pilih R=200.
\end{eulercomment}
\begin{eulerprompt}
>R=200; iterate("onepay",5000,10)
\end{eulerprompt}
\begin{euleroutput}
  Real 1 x 11 matrix
  
      5000.00     5350.00     5710.50     6081.82     ...
\end{euleroutput}
\begin{eulercomment}
Bagaimana jika kita menghilangkan jumlah yang sama setiap tahun?
\end{eulercomment}
\begin{eulerprompt}
>R=-200; iterate("onepay",5000,10)
\end{eulerprompt}
\begin{euleroutput}
  Real 1 x 11 matrix
  
      5000.00     4950.00     4898.50     4845.45     ...
\end{euleroutput}
\begin{eulercomment}
Kita melihat bahwa uang berkurang. Jelas, jika kita hanya memperoleh
bunga sebesar 150 pada tahun pertama, tetapi mengurangi 200, kita akan
kehilangan uang setiap tahun.

Bagaimana kita dapat menentukan berapa tahun uang tersebut akan
bertahan? Kita harus menulis perulangan untuk ini. Cara termudah
adalah dengan mengulanginya cukup lama.
\end{eulercomment}
\begin{eulerprompt}
>VKR=iterate("onepay",5000,50)
\end{eulerprompt}
\begin{euleroutput}
  Real 1 x 51 matrix
  
      5000.00     4950.00     4898.50     4845.45     ...
\end{euleroutput}
\begin{eulercomment}
Dengan menggunakan bahasa matriks, kita dapat menentukan nilai negatif
pertama dengan cara berikut.
\end{eulercomment}
\begin{eulerprompt}
>min(nonzeros(VKR<0))
\end{eulerprompt}
\begin{euleroutput}
        48.00 
\end{euleroutput}
\begin{eulercomment}
Alasannya adalah nonzeros(VKR\textless{}0) mengembalikan vektor indeks i, di
mana VKR[i]\textless{}0, dan min menghitung indeks minimal.

Karena vektor selalu dimulai dengan indeks 1, jawabannya adalah 47
tahun.

Fungsi iterate() memiliki satu trik lagi. Fungsi ini dapat mengambil
kondisi akhir sebagai argumen. Kemudian, fungsi ini akan mengembalikan
nilai dan jumlah iterasi.
\end{eulercomment}
\begin{eulerprompt}
>\{x,n\}=iterate("onepay",5000,till="x<0"); x, n,
\end{eulerprompt}
\begin{euleroutput}
       -19.83 
        47.00 
\end{euleroutput}
\begin{eulercomment}
Mari kita coba menjawab pertanyaan yang lebih ambigu. Asumsikan kita
tahu bahwa nilainya adalah 0 setelah 50 tahun. Berapa tingkat
bunganya?

Ini adalah pertanyaan yang hanya dapat dijawab secara numerik. Di
bawah ini, kita akan memperoleh rumus yang diperlukan. Kemudian Anda
akan melihat bahwa tidak ada rumus yang mudah untuk tingkat bunga.
Namun untuk saat ini, kita bertujuan untuk mencari solusi numerik.

Langkah pertama adalah mendefinisikan fungsi yang melakukan iterasi
sebanyak n kali. Kita menambahkan semua parameter ke fungsi ini.
\end{eulercomment}
\begin{eulerprompt}
>function f(K,R,P,n) := iterate("x*(1+P/100)+R",K,n;P,R)[-1]
\end{eulerprompt}
\begin{eulercomment}
Iterasinya sama seperti di atas

\end{eulercomment}
\begin{eulerformula}
\[
x_{n+1} = x_n \cdot \left(1+ \frac{P}{100}\right) + R
\]
\end{eulerformula}
\begin{eulercomment}
Namun, kita tidak lagi menggunakan nilai global R dalam ekspresi kita.
Fungsi seperti iterate() memiliki trik khusus di Euler. Anda dapat
meneruskan nilai variabel dalam ekspresi sebagai parameter titik koma.
Dalam kasus ini P dan R.

Selain itu, kita hanya tertarik pada nilai terakhir. Jadi, kita ambil
indeks [-1].

Mari kita coba uji coba.
\end{eulercomment}
\begin{eulerprompt}
>f(5000,-200,3,47)
\end{eulerprompt}
\begin{euleroutput}
       -19.83 
\end{euleroutput}
\begin{eulercomment}
Sekarang kita bisa memecahkan masalah kita.
\end{eulercomment}
\begin{eulerprompt}
>solve("f(5000,-200,x,50)",3)
\end{eulerprompt}
\begin{euleroutput}
         3.15 
\end{euleroutput}
\begin{eulercomment}
Rutin solve menyelesaikan ekspresi=0 untuk variabel x. Jawabannya
adalah 3,15\% per tahun. Kita ambil nilai awal 3\% untuk algoritma
tersebut. Fungsi solve() selalu membutuhkan nilai awal.

Kita dapat menggunakan fungsi yang sama untuk menyelesaikan pertanyaan
berikut: Berapa banyak yang dapat kita hapus per tahun sehingga modal
awal habis setelah 20 tahun dengan asumsi suku bunga 3\% per tahun.
\end{eulercomment}
\begin{eulerprompt}
>solve("f(5000,x,3,20)",-200)
\end{eulerprompt}
\begin{euleroutput}
      -336.08 
\end{euleroutput}
\begin{eulercomment}
Perhatikan bahwa Anda tidak dapat memecahkan masalah jumlah tahun,
karena fungsi kita mengasumsikan n sebagai nilai integer.

\end{eulercomment}
\eulersubheading{Solusi Simbolis untuk Masalah Suku Bunga}
\begin{eulercomment}
Kita dapat menggunakan bagian simbolis Euler untuk mempelajari masalah
tersebut. Pertama, kita mendefinisikan fungsi onepay() secara
simbolis.
\end{eulercomment}
\begin{eulerprompt}
>function op(K) &= K*q+R; $&op(K)
\end{eulerprompt}
\begin{eulerformula}
\[
R+q\,K
\]
\end{eulerformula}
\begin{eulercomment}
Sekarang kita dapat mengulanginya.
\end{eulercomment}
\begin{eulerprompt}
>$&op(op(op(op(K)))), $&expand(%)
\end{eulerprompt}
\begin{eulerformula}
\[
q^3\,R+q^2\,R+q\,R+R+q^4\,K
\]
\end{eulerformula}
\eulerimg{0}{images/Alifia Rahmawati_23030630044_LaTex & Markdown-100-large.png}
\begin{eulercomment}
Kita melihat suatu pola. Setelah n periode kita memiliki

\end{eulercomment}
\begin{eulerformula}
\[
K_n = q^n K + R (1+q+\ldots+q^{n-1}) = q^n K + \frac{q^n-1}{q-1} R
\]
\end{eulerformula}
\begin{eulercomment}
Rumus tersebut adalah rumus untuk jumlah geometrik, yang diketahui
oleh Maxima.
\end{eulercomment}
\begin{eulerprompt}
>&sum(q^k,k,0,n-1); $& % = ev(%,simpsum)
\end{eulerprompt}
\begin{eulerformula}
\[
\sum_{k=0}^{n-1}{q^{k}}=\frac{q^{n}-1}{q-1}
\]
\end{eulerformula}
\begin{eulercomment}
Ini agak rumit. Jumlahnya dievaluasi dengan tanda "simpsum" untuk
mereduksinya menjadi hasil bagi.

Mari kita buat fungsi untuk ini.
\end{eulercomment}
\begin{eulerprompt}
>function fs(K,R,P,n) &= (1+P/100)^n*K + ((1+P/100)^n-1)/(P/100)*R; $&fs(K,R,P,n)
\end{eulerprompt}
\begin{eulerformula}
\[
\frac{100\,\left(\left(\frac{P}{100}+1\right)^{n}-1\right)\,R}{P}+K  \,\left(\frac{P}{100}+1\right)^{n}
\]
\end{eulerformula}
\begin{eulercomment}
Fungsi ini melakukan hal yang sama seperti fungsi f sebelumnya. Namun,
fungsinya lebih efektif.
\end{eulercomment}
\begin{eulerprompt}
>longest f(5000,-200,3,47), longest fs(5000,-200,3,47)
\end{eulerprompt}
\begin{euleroutput}
       -19.82504734650985 
       -19.82504734652684 
\end{euleroutput}
\begin{eulercomment}
Kita sekarang dapat menggunakannya untuk menanyakan waktu n. Kapan
modal kita habis? Perkiraan awal kita adalah 30 tahun.
\end{eulercomment}
\begin{eulerprompt}
>solve("fs(5000,-330,3,x)",30)
\end{eulerprompt}
\begin{euleroutput}
        20.51 
\end{euleroutput}
\begin{eulercomment}
Jawaban ini menyatakan bahwa akan negatif setelah 21 tahun.

Kita juga dapat menggunakan sisi simbolik Euler untuk menghitung rumus
pembayaran.

Asumsikan kita mendapatkan pinjaman sebesar K, dan membayar n kali
cicilan sebesar R (dimulai setelah tahun pertama) sehingga menyisakan
utang residual sebesar Kn (pada saat pembayaran terakhir). Rumus untuk
ini jelas
\end{eulercomment}
\begin{eulerprompt}
>equ &= fs(K,R,P,n)=Kn; $&equ
\end{eulerprompt}
\begin{eulerformula}
\[
\frac{100\,\left(\left(\frac{P}{100}+1\right)^{n}-1\right)\,R}{P}+K  \,\left(\frac{P}{100}+1\right)^{n}={\it Kn}
\]
\end{eulerformula}
\begin{eulercomment}
Biasanya rumus ini diberikan dalam bentuk\\
\end{eulercomment}
\begin{eulerformula}
\[
i = \frac{P}{100}
\]
\end{eulerformula}
\begin{eulerprompt}
>equ &= (equ with P=100*i); $&equ
\end{eulerprompt}
\begin{eulerformula}
\[
\frac{\left(\left(i+1\right)^{n}-1\right)\,R}{i}+\left(i+1\right)^{  n}\,K={\it Kn}
\]
\end{eulerformula}
\begin{eulercomment}
Kita dapat mencari laju R secara simbolis.
\end{eulercomment}
\begin{eulerprompt}
>$&solve(equ,R)
\end{eulerprompt}
\begin{eulerformula}
\[
\left[ R=\frac{i\,{\it Kn}-i\,\left(i+1\right)^{n}\,K}{\left(i+1  \right)^{n}-1} \right] 
\]
\end{eulerformula}
\begin{eulercomment}
Seperti yang dapat Anda lihat dari rumus, fungsi ini mengembalikan
kesalahan floating point untuk i=0. Namun, Euler memplotnya.

Tentu saja, kita memiliki limit berikut.
\end{eulercomment}
\begin{eulerprompt}
>$&limit(R(5000,0,x,10),x,0)
\end{eulerprompt}
\begin{eulerformula}
\[
\lim_{x\rightarrow 0}{R\left(5000 , 0 , x , 10\right)}
\]
\end{eulerformula}
\begin{eulercomment}
Jelas, tanpa bunga, kita harus membayar kembali 10 suku bunga sebesar
500.

Persamaan ini juga dapat diselesaikan untuk n. Akan terlihat lebih
bagus jika kita menerapkan beberapa penyederhanaan.
\end{eulercomment}
\begin{eulerprompt}
>fn &= solve(equ,n) | ratsimp; $&fn
\end{eulerprompt}
\begin{eulerformula}
\[
\left[ n=\frac{\log \left(\frac{R+i\,{\it Kn}}{R+i\,K}\right)}{  \log \left(i+1\right)} \right] 
\]
\end{eulerformula}
\eulerheading{R.2 Latihan Soal}
\begin{eulercomment}
Soal No 49\\
Menyederhanakan:\\
\end{eulercomment}
\begin{eulerformula}
\[
\left(\frac{24a^{10}b^{-8}c^7}{12a^6b^{-3}c^5}\right)^{-5}
\]
\end{eulerformula}
\begin{eulerprompt}
>$&((24*a^10*b^(-8)*c^7)/(12*a^6*b^-3*c^5))^(-5)
\end{eulerprompt}
\begin{eulerformula}
\[
\frac{b^{25}}{32\,a^{20}\,c^{10}}
\]
\end{eulerformula}
\begin{eulercomment}
Soal No 50\\
Menyederhanakan:\\
\end{eulercomment}
\begin{eulerformula}
\[
\left(\frac{125p^{12}q^{-14}r^{22}}{25p^8q^6r^{-15}}\right)^{-4}
\]
\end{eulerformula}
\begin{eulerprompt}
>$&((125*p^(12)*q^(-14)*r^(22))/25*p^8*q^6*r^(-15))^(-4)
\end{eulerprompt}
\begin{eulerformula}
\[
\frac{q^{32}}{625\,p^{80}\,r^{28}}
\]
\end{eulerformula}
\begin{eulercomment}
Soal No 90\\
calculate\\
\end{eulercomment}
\begin{eulerformula}
\[
2^6*2^{-3}/2^{10}/2^{-8}
\]
\end{eulerformula}
\begin{eulerprompt}
>2^6*2^-3/2^10/2^-8
\end{eulerprompt}
\begin{euleroutput}
         2.00 
\end{euleroutput}
\begin{eulercomment}
Soal No 91\\
Calculate\\
\end{eulercomment}
\begin{eulerformula}
\[
\left(\frac{4(8-6)^2-4*3+2*8}{3^1+9^0}\right)
\]
\end{eulerformula}
\begin{eulerprompt}
>(4*(8-6)^2 - 4*3 + 2*8)/(3^1+19^0)
\end{eulerprompt}
\begin{euleroutput}
         5.00 
\end{euleroutput}
\begin{eulercomment}
Soal No 92\\
Calculate\\
\end{eulercomment}
\begin{eulerformula}
\[
\left(\frac{[4(8-6)^2-4](3+2*8)}{2^2(2^5+5)}\right)
\]
\end{eulerformula}
\begin{eulerprompt}
>((4*(8-6)^2-4)*3+2*8)/(3^1+9^0)
\end{eulerprompt}
\begin{euleroutput}
        13.00 
\end{euleroutput}
\eulerheading{R.3 Latihan soal}
\begin{eulercomment}
no 27\\
\end{eulercomment}
\begin{eulerformula}
\[
(x+3)^2
\]
\end{eulerformula}
\begin{eulerprompt}
>$&showev('expand((x+3)^2))
\end{eulerprompt}
\begin{eulerformula}
\[
{\it expand}\left(\left(x+3\right)^2\right)=x^2+6\,x+9
\]
\end{eulerformula}
\begin{eulercomment}
no 29\\
\end{eulercomment}
\begin{eulerformula}
\[
(y-5)^2
\]
\end{eulerformula}
\begin{eulerprompt}
>$&showev('expand((y-5)^2))
\end{eulerprompt}
\begin{eulerformula}
\[
{\it expand}\left(\left(y-5\right)^2\right)=y^2-10\,y+25
\]
\end{eulerformula}
\begin{eulercomment}
no 33\\
\end{eulercomment}
\begin{eulerformula}
\[
(2x+3y)^2
\]
\end{eulerformula}
\begin{eulerprompt}
>$&showev('expand((2*x+3*y)^2))
\end{eulerprompt}
\begin{eulerformula}
\[
{\it expand}\left(\left(3\,y+2\,x\right)^2\right)=9\,y^2+12\,x\,y+4  \,x^2
\]
\end{eulerformula}
\begin{eulercomment}
no 39\\
\end{eulercomment}
\begin{eulerformula}
\[
(3y+4)(3y-4)
\]
\end{eulerformula}
\begin{eulerprompt}
>$&showev('expand((3*y+4)*(3*y-4)))
\end{eulerprompt}
\begin{eulerformula}
\[
{\it expand}\left(\left(3\,y-4\right)\,\left(3\,y+4\right)\right)=9  \,y^2-16
\]
\end{eulerformula}
\begin{eulercomment}
no 42\\
\end{eulercomment}
\begin{eulerformula}
\[
(3x+5y)(3x-5y)
\]
\end{eulerformula}
\begin{eulerprompt}
>$&showev('expand ((3*x + 5*y)*(3*x - 5*y)))
\end{eulerprompt}
\begin{eulerformula}
\[
{\it expand}\left(\left(3\,x-5\,y\right)\,\left(5\,y+3\,x\right)  \right)=9\,x^2-25\,y^2
\]
\end{eulerformula}
\eulerheading{R.4 Latihan Soal }
\begin{eulercomment}
Faktor Trinomial \\
Nomor 23\\
\end{eulercomment}
\begin{eulerformula}
\[
t^2+8t+15
\]
\end{eulerformula}
\begin{eulerprompt}
>$&solve(t^2+8*t+15)
\end{eulerprompt}
\begin{eulerformula}
\[
\left[ t=-3 , t=-5 \right] 
\]
\end{eulerformula}
\begin{eulercomment}
Nomor 24\\
\end{eulercomment}
\begin{eulerformula}
\[
y^2+12y+27
\]
\end{eulerformula}
\begin{eulerprompt}
>$&solve(y^2+12*y+27)
\end{eulerprompt}
\begin{eulerformula}
\[
\left[ y=-9 , y=-3 \right] 
\]
\end{eulerformula}
\begin{eulercomment}
Nomor 47\\
\end{eulercomment}
\begin{eulerformula}
\[
z^2-81
\]
\end{eulerformula}
\begin{eulerprompt}
>$&solve(z^2-81)
\end{eulerprompt}
\begin{eulerformula}
\[
\left[ z=-9 , z=9 \right] 
\]
\end{eulerformula}
\begin{eulercomment}
Nomor 48\\
\end{eulercomment}
\begin{eulerformula}
\[
m^2-4
\]
\end{eulerformula}
\begin{eulerprompt}
>$&solve(m^2-4)
\end{eulerprompt}
\begin{eulerformula}
\[
\left[ m=-2 , m=2 \right] 
\]
\end{eulerformula}
\begin{eulercomment}
Nomor 49\\
\end{eulercomment}
\begin{eulerformula}
\[
16x^2-9
\]
\end{eulerformula}
\begin{eulerprompt}
>$&solve(16*x^2-9)
\end{eulerprompt}
\begin{eulerformula}
\[
\left[ x=-\frac{3}{4} , x=\frac{3}{4} \right] 
\]
\end{eulerformula}
\eulersubheading{R.5 Exercise Set}
\begin{eulercomment}
soal no 36\\
tentukan nilai y\\
\end{eulercomment}
\begin{eulerformula}
\[
y^2-4y-45=0
\]
\end{eulerformula}
\begin{eulerprompt}
>$&solve(y^2-4*y-45,y)
\end{eulerprompt}
\begin{eulerformula}
\[
\left[ y=9 , y=-5 \right] 
\]
\end{eulerformula}
\begin{eulercomment}
soal no 38\\
tentukan nilai y\\
\end{eulercomment}
\begin{eulerformula}
\[
t^2+6t=0
\]
\end{eulerformula}
\begin{eulerprompt}
>$&solve(t^2+ 6*t,t)
\end{eulerprompt}
\begin{eulerformula}
\[
\left[ t=-6 , t=0 \right] 
\]
\end{eulerformula}
\begin{eulercomment}
soal no 41\\
tentukan nilai x\\
\end{eulercomment}
\begin{eulerformula}
\[
x^2 + 100 = 20x
\]
\end{eulerformula}
\begin{eulerprompt}
>$&solve(x^2+100=20*x,x)
\end{eulerprompt}
\begin{eulerformula}
\[
\left[ x=10 \right] 
\]
\end{eulerformula}
\begin{eulercomment}
soal no 42\\
tentukan nilai y\\
\end{eulercomment}
\begin{eulerformula}
\[
y^2+25=10y
\]
\end{eulerformula}
\begin{eulerprompt}
>$&solve(y^2+25=10*y,y)
\end{eulerprompt}
\begin{eulerformula}
\[
\left[ y=5 \right] 
\]
\end{eulerformula}
\begin{eulercomment}
soal no 45\\
tentukan nilai y\\
\end{eulercomment}
\begin{eulerformula}
\[
3y^2 + 8y + 4=0
\]
\end{eulerformula}
\begin{eulerprompt}
>$&solve(3*y^2 + 8*y + 4,y)
\end{eulerprompt}
\begin{eulerformula}
\[
\left[ y=-\frac{2}{3} , y=-2 \right] 
\]
\end{eulerformula}
\eulersubheading{R.6 Exercise Set}
\begin{eulercomment}
Nomor 9\\
Menyederhanakan\\
\end{eulercomment}
\begin{eulerformula}
\[
\frac{x^{2}-4}{x^{2}-4x+4}
\]
\end{eulerformula}
\begin{eulerprompt}
>$&((x^2)/(x^2-4*x+4)), $&factor(%)
\end{eulerprompt}
\begin{eulerformula}
\[
\frac{x^2}{\left(x-2\right)^2}
\]
\end{eulerformula}
\eulerimg{1}{images/Alifia Rahmawati_23030630044_LaTex & Markdown-126-large.png}
\begin{eulercomment}
Nomor 11\\
Menyederhanakan\\
\end{eulercomment}
\begin{eulerformula}
\[
\frac{x^{3}-6x^{2}+9x}{x^{3}-3x^{2}}
\]
\end{eulerformula}
\begin{eulerprompt}
>$&((x^3 - 6*x^2 + 9*x)/ (x^3 - 3*x^2)), $&factor(%)
\end{eulerprompt}
\begin{eulerformula}
\[
\frac{x-3}{x}
\]
\end{eulerformula}
\eulerimg{1}{images/Alifia Rahmawati_23030630044_LaTex & Markdown-128-large.png}
\begin{eulercomment}
Nomor 14\\
Menyederhanakan\\
\end{eulercomment}
\begin{eulerformula}
\[
\frac{2x^{2}-20x+50}{10x^{2}-30x-100}
\]
\end{eulerformula}
\begin{eulerprompt}
>$&((2*x^2-20*x+50)/(10*x^2-30*x-100)), $&factor(%)
\end{eulerprompt}
\begin{eulerformula}
\[
\frac{x-5}{5\,\left(x+2\right)}
\]
\end{eulerformula}
\eulerimg{1}{images/Alifia Rahmawati_23030630044_LaTex & Markdown-130-large.png}
\begin{eulercomment}
Nomor 15\\
Menyederhanakan\\
\end{eulercomment}
\begin{eulerformula}
\[
\frac{4-x}{x^{2}+4x-32}
\]
\end{eulerformula}
\begin{eulerprompt}
>$&((4-x)/(x^2 +4*x-32)), $&factor(%)
\end{eulerprompt}
\begin{eulerformula}
\[
-\frac{1}{x+8}
\]
\end{eulerformula}
\eulerimg{1}{images/Alifia Rahmawati_23030630044_LaTex & Markdown-132-large.png}
\begin{eulercomment}
Nomor 16\\
Menyederhanakan\\
\end{eulercomment}
\begin{eulerformula}
\[
\frac{6-x}{x^{2}-36}
\]
\end{eulerformula}
\begin{eulerprompt}
>$&((6-x)/(x^2-36)), $&factor(%)
\end{eulerprompt}
\begin{eulerformula}
\[
-\frac{1}{x+6}
\]
\end{eulerformula}
\eulerimg{1}{images/Alifia Rahmawati_23030630044_LaTex & Markdown-134-large.png}
\eulerheading{Latihan Ulangan}
\begin{eulercomment}
Multiply\\
Nomor 71\\
\end{eulercomment}
\begin{eulerformula}
\[
(t^a+\frac{1}{t^a})^n
\]
\end{eulerformula}
\begin{eulerprompt}
>function P(a,n) &= (t^a+t^(-a))^n; $&P(a,n)
\end{eulerprompt}
\begin{eulerformula}
\[
\left(t^{a}+\frac{1}{t^{a}}\right)^{n}
\]
\end{eulerformula}
\begin{eulerprompt}
>$&P(a,2), $&expand(%)
\end{eulerprompt}
\begin{eulerformula}
\[
t^{2\,a}+\frac{1}{t^{2\,a}}+2
\]
\end{eulerformula}
\eulerimg{1}{images/Alifia Rahmawati_23030630044_LaTex & Markdown-137-large.png}
\begin{eulercomment}
Nomor 72\\
\end{eulercomment}
\begin{eulerformula}
\[
(y^b-z^c)(y^b+z^c)
\]
\end{eulerformula}
\begin{eulerprompt}
>$& '((y^b-z^c)*(y^b+z^c)) = expand((y^b-z^c)*(y^b+z^c))
\end{eulerprompt}
\begin{eulerformula}
\[
\left(y^{b}-z^{c}\right)\,\left(z^{c}+y^{b}\right)=y^{2\,b}-z^{2\,c  }
\]
\end{eulerformula}
\begin{eulercomment}
Nomor 73\\
\end{eulercomment}
\begin{eulerformula}
\[
(a^n-b^n)^3
\]
\end{eulerformula}
\begin{eulerprompt}
>$& '((a^n-b^n)^3) = expand((a^n-b^n)^3)
\end{eulerprompt}
\begin{eulerformula}
\[
\left(a^{n}-b^{n}\right)^3=-b^{3\,n}+3\,a^{n}\,b^{2\,n}-3\,a^{2\,n}  \,b^{n}+a^{3\,n}
\]
\end{eulerformula}
\begin{eulercomment}
Nomor 32\\
\end{eulercomment}
\begin{eulerformula}
\[
\frac{x^2+x-6}{x^2+8x+15}*\frac{x^2-25}{x^2-4x+4}
\]
\end{eulerformula}
\begin{eulerprompt}
>$& '(((x^2+x-6)/(x^2+8*x+15))*((x^2-25)/(x^2-4*x+4))) = simplify(((x^2+x-6)/(x^2+8*x+15))*((x^2-25)/(x^2-4*x+4)))
\end{eulerprompt}
\begin{eulerformula}
\[
\frac{\left(x^2-25\right)\,\left(x^2+x-6\right)}{\left(x^2-4\,x+4  \right)\,\left(x^2+8\,x+15\right)}={\it simplify}\left(\frac{\left(x  ^2-25\right)\,\left(x^2+x-6\right)}{\left(x^2-4\,x+4\right)\,\left(x  ^2+8\,x+15\right)}\right)
\]
\end{eulerformula}
\begin{eulerprompt}
>$&solve(((x^2+x-6)/(x^2+8*x+15))*((x^2-25)/(x^2-4*x+4)),x)
\end{eulerprompt}
\begin{eulerformula}
\[
\left[ x=5 \right] 
\]
\end{eulerformula}
\begin{eulercomment}
Nomor 33\\
\end{eulercomment}
\begin{eulerformula}
\[
\frac{x}{x^2-1}-\frac{3}{x^2+4x-5}
\]
\end{eulerformula}
\begin{eulerprompt}
>$& '(((x)/(x^2-1))-((3)/(x^2+4*x-5)))=simplify(((x)/(x^2-1))-((3)/(x^2+4*x-5)))
\end{eulerprompt}
\begin{eulerformula}
\[
\frac{x}{x^2-1}-\frac{3}{x^2+4\,x-5}={\it simplify}\left(\frac{x}{x  ^2-1}-\frac{3}{x^2+4\,x-5}\right)
\]
\end{eulerformula}
\begin{eulerprompt}
>$&solve(((x)/(x^2-1))-((3)/(x^2+4*x-5)))
\end{eulerprompt}
\begin{eulerformula}
\[
\left[ x=-3 \right] 
\]
\end{eulerformula}
\eulerheading{2.3 Exercise Set}
\begin{eulercomment}
Cari\\
\end{eulercomment}
\begin{eulerformula}
\[
\left(f\circ g\right)\left(x\right) dan \left(g\circ f\right)\left(x\right)
\]
\end{eulerformula}
\begin{eulercomment}
dan domain nya !

Nomor 1\\
\end{eulercomment}
\begin{eulerformula}
\[
 f(x)=x+3\ ,\ g(x)=x-3
\]
\end{eulerformula}
\begin{eulercomment}
\end{eulercomment}
\begin{eulerformula}
\[
\left(f\circ g\right)\left(x\right)=
\]
\end{eulerformula}
\begin{eulerprompt}
>$&gx:=x-3; $&fx:=gx+3; $&fx
\end{eulerprompt}
\begin{eulerformula}
\[
x
\]
\end{eulerformula}
\begin{eulercomment}
dengan domainnya\\
\end{eulercomment}
\begin{eulerformula}
\[
D_{f\circ g}=\left\{x\in\mathbb{R}\right\}
\]
\end{eulerformula}
\begin{eulercomment}
\end{eulercomment}
\begin{eulerformula}
\[
\left(g\circ f\right)\left(x\right)=
\]
\end{eulerformula}
\begin{eulerprompt}
>$&fx:=x+3; $&gx:=fx-3; $&gx
\end{eulerprompt}
\begin{eulerformula}
\[
x
\]
\end{eulerformula}
\begin{eulercomment}
dengan domainnya\\
\end{eulercomment}
\begin{eulerformula}
\[
D_{g\circ f}=\left\{x\in\mathbb{R}\right\}
\]
\end{eulerformula}
\begin{eulercomment}
Nomor 2\\
\end{eulercomment}
\begin{eulerformula}
\[
f(x)=4/(1-5x)\ \ ,\ g(x)=1/x
\]
\end{eulerformula}
\begin{eulerformula}
\[
\left(f\circ g\right)\left(x\right)=
\]
\end{eulerformula}
\begin{eulerprompt}
>$&gx:=1/x; $&fx:=4/(1-5*gx); $&fx
\end{eulerprompt}
\begin{eulerformula}
\[
\frac{4}{1-\frac{5}{x}}
\]
\end{eulerformula}
\begin{eulercomment}
dengan domain\\
\end{eulercomment}
\begin{eulerformula}
\[
D_{f\circ g}=\left\{x\in\mathbb{R}|x\neq 0\cup x\neq 5\right\}
\]
\end{eulerformula}
\begin{eulercomment}
\end{eulercomment}
\begin{eulerformula}
\[
\left(g\circ f\right)\left(x\right)=
\]
\end{eulerformula}
\begin{eulerprompt}
>$&fx:=4/(1-5*x); $&gx:=1/fx; $&gx
\end{eulerprompt}
\begin{eulerformula}
\[
\frac{1-5\,x}{4}
\]
\end{eulerformula}
\begin{eulercomment}
dengan domainnya\\
\end{eulercomment}
\begin{eulerformula}
\[
D_{g\circ f}=\left\{x\in\mathbb{R}\right\}
\]
\end{eulerformula}
\begin{eulercomment}
Diberikan fungsi\\
\end{eulercomment}
\begin{eulerformula}
\[
f(x)=3x+1 , g(x)=x^2-2x-6 , h(x)=x^3
\]
\end{eulerformula}
\begin{eulercomment}
cari\\
Nomor 3\\
\end{eulercomment}
\begin{eulerformula}
\[
\left(f\circ g\right)\left(1/3\right)
\]
\end{eulerformula}
\begin{eulerprompt}
>$x:=1/3; $&gx:=x^2-2*x-6; $&fx:=3*gx+1; $&fx
\end{eulerprompt}
\begin{eulerformula}
\[
-\frac{56}{3}
\]
\end{eulerformula}
\begin{eulercomment}
Nomor 4\\
\end{eulercomment}
\begin{eulerformula}
\[
\left(g\circ h\right)\left(1/2\right)
\]
\end{eulerformula}
\begin{eulerprompt}
>$x:=1/2; $&hx:=x^3; $&gx:=hx^2-2*hx-6; $&gx
\end{eulerprompt}
\begin{eulerformula}
\[
-\frac{399}{64}
\]
\end{eulerformula}
\begin{eulercomment}
Nomor 5\\
\end{eulercomment}
\begin{eulerformula}
\[
\left(g\circ g\right)\left(5\right)
\]
\end{eulerformula}
\begin{eulerprompt}
>$x:=5; $&gx:=x^2-2*x-6; $&gx:=x^2-2*x-6; $&gx
\end{eulerprompt}
\begin{eulerformula}
\[
9
\]
\end{eulerformula}
\eulerheading{3.1 Latihan Soal }
\begin{eulercomment}
Gunakan rumus kuadrat untuk menemukan solusi yang tepat\\
Nomor 39\\
\end{eulercomment}
\begin{eulerformula}
\[
5m^2+3m=2
\]
\end{eulerformula}
\begin{eulerprompt}
>$&solve(5*m^2+3*m=2,m)
\end{eulerprompt}
\begin{eulerformula}
\[
\left[ m=\frac{2}{5} , m=-1 \right] 
\]
\end{eulerformula}
\begin{eulercomment}
Nomor 40\\
\end{eulercomment}
\begin{eulerformula}
\[
2y^2-3y-2=0
\]
\end{eulerformula}
\begin{eulerprompt}
>$&solve(2*y^2-3*y-2=0,y)
\end{eulerprompt}
\begin{eulerformula}
\[
\left[ y=-\frac{1}{2} , y=2 \right] 
\]
\end{eulerformula}
\begin{eulercomment}
Nomo4 48\\
\end{eulercomment}
\begin{eulerformula}
\[
2t^2-5t=1
\]
\end{eulerformula}
\begin{eulerprompt}
>$&solve(2*t^2-5*t=1,t)
\end{eulerprompt}
\begin{eulerformula}
\[
\left[ t=\frac{5-\sqrt{33}}{4} , t=\frac{\sqrt{33}+5}{4} \right] 
\]
\end{eulerformula}
\begin{eulercomment}
Solve.\\
Nomor 83\\
\end{eulercomment}
\begin{eulerformula}
\[
y^4+4y^2-5=0
\]
\end{eulerformula}
\begin{eulerprompt}
>$&solve(y^4+4*y^2-5=0,y)
\end{eulerprompt}
\begin{eulerformula}
\[
\left[ y=-1 , y=1 , y=-\sqrt{5}\,i , y=\sqrt{5}\,i \right] 
\]
\end{eulerformula}
\begin{eulercomment}
Nomor 84\\
\end{eulercomment}
\begin{eulerformula}
\[
y^4-15y^2-16=0
\]
\end{eulerformula}
\begin{eulerprompt}
>$&solve(y^4-15*y^2-16=0,y)
\end{eulerprompt}
\begin{eulerformula}
\[
\left[ y=-i , y=i , y=-4 , y=4 \right] 
\]
\end{eulerformula}
\eulerheading{3.4 Latihan Soal }
\begin{eulercomment}
Solve \\
Nomor 1 \\
\end{eulercomment}
\begin{eulerformula}
\[
\frac{1}{4}+\frac{1}{5}=\frac{1}{t}
\]
\end{eulerformula}
\begin{eulerprompt}
>$&solve(1/4+1/5=1/t,t)
\end{eulerprompt}
\begin{eulerformula}
\[
\left[ t=\frac{20}{9} \right] 
\]
\end{eulerformula}
\begin{eulercomment}
Nomor 4\\
\end{eulercomment}
\begin{eulerformula}
\[
\frac{t+1}{3}-\frac{t-1}{2}=1
\]
\end{eulerformula}
\begin{eulerprompt}
>$&solve((t+1)/1-(t-1)/2=1,t)
\end{eulerprompt}
\begin{eulerformula}
\[
\left[ t=-1 \right] 
\]
\end{eulerformula}
\begin{eulercomment}
Nomor 6
\end{eulercomment}
\begin{eulerprompt}
>$&solve(1/t+1/2*t+1/3*t=5,t)
\end{eulerprompt}
\begin{eulerformula}
\[
\left[ t=\frac{15-\sqrt{195}}{5} , t=\frac{\sqrt{195}+15}{5}   \right] 
\]
\end{eulerformula}
\begin{eulercomment}
Nomor 11\\
\end{eulercomment}
\begin{eulerformula}
\[
\frac{6}{y+3}+\frac{2}{y}=\frac{5y-3}{y^2-9}
\]
\end{eulerformula}
\begin{eulerprompt}
>$&solve((5/(y+3))+(2/y)=(5*y-3)/(y^2-9),y)
\end{eulerprompt}
\begin{eulerformula}
\[
\left[ y=3-3\,\sqrt{2} , y=3\,\sqrt{2}+3 \right] 
\]
\end{eulerformula}
\begin{eulercomment}
Nomor 18
\end{eulercomment}
\begin{eulerprompt}
>$&solve(3*y+5/y^2+5*y +y+4/y+5=y+1/y,y)
\end{eulerprompt}
\begin{eulerformula}
\[
\left[ y=-\frac{47\,\left(\frac{\sqrt{3}\,i}{2}-\frac{1}{2}\right)  }{576\,\left(\frac{\sqrt{35539}}{128\,3^{\frac{3}{2}}}-\frac{3905}{  13824}\right)^{\frac{1}{3}}}+\left(\frac{\sqrt{35539}}{128\,3^{  \frac{3}{2}}}-\frac{3905}{13824}\right)^{\frac{1}{3}}\,\left(-\frac{  \sqrt{3}\,i}{2}-\frac{1}{2}\right)-\frac{5}{24} , y=\left(\frac{  \sqrt{35539}}{128\,3^{\frac{3}{2}}}-\frac{3905}{13824}\right)^{  \frac{1}{3}}\,\left(\frac{\sqrt{3}\,i}{2}-\frac{1}{2}\right)-\frac{  47\,\left(-\frac{\sqrt{3}\,i}{2}-\frac{1}{2}\right)}{576\,\left(  \frac{\sqrt{35539}}{128\,3^{\frac{3}{2}}}-\frac{3905}{13824}\right)  ^{\frac{1}{3}}}-\frac{5}{24} , y=\left(\frac{\sqrt{35539}}{128\,3^{  \frac{3}{2}}}-\frac{3905}{13824}\right)^{\frac{1}{3}}-\frac{47}{576  \,\left(\frac{\sqrt{35539}}{128\,3^{\frac{3}{2}}}-\frac{3905}{13824}  \right)^{\frac{1}{3}}}-\frac{5}{24} \right] 
\]
\end{eulerformula}
\eulerheading{3.5 Latihan Soal}
\begin{eulerprompt}
>&load(fourier_elim)
\end{eulerprompt}
\begin{euleroutput}
  
          D:/New folder/Euler x64/maxima/share/maxima/5.35.1/share/four\(\backslash\)
  ier_elim/fourier_elim.lisp
  
\end{euleroutput}
\begin{eulercomment}
Nomor 44
\end{eulercomment}
\begin{eulerprompt}
>$&kill(x);
>$&fourier_elim([4*x]>20,[x])//4*x>20
\end{eulerprompt}
\begin{eulerformula}
\[
\left[ \left[ 4\,\left(x-5\right) \right] >0 \right] 
\]
\end{eulerformula}
\begin{eulercomment}
Nomor 45
\end{eulercomment}
\begin{eulerprompt}
>$&fourier_elim([x+8]<9,[x])// x+8<9
\end{eulerprompt}
\begin{eulerformula}
\[
\left[ \left[ 1-x \right] >0 \right] 
\]
\end{eulerformula}
\begin{eulercomment}
Nomor 47
\end{eulercomment}
\begin{eulerprompt}
>$&fourier_elim([x+8]>= 9,[x])//x+8 >=9
\end{eulerprompt}
\begin{eulerformula}
\[
\left[ \left[ x-1 \right] =0 \right] \lor \left[ \left[ x-1   \right] >0 \right] 
\]
\end{eulerformula}
\begin{eulercomment}
Nomor 52
\end{eulercomment}
\begin{eulerprompt}
>$&fourier_elim([x+8]>= 9,[x])//x+8 >=9
\end{eulerprompt}
\begin{eulerformula}
\[
\left[ \left[ x-1 \right] =0 \right] \lor \left[ \left[ x-1   \right] >0 \right] 
\]
\end{eulerformula}
\begin{eulercomment}
Nomor 62
\end{eulercomment}
\begin{eulerprompt}
>$&fourier_elim([3*x+5]<0,[x])//3*x+5<0
\end{eulerprompt}
\begin{eulerformula}
\[
\left[ \left[ -3\,x-5 \right] >0 \right] 
\]
\end{eulerformula}
\eulerheading{Chapter 3 Test }
\begin{eulerprompt}
>$&kill(x);
>&load(fourier_elim)
\end{eulerprompt}
\begin{euleroutput}
  
          D:/New folder/Euler x64/maxima/share/maxima/5.35.1/share/four\(\backslash\)
  ier_elim/fourier_elim.lisp
  
\end{euleroutput}
\begin{eulercomment}
Nomor 8
\end{eulercomment}
\begin{eulerprompt}
>$&solve(3/3*x+4 + 2/x-1 =2,x)
\end{eulerprompt}
\begin{eulerformula}
\[
\left[ x=\frac{-\sqrt{7}\,i-1}{2} , x=\frac{\sqrt{7}\,i-1}{2}   \right] 
\]
\end{eulerformula}
\begin{eulercomment}
Nomor 11
\end{eulercomment}
\begin{eulerprompt}
>$&fourier_elim([x+4]=7,[x])//x+4=7
\end{eulerprompt}
\begin{eulerformula}
\[
\left[ \left[ x-3 \right] =0 \right] 
\]
\end{eulerformula}
\begin{eulercomment}
Nomor 12
\end{eulercomment}
\begin{eulerprompt}
>$&fourier_elim([4*y-3]=5,[x])//4*y-3=5
\end{eulerprompt}
\begin{eulerformula}
\[
\left[ \left[ 4\,\left(y-2\right) \right] =0 \right] 
\]
\end{eulerformula}
\begin{eulercomment}
Solve\\
Nomor 13
\end{eulercomment}
\begin{eulerprompt}
>$&fourier_elim([x+3]<=4,[x])//x+3<=4
\end{eulerprompt}
\begin{eulerformula}
\[
\left[ \left[ x-1 \right] =0 \right] \lor \left[ \left[ 1-x   \right] >0 \right] 
\]
\end{eulerformula}
\begin{eulercomment}
Nomor 15
\end{eulercomment}
\begin{eulerprompt}
>$&fourier_elim([x+5]>2,[x])//x+5>2
\end{eulerprompt}
\begin{eulerformula}
\[
\left[ \left[ x+3 \right] >0 \right] 
\]
\end{eulerformula}
\begin{eulercomment}
Nomor 19
\end{eulercomment}
\begin{eulerprompt}
>$&solve(x^2+4*x =1,x)
\end{eulerprompt}
\begin{eulerformula}
\[
\left[ x=-\sqrt{5}-2 , x=\sqrt{5}-2 \right] 
\]
\end{eulerformula}
\eulerheading{4.1 Latihan Soal }
\begin{eulercomment}
Gunakan substitusi untuk menentukan apakah 2,3 dan -1 adalah nol dari\\
Nomor 23
\end{eulercomment}
\begin{eulerprompt}
>function P(x) &= (x^3-9*x^2+14*x+24); $&P(x)
\end{eulerprompt}
\begin{eulerformula}
\[
x^3-9\,x^2+14\,x+24
\]
\end{eulerformula}
\begin{eulerprompt}
>P(2)
\end{eulerprompt}
\begin{euleroutput}
        24.00 
\end{euleroutput}
\begin{eulerprompt}
>P(3)
\end{eulerprompt}
\begin{euleroutput}
        12.00 
\end{euleroutput}
\begin{eulerprompt}
>P(-1)
\end{eulerprompt}
\begin{euleroutput}
         0.00 
\end{euleroutput}
\begin{eulercomment}
Jadi hasil substitusi yang mempunyai nilai nol adalah dengan
mensubtsisusi angka -1

Nomor 24
\end{eulercomment}
\begin{eulerprompt}
>function P(x) &= (2*x^3-3*x^2+x+6);$&P(x)
\end{eulerprompt}
\begin{eulerformula}
\[
2\,x^3-3\,x^2+x+6
\]
\end{eulerformula}
\begin{eulerprompt}
>P(2)
\end{eulerprompt}
\begin{euleroutput}
        12.00 
\end{euleroutput}
\begin{eulerprompt}
>P(3)
\end{eulerprompt}
\begin{euleroutput}
        36.00 
\end{euleroutput}
\begin{eulerprompt}
>P(-1)
\end{eulerprompt}
\begin{euleroutput}
         0.00 
\end{euleroutput}
\begin{eulercomment}
Jadi hasil substitusi yang mempunyai nilai nol adalah dengan
mensubtsisusi angka -1

Nomor 25
\end{eulercomment}
\begin{eulerprompt}
>function P(x) &= (x^4-6*x^3+8*x^2+6*x-9);$&P(x)
\end{eulerprompt}
\begin{eulerformula}
\[
x^4-6\,x^3+8\,x^2+6\,x-9
\]
\end{eulerformula}
\begin{eulerprompt}
>P(2)
\end{eulerprompt}
\begin{euleroutput}
         3.00 
\end{euleroutput}
\begin{eulerprompt}
>P(3)
\end{eulerprompt}
\begin{euleroutput}
         0.00 
\end{euleroutput}
\begin{eulerprompt}
>P(-1)
\end{eulerprompt}
\begin{euleroutput}
         0.00 
\end{euleroutput}
\begin{eulercomment}
Jadi hasil substitusi yang menghasilkan persamaan mempunyai nilai nol
adalah dengan mensubtsisusi angka 3 dan -1

Nomor 37
\end{eulercomment}
\begin{eulerprompt}
>$&solve(x^4-4*x^2+3,x)
\end{eulerprompt}
\begin{eulerformula}
\[
\left[ x=-1 , x=1 , x=-\sqrt{3} , x=\sqrt{3} \right] 
\]
\end{eulerformula}
\begin{eulercomment}
Nomor 39
\end{eulercomment}
\begin{eulerprompt}
>$&solve(x^3+3*x^2-x-3,x)
\end{eulerprompt}
\begin{eulerformula}
\[
\left[ x=1 , x=-1 , x=-3 \right] 
\]
\end{eulerformula}
\eulerheading{4.3 Latihan Soal}
\begin{eulercomment}
Nomor 1\\
Untuk fungsi\\
\end{eulercomment}
\begin{eulerformula}
\[
f(x)= x^4-6x^3+x^2+24x-20
\]
\end{eulerformula}
\begin{eulercomment}
Gunakan pembagian panjang untuk menentukan apakah masing-masing dari
berikut ini merupakan faktor dari f(x)

a) x+1\\
b) x-2\\
c) x + 5
\end{eulercomment}
\begin{eulerprompt}
>function f(x) &= (x^4-6*x^3+x^2+24*x-20);$&f(x)
\end{eulerprompt}
\begin{eulerformula}
\[
x^4-6\,x^3+x^2+24\,x-20
\]
\end{eulerformula}
\begin{eulerprompt}
>$&f(x+1), $&expand(%)
\end{eulerprompt}
\begin{eulerformula}
\[
x^4-2\,x^3-11\,x^2+12\,x
\]
\end{eulerformula}
\eulerimg{0}{images/Alifia Rahmawati_23030630044_LaTex & Markdown-179-large.png}
\begin{eulerprompt}
>$&f(x-2), $&expand(%)
\end{eulerprompt}
\begin{eulerformula}
\[
x^4-14\,x^3+61\,x^2-84\,x
\]
\end{eulerformula}
\eulerimg{0}{images/Alifia Rahmawati_23030630044_LaTex & Markdown-181-large.png}
\begin{eulerprompt}
>$&f(x+5), $&expand(%)
\end{eulerprompt}
\begin{eulerformula}
\[
x^4+14\,x^3+61\,x^2+84\,x
\]
\end{eulerformula}
\eulerimg{0}{images/Alifia Rahmawati_23030630044_LaTex & Markdown-183-large.png}
\begin{eulercomment}
Nomor 23\\
Gunakan pembagian sintetis untuk menemukan nilai fungsi.\\
\end{eulercomment}
\begin{eulerformula}
\[
f(x) = x^3-6x^2+11x-6
\]
\end{eulerformula}
\begin{eulercomment}
cari f(1), f(-2), dan f(3)
\end{eulercomment}
\begin{eulerprompt}
>function f(x) &= (x^3-6*x^2+11*x-6);$&f(x)
\end{eulerprompt}
\begin{eulerformula}
\[
x^3-6\,x^2+11\,x-6
\]
\end{eulerformula}
\begin{eulerprompt}
>f(1)
\end{eulerprompt}
\begin{euleroutput}
         0.00 
\end{euleroutput}
\begin{eulerprompt}
>f(-2)
\end{eulerprompt}
\begin{euleroutput}
       -60.00 
\end{euleroutput}
\begin{eulerprompt}
>f(3)
\end{eulerprompt}
\begin{euleroutput}
         0.00 
\end{euleroutput}
\begin{eulercomment}
Nomor 24\\
\end{eulercomment}
\begin{eulerformula}
\[
f(x)=x63+7x^2-12x-3
\]
\end{eulerformula}
\begin{eulercomment}
cari f(-3),f(-2), dan f(1)
\end{eulercomment}
\begin{eulerprompt}
>function f(x) &= (x^3+7*x^2-12*x-3);$&f(x)
\end{eulerprompt}
\begin{eulerformula}
\[
x^3+7\,x^2-12\,x-3
\]
\end{eulerformula}
\begin{eulerprompt}
>f(-3)
\end{eulerprompt}
\begin{euleroutput}
        69.00 
\end{euleroutput}
\begin{eulerprompt}
>f(-2)
\end{eulerprompt}
\begin{euleroutput}
        41.00 
\end{euleroutput}
\begin{eulerprompt}
>f(1)
\end{eulerprompt}
\begin{euleroutput}
        -7.00 
\end{euleroutput}
\begin{eulercomment}
Nomor 25\\
\end{eulercomment}
\begin{eulerformula}
\[
f(x) = x^4-3x^2+2x+8
\]
\end{eulerformula}
\begin{eulercomment}
cari f(-1),f(4) dan f(-5)
\end{eulercomment}
\begin{eulerprompt}
>function f(x) &= (x^4-3*x^3+2*x+8);$&f(x)
\end{eulerprompt}
\begin{eulerformula}
\[
x^4-3\,x^3+2\,x+8
\]
\end{eulerformula}
\begin{eulerprompt}
>f(-1)
\end{eulerprompt}
\begin{euleroutput}
        10.00 
\end{euleroutput}
\begin{eulerprompt}
>f(4)
\end{eulerprompt}
\begin{euleroutput}
        80.00 
\end{euleroutput}
\begin{eulerprompt}
>f(-5)
\end{eulerprompt}
\begin{euleroutput}
       998.00 
\end{euleroutput}
\begin{eulercomment}
Faktorkan fungsi polinomial f(x). Kemudian selesaikan persamaan f(x)=0

Nomor 39\\
\end{eulercomment}
\begin{eulerformula}
\[
f(x)=x^3+4x^2+x-6
\]
\end{eulerformula}
\begin{eulerprompt}
>fx &= (x^3+4*x^2+x-6=0); $&fx
\end{eulerprompt}
\begin{eulerformula}
\[
x^3+4\,x^2+x-6=0
\]
\end{eulerformula}
\begin{eulerprompt}
>$&factor((fx,x^3+4*x^2+x-6=0))
\end{eulerprompt}
\begin{eulerformula}
\[
\left(x-1\right)\,\left(x+2\right)\,\left(x+3\right)=0
\]
\end{eulerformula}
\begin{eulerprompt}
>$&solve(x^3+4*x^2+x-6=0,x)
\end{eulerprompt}
\begin{eulerformula}
\[
\left[ x=-3 , x=-2 , x=1 \right] 
\]
\end{eulerformula}
\begin{euleroutput}
  
\end{euleroutput}
\begin{euleroutput}
  
\end{euleroutput}
\eulerheading{Ulasan Campuran Tengah Bab}
\begin{eulercomment}
unakan pembagian sintetis untuk menemukan nilai fungsi

Nomor 18\\
\end{eulercomment}
\begin{eulerformula}
\[
g(x) = x^3-9x^2+4x-10
\]
\end{eulerformula}
\begin{eulercomment}
cari g(-5)
\end{eulercomment}
\begin{eulerprompt}
>function g(x) &= (x^3-9*x^2+4*x-10);$&g(x)
>g(-5)
\end{eulerprompt}
\begin{euleroutput}
  -380
\end{euleroutput}
\begin{eulercomment}
Nomor 19\\
\end{eulercomment}
\begin{eulerformula}
\[
f(x)=20x^2-40x
\]
\end{eulerformula}
\begin{eulercomment}
cari f(1/2)
\end{eulercomment}
\begin{eulerprompt}
>function f(x) &= (20*x^2-40*x);$&f(x)
>f(1/2)
\end{eulerprompt}
\begin{euleroutput}
  -15
\end{euleroutput}
\begin{eulercomment}
Dengan menggunakan pembagian sintetis, tentukan apakah angka-angka
tersebut adalah nol dari fungsi polinomial.

Angka 22\\
-1,5;\\
lateks: f(x)=x\textasciicircum{}6-35x\textasciicircum{}4+259x\textasciicircum{}2-225
\end{eulercomment}
\begin{eulerprompt}
>function f(x) &= (x^6-35*x^4+259*x^2-225);$&f(x)
>f(-1.5)
\end{eulerprompt}
\begin{euleroutput}
  191.953125
\end{euleroutput}
\begin{eulercomment}
Faktorkan fungsi polinomial f(x). Kemudian selesaikan persamaan f(x)
=0.\\
Nomor 23\\
\end{eulercomment}
\begin{eulerformula}
\[
h(x) = x^3-2x^2-55x+56
\]
\end{eulerformula}
\begin{eulerprompt}
>hx &= (x^3-2*x^2-55*x+56=0); $&hx
>$&factor((hx,x^3-2*x^2-55*x+56=0))
>$&solve(x^3-2*x^2-55*x+56=0,x)
\end{eulerprompt}
\begin{eulercomment}
Nomor 24\\
\end{eulercomment}
\begin{eulerformula}
\[
g(x) = x^4-2x^3-13x^2+14x+24
\]
\end{eulerformula}
\begin{eulerprompt}
>gx &= (x^4-2*x^3-13*x^2+14*x+24=0); $&gx
>$&factor((gx,x^4-2*x^3-13*x^2+14*x+24=0))
>$&solve(x^4-2*x^3-13*x^2+14*x+24=0,x)
\end{eulerprompt}
\begin{euleroutput}
  
\end{euleroutput}
\begin{eulercomment}
Nama  : Alifia Rahmawati\\
NIM   : 23030630044\\
Kelas : Matematika E 2023\\
TUGAS INDIVIDU GRAFIK 2D

\begin{eulercomment}
\eulerheading{Menggambar Grafik 2D dengan EMT}
\begin{eulercomment}
Notebook ini menjelaskan tentang cara menggambar berbagaikurva dan
grafik 2D dengan software EMT. EMT menyediakan fungsi plot2d() untuk
menggambar berbagai kurva dan grafik dua dimensi (2D).\\
\end{eulercomment}
\eulersubheading{Plot Dasar}
\begin{eulercomment}
Ada beberapa fungsi dasar plot. Ada koordinat layar, yang selalu
berkisar dari 0 hingga 1024 di setiap sumbu, tidak peduli apakah
layarnya persegi atau tidak. Selain itu, ada koordinat plot, yang
dapat diatur dengan setplot(). Pemetaan antara koordinat bergantung
pada jendela plot saat ini. Misalnya, shrinkwindow() default
menyisakan ruang untuk label sumbu dan judul plot.

Dalam contoh, kita hanya menggambar beberapa garis acak dalam berbagai
warna. Untuk detail tentang fungsi-fungsi ini, pelajari fungsi inti
EMT.
\end{eulercomment}
\begin{eulerprompt}
>clg; // clear screen
>window(0,0,1024,1024); // use all of the window
>setplot(0,1,0,1); // set plot coordinates
>hold on; // start overwrite mode
>n=100; X=random(n,2); Y=random(n,2);  // get random points
>colors=rgb(random(n),random(n),random(n)); // get random colors
>loop 1 to n; color(colors[#]); plot(X[#],Y[#]); end; // plot
>insimg;
>hold off;
>reset;
\end{eulerprompt}
\begin{eulercomment}
Grafik perlu ditahan, karena perintah plot() akan membersihkan jendela
plot.

Untuk membersihkan semua yang telah kita lakukan, kita menggunakan
reset().

Untuk menampilkan gambar hasil plot di layar notebook, perintah
plot2d() dapat diakhiri dengan titik dua (:). Cara lainnya adalah
perintah plot2d() diakhiri dengan titik koma (;), kemudian menggunakan
perintah insimg() untuk menampilkan gambar hasil plot.

Sebagai contoh lain, kita menggambar plot sebagai inset di plot lain.
Ini dilakukan dengan mendefinisikan jendela plot yang lebih kecil.
Perhatikan bahwa jendela ini tidak menyediakan ruang untuk label sumbu
di luar jendela plot. Kita harus menambahkan beberapa margin untuk ini
sesuai kebutuhan. Perhatikan bahwa kita menyimpan dan memulihkan
jendela penuh, dan menahan plot saat ini saat kita memplot inset.
\end{eulercomment}
\begin{eulerprompt}
>plot2d("x^3-x");
>xw=200; yw=100; ww=300; hw=300;
>ow=window();
>window(xw,yw,xw+ww,yw+hw);
>hold on;
>barclear(xw-50,yw-10,ww+60,ww+60);
>plot2d("x^4-x",grid=6):
>hold off;
>window(ow);
\end{eulerprompt}
\begin{eulercomment}
Plot dengan beberapa gambar dicapai dengan cara yang sama. Ada fungsi
utilitas figure() untuk ini.

\end{eulercomment}
\eulersubheading{Plot Aspect}
\begin{eulercomment}
Plot default menggunakan jendela plot persegi. Anda dapat mengubahnya
dengan fungsi aspect(). Jangan lupa untuk mengatur ulang aspek nanti.
Anda juga dapat mengubah default ini di menu dengan "Set Aspect" ke
rasio aspek tertentu atau ke ukuran jendela grafik saat ini.

Tetapi Anda juga dapat mengubahnya untuk satu plot. Untuk ini, ukuran
area plot saat ini diubah, dan jendela diatur sehingga label memiliki
cukup ruang.
\end{eulercomment}
\begin{eulerprompt}
>aspect(2); // rasio panjang dan lebar 2:1
>plot2d(["sin(x)","cos(x)"],0,2pi):
>aspect();
>reset;
\end{eulerprompt}
\begin{eulercomment}
Fungsi reset() mengembalikan default plot termasuk rasio aspek.\\
Plot 2D di Euler

EMT Math Toolbox memiliki plot dalam 2D, baik untuk data maupun
fungsi. EMT menggunakan fungsi plot2d. Fungsi ini dapat memplot fungsi
dan data.

Dimungkinkan untuk memplot di Maxima menggunakan Gnuplot atau di
Python menggunakan Math Plot Lib.

Euler dapat memplot plot 2D dari

- ekspresi\\
- fungsi, variabel, atau kurva berparameter,\\
- vektor nilai x-y,\\
- awan titik di bidang,\\
- kurva implisit dengan level atau daerah level.\\
- Fungsi kompleks

Gaya plot mencakup berbagai gaya untuk garis dan titik, plot batang,
dan plot berbayang.

\begin{eulercomment}
\eulerheading{Plot Ekspresi atau Variabel}
\begin{eulercomment}
Ekspresi tunggal dalam "x" (misalnya "4*x\textasciicircum{}2") atau nama fungsi
(misalnya "f") menghasilkan grafik fungsi.

Berikut adalah contoh paling dasar, yang menggunakan rentang default
dan menetapkan rentang y yang tepat agar sesuai dengan plot fungsi.

Catatan: Jika Anda mengakhiri baris perintah dengan titik dua ":",
plot akan disisipkan ke dalam jendela teks. Jika tidak, tekan TAB
untuk melihat plot jika jendela plot tertutup.
\end{eulercomment}
\begin{eulerprompt}
>plot2d("x^2"):
>aspect(1.5); plot2d("x^3-x"):
>a:=5.6; plot2d("exp(-a*x^2)/a"); insimg(30);
\end{eulerprompt}
\begin{eulercomment}
Dari beberapa contoh sebelumnya Anda dapat melihat bahwa aslinya
gambar plot menggunakan sumbu X dengan rentang nilai dari -2 sampai
dengan 2. Untuk mengubah rentang nilai X dan Y, Anda dapat menambahkan
nilai-nilai batas X (dan Y) di belakang ekspresi yang digambar.

Rentang plot ditetapkan dengan parameter yang ditetapkan berikut

- a,b: rentang x (default -2,2)\\
- c,d: rentang y (default: skala dengan nilai)\\
- r: alternatifnya radius di sekitar pusat plot\\
- cx,cy: koordinat pusat plot (default 0,0)
\end{eulercomment}
\begin{eulerprompt}
>reset;
>figure(2,2); ...
>for n=1 to 4; figure(n); plot2d("x^"+n); end; ...
>figure(0):
\end{eulerprompt}
\begin{eulercomment}
Dalam plot2d(), tersedia gaya alternatif dengan grid=x. Sebagai
gambaran umum, kami menampilkan berbagai gaya grid dalam satu gambar
(lihat di bawah untuk perintah figure()). Gaya grid=0 tidak
disertakan. Gaya ini tidak menampilkan grid dan bingkai.
\end{eulercomment}
\begin{eulerprompt}
>figure(3,3); ...
>for k=1:9; figure(k); plot2d("x^3-x",-2,1,grid=k); end; ...
>figure(0):
\end{eulerprompt}
\begin{eulercomment}
Jika argumen untuk plot2d() adalah ekspresi yang diikuti oleh empat
angka, angka-angka ini adalah rentang x dan y untuk plot.

Atau, a, b, c, d dapat ditetapkan sebagai parameter yang ditetapkan
sebagai a=... dst.

Dalam contoh berikut, kami mengubah gaya kisi, menambahkan label, dan
menggunakan label vertikal untuk sumbu y.
\end{eulercomment}
\begin{eulerprompt}
>aspect(1.5); plot2d("sin(x)",0,2pi,-1.2,1.2,grid=3,xl="x",yl="sin(x)"):
>plot2d("x^x",r=1.2,cx=1,cy=1):
\end{eulerprompt}
\begin{eulercomment}
Perhatikan bahwa x\textasciicircum{}x tidak didefinisikan untuk x\textless{}=0. Fungsi plot2d
menangkap kesalahan ini, dan mulai memplot segera setelah fungsi
didefinisikan. Ini berfungsi untuk semua fungsi yang mengembalikan NAN
di luar rentang definisinya.
\end{eulercomment}
\begin{eulerprompt}
>plot2d("log(x)",-0.1,2):
\end{eulerprompt}
\begin{eulercomment}
Parameter square=true (atau \textgreater{}square) memilih rentang y secara otomatis
sehingga hasilnya adalah jendela plot persegi. Perhatikan bahwa secara
default, Euler menggunakan ruang persegi di dalam jendela plot.
\end{eulercomment}
\begin{eulerprompt}
>plot2d("x^3-x",>square):
>plot2d(''integrate("sin(x)*exp(-x^2)",0,x)'',0,2): // plot integral
\end{eulerprompt}
\begin{eulercomment}
Jika Anda memerlukan lebih banyak ruang untuk label-y, panggil
shrinkwindow() dengan parameter yang lebih kecil, atau tetapkan nilai
positif untuk "lebih kecil" di plot2d().
\end{eulercomment}
\begin{eulerprompt}
>plot2d("gamma(x)",1,10,yl="y-values",smaller=6,<vertical):
\end{eulerprompt}
\begin{eulercomment}
Ekspresi simbolik juga dapat digunakan, karena disimpan sebagai
ekspresi string sederhana.
\end{eulercomment}
\begin{eulerprompt}
>x=linspace(0,2pi,1000); plot2d(sin(5x),cos(7x)):
>a:=5.6; expr &= exp(-a*x^2)/a; // define expression
>plot2d(expr,-2,2): // plot from -2 to 2
>plot2d(expr,r=1,thickness=2): // plot in a square around (0,0)
>plot2d(&diff(expr,x),>add,style="--",color=red): // add another plot
>plot2d(&diff(expr,x,2),a=-2,b=2,c=-2,d=1): // plot in rectangle
>plot2d(&diff(expr,x),a=-2,b=2,>square): // keep plot square
>plot2d("x^2",0,1,steps=1,color=red,n=10):
>plot2d("x^2",>add,steps=2,color=blue,n=10):
\end{eulerprompt}
\eulerheading{Fungsi dalam satu Parameter}
\begin{eulercomment}
Fungsi plotting yang paling penting untuk plot planar adalah plot2d().
Fungsi ini diimplementasikan dalam bahasa Euler dalam file "plot.e",
yang dimuat di awal program.

Berikut ini beberapa contoh penggunaan fungsi. Seperti biasa dalam
EMT, fungsi yang berfungsi untuk fungsi atau ekspresi lain, Anda dapat
meneruskan parameter tambahan (selain x) yang bukan variabel global ke
fungsi dengan parameter titik koma atau dengan koleksi panggilan.
\end{eulercomment}
\begin{eulerprompt}
>function f(x,a) := x^2/a+a*x^2-x; // define a function
>a=0.3; plot2d("f",0,1;a): // plot with a=0.3
>plot2d("f",0,1;0.4): // plot with a=0.4
>plot2d(\{\{"f",0.2\}\},0,1): // plot with a=0.2
>plot2d(\{\{"f(x,b)",b=0.1\}\},0,1): // plot with 0.1
>function f(x) := x^3-x; ...
>plot2d("f",r=1):
\end{eulerprompt}
\begin{eulercomment}
Berikut ini adalah ringkasan fungsi yang diterima

- ekspresi atau ekspresi simbolik dalam x\\
- fungsi atau fungsi simbolik berdasarkan nama seperti "f"\\
- fungsi simbolik hanya berdasarkan nama f

Fungsi plot2d() juga menerima fungsi simbolik. Untuk fungsi simbolik,
hanya nama yang berfungsi.
\end{eulercomment}
\begin{eulerprompt}
>function f(x) &= diff(x^x,x)
\end{eulerprompt}
\begin{euleroutput}
  
                              x
                             x  (log(x) + 1)
  
\end{euleroutput}
\begin{eulerprompt}
>plot2d(f,0,2):
\end{eulerprompt}
\begin{eulercomment}
Tentu saja, untuk ekspresi atau ungkapan simbolik, nama variabel sudah
cukup untuk memplotnya.
\end{eulercomment}
\begin{eulerprompt}
>expr &= sin(x)*exp(-x)
\end{eulerprompt}
\begin{euleroutput}
  
                                - x
                               E    sin(x)
  
\end{euleroutput}
\begin{eulerprompt}
>plot2d(expr,0,3pi):
>function f(x) &= x^x;
>plot2d(f,r=1,cx=1,cy=1,color=blue,thickness=2);
>plot2d(&diff(f(x),x),>add,color=red,style="-.-"):
\end{eulerprompt}
\begin{eulercomment}
Untuk gaya garis, ada berbagai pilihan.

- style="...". Pilih dari "-", "--", "-.", ".", ".-.", "-.-".\\
- color: Lihat di bawah untuk warna.

thickness: Default adalah 1.

Warna dapat dipilih sebagai salah satu warna default, atau sebagai
warna RGB.

- 0..15: indeks warna default.\\
- konstanta warna: putih, hitam, merah, hijau, biru, cyan, zaitun,
abu-abu muda, abu-abu, abu-abu tua, oranye, hijau muda, biru
kehijauan, biru muda, oranye muda, kuning\\
- rgb(merah,hijau,biru): parameter adalah bilangan real dalam [0,1].
\end{eulercomment}
\begin{eulerprompt}
>plot2d("exp(-x^2)",r=2,color=red,thickness=3,style="--"):
\end{eulerprompt}
\begin{eulercomment}
Berikut ini tampilan warna EMT yang telah ditetapkan sebelumnya.
\end{eulercomment}
\begin{eulerprompt}
>aspect(2); columnsplot(ones(1,16),lab=0:15,grid=0,color=0:15):
\end{eulerprompt}
\begin{eulercomment}
Namun Anda dapat menggunakan warna apa pun.
\end{eulercomment}
\begin{eulerprompt}
>columnsplot(ones(1,16),grid=0,color=rgb(0,0,linspace(0,1,15))):
\end{eulerprompt}
\eulerheading{Menggambar Beberapa Kurva pada bidang koordinat yang sama}
\begin{eulercomment}
Plotting lebih dari satu fungsi (multifungsi) ke dalam satu jendela
dapat dilakukan dengan berbagai cara. Salah satu metodenya adalah
menggunakan \textgreater{}add untuk beberapa panggilan ke plot2d secara
keseluruhan, kecuali panggilan pertama. Kami telah menggunakan fitur
ini pada contoh di atas.
\end{eulercomment}
\begin{eulerprompt}
>aspect(); plot2d("cos(x)",r=2,grid=6); plot2d("x",style=".",>add):
>aspect(1.5); plot2d("sin(x)",0,2pi); plot2d("cos(x)",color=blue,style="--",>add):
\end{eulerprompt}
\begin{eulercomment}
Salah satu kegunaan \textgreater{}add adalah untuk menambahkan titik pada kurva.
\end{eulercomment}
\begin{eulerprompt}
>plot2d("sin(x)",0,pi); plot2d(2,sin(2),>points,>add):
\end{eulerprompt}
\begin{eulercomment}
Kami menambahkan titik potong dengan label (pada posisi "cl" untuk
tengah kiri), dan memasukkan hasilnya ke dalam buku catatan. Kami juga
menambahkan judul pada plot.
\end{eulercomment}
\begin{eulerprompt}
>plot2d(["cos(x)","x"],r=1.1,cx=0.5,cy=0.5, ...
>color=[black,blue],style=["-","."], ...
>grid=1);
>x0=solve("cos(x)-x",1);  ...
>plot2d(x0,x0,>points,>add,title="Intersection Demo");  ...
>label("cos(x) = x",x0,x0,pos="cl",offset=20):
\end{eulerprompt}
\begin{eulercomment}
Dalam demo berikut, kami memplot fungsi sinc(x)=sin(x)/x dan ekspansi
Taylor ke-8 dan ke-16. Kami menghitung ekspansi ini menggunakan Maxima
melalui ekspresi simbolik.\\
Plot ini dilakukan dalam perintah multi-baris berikut dengan tiga
panggilan ke plot2d(). Yang kedua dan ketiga memiliki set flag \textgreater{}add,
yang membuat plot menggunakan rentang sebelumnya.

Kami menambahkan kotak label yang menjelaskan fungsi-fungsi tersebut.
\end{eulercomment}
\begin{eulerprompt}
>$taylor(sin(x)/x,x,0,4)
>plot2d("sinc(x)",0,4pi,color=green,thickness=2); ...
>  plot2d(&taylor(sin(x)/x,x,0,8),>add,color=blue,style="--"); ...
>  plot2d(&taylor(sin(x)/x,x,0,16),>add,color=red,style="-.-"); ...
>  labelbox(["sinc","T8","T16"],styles=["-","--","-.-"], ...
>    colors=[black,blue,red]):
\end{eulerprompt}
\begin{eulercomment}
Dalam contoh berikut, kami menghasilkan Polinomial Bernstein.

\end{eulercomment}
\begin{eulerformula}
\[
B_i(x) = \binom{n}{i} x^i (1-x)^{n-i}
\]
\end{eulerformula}
\begin{eulerprompt}
>plot2d("(1-x)^10",0,1); // plot first function
>for i=1 to 10; plot2d("bin(10,i)*x^i*(1-x)^(10-i)",>add); end;
>insimg;
\end{eulerprompt}
\begin{eulercomment}
Metode kedua menggunakan sepasang matriks nilai-x dan matriks nilai-y
dengan ukuran yang sama.

Kita buat matriks nilai dengan satu Polinomial Bernstein di setiap
baris. Untuk ini, kita cukup menggunakan vektor kolom i. Lihat
pengantar tentang bahasa matriks untuk mempelajari lebih detail.
\end{eulercomment}
\begin{eulerprompt}
>x=linspace(0,1,500);
>n=10; k=(0:n)'; // n is row vector, k is column vector
>y=bin(n,k)*x^k*(1-x)^(n-k); // y is a matrix then
>plot2d(x,y):
\end{eulerprompt}
\begin{eulercomment}
Perhatikan bahwa parameter warna dapat berupa vektor. Maka setiap
warna digunakan untuk setiap baris matriks.
\end{eulercomment}
\begin{eulerprompt}
>x=linspace(0,1,200); y=x^(1:10)'; plot2d(x,y,color=1:10):
\end{eulerprompt}
\begin{eulercomment}
Metode lain adalah menggunakan vektor ekspresi (string). Anda kemudian
dapat menggunakan array warna, array gaya, dan array ketebalan dengan
panjang yang sama.
\end{eulercomment}
\begin{eulerprompt}
>plot2d(["sin(x)","cos(x)"],0,2pi,color=4:5): 
>plot2d(["sin(x)","cos(x)"],0,2pi): // plot vector of expressions
\end{eulerprompt}
\begin{eulercomment}
Kita bisa mendapatkan vektor tersebut dari Maxima menggunakan
makelist() dan mxm2str().
\end{eulercomment}
\begin{eulerprompt}
>v &= makelist(binomial(10,i)*x^i*(1-x)^(10-i),i,0,10) // make list
\end{eulerprompt}
\begin{euleroutput}
  
                 10            9              8  2             7  3
         [(1 - x)  , 10 (1 - x)  x, 45 (1 - x)  x , 120 (1 - x)  x , 
             6  4             5  5             4  6             3  7
  210 (1 - x)  x , 252 (1 - x)  x , 210 (1 - x)  x , 120 (1 - x)  x , 
            2  8              9   10
  45 (1 - x)  x , 10 (1 - x) x , x  ]
  
\end{euleroutput}
\begin{eulerprompt}
>mxm2str(v) // get a vector of strings from the symbolic vector
\end{eulerprompt}
\begin{euleroutput}
  (1-x)^10
  10*(1-x)^9*x
  45*(1-x)^8*x^2
  120*(1-x)^7*x^3
  210*(1-x)^6*x^4
  252*(1-x)^5*x^5
  210*(1-x)^4*x^6
  120*(1-x)^3*x^7
  45*(1-x)^2*x^8
  10*(1-x)*x^9
  x^10
\end{euleroutput}
\begin{eulerprompt}
>plot2d(mxm2str(v),0,1): // plot functions
\end{eulerprompt}
\begin{eulercomment}
Alternatif lain adalah dengan menggunakan bahasa matriks Euler.

Jika suatu ekspresi menghasilkan matriks fungsi, dengan satu fungsi di
setiap baris, semua fungsi ini akan diplot menjadi satu plot.

Untuk ini, gunakan vektor parameter dalam bentuk vektor kolom. Jika
array warna ditambahkan, array tersebut akan digunakan untuk setiap
baris plot.
\end{eulercomment}
\begin{eulerprompt}
>n=(1:10)'; plot2d("x^n",0,1,color=1:10):
\end{eulerprompt}
\begin{eulercomment}
Ekspresi dan fungsi satu baris dapat melihat variabel global.

Jika Anda tidak dapat menggunakan variabel global, Anda perlu
menggunakan fungsi dengan parameter tambahan, dan meneruskan parameter
ini sebagai parameter titik koma.

Berhati-hatilah, untuk meletakkan semua parameter yang ditetapkan di
akhir perintah plot2d. Dalam contoh ini, kami meneruskan a=5 ke fungsi
f, yang kami plot dari -10 hingga 10.
\end{eulercomment}
\begin{eulerprompt}
>function f(x,a) := 1/a*exp(-x^2/a); ...
>plot2d("f",-10,10;5,thickness=2,title="a=5"):
\end{eulerprompt}
\begin{eulercomment}
Atau, gunakan koleksi dengan nama fungsi dan semua parameter tambahan.
Daftar khusus ini disebut koleksi panggilan, dan itu adalah cara yang
lebih disukai untuk meneruskan argumen ke suatu fungsi yang diteruskan
sebagai argumen ke fungsi lain.

Dalam contoh berikut, kami menggunakan loop untuk memplot beberapa
fungsi (lihat tutorial tentang pemrograman untuk loop).
\end{eulercomment}
\begin{eulerprompt}
>plot2d(\{\{"f",1\}\},-10,10); ...
>for a=2:10; plot2d(\{\{"f",a\}\},>add); end:
\end{eulerprompt}
\begin{eulercomment}
Kita dapat memperoleh hasil yang sama dengan cara berikut menggunakan
bahasa matriks EMT. Setiap baris matriks f(x,a) adalah satu fungsi.
Selain itu, kita dapat mengatur warna untuk setiap baris matriks. Klik
dua kali pada fungsi getspectral() untuk penjelasannya.
\end{eulercomment}
\begin{eulerprompt}
>x=-10:0.01:10; a=(1:10)'; plot2d(x,f(x,a),color=getspectral(a/10)):
\end{eulerprompt}
\eulersubheading{Label Teks}
\begin{eulercomment}
Dekorasi sederhana dapat berupa

- judul dengan title="..."\\
- label x dan y dengan xl="...", yl="..."\\
- label teks lain dengan label("...",x,y)

Perintah label akan memplot ke dalam plot saat ini pada koordinat plot
(x,y). Perintah ini dapat mengambil argumen posisi.
\end{eulercomment}
\begin{eulerprompt}
>plot2d("x^3-x",-1,2,title="y=x^3-x",yl="y",xl="x"):
>expr := "log(x)/x"; ...
>  plot2d(expr,0.5,5,title="y="+expr,xl="x",yl="y"); ...
>  label("(1,0)",1,0); label("Max",E,expr(E),pos="lc"):
\end{eulerprompt}
\begin{eulercomment}
Ada juga fungsi labelbox(), yang dapat menampilkan fungsi dan teks.
Fungsi ini mengambil vektor string dan warna, satu item untuk setiap
fungsi.
\end{eulercomment}
\begin{eulerprompt}
>function f(x) &= x^2*exp(-x^2);  ...
>plot2d(&f(x),a=-3,b=3,c=-1,d=1);  ...
>plot2d(&diff(f(x),x),>add,color=blue,style="--"); ...
>labelbox(["function","derivative"],styles=["-","--"], ...
>   colors=[black,blue],w=0.4):
\end{eulerprompt}
\begin{eulercomment}
Kotak tersebut ditambatkan di kanan atas secara default, tetapi \textgreater{}left
menambatkannya di kiri atas. Anda dapat memindahkannya ke tempat mana
pun yang Anda suka. Posisi jangkar adalah sudut kanan atas kotak, dan
angka-angkanya adalah pecahan dari ukuran jendela grafik. Lebarnya
otomatis.

Untuk plot titik, kotak label juga berfungsi. Tambahkan parameter
\textgreater{}points, atau vektor bendera, satu untuk setiap label.

Dalam contoh berikut, hanya ada satu fungsi. Jadi, kita dapat
menggunakan string alih-alih vektor string. Kita tetapkan warna teks
menjadi hitam untuk contoh ini.
\end{eulercomment}
\begin{eulerprompt}
>n=10; plot2d(0:n,bin(n,0:n),>addpoints); ...
>labelbox("Binomials",styles="[]",>points,x=0.1,y=0.1, ...
>tcolor=black,>left):
\end{eulerprompt}
\begin{eulercomment}
Gaya plot ini juga tersedia di statplot(). Seperti di plot2d(), warna
dapat diatur untuk setiap baris plot. Ada plot yang lebih khusus untuk
keperluan statistik (lihat tutorial tentang statistik).
\end{eulercomment}
\begin{eulerprompt}
>statplot(1:10,random(2,10),color=[red,blue]):
\end{eulerprompt}
\begin{eulercomment}
Fitur serupa adalah fungsi textbox().

Lebar secara default adalah lebar maksimal baris teks. Namun, pengguna
juga dapat mengaturnya.
\end{eulercomment}
\begin{eulerprompt}
>function f(x) &= exp(-x)*sin(2*pi*x); ...
>plot2d("f(x)",0,2pi); ...
>textbox(latex("\(\backslash\)text\{Example of a damped oscillation\}\(\backslash\) f(x)=e^\{-x\}sin(2\(\backslash\)pi x)"),w=0.85):
\end{eulerprompt}
\begin{eulercomment}
Label teks, judul, kotak label, dan teks lainnya dapat berisi string
Unicode (lihat sintaksis EMT untuk informasi lebih lanjut tentang
string Unicode).
\end{eulercomment}
\begin{eulerprompt}
>plot2d("x^3-x",title=u"x &rarr; x&sup3; - x"):
\end{eulerprompt}
\begin{eulercomment}
Label pada sumbu x dan y dapat vertikal, begitu pula sumbunya.
\end{eulercomment}
\begin{eulerprompt}
>plot2d("sinc(x)",0,2pi,xl="x",yl=u"x &rarr; sinc(x)",>vertical):
\end{eulerprompt}
\eulersubheading{LaTeX}
\begin{eulercomment}
Anda juga dapat memplot rumus LaTeX jika Anda telah menginstal sistem
LaTeX. Saya merekomendasikan MiKTeX. Jalur ke biner "latex" dan
"dvipng" harus berada di jalur sistem, atau Anda harus mengatur LaTeX
di menu opsi.

Perlu dicatat, bahwa penguraian LaTeX lambat. Jika Anda ingin
menggunakan LaTeX dalam plot animasi, Anda harus memanggil latex()
sebelum loop sekali dan menggunakan hasilnya (gambar dalam matriks
RGB).

Dalam plot berikut, kami menggunakan LaTeX untuk label x dan y, label,
kotak label, dan judul plot.
\end{eulercomment}
\begin{eulerprompt}
>plot2d("exp(-x)*sin(x)/x",a=0,b=2pi,c=0,d=1,grid=6,color=blue, ...
> title=latex("\(\backslash\)text\{Function $\(\backslash\)Phi$\}"), ...
>  xl=latex("\(\backslash\)phi"),yl=latex("\(\backslash\)Phi(\(\backslash\)phi)")); ...
>textbox( ...
>  latex("\(\backslash\)Phi(\(\backslash\)phi) = e^\{-\(\backslash\)phi\} \(\backslash\)frac\{\(\backslash\)sin(\(\backslash\)phi)\}\{\(\backslash\)phi\}"),x=0.8,y=0.5); ...
>label(latex("\(\backslash\)Phi",color=blue),1,0.4):
\end{eulerprompt}
\begin{eulercomment}
Sering kali, kita menginginkan spasi nonkonformal dan label teks pada
sumbu x. Kita dapat menggunakan xaxis() dan yaxis() seperti yang akan
kita tunjukkan nanti.

Cara termudah adalah membuat plot kosong dengan bingkai menggunakan
grid=4, lalu menambahkan grid dengan ygrid() dan xgrid(). Dalam contoh
berikut, kita menggunakan tiga string LaTeX untuk label pada sumbu x
dengan xtick().
\end{eulercomment}
\begin{eulerprompt}
>plot2d("sinc(x)",0,2pi,grid=4,<ticks); ...
>ygrid(-2:0.5:2,grid=6); ...
>xgrid([0:2]*pi,<ticks,grid=6);  ...
>xtick([0,pi,2pi],["0","\(\backslash\)pi","2\(\backslash\)pi"],>latex):
\end{eulerprompt}
\begin{eulercomment}
Tentu saja, fungsi juga dapat digunakan.
\end{eulercomment}
\begin{eulerprompt}
>function map f(x) ...
\end{eulerprompt}
\begin{eulerudf}
  if x>0 then return x^4
  else return x^2
  endif
  endfunction
\end{eulerudf}
\begin{eulercomment}
Parameter "peta" membantu menggunakan fungsi untuk vektor. Untuk plot,
parameter tersebut tidak diperlukan. Namun, untuk menunjukkan bahwa
vektorisasi berguna, kami menambahkan beberapa poin penting ke plot
pada x=-1, x=0, dan x=1.\\
Dalam plot berikut, kami juga memasukkan beberapa kode LaTeX. Kami
menggunakannya untuk dua label dan kotak teks. Tentu saja, Anda hanya
dapat menggunakan LaTeX jika Anda telah menginstal LaTeX dengan benar.
\end{eulercomment}
\begin{eulerprompt}
>plot2d("f",-1,1,xl="x",yl="f(x)",grid=6);  ...
>plot2d([-1,0,1],f([-1,0,1]),>points,>add); ...
>label(latex("x^3"),0.72,f(0.72)); ...
>label(latex("x^2"),-0.52,f(-0.52),pos="ll"); ...
>textbox( ...
>  latex("f(x)=\(\backslash\)begin\{cases\} x^3 & x>0 \(\backslash\)\(\backslash\) x^2 & x \(\backslash\)le 0\(\backslash\)end\{cases\}"), ...
>  x=0.7,y=0.2):
\end{eulerprompt}
\begin{euleroutput}
  Variable f not found!
  Use global variables or parameters for string evaluation.
  Error in expression: f
   %ploteval:
      y0=f$(x[1],args());
  adaptiveevalone:
      s=%ploteval(g$,t;args());
  Try "trace errors" to inspect local variables after errors.
  plot2d:
      dw/n,dw/n^2,dw/n,auto;args());
\end{euleroutput}
\begin{eulercomment}
\end{eulercomment}
\eulersubheading{Interaksi Pengguna}
\begin{eulercomment}
Saat memplot fungsi atau ekspresi, parameter \textgreater{}user memungkinkan
pengguna untuk memperbesar dan menggeser plot dengan tombol kursor
atau tetikus. Pengguna dapat

- memperbesar dengan + atau -\\
- memindahkan plot dengan tombol kursor\\
- memilih jendela plot dengan tetikus\\
- mengatur ulang tampilan dengan spasi\\
- keluar dengan kembali

Tombol spasi akan mengatur ulang plot ke jendela plot asli.

Saat memplot data, tanda \textgreater{}user akan menunggu penekanan tombol.
\end{eulercomment}
\begin{eulerprompt}
>plot2d(\{\{"x^3-a*x",a=1\}\},>user,title="Press any key!"):
>plot2d("exp(x)*sin(x)",user=true, ...
>  title="+/- or cursor keys (return to exit)"):
\end{eulerprompt}
\begin{eulercomment}
The following demonstrates an advanced way of user interaction (see
the tutorial about programming for details).

The built-in function mousedrag() waits for mouse or keyboard events.
It reports mouse down, mouse moved or mouse up, and key presses. The
function dragpoints() makes use of this, and lets the user drag any
point in a plot.

We need a plot function first. For an example, we interpolate in 5
points with a polynomial. The function should plot into a fixed plot
area.
\end{eulercomment}
\begin{eulerprompt}
>function plotf(xp,yp,select) ...
\end{eulerprompt}
\begin{eulerudf}
  d=interp(xp,yp);
    plot2d("interpval(xp,d,x)";d,xp,r=2);
    plot2d(xp,yp,>points,>add);
    if select>0 then
      plot2d(xp[select],yp[select],color=red,>points,>add);
    endif;
    title("Drag one point, or press space or return!");
  endfunction
  
\end{eulerudf}
\begin{eulercomment}
Perhatikan parameter titik koma di plot2d (d dan xp), yang diteruskan
ke evaluasi fungsi interp(). Tanpa ini, kita harus menulis fungsi
plotinterp() terlebih dahulu, mengakses nilai secara global.

Sekarang kita buat beberapa nilai acak, dan biarkan pengguna menyeret
titik-titiknya.
\end{eulercomment}
\begin{eulerprompt}
>t=-1:0.5:1; dragpoints("plotf",t,random(size(t))-0.5):
\end{eulerprompt}
\begin{euleroutput}
  Variable plotf not found!
  Use global variables or parameters for string evaluation.
  Error in expression: plotf
  Try "trace errors" to inspect local variables after errors.
  dragpoints:
      f$(x,y,select,args());
\end{euleroutput}
\begin{eulercomment}
Ada juga fungsi yang memplot fungsi lain tergantung pada vektor
parameter dan memungkinkan pengguna menyesuaikan parameter tersebut.

Pertama, kita perlu fungsi plot.
\end{eulercomment}
\begin{eulerprompt}
>function plotf([a,b]) := plot2d("exp(a*x)*cos(2pi*b*x)",0,2pi;a,b);
\end{eulerprompt}
\begin{eulercomment}
Kemudian kita perlu nama untuk parameter, nilai awal, dan matriks
rentang nx2, secara opsional baris judul.\\
Ada slider interaktif, yang dapat mengatur nilai oleh pengguna. Fungsi
dragvalues() menyediakan ini.
\end{eulercomment}
\begin{eulerprompt}
>dragvalues("plotf",["a","b"],[-1,2],[[-2,2];[1,10]], ...
>  heading="Drag these values:",hcolor=black):
\end{eulerprompt}
\begin{eulercomment}
Dimungkinkan untuk membatasi nilai yang diseret ke bilangan bulat.
Misalnya, kita menulis fungsi plot, yang memplot polinomial Taylor
berderajat n ke fungsi kosinus.
\end{eulercomment}
\begin{eulerprompt}
>function plotf(n) ...
\end{eulerprompt}
\begin{eulerudf}
  plot2d("cos(x)",0,2pi,>square,grid=6);
  plot2d(&"taylor(cos(x),x,0,@n)",color=blue,>add);
  textbox("Taylor polynomial of degree "+n,0.1,0.02,style="t",>left);
  endfunction
\end{eulerudf}
\begin{eulercomment}
Sekarang kita biarkan derajat n bervariasi dari 0 hingga 20 dalam 20
stop. Hasil dragvalues() digunakan untuk memplot sketsa dengan n ini,
dan untuk memasukkan plot ke dalam buku catatan.
\end{eulercomment}
\begin{eulerprompt}
>nd=dragvalues("plotf","degree",2,[0,20],20,y=0.8, ...
>   heading="Drag the value:"); ...
>plotf(nd):
\end{eulerprompt}
\begin{eulercomment}
Berikut ini adalah demonstrasi sederhana dari fungsi tersebut.
Pengguna dapat menggambar di atas jendela plot, meninggalkan jejak
titik-titik.
\end{eulercomment}
\begin{eulerprompt}
>function dragtest ...
\end{eulerprompt}
\begin{eulerudf}
   plot2d(none,r=1,title="Drag with the mouse, or press any key!");
    start=0;
    repeat
      \{flag,m,time\}=mousedrag();
      if flag==0 then return; endif;
      if flag==2 then
        hold on; mark(m[1],m[2]); hold off;
      endif;
    end
  endfunction
\end{eulerudf}
\begin{eulerprompt}
>dragtest // lihat hasilnya dan cobalah lakukan!
\end{eulerprompt}
\eulersubheading{Gaya Plot 2D}
\begin{eulercomment}
Secara default, EMT menghitung tanda centang sumbu otomatis dan
menambahkan label ke setiap tanda centang. Ini dapat diubah dengan
parameter grid. Gaya default sumbu dan label dapat dimodifikasi.
Selain itu, label dan judul dapat ditambahkan secara manual. Untuk
mengatur ulang ke gaya default, gunakan reset().
\end{eulercomment}
\begin{eulerprompt}
>aspect(); 
>figure(3,4); ...
> figure(1); plot2d("x^3-x",grid=0); ... // no grid, frame or axis
> figure(2); plot2d("x^3-x",grid=1); ... // x-y-axis
> figure(3); plot2d("x^3-x",grid=2); ... // default ticks
> figure(4); plot2d("x^3-x",grid=3); ... // x-y- axis with labels inside
> figure(5); plot2d("x^3-x",grid=4); ... // no ticks, only labels
> figure(6); plot2d("x^3-x",grid=5); ... // default, but no margin
> figure(7); plot2d("x^3-x",grid=6); ... // axes only
> figure(8); plot2d("x^3-x",grid=7); ... // axes only, ticks at axis
> figure(9); plot2d("x^3-x",grid=8); ... // axes only, finer ticks at axis
> figure(10); plot2d("x^3-x",grid=9); ... // default, small ticks inside
> figure(11); plot2d("x^3-x",grid=10); ...// no ticks, axes only
> figure(0):
\end{eulerprompt}
\begin{eulercomment}
Parameter \textless{}frame menonaktifkan bingkai, dan framecolor=blue menyetel
bingkai ke warna biru.

Jika Anda menginginkan tanda centang Anda sendiri, Anda dapat
menggunakan style=0, dan menambahkan semuanya nanti.
\end{eulercomment}
\begin{eulerprompt}
>aspect(1.5);
>plot2d("x^3-x",grid=0); // plot
>frame; xgrid([-1,0,1]); ygrid(0): // add frame and grid
\end{eulerprompt}
\begin{eulercomment}
Untuk judul plot dan label sumbu, lihat contoh berikut.
\end{eulercomment}
\begin{eulerprompt}
>plot2d("exp(x)",-1,1);
>textcolor(black); // set the text color to black
>title(latex("y=e^x")); // title above the plot
>xlabel(latex("x")); // "x" for x-axis
>ylabel(latex("y"),>vertical); // vertical "y" for y-axis
>label(latex("(0,1)"),0,1,color=blue): // label a point
\end{eulerprompt}
\begin{eulercomment}
Sumbu dapat digambar secara terpisah dengan xaxis() dan yaxis().
\end{eulercomment}
\begin{eulerprompt}
>plot2d("x^3-x",<grid,<frame);
>xaxis(0,xx=-2:1,style="->"); yaxis(0,yy=-5:5,style="->"):
\end{eulerprompt}
\begin{eulercomment}
Teks pada plot dapat diatur dengan label(). Dalam contoh berikut, "lc"
berarti tengah bawah. Ini mengatur posisi label relatif terhadap
koordinat plot.
\end{eulercomment}
\begin{eulerprompt}
>function f(x) &= x^3-x
\end{eulerprompt}
\begin{euleroutput}
  
                                   3
                                  x  - x
  
\end{euleroutput}
\begin{eulerprompt}
>plot2d(f,-1,1,>square);
>x0=fmin(f,0,1); // compute point of minimum
>label("Rel. Min.",x0,f(x0),pos="lc"): // add a label there
\end{eulerprompt}
\begin{eulercomment}
Ada juga kotak teks.
\end{eulercomment}
\begin{eulerprompt}
>plot2d(&f(x),-1,1,-2,2); // function
>plot2d(&diff(f(x),x),>add,style="--",color=red); // derivative
>labelbox(["f","f'"],["-","--"],[black,red]): // label box
>plot2d(["exp(x)","1+x"],color=[black,blue],style=["-","-.-"]):
>gridstyle("->",color=gray,textcolor=gray,framecolor=gray);  ...
> plot2d("x^3-x",grid=1);   ...
> settitle("y=x^3-x",color=black); ...
> label("x",2,0,pos="bc",color=gray);  ...
> label("y",0,6,pos="cl",color=gray); ...
> reset():
\end{eulerprompt}
\begin{eulercomment}
Untuk kontrol yang lebih baik, sumbu x dan sumbu y dapat dilakukan
secara manual.

Perintah fullwindow() memperluas jendela plot karena kita tidak lagi
memerlukan tempat untuk label di luar jendela plot. Gunakan
shrinkwindow() atau reset() untuk mengatur ulang ke default.
\end{eulercomment}
\begin{eulerprompt}
>fullwindow; ...
>gridstyle(color=darkgray,textcolor=darkgray); ...
> plot2d(["2^x","1","2^(-x)"],a=-2,b=2,c=0,d=4,<grid,color=4:6,<frame); ...
> xaxis(0,-2:1,style="->"); xaxis(0,2,"x",<axis); ...
> yaxis(0,4,"y",style="->"); ...
> yaxis(-2,1:4,>left); ...
> yaxis(2,2^(-2:2),style=".",<left); ...
> labelbox(["2^x","1","2^-x"],colors=4:6,x=0.8,y=0.2); ...
> reset:
\end{eulerprompt}
\begin{eulercomment}
Berikut contoh lain, di mana string Unicode digunakan dan sumbu berada
di luar area plot.
\end{eulercomment}
\begin{eulerprompt}
>aspect(1.5); 
>plot2d(["sin(x)","cos(x)"],0,2pi,color=[red,green],<grid,<frame); ...
> xaxis(-1.1,(0:2)*pi,xt=["0",u"&pi;",u"2&pi;"],style="-",>ticks,>zero);  ...
> xgrid((0:0.5:2)*pi,<ticks); ...
> yaxis(-0.1*pi,-1:0.2:1,style="-",>zero,>grid); ...
> labelbox(["sin","cos"],colors=[red,green],x=0.5,y=0.2,>left); ...
> xlabel(u"&phi;"); ylabel(u"f(&phi;)"):
\end{eulerprompt}
\eulerheading{Plotting Data 2D}
\begin{eulercomment}
Jika x dan y adalah vektor data, data ini akan digunakan sebagai
koordinat x dan y dari sebuah kurva. Dalam kasus ini, a, b, c, dan d,
atau radius r dapat ditentukan, atau jendela plot akan menyesuaikan
secara otomatis dengan data. Atau, \textgreater{}square dapat diatur untuk
mempertahankan rasio aspek persegi.

Plotting ekspresi hanyalah singkatan untuk plot data. Untuk plot data,
Anda memerlukan satu atau beberapa baris nilai-x, dan satu atau
beberapa baris nilai-y. Dari rentang dan nilai-x, fungsi plot2d akan
menghitung data yang akan diplot, secara default dengan evaluasi
adaptif fungsi. Untuk plot titik gunakan "\textgreater{}points", untuk garis dan
titik campuran gunakan "\textgreater{}addpoints".

Namun, Anda dapat memasukkan data secara langsung.

- Gunakan vektor baris untuk x dan y untuk satu fungsi.\\
- Matriks untuk x dan y diplot baris demi baris.

Berikut adalah contoh dengan satu baris untuk x dan y.
\end{eulercomment}
\begin{eulerprompt}
>x=-10:0.1:10; y=exp(-x^2)*x; plot2d(x,y):
\end{eulerprompt}
\begin{eulerudf}
  
\end{eulerudf}
\begin{eulercomment}
Data juga dapat diplot sebagai titik. Gunakan points=true untuk ini.
Plot bekerja seperti poligon, tetapi hanya menggambar sudutnya.

- style="...": Pilih dari "[]", "\textless{}\textgreater{}", "o", ".", "..", "+", "*", "[]#",
"\textless{}\textgreater{}#", "o#", "..#", "#", "\textbar{}".

Untuk memplot kumpulan titik, gunakan \textgreater{}points. Jika warnanya adalah
vektor warna, setiap titik\\
mendapatkan warna yang berbeda. Untuk matriks koordinat dan vektor
kolom, warna berlaku untuk baris matriks.\\
Parameter \textgreater{}addpoints menambahkan titik ke segmen garis untuk plot
data.
\end{eulercomment}
\begin{eulerprompt}
>xdata=[1,1.5,2.5,3,4]; ydata=[3,3.1,2.8,2.9,2.7]; // data
>plot2d(xdata,ydata,a=0.5,b=4.5,c=2.5,d=3.5,style="."); // lines
>plot2d(xdata,ydata,>points,>add,style="o"): // add points
>p=polyfit(xdata,ydata,1); // get regression line
>plot2d("polyval(p,x)",>add,color=red): // add plot of line
\end{eulerprompt}
\begin{euleroutput}
  Variable or function p not found.
  Error in expression: polyval(p,x)
   %ploteval:
      y0=f$(x[1],args());
  adaptiveevalone:
      s=%ploteval(g$,t;args());
  Try "trace errors" to inspect local variables after errors.
  plot2d:
      dw/n,dw/n^2,dw/n,auto;args());
\end{euleroutput}
\eulerheading{Menggambar Daerah Yang Berbatasan Kurva}
\begin{eulercomment}
Plot data sebenarnya adalah poligon. Kita juga dapat memplot kurva
atau kurva terisi.

- terisi=benar mengisi plot.\\
- style="...": Pilih dari "#", "/", "\textbackslash{}", "\textbackslash{}/".\\
- fillcolor: Lihat di atas untuk warna yang tersedia.

Warna isian ditentukan oleh argumen "fillcolor", dan pada \textless{}outline
opsional mencegah penggambaran batas untuk semua gaya kecuali yang
default.
\end{eulercomment}
\begin{eulerprompt}
>t=linspace(0,2pi,1000); // parameter for curve
>x=sin(t)*exp(t/pi); y=cos(t)*exp(t/pi); // x(t) and y(t)
>figure(1,2); aspect(16/9)
>figure(1); plot2d(x,y,r=10); // plot curve
>figure(2); plot2d(x,y,r=10,>filled,style="/",fillcolor=red); // fill curve
>figure(0):
\end{eulerprompt}
\begin{eulercomment}
Pada contoh berikut ini kami memplot elips yang terisi dan dua segi
enam yang terisi menggunakan kurva tertutup dengan 6 titik dengan gaya
isian yang berbeda.
\end{eulercomment}
\begin{eulerprompt}
>dx=linspace(0,2pi,1000); plot2d(sin(x),cos(x)*0.5,r=1,>filled,style="/"): insimg;
\end{eulerprompt}
\begin{euleroutput}
  The : allows only real arguments!
  Error in:
  ... (sin(x),cos(x)*0.5,r=1,>filled,style="/"): insimg; ...
                                                       ^
\end{euleroutput}
\begin{eulerprompt}
>t=linspace(0,2pi,6); ...
>plot2d(cos(t),sin(t),>filled,style="/",fillcolor=red,r=1.2):
>t=linspace(0,2pi,6); plot2d(cos(t),sin(t),>filled,style="#"):
\end{eulerprompt}
\begin{eulercomment}
Contoh lain adalah septagon, yang kita buat dengan 7 titik pada
lingkaran satuan.
\end{eulercomment}
\begin{eulerprompt}
>t=linspace(0,2pi,7);  ...
>plot2d(cos(t),sin(t),r=1,>filled,style="/",fillcolor=red):
\end{eulerprompt}
\begin{eulercomment}
Berikut ini adalah himpunan nilai maksimal dari empat kondisi linier
yang kurang dari atau sama dengan 3. Ini adalah A[k].v\textless{}=3 untuk semua
baris A. Untuk mendapatkan sudut yang bagus, kita menggunakan n yang
relatif besar.
\end{eulercomment}
\begin{eulerprompt}
>A=[2,1;1,2;-1,0;0,-1];
>function f(x,y) := max([x,y].A');
>plot2d("f",r=4,level=[0;3],color=green,n=111):
\end{eulerprompt}
\begin{eulercomment}
Poin utama dari bahasa matriks adalah memungkinkan pembuatan tabel
fungsi dengan mudah.
\end{eulercomment}
\begin{eulerprompt}
>t=linspace(0,2pi,1000); x=cos(3*t); y=sin(4*t);
\end{eulerprompt}
\begin{eulercomment}
Sekarang kita memiliki vektor x dan y dari nilai-nilai. plot2d() dapat
memplot nilai-nilai ini sebagai kurva yang menghubungkan titik-titik.
Plot dapat diisi. Dalam hal ini menghasilkan hasil yang bagus karena
aturan lilitan, yang digunakan untuk pengisian.
\end{eulercomment}
\begin{eulerprompt}
>plot2d(x,y,<grid,<frame,>filled):
\end{eulerprompt}
\begin{eulercomment}
Vektor interval diplot terhadap nilai x sebagai daerah terisi antara
nilai interval yang lebih rendah dan lebih tinggi.

Hal ini dapat berguna untuk memplot kesalahan perhitungan. Namun, hal
ini juga dapat digunakan untuk memplot kesalahan statistik.
\end{eulercomment}
\begin{eulerprompt}
>t=0:0.1:1; ...
> plot2d(t,interval(t-random(size(t)),t+random(size(t))),style="|");  ...
> plot2d(t,t,add=true):
\end{eulerprompt}
\begin{eulercomment}
Jika x adalah vektor yang diurutkan, dan y adalah vektor interval,
maka plot2d akan memplot rentang interval yang terisi pada bidang.
Gaya isiannya sama dengan gaya poligon.
\end{eulercomment}
\begin{eulerprompt}
>t=-1:0.01:1; x=~t-0.01,t+0.01~; y=x^3-x;
>plot2d(t,y):
\end{eulerprompt}
\begin{eulercomment}
Dimungkinkan untuk mengisi wilayah nilai untuk fungsi tertentu. Untuk
ini, level harus berupa matriks 2xn. Baris pertama adalah batas bawah
dan baris kedua berisi batas atas.
\end{eulercomment}
\begin{eulerprompt}
>expr := "2*x^2+x*y+3*y^4+y"; // define an expression f(x,y)
>plot2d(expr,level=[0;1],style="-",color=blue): // 0 <= f(x,y) <= 1
\end{eulerprompt}
\begin{eulercomment}
Kita juga dapat mengisi rentang nilai seperti

\end{eulercomment}
\begin{eulerformula}
\[
-1 \le (x^2+y^2)^2-x^2+y^2 \le 0.
\]
\end{eulerformula}
\begin{eulercomment}
\end{eulercomment}
\begin{eulerprompt}
>plot2d("(x^2+y^2)^2-x^2+y^2",r=1.2,level=[-1;0],style="/"):
>plot2d("cos(x)","sin(x)^3",xmin=0,xmax=2pi,>filled,style="/"):
\end{eulerprompt}
\eulerheading{Grafik Fungsi Parametrik}
\begin{eulercomment}
Nilai-nilai x tidak perlu diurutkan. (x,y) hanya menggambarkan kurva.
Jika x diurutkan, kurva tersebut adalah grafik fungsi.

Dalam contoh berikut, kita memplot spiral

\end{eulercomment}
\begin{eulerformula}
\[
\gamma(t) = t \cdot (\cos(2\pi t),\sin(2\pi t))
\]
\end{eulerformula}
\begin{eulercomment}
Kita perlu menggunakan banyak titik untuk tampilan yang halus atau
fungsi adaptive() untuk mengevaluasi ekspresi (lihat fungsi adaptive()
untuk detail lebih lanjut).
\end{eulercomment}
\begin{eulerprompt}
>t=linspace(0,1,1000); ...
>plot2d(t*cos(2*pi*t),t*sin(2*pi*t),r=1):
\end{eulerprompt}
\begin{eulercomment}
Atau, ada kemungkinan untuk menggunakan dua ekspresi untuk kurva.
Berikut ini memplot kurva yang sama seperti di atas.
\end{eulercomment}
\begin{eulerprompt}
>plot2d("x*cos(2*pi*x)","x*sin(2*pi*x)",xmin=0,xmax=1,r=1):
>t=linspace(0,1,1000); r=exp(-t); x=r*cos(2pi*t); y=r*sin(2pi*t);
>plot2d(x,y,r=1):
\end{eulerprompt}
\begin{eulercomment}
Pada contoh berikut, kita plot kurva

\end{eulercomment}
\begin{eulerformula}
\[
\gamma(t) = (r(t) \cos(t), r(t) \sin(t))
\]
\end{eulerformula}
\begin{eulercomment}
dengan

\end{eulercomment}
\begin{eulerformula}
\[
r(t) = 1 + \dfrac{\sin(3t)}{2}.
\]
\end{eulerformula}
\begin{eulerprompt}
>t=linspace(0,2pi,1000); r=1+sin(3*t)/2; x=r*cos(t); y=r*sin(t); ...
>plot2d(x,y,>filled,fillcolor=red,style="/",r=1.5):
\end{eulerprompt}
\eulerheading{Menggambar Grafik Bilangan Kompleks}
\begin{eulercomment}
Susunan bilangan kompleks juga dapat diplot. Kemudian titik-titik grid
akan dihubungkan. Jika sejumlah garis grid ditentukan (atau vektor
garis grid 1x2) dalam argumen cgrid, hanya garis grid tersebut yang
terlihat.

Matriks bilangan kompleks akan secara otomatis diplot sebagai grid
dalam bidang kompleks.

Dalam contoh berikut, kami memplot gambar lingkaran satuan di bawah
fungsi eksponensial. Parameter cgrid menyembunyikan beberapa kurva
grid.
\end{eulercomment}
\begin{eulerprompt}
>aspect(); r=linspace(0,1,50); a=linspace(0,2pi,80)'; z=r*exp(I*a);...
>plot2d(z,a=-1.25,b=1.25,c=-1.25,d=1.25,cgrid=10):
>aspect(1.25); r=linspace(0,1,50); a=linspace(0,2pi,200)'; z=r*exp(I*a);
>plot2d(exp(z),cgrid=[40,10]):
>r=linspace(0,1,10); a=linspace(0,2pi,40)'; z=r*exp(I*a);
>plot2d(exp(z),>points,>add):
\end{eulerprompt}
\begin{eulercomment}
Vektor bilangan kompleks secara otomatis diplot sebagai kurva pada
bidang kompleks dengan bagian riil dan bagian imajiner.

Dalam contoh, kita memplot lingkaran satuan dengan

\end{eulercomment}
\begin{eulerformula}
\[
\gamma(t) = e^{it}
\]
\end{eulerformula}
\begin{eulerprompt}
>t=linspace(0,2pi,1000); ...
>plot2d(exp(I*t)+exp(4*I*t),r=2):
\end{eulerprompt}
\eulerheading{Plot Statistik}
\begin{eulercomment}
Ada banyak fungsi yang dikhususkan pada plot statistik. Salah satu
plot yang sering digunakan adalah plot kolom.

Penjumlahan kumulatif dari nilai-nilai berdistribusi normal 0-1
menghasilkan pergerakan acak.
\end{eulercomment}
\begin{eulerprompt}
>plot2d(cumsum(randnormal(1,1000))):
\end{eulerprompt}
\begin{eulercomment}
Menggunakan dua baris menunjukkan jalan dalam dua dimensi.
\end{eulercomment}
\begin{eulerprompt}
>X=cumsum(randnormal(2,1000)); plot2d(X[1],X[2]):
>columnsplot(cumsum(random(10)),style="/",color=blue):
\end{eulerprompt}
\begin{eulercomment}
Ia juga dapat menampilkan string sebagai label.
\end{eulercomment}
\begin{eulerprompt}
>months=["Jan","Feb","Mar","Apr","May","Jun", ...
>"Jul","Aug","Sep","Oct","Nov","Dec"];
>values=[10,12,12,18,22,28,30,26,22,18,12,8];
>columnsplot(values,lab=months,color=red,style="-");
>title("Temperature"):
>k=0:10;
>plot2d(k,bin(10,k),>bar):
>plot2d(k,bin(10,k)); plot2d(k,bin(10,k),>points,>add):
>plot2d(normal(1000),normal(1000),>points,grid=6,style=".."):
>plot2d(normal(1,1000),>distribution,style="O"):
>plot2d("qnormal",0,5;2.5,0.5,>filled):
\end{eulerprompt}
\begin{eulercomment}
Untuk memplot distribusi statistik eksperimental, Anda dapat
menggunakan distribution=n dengan plot2d.
\end{eulercomment}
\begin{eulerprompt}
>w=randexponential(1,1000); // exponential distribution
>plot2d(w,>distribution): // or distribution=n with n intervals
\end{eulerprompt}
\begin{eulercomment}
Atau Anda dapat menghitung distribusi dari data dan memplot hasilnya
dengan \textgreater{}bar di plot3d, atau dengan plot kolom.
\end{eulercomment}
\begin{eulerprompt}
>w=normal(1000); // 0-1-normal distribution
>\{x,y\}=histo(w,10,v=[-6,-4,-2,-1,0,1,2,4,6]); // interval bounds v
>plot2d(x,y,>bar):
\end{eulerprompt}
\begin{eulercomment}
Fungsi statplot() mengatur gaya dengan string sederhana.
\end{eulercomment}
\begin{eulerprompt}
>statplot(1:10,cumsum(random(10)),"b"):
>n=10; i=0:n; ...
>plot2d(i,bin(n,i)/2^n,a=0,b=10,c=0,d=0.3); ...
>plot2d(i,bin(n,i)/2^n,points=true,style="ow",add=true,color=blue):
\end{eulerprompt}
\begin{eulercomment}
Selain itu, data dapat diplot sebagai batang. Dalam kasus ini, x harus
diurutkan dan satu elemen lebih panjang dari y. Batang akan memanjang
dari x[i] ke x[i+1] dengan nilai y[i]. Jika x memiliki ukuran yang
sama dengan y, maka akan memanjang satu elemen dengan spasi terakhir.

Gaya isian dapat digunakan seperti di atas.
\end{eulercomment}
\begin{eulerprompt}
>n=10; k=bin(n,0:n); ...
>plot2d(-0.5:n+0.5,k,bar=true,fillcolor=lightgray):
\end{eulerprompt}
\begin{eulercomment}
Data untuk diagram batang (batang=1) dan histogram (histogram=1) dapat
diberikan secara eksplisit dalam xv dan yv, atau dapat dihitung dari
distribusi empiris dalam xv dengan \textgreater{}distribusi (atau distribusi=n).
Histogram nilai xv akan dihitung secara otomatis dengan \textgreater{}histogram.
Jika \textgreater{}even ditentukan, nilai xv akan dihitung dalam interval integer.
\end{eulercomment}
\begin{eulerprompt}
>plot2d(normal(10000),distribution=50):
>k=0:10; m=bin(10,k); x=(0:11)-0.5; plot2d(x,m,>bar):
>columnsplot(m,k):
>plot2d(random(600)*6,histogram=6):
\end{eulerprompt}
\begin{eulercomment}
Untuk distribusi, ada parameter distribution=n, yang menghitung nilai
secara otomatis dan mencetak distribusi relatif dengan n sub-interval.
\end{eulercomment}
\begin{eulerprompt}
>plot2d(normal(1,1000),distribution=10,style="\(\backslash\)/"):
\end{eulerprompt}
\begin{eulercomment}
Dengan parameter even=true, ini akan menggunakan interval integer.
\end{eulercomment}
\begin{eulerprompt}
>plot2d(intrandom(1,1000,10),distribution=10,even=true):
\end{eulerprompt}
\begin{eulercomment}
Perhatikan bahwa ada banyak plot statistik yang mungkin berguna.
Lihatlah tutorial tentang statistik.
\end{eulercomment}
\begin{eulerprompt}
>columnsplot(getmultiplicities(1:6,intrandom(1,6000,6))):
>plot2d(normal(1,1000),>distribution); ...
>  plot2d("qnormal(x)",color=red,thickness=2,>add):
\end{eulerprompt}
\begin{eulercomment}
Ada juga banyak plot khusus untuk statistik. Boxplot menunjukkan
kuartil dari distribusi ini dan banyak outlier. Menurut definisi,
outlier dalam boxplot adalah data yang melebihi 1,5 kali rentang
tengah 50\% dari plot.
\end{eulercomment}
\begin{eulerprompt}
>M=normal(5,1000); boxplot(quartiles(M)):
\end{eulerprompt}
\eulerheading{Fungsi Implisit}
\begin{eulercomment}
Plot implisit menunjukkan garis level yang menyelesaikan f(x,y)=level,
di mana "level" dapat berupa nilai tunggal atau vektor nilai. Jika
level="auto", akan ada garis level nc, yang akan menyebar antara
minimum dan maksimum fungsi secara merata. Warna yang lebih gelap atau
lebih terang dapat ditambahkan dengan \textgreater{}hue untuk menunjukkan nilai
fungsi. Untuk fungsi implisit, xv harus berupa fungsi atau ekspresi
parameter x dan y, atau, sebagai alternatif, xv dapat berupa matriks
nilai.

Euler dapat menandai garis level

\end{eulercomment}
\begin{eulerformula}
\[
f(x,y) = c
\]
\end{eulerformula}
\begin{eulercomment}
dari fungsi apa pun.

Untuk menggambar himpunan f(x,y)=c untuk satu atau lebih konstanta c,
Anda dapat menggunakan plot2d() dengan plot implisitnya di bidang.
Parameter untuk c adalah level=c, di mana c dapat berupa vektor garis
level. Selain itu, skema warna dapat digambar di latar belakang untuk
menunjukkan nilai fungsi untuk setiap titik dalam plot. Parameter "n"
menentukan kehalusan plot.
\end{eulercomment}
\begin{eulerprompt}
>aspect(1.5); 
>plot2d("x^2+y^2-x*y-x",r=1.5,level=0,contourcolor=red):
>expr := "2*x^2+x*y+3*y^4+y"; // define an expression f(x,y)
>plot2d(expr,level=0): // Solutions of f(x,y)=0
>plot2d(expr,level=0:0.5:20,>hue,contourcolor=white,n=200): // nice
>plot2d(expr,level=0:0.5:20,>hue,>spectral,n=200,grid=4): // nicer
\end{eulerprompt}
\begin{eulercomment}
Ini juga berlaku untuk plot data. Namun, Anda harus menentukan rentang\\
untuk label sumbu.
\end{eulercomment}
\begin{eulerprompt}
>x=-2:0.05:1; y=x'; z=expr(x,y);
>plot2d(z,level=0,a=-1,b=2,c=-2,d=1,>hue):
>plot2d("x^3-y^2",>contour,>hue,>spectral):
>plot2d("x^3-y^2",level=0,contourwidth=3,>add,contourcolor=red):
>z=z+normal(size(z))*0.2;
>plot2d(z,level=0.5,a=-1,b=2,c=-2,d=1):
>plot2d(expr,level=[0:0.2:5;0.05:0.2:5.05],color=lightgray):
>plot2d("x^2+y^3+x*y",level=1,r=4,n=100):
>plot2d("x^2+2*y^2-x*y",level=0:0.1:10,n=100,contourcolor=white,>hue):
\end{eulerprompt}
\begin{eulercomment}
Dimungkinkan juga untuk mengisi himpunan

\end{eulercomment}
\begin{eulerformula}
\[
a \le f(x,y) \le b
\]
\end{eulerformula}
\begin{eulercomment}
dengan rentang level.

Dimungkinkan untuk mengisi wilayah nilai untuk fungsi tertentu. Untuk
ini, level harus berupa matriks 2xn. Baris pertama adalah batas bawah
dan baris kedua berisi batas atas.
\end{eulercomment}
\begin{eulerprompt}
>plot2d(expr,level=[0;1],style="-",color=blue): // 0 <= f(x,y) <= 1
\end{eulerprompt}
\begin{eulercomment}
Plot implisit juga dapat menunjukkan rentang level. Level harus berupa
matriks 2xn interval level, di mana baris pertama berisi awal dan
baris kedua berisi akhir setiap interval. Atau, vektor baris sederhana
dapat digunakan untuk level, dan parameter dl memperluas nilai level
ke interval.
\end{eulercomment}
\begin{eulerprompt}
>plot2d("x^4+y^4",r=1.5,level=[0;1],color=blue,style="/"):
>plot2d("x^2+y^3+x*y",level=[0,2,4;1,3,5],style="/",r=2,n=100):
>plot2d("x^2+y^3+x*y",level=-10:20,r=2,style="-",dl=0.1,n=100):
>plot2d("sin(x)*cos(y)",r=pi,>hue,>levels,n=100):
\end{eulerprompt}
\begin{eulercomment}
Dimungkinkan juga untuk menandai suatu wilayah

\end{eulercomment}
\begin{eulerformula}
\[
a \le f(x,y) \le b.
\]
\end{eulerformula}
\begin{eulercomment}
Hal ini dilakukan dengan menambahkan level dengan dua baris.
\end{eulercomment}
\begin{eulerprompt}
>plot2d("(x^2+y^2-1)^3-x^2*y^3",r=1.3, ...
>  style="#",color=red,<outline, ...
>  level=[-2;0],n=100):
\end{eulerprompt}
\begin{eulercomment}
Dimungkinkan untuk menentukan level tertentu. Misalnya, kita dapat
memplot solusi persamaan seperti

\end{eulercomment}
\begin{eulerformula}
\[
x^3-xy+x^2y^2=6
\]
\end{eulerformula}
\begin{eulerprompt}
>plot2d("x^3-x*y+x^2*y^2",r=6,level=1,n=100):
>function starplot1 (v, style="/", color=green, lab=none) ...
\end{eulerprompt}
\begin{eulerudf}
  if !holding() then clg; endif;
    w=window(); window(0,0,1024,1024);
    h=holding(1);
    r=max(abs(v))*1.2;
    setplot(-r,r,-r,r);
    n=cols(v); t=linspace(0,2pi,n);
    v=v|v[1]; c=v*cos(t); s=v*sin(t);
    cl=barcolor(color); st=barstyle(style);
    loop 1 to n
      polygon([0,c[#],c[#+1]],[0,s[#],s[#+1]],1);
   if lab!=none then
        rlab=v[#]+r*0.1;
        \{col,row\}=toscreen(cos(t[#])*rlab,sin(t[#])*rlab);
        ctext(""+lab[#],col,row-textheight()/2);
      endif;
    end;
    barcolor(cl); barstyle(st);
    holding(h);
    window(w);
  endfunction
\end{eulerudf}
\begin{eulercomment}
Tidak ada tanda centang pada grid atau sumbu di sini. Selain itu, kami
menggunakan jendela penuh untuk plot.

Kami memanggil reset sebelum menguji plot ini untuk mengembalikan
grafik ke default. Ini tidak perlu, jika Anda yakin bahwa plot Anda
berfungsi.
\end{eulercomment}
\begin{eulerprompt}
>reset; starplot1(normal(1,10)+5,color=red,lab=1:10):
\end{eulerprompt}
\begin{eulercomment}
Terkadang, Anda mungkin ingin memplot sesuatu yang tidak dapat
dilakukan oleh plot2d, tetapi hampir.

Dalam fungsi berikut, kita melakukan plot impuls logaritmik. plot2d
dapat melakukan plot logaritmik, tetapi tidak untuk batang impuls.
\end{eulercomment}
\begin{eulerprompt}
>function logimpulseplot1 (x,y) ...
\end{eulerprompt}
\begin{eulerudf}
  \{x0,y0\}=makeimpulse(x,log(y)/log(10));
    plot2d(x0,y0,>bar,grid=0);
    h=holding(1);
    frame();
    xgrid(ticks(x));
    p=plot();
    for i=-10 to 10;
      if i<=p[4] and i>=p[3] then
         ygrid(i,yt="10^"+i);
      endif;
    end;
    holding(h);
  endfunction
\end{eulerudf}
\begin{eulercomment}
Mari kita mengujinya dengan nilai-nilai yang terdistribusi secara
eksponensial.
\end{eulercomment}
\begin{eulerprompt}
>aspect(1.5); x=1:10; y=-log(random(size(x)))*200; ...
>logimpulseplot1(x,y):
\end{eulerprompt}
\begin{eulercomment}
Mari kita animasikan kurva 2D menggunakan plot langsung. Perintah
plot(x,y) cukup memplot kurva ke dalam jendela plot. setplot(a,b,c,d)
mengatur jendela ini.

Fungsi wait(0) memaksa plot untuk muncul di jendela grafik. Jika
tidak, penggambaran ulang akan dilakukan dalam interval waktu yang
jarang.
\end{eulercomment}
\begin{eulerprompt}
>function animliss (n,m) ...
\end{eulerprompt}
\begin{eulerudf}
  t=linspace(0,2pi,500);
  f=0;
  c=framecolor(0);
  l=linewidth(2);
  setplot(-1,1,-1,1);
  repeat
    clg;
    plot(sin(n*t),cos(m*t+f));
    wait(0);
    if testkey() then break; endif;
    f=f+0.02;
  end;
  framecolor(c);
  linewidth(l);
  endfunction
\end{eulerudf}
\begin{eulercomment}
Tekan tombol apa saja untuk menghentikan animasi ini.
\end{eulercomment}
\begin{eulerprompt}
>animliss(2,3); // lihat hasilnya, jika sudah puas, tekan ENTER
\end{eulerprompt}
\eulerheading{Plot Logaritmik}
\begin{eulercomment}
EMT menggunakan parameter "logplot" untuk skala logaritmik.\\
Plot logaritmik dapat diplot menggunakan skala logaritmik dalam y
dengan logplot=1, atau menggunakan skala logaritmik dalam x dan y
dengan logplot=2, atau dalam x dengan logplot=3.

- logplot=1: y-logaritmik\\
- logplot=2: x-y-logaritmik\\
- logplot=3: x-logaritmik
\end{eulercomment}
\begin{eulerprompt}
>plot2d("exp(x^3-x)*x^2",1,5,logplot=1):
>plot2d("exp(x+sin(x))",0,100,logplot=1):
>plot2d("exp(x+sin(x))",10,100,logplot=2):
>plot2d("gamma(x)",1,10,logplot=1):
>plot2d("log(x*(2+sin(x/100)))",10,1000,logplot=3):
\end{eulerprompt}
\begin{eulercomment}
Hal ini juga berlaku untuk plot data.
\end{eulercomment}
\begin{eulerprompt}
>x=10^(1:20); y=x^2-x;
>plot2d(x,y,logplot=2):
\end{eulerprompt}
\begin{euleroutput}
  x = -10:0.01:10;
  a = (1:5)';
  plot2d(x, a*sin(x), color=getspectral(a/5)):
\end{euleroutput}
\eulerheading{Rujukan Lengkap Fungsi plot2d()}
\begin{eulercomment}
fungsi plot2d (xv, yv, btest, a, b, c, d, xmin, xmax, r, n,logplot,
grid, bingkai, framecolor, kotak, warna, ketebalan, gaya,tomatis,
tambah, pengguna, delta, titik, addpoints, pointstyle, batang,
histogram,istribusi, merata, langkah, sendiri, adaptif, rona, level,
kontur,nc, terisi, warna isian, garis besar, judul, xl, yl, peta,
warna kontur,lebar kontur, tanda centang, margin, kliping, cx, cy,
insimg, spektral,cgrid, vertikal, lebih kecil, dl, niveau, level)

Fungsi plot multiguna untuk plot pada bidang (plot 2D). Fungsi ini
dapat melakukan plot fungsi satu variabel, plot data, kurva pada
bidang, plot batang, kisi bilangan kompleks, dan plot implisit fungsi
dua variabel.

Parameter

x,y       : persamaan, fungsi atau vektor data\\
a,b,c,d   : Luas plot (default a=-2,b=2)\\
r         : jika r disetel, maka a=cx-r, b=cx+r, c=cy-r, d=cy+r\\
r dapat berupa vektor [rx,ry] atau vektor [rx1,rx2,ry1,ry2].\\
xmin,xmax : rentang parameter untuk kurva\\
auto      : Menentukan rentang y secara otomatis (default)\\
aquare    : jika benar, cobalah untuk mempertahankan rentang x-y
persegi\\
n         : jumlah interval (standarnya adaptif)\\
grid      : 0 = tidak ada kisi dan label,\\
\end{eulercomment}
\begin{eulerttcomment}
            1 = sumbu saja,
            2 = grid normal (lihat di bawah untuk jumlah garis grid)
            3 = sumbu dalam
            4 = tidak ada grid
            5 = grid penuh termasuk margin
            6 = centang pada bingkai
            7 = sumbu saja
            8 = sumbu saja, sub-centang
\end{eulerttcomment}
\begin{eulercomment}
frame     : 0 = tanpa bingkai\\
framecolor: warna bingkai dan grid\\
margin    : angka antara 0 dan 0,4 untuk margin di sekitar plot\\
color     : Warna kurva. Jika ini adalah vektor warna,\\
\end{eulercomment}
\begin{eulerttcomment}
            itu akan digunakan untuk setiap baris matriks plot. Dalam
            kasus plot titik, itu harus berupa vektor kolom. Jika
            suatu vektor baris atau a matriks warna penuh
            digunakan untuk plot titik, itu akan digunakan untuk
            setiap titik data.
\end{eulerttcomment}
\begin{eulercomment}
thickness : ketebalan garis untuk kurva\\
\end{eulercomment}
\begin{eulerttcomment}
            Nilai ini bisa lebih kecil dari 1 untuk garis yang sangat
            tipis.
\end{eulerttcomment}
\begin{eulercomment}
style     : Gaya plot untuk garis, penanda, dan isian.\\
\end{eulercomment}
\begin{eulerttcomment}
            Untuk penggunaan poin
            "[]", "<>", ".", "..", "...",
            "*", "+", "|", "-", "o"
            "[]#", "<>#", "o#" (bentuk terisi)
            "[]w", "<>w", "ow" (tidak transparan)
            Untuk penggunaan garis
            "-", "--", "-.", ".", ".-.", "-.-", "->"
            Untuk penggunaan poligon atau plot batang yang terisi
            "#", "#O", "O", "/", "\(\backslash\)", "\(\backslash\)/",
            "+", "|", "-", "t"
\end{eulerttcomment}
\begin{eulercomment}
points    : plot titik tunggal, bukan segmen garis\\
addpoints : jika benar, buat plot segmen garis dan titik\\
add       : menambahkan plot ke plot yang ada\\
user      : memungkinkan interaksi pengguna untuk fungsi\\
delta     : ukuran langkah untuk interaksi pengguna\\
bar       : plot batang (x adalah batas interval, y adalah nilai
interval)\\
histogram : memplot frekuensi x dalam n subinterval\\
distribution=n : memplot distribusi x dengan n subinterval\\
even      : gunakan nilai antar untuk histogram otomatis.\\
steps     : memplot fungsi sebagai fungsi langkah (langkah=1,2)\\
adaptive  : gunakan plot adaptif (n adalah jumlah langkah minimal)\\
level     : garis tingkat plot dari fungsi implisit dua variabel\\
outline   : menarik batas rentang level

Jika nilai level adalah matriks 2xn, rentang level akan digambarkan
dalam warna menggunakan gaya isian yang diberikan. Jika garis besarnya
benar, maka itu benar akan digambar dalam warna kontur. Dengan
menggunakan fitur ini, wilayah f(x,y) antar batas dapat ditandai.

hue       : tambahkan warna rona ke plot level untuk menunjukkan nilai\\
\end{eulercomment}
\begin{eulerttcomment}
            fungsi
\end{eulerttcomment}
\begin{eulercomment}
contour   : Gunakan plot level dengan level otomatis\\
nc        : jumlah garis level otomatis\\
title     : judul plot (default "")\\
xl, yl    : label untuk sumbu x dan y\\
smaller   : jika \textgreater{}0, akan ada lebih banyak ruang di sebelah kiri untuk
label.\\
vertical  : Mengaktifkan atau menonaktifkan label vertikal. Ini
mengubah label vertikal variabel global secara lokal untuk satu plot.
Nilai 1 hanya menetapkan teks vertikal, nilai 2 menggunakan label
numerik vertikal pada sumbu y.

filled    : mengisi plot kurva\\
fillcolor : warna isian untuk batang dan kurva yang terisi\\
outline   : batas poligon terisi\\
logplot   : mengatur plot logaritmik\\
\end{eulercomment}
\begin{eulerttcomment}
            1 = logplot in y,
            2 = logplot in xy,
            3 = logplot in x
\end{eulerttcomment}
\begin{eulercomment}
own       :\\
\end{eulercomment}
\begin{eulerttcomment}
  Sebuah string, yang menunjuk ke rutinitas plotnya sendiri.   Dengan>
\end{eulerttcomment}
\begin{eulercomment}
pengguna, Anda mendapatkan\\
\end{eulercomment}
\begin{eulerttcomment}
  interaksi pengguna yang sama seperti di plot2d. Kisarannya akan
\end{eulerttcomment}
\begin{eulercomment}
ditentukan sebelum setiap panggilan ke fungsi Anda.\\
maps      : map expressions (0 is faster), functions are always
mapped.\\
contourcolor : color of contour lines\\
contourwidth : width of contour lines\\
clipping  : toggles the clipping (default is true)\\
title     :\\
\end{eulercomment}
\begin{eulerttcomment}
 Ini dapat digunakan untuk mendeskripsikan plot. Judul akan  muncul di
\end{eulerttcomment}
\begin{eulercomment}
atas plot. Selain itu, label untuk sumbu x dan y dapat  ditambahkan
dengan xl="string" atau yl="string". Label lain   dapat ditambahkan
dengan fungsi label() atau labelbox().  Judulnya bisa berupa unicode
string atau gambar rumus Lateks.\\
cgrid     :\\
\end{eulercomment}
\begin{eulerttcomment}
 Menentukan jumlah garis grid untuk plot grid yang kompleks.
 Harus berupa pembagi ukuran matriks dikurangi 1 (jumlah subinterval).
\end{eulerttcomment}
\begin{eulercomment}
cgrid dapat berupa vektor [cx,cy].

Ringkasan

Fungsinya dapat diplot

- ekspresi, kumpulan panggilan atau fungsi dari satu variabel,\\
- kurva parametrik,\\
- x data versus y data,\\
- fungsi implisit,\\
- plot batang,\\
- jaringan kompleks,\\
- poligon.

Jika fungsi atau ekspresi untuk xv diberikan, plot2d() akan menghitung
nilai dalam rentang tertentu menggunakan fungsi atau ekspresi
tersebut. Ekspresinya harus berupa ekspresi dalam variabel x. Rentang
harus ditentukan dalam parameter a dan b kecuali rentang default
[-2,2] harus digunakan. Rentang y akan dihitung secara otomatis,
kecuali c dan d ditentukan, atau radius r, yang menghasilkan rentang
[-r,r] untuk x dan y. Untuk plot fungsi, plot2d akan menggunakan
evaluasi adaptif fungsi secara default. Untuk mempercepat plot fungsi
yang rumit, nonaktifkan ini dengan \textless{}adaptive, dan secara opsional
mengurangi jumlah interval n. Selain itu, plot2d() akan secara default
menggunakan pemetaan. Yaitu, ini akan menghitung elemen plot untuk
elemen. Jika ekspresi atau fungsi Anda dapat menangani a vektor x,
Anda dapat mematikannya dengan \textless{}maps untuk evaluasi lebih cepat.

Perhatikan bahwa plot adaptif selalu dihitung elemen demi elemen.\\
Jika fungsi atau ekspresi untuk xv dan yv ditentukan,plot2d() akan
menghitung kurva dengan nilai xv sebagai koordinat x dan nilai yv
sebagai koordinat y. Dalam hal ini, seharusnya ada kisaran
didefinisikan untuk parameter menggunakan xmin, xmax. Ekspresi
terkandung dalam string harus selalu berupa ekspresi dalam variabel
parameter x.


\begin{eulercomment}
\eulerheading{Latihan Soal}
\begin{eulercomment}
1. Gunakan Euler Math Toolbox untuk memplot grafik fungsi\\
\end{eulercomment}
\begin{eulerformula}
\[
f(x)=a*sin
\]
\end{eulerformula}
\begin{eulercomment}
dengan nilai parameter aa yang bervariasi dari 1 hingga 5. Buat grafik
dalam rentang x=-10 hingga x=10, dengan setiap grafik diberi warna
yang berbeda berdasarkan spektrum warna.
\end{eulercomment}
\begin{eulerprompt}
>x = -10:0.01:10;
>a = (1:5)';
>plot2d(x, a*sin(x), color=getspectral(a/5)):
\end{eulerprompt}
\begin{euleroutput}
  
\end{euleroutput}
\eulerheading{Tugas Individu Visualisasi 3D}
\begin{eulercomment}
Nama : Alifia Rahmawati\\
NIM  : 23030630044\\
Kelas: Matematika E 2023


\begin{eulercomment}
\eulerheading{Menggambar Plot 3D dengan EMT}
\begin{eulercomment}
Ini adalah pengantar plot 3D di Euler. Kita memerlukan plot 3D untuk
memvisualisasikan fungsi dua variabel.

Euler menggambar fungsi tersebut menggunakan algoritma pengurutan
untuk menyembunyikan bagian di latar belakang. Secara umum, Euler
menggunakan proyeksi pusat. Defaultnya adalah dari kuadran x-y positif
ke arah titik asal x=y=z=0, tetapi sudut=0° terlihat dari arah sumbu
y. Sudut pandang dan tinggi dapat diubah.

Euler dapat memplot

- permukaan dengan bayangan dan garis datar atau rentang datar,\\
- awan titik,\\
- kurva parametrik,\\
- permukaan implisit.

Plot 3D suatu fungsi menggunakan plot3d. Cara termudah adalah memplot
ekspresi dalam x dan y. Parameter r mengatur rentang plot di sekitar
(0,0).
\end{eulercomment}
\begin{eulerprompt}
>aspect(1.5); plot3d("x^2+sin(y)",-5,5,0,6*pi):
>plot3d("x^2+x*sin(y)",-5,5,0,6*pi):
\end{eulerprompt}
\begin{eulercomment}
Silakan lakukan modifikasi agar gambar "talang bergelombang" tersebut
tidak lurus melainkan melengkung/melingkar, baik melingkar secara
mendatar maupun melingkar turun/naik (seperti papan peluncur pada
kolam renang. Temukan rumusnya.
\end{eulercomment}
\eulerheading{Fungsi Dua Variabel}
\begin{eulercomment}
Untuk grafik fungsi, gunakan

- ekspresi sederhana dalam x dan y,\\
- nama fungsi dua variabel\\
- atau matriks data.

Defaultnya adalah kisi kawat yang terisi dengan warna berbeda di kedua
sisinya. Perhatikan bahwa jumlah interval kisi default adalah 10,
tetapi plot menggunakan jumlah persegi panjang 40x40 default untuk
membuat permukaan. Ini dapat diubah.

- n=40, n=[40,40]: jumlah garis kisi di setiap arah\\
- kisi=10, kisi=[10,10]: jumlah garis kisi di setiap arah.

Kami menggunakan default n=40 dan kisi=10.
\end{eulercomment}
\begin{eulerprompt}
>plot3d("x^2+y^2"):
\end{eulerprompt}
\begin{eulercomment}
Interaksi pengguna dimungkinkan dengan parameter \textgreater{}user. Pengguna dapat
menekan tombol berikut.

- kiri, kanan, atas, bawah: mengubah sudut pandang\\
- +, -: memperbesar atau memperkecil\\
- a: menghasilkan anaglif (lihat di bawah)\\
- l: mengubah arah sumber cahaya (lihat di bawah)\\
- spasi: mengatur ulang ke default\\
- return: mengakhiri interaksi
\end{eulercomment}
\begin{eulerprompt}
>plot3d("exp(-x^2+y^2)",>user, ...
>  title="Turn with the vector keys (press return to finish)"):
\end{eulerprompt}
\begin{eulercomment}
Rentang plot untuk fungsi dapat ditentukan dengan

- a,b: rentang x\\
- c,d: rentang y\\
- r: persegi simetris di sekitar (0,0).\\
- n: jumlah subinterval untuk plot.

Ada beberapa parameter untuk menskalakan fungsi atau mengubah tampilan
grafik.

fscale: skala ke nilai fungsi (default adalah \textless{}fscale).\\
scale: angka atau vektor 1x2 untuk diskalakan ke arah x dan y.\\
frame: jenis frame (default 1).
\end{eulercomment}
\begin{eulerprompt}
>plot3d("exp(-(x^2+y^2)/5)",r=10,n=80,fscale=4,scale=1.2,frame=3,>user):
\end{eulerprompt}
\begin{eulercomment}
Tampilan dapat diubah dengan berbagai cara.

- jarak: jarak pandang ke plot.\\
- perbesaran: nilai perbesaran.\\
- sudut: sudut ke sumbu y negatif dalam radian.\\
- tinggi: tinggi tampilan dalam radian.

Nilai default dapat diperiksa atau diubah dengan fungsi view(). Fungsi
ini mengembalikan parameter dalam urutan di atas.
\end{eulercomment}
\begin{eulerprompt}
>view
\end{eulerprompt}
\begin{euleroutput}
  [5,  2.6,  2,  0.4]
\end{euleroutput}
\begin{eulercomment}
Jarak yang lebih dekat membutuhkan zoom yang lebih sedikit. Efeknya
lebih seperti lensa sudut lebar.

Dalam contoh berikut, sudut=0 dan tinggi=0 terlihat dari sumbu y
negatif. Label sumbu untuk y disembunyikan dalam kasus ini.
\end{eulercomment}
\begin{eulerprompt}
>plot3d("x^2+y",distance=3,zoom=1,angle=pi/2,height=0):
\end{eulerprompt}
\begin{eulercomment}
Plot selalu mengarah ke tengah kubus plot. Anda dapat memindahkan
bagian tengah dengan parameter center.
\end{eulercomment}
\begin{eulerprompt}
>plot3d("x^4+y^2",a=0,b=1,c=-1,d=1,angle=-20°,height=20°, ...
>  center=[0.4,0,0],zoom=5):
\end{eulerprompt}
\begin{eulercomment}
Plot diskalakan agar sesuai dengan kubus satuan untuk dilihat. Jadi
tidak perlu mengubah jarak atau zoom tergantung pada ukuran plot.
Namun, label merujuk pada ukuran sebenarnya.

Jika Anda menonaktifkannya dengan scale=false, Anda perlu berhati-hati
agar plot tetap sesuai dengan jendela plot, dengan mengubah jarak
tampilan atau zoom, dan memindahkan bagian tengah.
\end{eulercomment}
\begin{eulerprompt}
>plot3d("5*exp(-x^2-y^2)",r=2,<fscale,<scale,distance=13,height=50°, ...
>  center=[0,0,-2],frame=3):
\end{eulerprompt}
\begin{eulercomment}
Plot polar juga tersedia. Parameter polar=true menggambar plot polar.
Fungsi tersebut harus tetap berupa fungsi x dan y. Parameter "fscale"
menskalakan fungsi dengan skalanya sendiri. Jika tidak, fungsi
tersebut diskalakan agar sesuai dengan kubus.
\end{eulercomment}
\begin{eulerprompt}
>plot3d("1/(x^2+y^2+1)",r=5,>polar, ...
>fscale=2,>hue,n=100,zoom=4,>contour,color=blue):
>function f(r) := exp(-r/2)*cos(r); ...
>plot3d("f(x^2+y^2)",>polar,scale=[1,1,0.4],r=pi,frame=3,zoom=4):
\end{eulerprompt}
\begin{eulercomment}
Parameter rotate memutar fungsi dalam x di sekitar sumbu x.

- rotate=1: Menggunakan sumbu x\\
- rotate=2: Menggunakan sumbu z
\end{eulercomment}
\begin{eulerprompt}
>plot3d("x^2+1",a=-1,b=1,rotate=true,grid=5):
>plot3d("x^2+1",a=-1,b=1,rotate=2,grid=5):
>plot3d("sqrt(25-x^2)",a=0,b=5,rotate=1):
>plot3d("x*sin(x)",a=0,b=6pi,rotate=2):
\end{eulerprompt}
\begin{eulercomment}
Berikut adalah plot dengan tiga fungsi.
\end{eulercomment}
\begin{eulerprompt}
>plot3d("x","x^2+y^2","y",r=2,zoom=3.5,frame=3):
\end{eulerprompt}
\eulerheading{Plot Kontur}
\begin{eulercomment}
Untuk plot, Euler menambahkan garis kisi. Sebagai gantinya,
dimungkinkan untuk menggunakan garis level dan rona satu warna atau
rona warna spektral. Euler dapat menggambar tinggi fungsi pada plot
dengan bayangan. Dalam semua plot 3D, Euler dapat menghasilkan anaglif
merah/sian.

- \textgreater{}hue: Mengaktifkan bayangan terang alih-alih kabel.\\
- \textgreater{}contour: Memplot garis kontur otomatis pada plot.\\
- level=... (atau level): Vektor nilai untuk garis kontur.

Nilai default adalah level="auto", yang menghitung beberapa garis
level secara otomatis. Seperti yang Anda lihat di plot, level
sebenarnya adalah rentang level.

Gaya default dapat diubah. Untuk plot kontur berikut, kami menggunakan
kisi yang lebih halus untuk titik 100x100, menskalakan fungsi dan
plot, dan menggunakan sudut pandang yang berbeda.
\end{eulercomment}
\begin{eulerprompt}
>plot3d("exp(-x^2-y^2)",r=2,n=100,level="thin", ...
> >contour,>spectral,fscale=1,scale=1.1,angle=45°,height=20°):
>plot3d("exp(x*y)",angle=100°,>contour,color=green):
\end{eulerprompt}
\begin{eulercomment}
Shading default menggunakan warna abu-abu. Namun, rentang warna
spektral juga tersedia.

- \textgreater{}spectral: Menggunakan skema spektral default\\
- color=...: Menggunakan warna khusus atau skema spektral

Untuk plot berikut, kami menggunakan skema spektral default dan
menambah jumlah titik untuk mendapatkan tampilan yang sangat halus.
\end{eulercomment}
\begin{eulerprompt}
>plot3d("x^2+y^2",>spectral,>contour,n=100):
\end{eulerprompt}
\begin{eulercomment}
Alih-alih garis level otomatis, kita juga dapat mengatur nilai garis
level. Ini akan menghasilkan garis level tipis alih-alih rentang
level.
\end{eulercomment}
\begin{eulerprompt}
>plot3d("x^2-y^2",0,5,0,5,level=-1:0.1:1,color=redgreen):
\end{eulerprompt}
\begin{eulercomment}
Dalam plot berikut, kami menggunakan dua pita level yang sangat lebar
dari -0,1 hingga 1, dan dari 0,9 hingga 1. Ini dimasukkan sebagai
matriks dengan batas level sebagai kolom.

Selain itu, kami melapisi kisi dengan 10 interval di setiap arah.
\end{eulercomment}
\begin{eulerprompt}
>plot3d("x^2+y^3",level=[-0.1,0.9;0,1], ...
>  >spectral,angle=30°,grid=10,contourcolor=gray):
\end{eulerprompt}
\begin{eulercomment}
Dalam contoh berikut, kami memplot himpunan, di mana

\end{eulercomment}
\begin{eulerformula}
\[
f(x,y) = x^y-y^x = 0
\]
\end{eulerformula}
\begin{eulercomment}
Kami menggunakan satu garis tipis untuk garis datar.
\end{eulercomment}
\begin{eulerprompt}
>plot3d("x^y-y^x",level=0,a=0,b=6,c=0,d=6,contourcolor=red,n=100):
\end{eulerprompt}
\begin{eulercomment}
Dimungkinkan untuk menunjukkan bidang kontur di bawah plot. Warna dan
jarak ke plot dapat ditentukan.
\end{eulercomment}
\begin{eulerprompt}
>plot3d("x^2+y^4",>cp,cpcolor=green,cpdelta=0.2):
\end{eulerprompt}
\begin{eulercomment}
Berikut ini beberapa gaya lainnya. Kami selalu menonaktifkan bingkai,
dan menggunakan berbagai skema warna untuk plot dan kisi.
\end{eulercomment}
\begin{eulerprompt}
>figure(2,2); ...
>expr="y^3-x^2"; ...
>figure(1);  ...
>  plot3d(expr,<frame,>cp,cpcolor=spectral); ...
>figure(2);  ...
>  plot3d(expr,<frame,>spectral,grid=10,cp=2); ...
>figure(3);  ...
>  plot3d(expr,<frame,>contour,color=gray,nc=5,cp=3,cpcolor=greenred); ...
>figure(4);  ...
>  plot3d(expr,<frame,>hue,grid=10,>transparent,>cp,cpcolor=gray); ...
>figure(0):
\end{eulerprompt}
\begin{eulercomment}
Ada beberapa skema spektral lain, yang diberi nomor dari 1 hingga 9.
Namun, Anda juga dapat menggunakan color=value, di mana value

- spektral: untuk rentang dari biru hingga merah\\
- putih: untuk rentang yang lebih redup\\
- kuning biru, ungu hijau, biru kuning, hijau merah\\
- biru kuning, hijau ungu, kuning biru, merah hijau
\end{eulercomment}
\begin{eulerprompt}
>figure(3,3); ...
>for i=1:9;  ...
>  figure(i); plot3d("x^2+y^2",spectral=i,>contour,>cp,<frame,zoom=4);  ...
>end; ...
>figure(0):
\end{eulerprompt}
\begin{eulercomment}
Sumber cahaya dapat diubah dengan l dan tombol kursor selama interaksi
pengguna. Sumber cahaya juga dapat diatur dengan parameter.

- light: arah cahaya\\
- amb: cahaya sekitar antara 0 dan 1

Perlu dicatat bahwa program tidak membuat perbedaan antara sisi plot.
Tidak ada bayangan. Untuk ini, Anda memerlukan Povray.
\end{eulercomment}
\begin{eulerprompt}
>plot3d("-x^2-y^2", ...
>  hue=true,light=[0,1,1],amb=0,user=true, ...
>  title="Press l and cursor keys (return to exit)"):
\end{eulerprompt}
\begin{eulercomment}
Parameter warna mengubah warna permukaan. Warna garis level juga dapat
diubah.
\end{eulercomment}
\begin{eulerprompt}
>plot3d("-x^2-y^2",color=rgb(0.7,0.7,0),hue=true,frame=false, ...
>  zoom=3,contourcolor=red,level=-2:0.1:1,dl=0.01):
\end{eulerprompt}
\begin{eulercomment}
Warna 0 memberikan efek pelangi khusus.
\end{eulercomment}
\begin{eulerprompt}
>plot3d("x^2/(x^2+y^2+1)",color=0,hue=true,grid=10):
\end{eulerprompt}
\begin{eulercomment}
Permukaannya juga bisa transparan.
\end{eulercomment}
\begin{eulerprompt}
>plot3d("x^2+y^2",>transparent,grid=10,wirecolor=blue):
\end{eulerprompt}
\eulerheading{Plot Implisit}
\begin{eulercomment}
Ada juga plot implisit dalam tiga dimensi. Euler menghasilkan potongan
melalui objek. Fitur plot3d mencakup plot implisit. Plot ini
menunjukkan himpunan nol dari suatu fungsi dalam tiga variabel.

Solusi dari

\end{eulercomment}
\begin{eulerformula}
\[
f(x,y,z) = 0
\]
\end{eulerformula}
\begin{eulercomment}
dapat divisualisasikan dalam potongan yang sejajar dengan bidang x-y,
x-z, dan y-z.

- implisit=1: potongan sejajar dengan bidang y-z\\
- implisit=2: potongan sejajar dengan bidang x-z\\
- implisit=4: potongan sejajar dengan bidang x-y

Tambahkan nilai-nilai ini, jika Anda suka. Dalam contoh ini, kami
memplot

\end{eulercomment}
\begin{eulerformula}
\[
M = \{ (x,y,z) : x^2+y^3+zy=1 \}
\]
\end{eulerformula}
\begin{eulerprompt}
>plot3d("x^2+y^3+z*y-1",r=5,implicit=3):
>c=1; d=1;
>plot3d("((x^2+y^2-c^2)^2+(z^2-1)^2)*((y^2+z^2-c^2)^2+(x^2-1)^2)*((z^2+x^2-c^2)^2+(y^2-1)^2)-d",r=2,<frame,>implicit,>user): 
>plot3d("x^2+y^2+4*x*z+z^3",>implicit,r=2,zoom=2.5):
>plot3d("x^2+y^2+4*x*z+z^3",>implicit,r=5,zoom=3):
\end{eulerprompt}
\eulerheading{Membuat Plot Data 3D}
\begin{eulercomment}
Sama seperti plot2d, plot3d menerima data. Untuk objek 3D, Anda perlu
menyediakan matriks nilai x, y, dan z, atau tiga fungsi atau ekspresi
fx(x,y), fy(x,y), fz(x,y).

\end{eulercomment}
\begin{eulerformula}
\[
\gamma(t,s) = (x(t,s),y(t,s),z(t,s))
\]
\end{eulerformula}
\begin{eulercomment}
Karena x,y,z adalah matriks, kami berasumsi bahwa (t,s) berjalan
melalui kisi persegi. Hasilnya, Anda dapat membuat plot gambar persegi
panjang di ruang angkasa.

Anda dapat menggunakan bahasa matriks Euler untuk menghasilkan
koordinat secara efektif.

Dalam contoh berikut, kami menggunakan vektor nilai t dan vektor kolom
nilai s untuk membuat parameter permukaan bola. Dalam gambar, kami
dapat menandai wilayah, dalam kasus kami wilayah kutub.
\end{eulercomment}
\begin{eulerprompt}
>t=linspace(0,2pi,180); s=linspace(-pi/2,pi/2,90)'; ...
>x=cos(s)*cos(t); y=cos(s)*sin(t); z=sin(s); ...
>plot3d(x,y,z,>hue, ...
>color=blue,<frame,grid=[10,20], ...
>values=s,contourcolor=red,level=[90°-24°;90°-22°], ...
>scale=1.4,height=50°):
\end{eulerprompt}
\begin{eulercomment}
Berikut adalah contoh, yang merupakan grafik suatu fungsi.
\end{eulercomment}
\begin{eulerprompt}
>t=-1:0.1:1; s=(-1:0.1:1)'; plot3d(t,s,t*s,grid=10):
\end{eulerprompt}
\begin{eulercomment}
However, we can make all sorts of surfaces. Here is the same surface
as a function

\end{eulercomment}
\begin{eulerformula}
\[
x = y \, z
\]
\end{eulerformula}
\begin{eulerprompt}
>plot3d(t*s,t,s,angle=180°,grid=10):
\end{eulerprompt}
\begin{eulercomment}
Dengan usaha lebih, kita dapat menghasilkan banyak permukaan.

Dalam contoh berikut, kita membuat tampilan berbayang dari bola yang
terdistorsi. Koordinat yang biasa untuk bola adalah

\end{eulercomment}
\begin{eulerformula}
\[
\gamma(t,s) = (\cos(t)\cos(s),\sin(t)\sin(s),\cos(s))
\]
\end{eulerformula}
\begin{eulercomment}
dengan

\end{eulercomment}
\begin{eulerformula}
\[
0 \le t \le 2\pi, \quad \frac{-\pi}{2} \le s \le \frac{\pi}{2}.
\]
\end{eulerformula}
\begin{eulercomment}
Kita mendistorsi ini dengan faktor

\end{eulercomment}
\begin{eulerformula}
\[
d(t,s) = \frac{\cos(4t)+\cos(8s)}{4}.
\]
\end{eulerformula}
\begin{eulerprompt}
>t=linspace(0,2pi,320); s=linspace(-pi/2,pi/2,160)'; ...
>d=1+0.2*(cos(4*t)+cos(8*s)); ...
>plot3d(cos(t)*cos(s)*d,sin(t)*cos(s)*d,sin(s)*d,hue=1, ...
>  light=[1,0,1],frame=0,zoom=5):
\end{eulerprompt}
\begin{eulercomment}
Tentu saja, titik awan juga memungkinkan. Untuk memplot data titik di
ruang, kita memerlukan tiga vektor untuk koordinat titik.

Gayanya sama seperti di plot2d dengan points=true;
\end{eulercomment}
\begin{eulerprompt}
>n=500;  ...
>  plot3d(normal(1,n),normal(1,n),normal(1,n),points=true,style="."):
\end{eulerprompt}
\begin{eulercomment}
Anda juga dapat memplot kurva dalam 3D. Dalam kasus ini, lebih mudah
untuk menghitung titik-titik kurva terlebih dahulu. Untuk kurva dalam
bidang, kami menggunakan urutan koordinat dan parameter wire=true.
\end{eulercomment}
\begin{eulerprompt}
>t=linspace(0,8pi,500); ...
>plot3d(sin(t),cos(t),t/10,>wire,zoom=3):
>t=linspace(0,4pi,1000); plot3d(cos(t),sin(t),t/2pi,>wire, ...
>linewidth=3,wirecolor=blue):
>X=cumsum(normal(3,100)); ...
> plot3d(X[1],X[2],X[3],>anaglyph,>wire):
\end{eulerprompt}
\begin{eulercomment}
EMT juga dapat membuat grafik dalam mode anaglif. Untuk melihat grafik
tersebut, Anda memerlukan kacamata merah/sian.
\end{eulercomment}
\begin{eulerprompt}
> plot3d("x^2+y^3",>anaglyph,>contour,angle=30°):
\end{eulerprompt}
\begin{eulercomment}
Seringkali, skema warna spektral digunakan untuk plot. Ini menekankan
tinggi fungsi.
\end{eulercomment}
\begin{eulerprompt}
>plot3d("x^2*y^3-y",>spectral,>contour,zoom=3.2):
\end{eulerprompt}
\begin{eulercomment}
Euler juga dapat memplot permukaan berparameter, ketika parameternya
adalah nilai x, y, dan z dari gambar kotak persegi panjang di ruang
tersebut.

Untuk demo berikut, kami menyiapkan parameter u dan v, dan
menghasilkan koordinat ruang dari parameter tersebut.
\end{eulercomment}
\begin{eulerprompt}
>u=linspace(-1,1,10); v=linspace(0,2*pi,50)'; ...
>X=(3+u*cos(v/2))*cos(v); Y=(3+u*cos(v/2))*sin(v); Z=u*sin(v/2); ...
>plot3d(X,Y,Z,>anaglyph,<frame,>wire,scale=2.3):
\end{eulerprompt}
\begin{eulercomment}
Berikut adalah contoh yang lebih rumit, yang tampak megah dengan kaca
merah/cyan.
\end{eulercomment}
\begin{eulerprompt}
>u:=linspace(-pi,pi,160); v:=linspace(-pi,pi,400)';  ...
>x:=(4*(1+.25*sin(3*v))+cos(u))*cos(2*v); ...
>y:=(4*(1+.25*sin(3*v))+cos(u))*sin(2*v); ...
> z=sin(u)+2*cos(3*v); ...
>plot3d(x,y,z,frame=0,scale=1.5,hue=1,light=[1,0,-1],zoom=2.8,>anaglyph):
\end{eulerprompt}
\eulerheading{Plot Statistik}
\begin{eulercomment}
Plot batang juga dimungkinkan. Untuk ini, kita harus menyediakan

- x: vektor baris dengan n+1 elemen\\
- y: vektor kolom dengan n+1 elemen\\
- z: matriks nilai nxn.

z dapat lebih besar, tetapi hanya nilai nxn yang akan digunakan.

Dalam contoh, pertama-tama kita menghitung nilai. Kemudian kita
menyesuaikan x dan y, sehingga vektor berpusat pada nilai yang
digunakan.
\end{eulercomment}
\begin{eulerprompt}
>x=-1:0.1:1; y=x'; z=x^2+y^2; ...
>xa=(x|1.1)-0.05; ya=(y_1.1)-0.05; ...
>plot3d(xa,ya,z,bar=true):
\end{eulerprompt}
\begin{eulercomment}
Dimungkinkan untuk membagi bidang permukaan menjadi dua bagian atau
lebih.
\end{eulercomment}
\begin{eulerprompt}
>x=-1:0.1:1; y=x'; z=x+y; d=zeros(size(x)); ...
>plot3d(x,y,z,disconnect=2:2:20):
\end{eulerprompt}
\begin{eulercomment}
Jika memuat atau membuat matriks data M dari sebuah file dan perlu
memplotnya dalam 3D, Anda dapat menskalakan matriks ke [-1,1] dengan
scale(M), atau menskalakan matriks dengan \textgreater{}zscale. Ini dapat
dikombinasikan dengan faktor penskalaan individual yang diterapkan
sebagai tambahan.
\end{eulercomment}
\begin{eulerprompt}
>i=1:20; j=i'; ...
>plot3d(i*j^2+100*normal(20,20),>zscale,scale=[1,1,1.5],angle=-40°,zoom=1.8):
>Z=intrandom(5,100,6); v=zeros(5,6); ...
>loop 1 to 5; v[#]=getmultiplicities(1:6,Z[#]); end; ...
>columnsplot3d(v',scols=1:5,ccols=[1:5]):
\end{eulerprompt}
\eulerheading{Permukaan Benda Putar}
\begin{eulerprompt}
>plot2d("(x^2+y^2-1)^3-x^2*y^3",r=1.3, ...
>style="#",color=red,<outline, ...
>level=[-2;0],n=100):
>ekspresi &= (x^2+y^2-1)^3-x^2*y^3; $ekspresi
\end{eulerprompt}
\begin{eulercomment}
Kita ingin memutar kurva jantung di sekitar sumbu y. Berikut ini
adalah ekspresi yang mendefinisikan jantung:

\end{eulercomment}
\begin{eulerformula}
\[
f(x,y)=(x^2+y^2-1)^3-x^2.y^3.
\]
\end{eulerformula}
\begin{eulercomment}
Selanjutnya kita tetapkan

\end{eulercomment}
\begin{eulerformula}
\[
x=r.cos(a),\quad y=r.sin(a).
\]
\end{eulerformula}
\begin{eulerprompt}
>function fr(r,a) &= ekspresi with [x=r*cos(a),y=r*sin(a)] | trigreduce; $fr(r,a)
\end{eulerprompt}
\begin{eulercomment}
Hal ini memungkinkan untuk mendefinisikan fungsi numerik, yang
memecahkan r, jika a diberikan. Dengan fungsi itu kita dapat memplot
jantung yang diputar sebagai permukaan parametrik.
\end{eulercomment}
\begin{eulerprompt}
>function map f(a) := bisect("fr",0,2;a); ...
>t=linspace(-pi/2,pi/2,100); r=f(t);  ...
>s=linspace(pi,2pi,100)'; ...
>plot3d(r*cos(t)*sin(s),r*cos(t)*cos(s),r*sin(t), ...
>>hue,<frame,color=red,zoom=4,amb=0,max=0.7,grid=12,height=50°):
\end{eulerprompt}
\begin{eulercomment}
Berikut ini adalah plot 3D dari gambar di atas yang diputar di sekitar
sumbu z. Kami mendefinisikan fungsi yang menggambarkan objek tersebut.
\end{eulercomment}
\begin{eulerprompt}
>function f(x,y,z) ...
\end{eulerprompt}
\begin{eulerudf}
  r=x^2+y^2;
  return (r+z^2-1)^3-r*z^3;
   endfunction
\end{eulerudf}
\begin{eulerprompt}
>plot3d("f(x,y,z)", ...
>xmin=0,xmax=1.2,ymin=-1.2,ymax=1.2,zmin=-1.2,zmax=1.4, ...
>implicit=1,angle=-30°,zoom=2.5,n=[10,100,60],>anaglyph):
\end{eulerprompt}
\eulerheading{Plot 3D Khusus}
\begin{eulercomment}
Fungsi plot3d memang bagus, tetapi tidak memenuhi semua kebutuhan.
Selain rutinitas yang lebih mendasar, Anda dapat memperoleh plot
berbingkai dari objek apa pun yang Anda suka.

Meskipun Euler bukanlah program 3D, ia dapat menggabungkan beberapa
objek dasar. Kami mencoba memvisualisasikan parabola dan garis
singgungnya.
\end{eulercomment}
\begin{eulerprompt}
>function myplot ...
\end{eulerprompt}
\begin{eulerudf}
    y=-1:0.01:1; x=(-1:0.01:1)';
    plot3d(x,y,0.2*(x-0.1)/2,<scale,<frame,>hue, ..
      hues=0.5,>contour,color=orange);
    h=holding(1);
    plot3d(x,y,(x^2+y^2)/2,<scale,<frame,>contour,>hue);
    holding(h);
  endfunction
\end{eulerudf}
\begin{eulercomment}
Sekarang framedplot() menyediakan bingkai dan mengatur tampilan.
\end{eulercomment}
\begin{eulerprompt}
>framedplot("myplot",[-1,1,-1,1,0,1],height=0,angle=-30°, ...
>  center=[0,0,-0.7],zoom=3):
\end{eulerprompt}
\begin{eulercomment}
Dengan cara yang sama, Anda dapat memplot bidang kontur secara manual.
Perhatikan bahwa plot3d() menetapkan jendela ke fullwindow() secara
default, tetapi plotcontourplane() mengasumsikannya.
\end{eulercomment}
\begin{eulerprompt}
>x=-1:0.02:1.1; y=x'; z=x^2-y^4;
>function myplot (x,y,z) ...
\end{eulerprompt}
\begin{eulerudf}
    zoom(2);
    wi=fullwindow();
    plotcontourplane(x,y,z,level="auto",<scale);
    plot3d(x,y,z,>hue,<scale,>add,color=white,level="thin");
    window(wi);
    reset();
  endfunction
\end{eulerudf}
\begin{eulerprompt}
>myplot(x,y,z):
\end{eulerprompt}
\eulerheading{Animasi}
\begin{eulercomment}
Euler dapat menggunakan bingkai untuk melakukan pra-komputasi animasi.

Salah satu fungsi yang memanfaatkan teknik ini adalah rotate. Fungsi
ini dapat mengubah sudut pandang dan menggambar ulang plot 3D. Fungsi
ini memanggil addpage() untuk setiap plot baru. Terakhir, fungsi ini
menganimasikan plot tersebut.

Silakan pelajari sumber rotate untuk melihat detail selengkapnya.
\end{eulercomment}
\begin{eulerprompt}
>function testplot () := plot3d("x^2+y^3"); ...
>rotate("testplot"); testplot():
\end{eulerprompt}
\eulerheading{Menggambar Povray}
\begin{eulercomment}
Dengan bantuan file Euler povray.e, Euler dapat membuat file Povray.
Hasilnya sangat bagus untuk dilihat.

Anda perlu menginstal Povray (32bit atau 64bit) dari
http://www.povray.org/, dan meletakkan subdirektori "bin" dari Povray ke dalam jalur lingkungan, atau mengatur variabel "defaultpovray" dengan jalur lengkap yang mengarah ke "pvengine.exe".

Antarmuka Povray dari Euler membuat file Povray di direktori home
pengguna, dan memanggil Povray untuk mengurai file-file ini. Nama file
default adalah current.pov, dan direktori default adalah eulerhome(),
biasanya c:\textbackslash{}Users\textbackslash{}Username\textbackslash{}Euler. Povray membuat file PNG, yang dapat
dimuat oleh Euler ke dalam buku catatan. Untuk membersihkan file-file
ini, gunakan povclear().

Fungsi pov3d memiliki semangat yang sama dengan plot3d. Fungsi ini
dapat menghasilkan grafik fungsi f(x,y), atau permukaan dengan
koordinat X,Y,Z dalam matriks, termasuk garis level opsional. Fungsi
ini memulai raytracer secara otomatis, dan memuat adegan ke dalam buku
catatan Euler.

Selain pov3d(), ada banyak fungsi, yang menghasilkan objek Povray.
Fungsi-fungsi ini mengembalikan string, yang berisi kode Povray untuk
objek. Untuk menggunakan fungsi-fungsi ini, mulai file Povray dengan
povstart(). Kemudian gunakan writeln(...) untuk menulis objek ke file
adegan. Terakhir, akhiri file dengan povend(). Secara default,
raytracer akan mulai, dan PNG akan dimasukkan ke dalam buku catatan
Euler.

Fungsi objek memiliki parameter yang disebut "look", yang memerlukan
string dengan kode Povray untuk tekstur dan penyelesaian objek. Fungsi
povlook() dapat digunakan untuk menghasilkan string ini. Fungsi ini
memiliki parameter untuk warna, transparansi, Phong Shading, dll.

Perhatikan bahwa alam semesta Povray memiliki sistem koordinat lain.
Antarmuka ini menerjemahkan semua koordinat ke sistem Povray. Jadi
Anda dapat terus berpikir dalam sistem koordinat Euler dengan z
menunjuk vertikal ke atas, dan sumbu x, y, z dalam arah kanan.\\
Anda perlu memuat berkas povray.
\end{eulercomment}
\begin{eulerprompt}
>load povray;
\end{eulerprompt}
\begin{eulercomment}
Pastikan direktori bin Povray ada di jalur tersebut. Jika tidak, edit
variabel berikut sehingga berisi jalur ke povray yang dapat
dieksekusi.
\end{eulercomment}
\begin{eulerprompt}
>defaultpovray="C:\(\backslash\)Program Files\(\backslash\)POV-Ray\(\backslash\)v3.7\(\backslash\)bin\(\backslash\)pvengine.exe"
\end{eulerprompt}
\begin{euleroutput}
  C:\(\backslash\)Program Files\(\backslash\)POV-Ray\(\backslash\)v3.7\(\backslash\)bin\(\backslash\)pvengine.exe
\end{euleroutput}
\begin{eulercomment}
Untuk kesan pertama, kami membuat fungsi sederhana. Perintah berikut
menghasilkan file povray di direktori pengguna Anda, dan menjalankan
Povray untuk melakukan ray tracing pada file ini.

Jika Anda menjalankan perintah berikut, GUI Povray akan terbuka,
menjalankan file, dan menutup secara otomatis. Karena alasan keamanan,
Anda akan ditanya apakah Anda ingin mengizinkan file exe untuk
berjalan. Anda dapat menekan batal untuk menghentikan pertanyaan lebih
lanjut. Anda mungkin harus menekan OK di jendela Povray untuk mengakui
dialog awal Povray.
\end{eulercomment}
\begin{eulerprompt}
>plot3d("x^2+y^2",zoom=2):
>pov3d("x^2+y^2",zoom=3);     
\end{eulerprompt}
\begin{eulercomment}
Kita dapat membuat fungsi tersebut transparan dan menambahkan
penyelesaian lainnya. Kita juga dapat menambahkan garis level pada
plot fungsi.
\end{eulercomment}
\begin{eulerprompt}
>pov3d("x^2+y^3",axiscolor=red,angle=-45°,>anaglyph, ...
>  look=povlook(cyan,0.2),level=-1:0.5:1,zoom=3.8);
\end{eulerprompt}
\begin{eulercomment}
Terkadang perlu untuk mencegah penskalaan fungsi, dan menskalakan
fungsi secara manual.

Kami memplot himpunan titik pada bidang kompleks, di mana hasil kali
jarak ke 1 dan -1 sama dengan 1.
\end{eulercomment}
\begin{eulerprompt}
>pov3d("((x-1)^2+y^2)*((x+1)^2+y^2)/40",r=2, ...
>  angle=-120°,level=1/40,dlevel=0.005,light=[-1,1,1],height=10°,n=50, ...
>  <fscale,zoom=3.8);
\end{eulerprompt}
\eulerheading{Membuat Plot dengan Koordinat}
\begin{eulercomment}
Alih-alih menggunakan fungsi, kita dapat membuat plot dengan
koordinat. Seperti pada plot3d, kita memerlukan tiga matriks untuk
menentukan objek.

Dalam contoh ini, kita memutar fungsi di sekitar sumbu z.
\end{eulercomment}
\begin{eulerprompt}
>function f(x) := x^3-x+1; ...
>x=-1:0.01:1; t=linspace(0,2pi,50)'; ...
>Z=x; X=cos(t)*f(x); Y=sin(t)*f(x); ...
>pov3d(X,Y,Z,angle=40°,look=povlook(red,0.1),height=50°,axis=0,zoom=4,light=[10,5,15]);
\end{eulerprompt}
\begin{eulercomment}
Dalam contoh berikut, kami memplot gelombang yang diredam. Kami
menghasilkan gelombang dengan bahasa matriks Euler.

Kami juga menunjukkan, bagaimana objek tambahan dapat ditambahkan ke
adegan pov3d. Untuk pembuatan objek, lihat contoh berikut. Perhatikan
bahwa plot3d menskalakan plot, sehingga sesuai dengan kubus satuan.
\end{eulercomment}
\begin{eulerprompt}
>r=linspace(0,1,80); phi=linspace(0,2pi,80)'; ...
>x=r*cos(phi); y=r*sin(phi); z=exp(-5*r)*cos(8*pi*r)/3;  ...
>pov3d(x,y,z,zoom=6,axis=0,height=30°,add=povsphere([0.5,0,0.25],0.15,povlook(red)), ...
>  w=500,h=300);
\end{eulerprompt}
\begin{eulercomment}
Dengan metode shading Povray yang canggih, hanya sedikit titik yang
dapat menghasilkan permukaan yang sangat halus. Hanya pada batas dan
bayangan, triknya mungkin menjadi jelas.

Untuk ini, kita perlu menambahkan vektor normal di setiap titik
matriks.
\end{eulercomment}
\begin{eulerprompt}
>Z &= x^2*y^3
\end{eulerprompt}
\begin{euleroutput}
  
                                   2  3
                                  x  y
  
\end{euleroutput}
\begin{eulercomment}
Persamaan permukaannya adalah [x,y,Z]. Kita hitung dua turunan x dan y
dari persamaan ini dan ambil perkalian silang sebagai normalnya.
\end{eulercomment}
\begin{eulerprompt}
>dx &= diff([x,y,Z],x); dy &= diff([x,y,Z],y);
\end{eulerprompt}
\begin{eulercomment}
Kami mendefinisikan normal sebagai perkalian silang turunan-turunan
ini dan mendefinisikan fungsi koordinat.
\end{eulercomment}
\begin{eulerprompt}
>N &= crossproduct(dx,dy); NX &= N[1]; NY &= N[2]; NZ &= N[3]; N,
\end{eulerprompt}
\begin{euleroutput}
  
                                 3       2  2
                         [- 2 x y , - 3 x  y , 1]
  
\end{euleroutput}
\begin{eulercomment}
Kami hanya menggunakan 25 poin.
\end{eulercomment}
\begin{eulerprompt}
>x=-1:0.5:1; y=x';
>pov3d(x,y,Z(x,y),angle=10°, ...
>  xv=NX(x,y),yv=NY(x,y),zv=NZ(x,y),<shadow);
\end{eulerprompt}
\begin{eulercomment}
Berikut ini adalah simpul Trefoil yang dibuat oleh A. Busser di
Povray. Ada versi yang lebih baik dari simpul ini dalam
contoh-contohnya.

Lihat: Contoh\textbackslash{}Simpul Trefoil \textbar{} Simpul Trefoil

Untuk tampilan yang bagus dengan tidak terlalu banyak titik, kami
menambahkan vektor normal di sini. Kami menggunakan Maxima untuk
menghitung normal bagi kami. Pertama, tiga fungsi untuk koordinat
sebagai ekspresi simbolik.
\end{eulercomment}
\begin{eulerprompt}
>X &= ((4+sin(3*y))+cos(x))*cos(2*y); ...
>Y &= ((4+sin(3*y))+cos(x))*sin(2*y); ...
>Z &= sin(x)+2*cos(3*y);
\end{eulerprompt}
\begin{eulercomment}
Kemudian dua vektor turunan ke x dan y.
\end{eulercomment}
\begin{eulerprompt}
>dx &= diff([X,Y,Z],x); dy &= diff([X,Y,Z],y);
\end{eulerprompt}
\begin{eulercomment}
Sekarang normal, yang merupakan perkalian silang dari dua turunan.
\end{eulercomment}
\begin{eulerprompt}
>dn &= crossproduct(dx,dy);
\end{eulerprompt}
\begin{eulercomment}
Sekarang mari kita evaluasi semua ini secara numerik.
\end{eulercomment}
\begin{eulerprompt}
>x:=linspace(-%pi,%pi,40); y:=linspace(-%pi,%pi,100)';
\end{eulerprompt}
\begin{eulercomment}
Vektor normal adalah evaluasi ekspresi simbolik dn[i] untuk i=1,2,3.
Sintaks untuk ini adalah \&"ekspresi"(parameter). Ini adalah alternatif
untuk metode pada contoh sebelumnya, di mana kita mendefinisikan
ekspresi simbolik NX, NY, NZ terlebih dahulu.
\end{eulercomment}
\begin{eulerprompt}
>pov3d(X(x,y),Y(x,y),Z(x,y),>anaglyph,axis=0,zoom=5,w=450,h=350, ...
>  <shadow,look=povlook(blue), ...
>  xv=&"dn[1]"(x,y), yv=&"dn[2]"(x,y), zv=&"dn[3]"(x,y));
\end{eulerprompt}
\begin{eulercomment}
Kita juga dapat membuat grid dalam 3D.
\end{eulercomment}
\begin{eulerprompt}
>povstart(zoom=4); ...
>x=-1:0.5:1; r=1-(x+1)^2/6; ...
>t=(0°:30°:360°)'; y=r*cos(t); z=r*sin(t); ...
>writeln(povgrid(x,y,z,d=0.02,dballs=0.05)); ...
>povend();
\end{eulerprompt}
\begin{eulercomment}
Dengan povgrid(), kurva dimungkinkan.
\end{eulercomment}
\begin{eulerprompt}
>povstart(center=[0,0,1],zoom=3.6); ...
>t=linspace(0,2,1000); r=exp(-t); ...
>x=cos(2*pi*10*t)*r; y=sin(2*pi*10*t)*r; z=t; ...
>writeln(povgrid(x,y,z,povlook(red))); ...
>writeAxis(0,2,axis=3); ...
>povend();
\end{eulerprompt}
\eulerheading{Objek Povray}
\begin{eulercomment}
Di atas, kami menggunakan pov3d untuk memplot permukaan. Antarmuka
povray di Euler juga dapat menghasilkan objek Povray. Objek-objek ini
disimpan sebagai string di Euler, dan perlu ditulis ke berkas Povray.

Kami memulai output dengan povstart().
\end{eulercomment}
\begin{eulerprompt}
>povstart(zoom=4);
\end{eulerprompt}
\begin{eulercomment}
Pertama, kita mendefinisikan tiga silinder, dan menyimpannya dalam
string di Euler.

Fungsi povx() dll. hanya mengembalikan vektor [1,0,0], yang dapat
digunakan sebagai gantinya.
\end{eulercomment}
\begin{eulerprompt}
>c1=povcylinder(-povx,povx,1,povlook(red)); ...
>c2=povcylinder(-povy,povy,1,povlook(yellow)); ...
>c3=povcylinder(-povz,povz,1,povlook(blue)); ...
\end{eulerprompt}
\begin{eulercomment}
Rangkaian tersebut berisi kode Povray, yang tidak perlu kita pahami
saat itu.
\end{eulercomment}
\begin{eulerprompt}
>c2
\end{eulerprompt}
\begin{euleroutput}
  cylinder \{ <0,0,-1>, <0,0,1>, 1
   texture \{ pigment \{ color rgb <0.941176,0.941176,0.392157> \}  \} 
   finish \{ ambient 0.2 \} 
   \}
\end{euleroutput}
\begin{eulercomment}
Seperti yang Anda lihat, kami menambahkan tekstur ke objek dalam tiga
warna berbeda.

Hal itu dilakukan oleh povlook(), yang mengembalikan string dengan
kode Povray yang relevan. Kita dapat menggunakan warna Euler default,
atau menentukan warna kita sendiri. Kita juga dapat menambahkan
transparansi, atau mengubah cahaya sekitar.
\end{eulercomment}
\begin{eulerprompt}
>povlook(rgb(0.1,0.2,0.3),0.1,0.5)
\end{eulerprompt}
\begin{euleroutput}
   texture \{ pigment \{ color rgbf <0.101961,0.2,0.301961,0.1> \}  \} 
   finish \{ ambient 0.5 \} 
  
\end{euleroutput}
\begin{eulercomment}
Sekarang kita mendefinisikan objek persimpangan dan menulis hasilnya
ke berkas.
\end{eulercomment}
\begin{eulerprompt}
>writeln(povintersection([c1,c2,c3]));
\end{eulerprompt}
\begin{eulercomment}
Persimpangan tiga silinder sulit dibayangkan, jika Anda belum pernah
melihatnya sebelumnya.
\end{eulercomment}
\begin{eulerprompt}
>povend;
\end{eulerprompt}
\begin{eulercomment}
Fungsi-fungsi berikut menghasilkan fraktal secara rekursif.

Fungsi pertama menunjukkan bagaimana Euler menangani objek-objek
Povray sederhana. Fungsi povbox() mengembalikan string yang berisi
koordinat kotak, tekstur, dan hasil akhir.
\end{eulercomment}
\begin{eulerprompt}
>function onebox(x,y,z,d) := povbox([x,y,z],[x+d,y+d,z+d],povlook());
>function fractal (x,y,z,h,n) ...
\end{eulerprompt}
\begin{eulerudf}
   if n==1 then writeln(onebox(x,y,z,h));
   else
     h=h/3;
     fractal(x,y,z,h,n-1);
     fractal(x+2*h,y,z,h,n-1);
     fractal(x,y+2*h,z,h,n-1);
     fractal(x,y,z+2*h,h,n-1);
     fractal(x+2*h,y+2*h,z,h,n-1);
     fractal(x+2*h,y,z+2*h,h,n-1);
     fractal(x,y+2*h,z+2*h,h,n-1);
     fractal(x+2*h,y+2*h,z+2*h,h,n-1);
     fractal(x+h,y+h,z+h,h,n-1);
   endif;
  endfunction
\end{eulerudf}
\begin{eulerprompt}
>povstart(fade=10,<shadow);
>fractal(-1,-1,-1,2,4);
>povend();
\end{eulerprompt}
\begin{eulercomment}
Perbedaan memungkinkan pemisahan satu objek dari objek lainnya.
Seperti halnya persimpangan, ada beberapa objek CSG dari Povray.
\end{eulercomment}
\begin{eulerprompt}
>povstart(light=[5,-5,5],fade=10);
\end{eulerprompt}
\begin{eulercomment}
Untuk demonstrasi ini, kami mendefinisikan objek dalam Povray,
alih-alih menggunakan string dalam Euler. Definisi langsung ditulis ke
berkas.

Koordinat kotak -1 berarti [-1,-1,-1].
\end{eulercomment}
\begin{eulerprompt}
>povdefine("mycube",povbox(-1,1));
\end{eulerprompt}
\begin{eulercomment}
Kita dapat menggunakan objek ini dalam povobject(), yang mengembalikan
string seperti biasa.
\end{eulercomment}
\begin{eulerprompt}
>c1=povobject("mycube",povlook(red));
\end{eulerprompt}
\begin{eulercomment}
Kita buat kubus kedua, lalu putar dan ubah skalanya sedikit.
\end{eulercomment}
\begin{eulerprompt}
>c2=povobject("mycube",povlook(yellow),translate=[1,1,1], ...
>  rotate=xrotate(10°)+yrotate(10°), scale=1.2);
\end{eulerprompt}
\begin{eulercomment}
Lalu kita ambil selisih kedua benda tersebut.
\end{eulercomment}
\begin{eulerprompt}
>writeln(povdifference(c1,c2));
\end{eulerprompt}
\begin{eulercomment}
Sekarang tambahkan tiga sumbu.
\end{eulercomment}
\begin{eulerprompt}
>writeAxis(-1.2,1.2,axis=1); ...
>writeAxis(-1.2,1.2,axis=2); ...
>writeAxis(-1.2,1.2,axis=4); ...
>povend();
\end{eulerprompt}
\eulerheading{Fungsi Implisit}
\begin{eulercomment}
Povray dapat memplot himpunan di mana f(x,y,z)=0, sama seperti
parameter implisit dalam plot3d. Namun, hasilnya terlihat jauh lebih
baik.

Sintaks untuk fungsi-fungsi tersebut sedikit berbeda. Anda tidak dapat
menggunakan output dari ekspresi Maxima atau Euler.

\end{eulercomment}
\begin{eulerformula}
\[
((x^2+y^2-c^2)^2+(z^2-1)^2)*((y^2+z^2-c^2)^2+(x^2-1)^2)*((z^2+x^2-c^2)^2+(y^2-1)^2)=d
\]
\end{eulerformula}
\begin{eulerprompt}
>povstart(angle=70°, height=50°, zoom=4);
>c = 0.1;
>d = 0.1;
>writeln(povsurface("(pow(pow(x, 2) + pow(y, 2) - pow(c, 2), 2) + pow(pow(z, 2) - 1, 2))*(pow(pow(y, 2) + pow(z, 2) - pow(c, 2), 2) + pow(pow(x, 2) - 1, 2))*(pow(pow(z, 2) + pow(x, 2) - pow(c, 2), 2) + pow(pow(y, 2) - 1, 2)) - d",povlook(red)));
>povend();
\end{eulerprompt}
\begin{euleroutput}
  Error : Povray error!
  
  Error generated by error() command
  
  povray:
      error("Povray error!");
  Try "trace errors" to inspect local variables after errors.
  povend:
      povray(file,w,h,aspect,exit); 
\end{euleroutput}
\begin{eulerprompt}
>povstart(angle=25°,height=10°); 
>writeln(povsurface("pow(x,2)+pow(y,2)*pow(z,2)-1",povlook(blue),povbox(-2,2,"")));
>povend();
>povstart(angle=70°,height=50°,zoom=4);
\end{eulerprompt}
\begin{eulercomment}
Buat permukaan implisit. Perhatikan sintaksis yang berbeda dalam
ekspresi.
\end{eulercomment}
\begin{eulerprompt}
>writeln(povsurface("pow(x,2)*y-pow(y,3)-pow(z,2)",povlook(green))); ...
>writeAxes(); ...
>povend();
\end{eulerprompt}
\eulerheading{Objek Mesh}
\begin{eulercomment}
Dalam contoh ini, kami menunjukkan cara membuat objek mesh, dan
menggambarnya dengan informasi tambahan.

Kami ingin memaksimalkan xy dalam kondisi x+y=1 dan menunjukkan
sentuhan tangensial garis-garis level.
\end{eulercomment}
\begin{eulerprompt}
>povstart(angle=-10°,center=[0.5,0.5,0.5],zoom=7);
\end{eulerprompt}
\begin{eulercomment}
Kita tidak dapat menyimpan objek dalam string seperti sebelumnya,
karena terlalu besar. Jadi kita mendefinisikan objek dalam file Povray
menggunakan #declare. Fungsi povtriangle() melakukan ini secara
otomatis. Fungsi ini dapat menerima vektor normal seperti pov3d().

Berikut ini mendefinisikan objek mesh, dan langsung menuliskannya ke
dalam file.
\end{eulercomment}
\begin{eulerprompt}
>x=0:0.02:1; y=x'; z=x*y; vx=-y; vy=-x; vz=1;
>mesh=povtriangles(x,y,z,"",vx,vy,vz);
\end{eulerprompt}
\begin{eulercomment}
Sekarang kita definisikan dua cakram, yang akan berpotongan dengan
permukaan.
\end{eulercomment}
\begin{eulerprompt}
>cl=povdisc([0.5,0.5,0],[1,1,0],2); ...
>ll=povdisc([0,0,1/4],[0,0,1],2);
\end{eulerprompt}
\begin{eulercomment}
Tuliskan permukaan dikurangi kedua cakram.
\end{eulercomment}
\begin{eulerprompt}
>writeln(povdifference(mesh,povunion([cl,ll]),povlook(green)));
\end{eulerprompt}
\begin{eulercomment}
Tuliskan dua titik potongnya.
\end{eulercomment}
\begin{eulerprompt}
>writeln(povintersection([mesh,cl],povlook(red))); ...
>writeln(povintersection([mesh,ll],povlook(gray)));
\end{eulerprompt}
\begin{eulercomment}
Tuliskan titik maksimumnya.
\end{eulercomment}
\begin{eulerprompt}
>writeln(povpoint([1/2,1/2,1/4],povlook(gray),size=2*defaultpointsize));
\end{eulerprompt}
\begin{eulercomment}
Tambahkan sumbu dan selesaikan.
\end{eulercomment}
\begin{eulerprompt}
>writeAxes(0,1,0,1,0,1,d=0.015); ...
>povend();
\end{eulerprompt}
\eulerheading{Anaglif dalam Povray}
\begin{eulercomment}
Untuk menghasilkan anaglif untuk kacamata merah/sian, Povray harus
dijalankan dua kali dari posisi kamera yang berbeda. Ia menghasilkan
dua file Povray dan dua file PNG, yang dimuat dengan fungsi
loadanaglyph().

Tentu saja, Anda memerlukan kacamata merah/sian untuk melihat contoh
berikut dengan benar.

Fungsi pov3d() memiliki sakelar sederhana untuk menghasilkan anaglif.
\end{eulercomment}
\begin{eulerprompt}
>pov3d("-exp(-x^2-y^2)/2",r=2,height=45°,>anaglyph, ...
>  center=[0,0,0.5],zoom=3.5);
\end{eulerprompt}
\begin{eulercomment}
Jika Anda membuat suatu pemandangan dengan objek, Anda perlu
memasukkan pembuatan pemandangan tersebut ke dalam suatu fungsi, dan
menjalankannya dua kali dengan nilai yang berbeda untuk parameter
anaglyph.
\end{eulercomment}
\begin{eulerprompt}
>function myscene ...
\end{eulerprompt}
\begin{eulerudf}
    s=povsphere(povc,1);
    cl=povcylinder(-povz,povz,0.5);
    clx=povobject(cl,rotate=xrotate(90°));
    cly=povobject(cl,rotate=yrotate(90°));
    c=povbox([-1,-1,0],1);
    un=povunion([cl,clx,cly,c]);
    obj=povdifference(s,un,povlook(red));
    writeln(obj);
    writeAxes();
  endfunction
\end{eulerudf}
\begin{eulercomment}
Fungsi povanaglyph() melakukan semua ini. Parameternya seperti pada
povstart() dan povend() yang digabungkan.
\end{eulercomment}
\begin{eulerprompt}
>povanaglyph("myscene",zoom=4.5);
\end{eulerprompt}
\eulerheading{Menentukan Objek Sendiri}
\begin{eulercomment}
Antarmuka povray Euler berisi banyak objek. Namun, Anda tidak terbatas
pada objek-objek ini. Anda dapat membuat objek sendiri, yang
menggabungkan objek lain, atau objek yang sama sekali baru.

Kami mendemonstrasikan sebuah torus. Perintah Povray untuk ini adalah
"torus". Jadi, kami mengembalikan string dengan perintah ini dan
parameternya. Perhatikan bahwa torus selalu berpusat di titik asal.
\end{eulercomment}
\begin{eulerprompt}
>function povdonat (r1,r2,look="") ...
\end{eulerprompt}
\begin{eulerudf}
    return "torus \{"+r1+","+r2+look+"\}";
  endfunction
\end{eulerudf}
\begin{eulercomment}
Inilah torus pertama kita.
\end{eulercomment}
\begin{eulerprompt}
>t1=povdonat(0.8,0.2)
\end{eulerprompt}
\begin{euleroutput}
  torus \{0.8,0.2\}
\end{euleroutput}
\begin{eulercomment}
Mari kita gunakan objek ini untuk membuat torus kedua, diterjemahkan
dan diputar.
\end{eulercomment}
\begin{eulerprompt}
>t2=povobject(t1,rotate=xrotate(90°),translate=[0.8,0,0])
\end{eulerprompt}
\begin{euleroutput}
  object \{ torus \{0.8,0.2\}
   rotate 90 *x 
   translate <0.8,0,0>
   \}
\end{euleroutput}
\begin{eulercomment}
Sekarang kita tempatkan objek-objek ini ke dalam sebuah scene. Untuk
tampilannya, kita gunakan Phong Shading.
\end{eulercomment}
\begin{eulerprompt}
>povstart(center=[0.4,0,0],angle=0°,zoom=3.8,aspect=1.5); ...
>writeln(povobject(t1,povlook(green,phong=1))); ...
>writeln(povobject(t2,povlook(green,phong=1))); ...
\end{eulerprompt}
\begin{eulercomment}
\textgreater{}povend();

memanggil program Povray. Namun, jika terjadi kesalahan, program
tersebut tidak menampilkan kesalahan tersebut. Oleh karena itu, Anda
harus menggunakan

\textgreater{}povend(\textless{}exit);

jika ada yang tidak berhasil. Ini akan membiarkan jendela Povray
terbuka.
\end{eulercomment}
\begin{eulerprompt}
>povend(h=320,w=480);
\end{eulerprompt}
\begin{eulercomment}
Berikut adalah contoh yang lebih rinci. Kami memecahkan

\end{eulercomment}
\begin{eulerformula}
\[
Ax \le b, \quad x \ge 0, \quad c.x \to \text{Max.}
\]
\end{eulerformula}
\begin{eulercomment}
dan menunjukkan titik-titik yang layak dan titik-titik optimum dalam
plot 3D.
\end{eulercomment}
\begin{eulerprompt}
>A=[10,8,4;5,6,8;6,3,2;9,5,6];
>b=[10,10,10,10]';
>c=[1,1,1];
\end{eulerprompt}
\begin{eulercomment}
Pertama, mari kita periksa, apakah contoh ini punya solusi.
\end{eulercomment}
\begin{eulerprompt}
>x=simplex(A,b,c,>max,>check)'
\end{eulerprompt}
\begin{euleroutput}
  [0,  1,  0.5]
\end{euleroutput}
\begin{eulercomment}
Ya, benar.

Berikutnya kita mendefinisikan dua objek. Yang pertama adalah bidang
datar

\end{eulercomment}
\begin{eulerformula}
\[
a \cdot x \le b
\]
\end{eulerformula}
\begin{eulerprompt}
>function oneplane (a,b,look="") ...
\end{eulerprompt}
\begin{eulerudf}
    return povplane(a,b,look)
  endfunction
\end{eulerudf}
\begin{eulercomment}
Kemudian kita mendefinisikan irisan semua ruang setengah dan sebuah
kubus.
\end{eulercomment}
\begin{eulerprompt}
>function adm (A, b, r, look="") ...
\end{eulerprompt}
\begin{eulerudf}
    ol=[];
    loop 1 to rows(A); ol=ol|oneplane(A[#],b[#]); end;
    ol=ol|povbox([0,0,0],[r,r,r]);
    return povintersection(ol,look);
  endfunction
\end{eulerudf}
\begin{eulercomment}
Sekarang, kita dapat merencanakan adegannya.
\end{eulercomment}
\begin{eulerprompt}
>povstart(angle=120°,center=[0.5,0.5,0.5],zoom=3.5); ...
>writeln(adm(A,b,2,povlook(green,0.4))); ...
>writeAxes(0,1.3,0,1.6,0,1.5); ...
\end{eulerprompt}
\begin{eulercomment}
Berikut ini adalah lingkaran di sekitar titik optimum.
\end{eulercomment}
\begin{eulerprompt}
>writeln(povintersection([povsphere(x,0.5),povplane(c,c.x')], ...
>  povlook(red,0.9)));
\end{eulerprompt}
\begin{eulercomment}
Dan kesalahan dalam arah yang optimum.
\end{eulercomment}
\begin{eulerprompt}
>writeln(povarrow(x,c*0.5,povlook(red)));
\end{eulerprompt}
\begin{eulercomment}
Kita menambahkan teks ke layar. Teks hanyalah objek 3D. Kita perlu
menempatkan dan memutarnya sesuai dengan pandangan kita.
\end{eulercomment}
\begin{eulerprompt}
>writeln(povtext("Linear Problem",[0,0.2,1.3],size=0.05,rotate=5°)); ...
>povend();
\end{eulerprompt}
\eulerheading{Contoh Lainnya}
\begin{eulercomment}
Anda dapat menemukan beberapa contoh lainnya untuk Povray di Euler
dalam berkas berikut.

Lihat: Contoh/Bola Dandelin\\
Lihat: Contoh/Donat Math\\
Lihat: Contoh/Simpul Trefoil\\
Lihat: Contoh/Optimasi dengan Penskalaan Afinitas

\begin{eulercomment}
\eulerheading{Contoh Soal Diberikan fungsi kuadrat g(x):=x2-2x+3g(x):=x2-2x+3,}
\begin{eulercomment}
lakukan visualisasi grafik 3D dari fungsi tersebut dengan
langkah-langkah berikut:

1. Definisikan fungsi g(x) dan variabel x=-2:0.02:2 serta\\
\end{eulercomment}
\begin{eulerttcomment}
   u=linspace(0,2p,100)'.
\end{eulerttcomment}
\begin{eulercomment}
2. Buat grid 3D berdasarkan variabel:\\
\end{eulercomment}
\begin{eulerttcomment}
   Z=x,
   X=cos(u)·g(x),
   Y=sin(u)·g(x).
\end{eulerttcomment}
\begin{eulercomment}

Definisikan fungsi g(x)
\end{eulercomment}
\begin{eulerprompt}
>function g(x) := x^2 - 2*x + 3;
\end{eulerprompt}
\begin{eulercomment}
Rentan nilai x dan u
\end{eulercomment}
\begin{eulerprompt}
>x = -2:0.02:2;
>u = linspace(0, 2*pi, 100)';
\end{eulerprompt}
\begin{eulercomment}
Buat grid untuk X dan U
\end{eulercomment}
\begin{eulerprompt}
>Z = x;
>X = cos(u)*g(x);
>Y = sin(u)*g(x);
\end{eulerprompt}
\begin{eulercomment}
Visualisasi 3D
\end{eulercomment}
\begin{eulerprompt}
>pov3d(X, Y, Z, angle=30°, look=povlook(blue, 0.2), height=60°, axis=0, zoom=3, light=[15,10,20]);
\end{eulerprompt}
\begin{euleroutput}
  
\end{euleroutput}
\begin{eulercomment}
Nama  : Alifia Rahmawati\\
NIM   : 23030630044\\
Kelas : Matematiak E

\end{eulercomment}
\eulersubheading{}
\begin{eulercomment}
Kalkulus dengan EMT

Materi Kalkulus mencakup di antaranya:

- Fungsi (fungsi aljabar, trigonometri, eksponensial, logaritma,
komposisi fungsi)\\
- Limit Fungsi,\\
- Turunan Fungsi,\\
- Integral Tak Tentu,\\
- Integral Tentu dan Aplikasinya,\\
- Barisan dan Deret (kekonvergenan barisan dan deret).

EMT (bersama Maxima) dapat digunakan untuk melakukan semua perhitungan
di dalam kalkulus, baik secara numerik maupun analitik (eksak).

\end{eulercomment}
\eulersubheading{Mendefinisikan Fungsi}
\begin{eulercomment}
Terdapat beberapa cara mendefinisikan fungsi pada EMT, yakni:

- Menggunakan format nama\_fungsi := rumus fungsi (untuk fungsi
numerik),\\
- Menggunakan format nama\_fungsi \&= rumus fungsi (untuk fungsi
simbolik, namun dapat dihitung secara numerik),\\
- Menggunakan format nama\_fungsi \&\&= rumus fungsi (untuk fungsi
simbolik murni, tidak dapat dihitung langsung),\\
- Fungsi sebagai program EMT.

Setiap format harus diawali dengan perintah function (bukan sebagai
ekspresi).

Berikut adalah adalah beberapa contoh cara mendefinisikan fungsi:

\end{eulercomment}
\begin{eulerformula}
\[
f(x)=2x^2+e^{\sin(x)}.
\]
\end{eulerformula}
\begin{eulerprompt}
>function f(x) := 2*x^2+exp(sin(x)) // fungsi numerik
>f(0), f(1), f(pi)
\end{eulerprompt}
\begin{euleroutput}
  1
  4.31977682472
  20.7392088022
\end{euleroutput}
\begin{eulerprompt}
>f(a) // tidak dapat dihitung nilainya
\end{eulerprompt}
\begin{euleroutput}
  Variable or function a not found.
  Error in:
  f(a) // tidak dapat dihitung nilainya ...
     ^
\end{euleroutput}
\begin{eulercomment}
Silakan Anda plot kurva fungsi di atas!
\end{eulercomment}
\begin{eulerprompt}
>plot2d("2*x^2+exp(sin(x))"):
\end{eulerprompt}
\begin{eulercomment}
Berikutnya kita definisikan fungsi:

\end{eulercomment}
\begin{eulerformula}
\[
g(x)=\frac{\sqrt{x^2-3x}}{x+1}.
\]
\end{eulerformula}
\begin{eulerprompt}
>function g(x) := sqrt(x^2-3*x)/(x+1)
>g(3)
\end{eulerprompt}
\begin{euleroutput}
  0
\end{euleroutput}
\begin{eulerprompt}
>g(0)
\end{eulerprompt}
\begin{euleroutput}
  0
\end{euleroutput}
\begin{eulerprompt}
>g(1) // kompleks, tidak dapat dihitung oleh fungsi numerik
\end{eulerprompt}
\begin{euleroutput}
  Floating point error!
  Error in sqrt
  Try "trace errors" to inspect local variables after errors.
  g:
      useglobal; return sqrt(x^2-3*x)/(x+1) 
  Error in:
  g(1) // kompleks, tidak dapat dihitung oleh fungsi numerik ...
      ^
\end{euleroutput}
\begin{eulercomment}
Silakan Anda plot kurva fungsi di atas!
\end{eulercomment}
\begin{eulerprompt}
>plot2d("sqrt(x^2-3*x)/(x+1)"):
>f(g(5)) // komposisi fungsi
\end{eulerprompt}
\begin{euleroutput}
  2.20920171961
\end{euleroutput}
\begin{eulerprompt}
> g(f(5))
\end{eulerprompt}
\begin{euleroutput}
  0.950898070639
\end{euleroutput}
\begin{eulerprompt}
>function h(x) := f(g(x)) // definisi komposisi fungsi 
>h(5) // sama dengan f(g(5))
\end{eulerprompt}
\begin{euleroutput}
  2.20920171961
\end{euleroutput}
\begin{eulercomment}
Silakan Anda plot kurva fungsi komposisi fungsi f dan g:

\end{eulercomment}
\begin{eulerformula}
\[
h(x)=f(g(x))
\]
\end{eulerformula}
\begin{eulercomment}
dan

\end{eulercomment}
\begin{eulerformula}
\[
u(x)=g(f(x))
\]
\end{eulerformula}
\begin{eulercomment}
bersama-sama kurva fungsi f dan g dalam satu bidang koordinat.
\end{eulercomment}
\begin{eulerprompt}
>f(0:10) // nilai-nilai f(0), f(1), f(2), ..., f(10)
\end{eulerprompt}
\begin{euleroutput}
  [1,  4.31978,  10.4826,  19.1516,  32.4692,  50.3833,  72.7562,
  99.929,  130.69,  163.51,  200.58]
\end{euleroutput}
\begin{eulerprompt}
>fmap(0:10) // sama dengan f(0:10), berlaku untuk semua fungsi
\end{eulerprompt}
\begin{euleroutput}
  [1,  4.31978,  10.4826,  19.1516,  32.4692,  50.3833,  72.7562,
  99.929,  130.69,  163.51,  200.58]
\end{euleroutput}
\begin{eulerprompt}
>gmap(200:210)
\end{eulerprompt}
\begin{euleroutput}
  [0.987534,  0.987596,  0.987657,  0.987718,  0.987778,  0.987837,
  0.987896,  0.987954,  0.988012,  0.988069,  0.988126]
\end{euleroutput}
\begin{eulercomment}
Misalkan kita akan mendefinisikan fungsi

\end{eulercomment}
\begin{eulerformula}
\[
f(x) = \begin{cases} x^3 & x>0 \\ x^2 & x\le 0. \end{cases}
\]
\end{eulerformula}
\begin{eulercomment}
Fungsi tersebut tidak dapat didefinisikan sebagai fungsi numerik
secara "inline" menggunakan format :=, melainkan didefinisikan sebagai
program. Perhatikan, kata "map" digunakan agar fungsi dapat menerima
vektor sebagai input, dan hasilnya berupa vektor. Jika tanpa kata
"map" fungsinya hanya dapat menerima input satu nilai.
\end{eulercomment}
\begin{eulerprompt}
>function map f(x) ...
\end{eulerprompt}
\begin{eulerudf}
    if x>0 then return x^3
    else return x^2
    endif;
  endfunction
\end{eulerudf}
\begin{eulerprompt}
>f(1)
\end{eulerprompt}
\begin{euleroutput}
  1
\end{euleroutput}
\begin{eulerprompt}
>f(-2)
\end{eulerprompt}
\begin{euleroutput}
  4
\end{euleroutput}
\begin{eulerprompt}
>f(-5:5)
\end{eulerprompt}
\begin{euleroutput}
  [25,  16,  9,  4,  1,  0,  1,  8,  27,  64,  125]
\end{euleroutput}
\begin{eulerprompt}
>aspect(1.5); plot2d("f(x)",-5,5):
>function f(x) &= 2*E^x // fungsi simbolik
\end{eulerprompt}
\begin{euleroutput}
  
                                      x
                                   2 E
  
\end{euleroutput}
\begin{eulerprompt}
>$f(a) // nilai fungsi secara simbolik
>f(E) // nilai fungsi berupa bilangan desimal
\end{eulerprompt}
\begin{euleroutput}
  30.308524483
\end{euleroutput}
\begin{eulerprompt}
>$f(E), $float(%)
>function g(x) &= 3*x+1
\end{eulerprompt}
\begin{euleroutput}
  
                                 3 x + 1
  
\end{euleroutput}
\begin{eulerprompt}
>function h(x) &= f(g(x)) // komposisi fungsi
\end{eulerprompt}
\begin{euleroutput}
  
                                   3 x + 1
                                2 E
  
\end{euleroutput}
\begin{eulerprompt}
>plot2d("h(x)",-1,1):
\end{eulerprompt}
\eulerheading{Latihan}
\begin{eulercomment}
Bukalah buku Kalkulus. Cari dan pilih beberapa (paling sedikit 5
fungsi berbeda tipe/bentuk/jenis) fungsi dari buku tersebut, kemudian
definisikan fungsi-fungsi tersebut dan komposisinya di EMT pada
baris-baris perintah berikut (jika perlu tambahkan lagi). Untuk setiap
fungsi, hitung beberapa nilainya, baik untuk satu nilai maupun vektor.
Gambar grafik fungsi-fungsi tersebut dan komposisi-komposisi 2 fungsi.

Juga, carilah fungsi beberapa (dua) variabel. Lakukan hal sama seperti
di atas.


1. Misalkan kita punya fungsi aljabar:\\
\end{eulercomment}
\begin{eulerformula}
\[
g(x)=\frac{\sqrt{x^2-5x}}{2x-3}
\]
\end{eulerformula}
\begin{eulercomment}
cari g(0), g(5) dan buatkan plot 2d
\end{eulercomment}
\begin{eulerprompt}
>function g(x) := sqrt(x^2-5x)/(2x-3)
>g(0), g(5)
\end{eulerprompt}
\begin{euleroutput}
  0
  0
\end{euleroutput}
\begin{eulerprompt}
>plot2d("sqrt(x^2-5x)/(2x-3)"):
\end{eulerprompt}
\begin{eulercomment}
2.

\end{eulercomment}
\begin{eulerformula}
\[
g(x)=\sqrt{x^2+36}
\]
\end{eulerformula}
\begin{eulerprompt}
>function g(x) := (sqrt(x^2+36)) // fungsi numerik
>g(6)
\end{eulerprompt}
\begin{euleroutput}
  8.48528137424
\end{euleroutput}
\begin{eulerprompt}
>g=(-3:3)
\end{eulerprompt}
\begin{euleroutput}
  [-3,  -2,  -1,  0,  1,  2,  3]
\end{euleroutput}
\begin{eulerprompt}
>aspect(3); plot2d("g(x)",-3,3):
\end{eulerprompt}
\begin{eulercomment}
3.\\
\end{eulercomment}
\begin{eulerformula}
\[
d(x)= \frac{x^2 + 6}{2x}
\]
\end{eulerformula}
\begin{eulerprompt}
>function d(x) := ((x^2+6)/(2*x)) // fungsi numerik
>d(2)
\end{eulerprompt}
\begin{euleroutput}
  2.5
\end{euleroutput}
\begin{eulerprompt}
>d=(-2:2)
\end{eulerprompt}
\begin{euleroutput}
  [-2,  -1,  0,  1,  2]
\end{euleroutput}
\begin{eulerprompt}
> aspect(3); plot2d("d(x)",-2,2):
\end{eulerprompt}
\begin{eulercomment}
4.\\
\end{eulercomment}
\begin{eulerformula}
\[
f(x)= \cos{x}
\]
\end{eulerformula}
\begin{eulerformula}
\[
g(x)= \sin{x}
\]
\end{eulerformula}
\begin{eulerprompt}
>function f(x) &= (cos(x)) // fungsi numerik
\end{eulerprompt}
\begin{euleroutput}
  
                                  cos(x)
  
\end{euleroutput}
\begin{eulerprompt}
>f(pi)
\end{eulerprompt}
\begin{euleroutput}
  -1
\end{euleroutput}
\begin{eulerprompt}
>f(3*pi)
\end{eulerprompt}
\begin{euleroutput}
  -1
\end{euleroutput}
\begin{eulerprompt}
>function g(x) &= (sin(x)) // fungsi numerik
\end{eulerprompt}
\begin{euleroutput}
  
                                  sin(x)
  
\end{euleroutput}
\begin{eulerprompt}
>g(pi)
\end{eulerprompt}
\begin{euleroutput}
  0
\end{euleroutput}
\begin{eulerprompt}
>g(2*pi)
\end{eulerprompt}
\begin{euleroutput}
  0
\end{euleroutput}
\begin{eulerprompt}
>f(g(pi)) // komposisi fungsi
\end{eulerprompt}
\begin{euleroutput}
  1
\end{euleroutput}
\begin{eulerprompt}
>g(f(pi))
\end{eulerprompt}
\begin{euleroutput}
  -0.841470984808
\end{euleroutput}
\begin{eulerprompt}
>function h(x) &= f(g(pi))
\end{eulerprompt}
\begin{euleroutput}
  
                                    1
  
\end{euleroutput}
\begin{eulerprompt}
>plot2d("h(x)",-4,4):
\end{eulerprompt}
\begin{eulercomment}
5.\\
\end{eulercomment}
\begin{eulerformula}
\[
e(x) = \begin{cases} x-3 & x<2 \\ 1-x & x\ge2. \end{cases}
\]
\end{eulerformula}
\begin{eulerprompt}
>function map e(x) ...
\end{eulerprompt}
\begin{eulerudf}
  if x<2 then return x-3
  else return 1-x
  endif;e(5)
  endfunction
\end{eulerudf}
\begin{eulerprompt}
>e(5)
\end{eulerprompt}
\begin{euleroutput}
  -4
\end{euleroutput}
\begin{eulerprompt}
>e=(-3:3)
\end{eulerprompt}
\begin{euleroutput}
  [-3,  -2,  -1,  0,  1,  2,  3]
\end{euleroutput}
\begin{eulerprompt}
>aspect(2); plot2d("e(x)",-3,3):
\end{eulerprompt}
\begin{eulercomment}
\begin{eulercomment}
\eulerheading{Menghitung Limit}
\begin{eulercomment}
Perhitungan limit pada EMT dapat dilakukan dengan menggunakan fungsi
Maxima, yakni "limit". Fungsi "limit" dapat digunakan untuk menghitung
limit fungsi dalam bentuk ekspresi maupun fungsi yang sudah
didefinisikan sebelumnya. Nilai limit dapat dihitung pada sebarang
nilai atau pada tak hingga (-inf, minf, dan inf). Limit kiri dan limit
kanan juga dapat dihitung, dengan cara memberi opsi "plus" atau
"minus". Hasil limit dapat berupa nilai, "und" (tak definisi), "ind"
(tak tentu namun terbatas), "infinity" (kompleks tak hingga).

Perhatikan beberapa contoh berikut. Perhatikan cara menampilkan
perhitungan secara lengkap, tidak hanya menampilkan hasilnya saja.
\end{eulercomment}
\begin{eulerprompt}
>$showev('limit(sqrt(x^2-3*x)/(x+1),x,inf))
>$limit((x^3-13*x^2+51*x-63)/(x^3-4*x^2-3*x+18),x,3)
\end{eulerprompt}
\begin{eulercomment}
maxima: 'limit((x\textasciicircum{}3-13*x\textasciicircum{}2+51*x-63)/(x\textasciicircum{}3-4*x\textasciicircum{}2-3*x+18),x,3)=limit((x\textasciicircum{}3-13*x\textasciicircum{}2+51*x-63)/(x\textasciicircum{}3-4*x\textasciicircum{}2-3*x+18),x,3)

Fungsi tersebut diskontinu di titik x=3. Berikut adalah grafik
fungsinya.
\end{eulercomment}
\begin{eulerprompt}
>aspect(1.5); plot2d("(x^3-13*x^2+51*x-63)/(x^3-4*x^2-3*x+18)",0,4); plot2d(3,-4/5,>points,style="ow",>add):
>$limit(2*x*sin(x)/(1-cos(x)),x,0)
\end{eulerprompt}
\begin{eulercomment}
maxima: 'limit(2*x*sin(x)/(1-cos(x)),x,0)=limit(2*x*sin(x)/(1-cos(x)),x,0)

Fungsi tersebut diskontinu di titik x=0. Berikut adalah grafik
fungsinya.
\end{eulercomment}
\begin{eulerprompt}
>plot2d("2*x*sin(x)/(1-cos(x))",-pi,pi); plot2d(0,4,>points,style="ow",>add):
>$limit(cot(7*h)/cot(5*h),h,0)
\end{eulerprompt}
\begin{eulercomment}
maxima: showev('limit(cot(7*h)/cot(5*h),h,0))

Fungsi tersebut juga diskontinu (karena tidak terdefinisi) di x=0.
Berikut adalah grafiknya.
\end{eulercomment}
\begin{eulerprompt}
>plot2d("cot(7*x)/cot(5*x)",-0.001,0.001); plot2d(0,5/7,>points,style="ow",>add):
>$showev('limit(((x/8)^(1/3)-1)/(x-8),x,8))
\end{eulerprompt}
\begin{eulercomment}
Tunjukkan limit tersebut dengan grafik, seperti contoh-contoh
sebelumnya.
\end{eulercomment}
\begin{eulerprompt}
>plot2d("(x/8)^(1/3)-1/(x-8)",-5,5); plot2d(8,1/24,>points,style="ow",>add):
>$showev('limit(1/(2*x-1),x,0))
\end{eulerprompt}
\begin{eulercomment}
Tunjukkan limit tersebut dengan grafik, seperti contoh-contoh sebelumnya.
\end{eulercomment}
\begin{eulerprompt}
>plot2d("1/(2*x-1)",-5,5); plot2d(0,-1,>points,style="ow",>add):
>$showev('limit((x^2-3*x-10)/(x-5),x,5))
\end{eulerprompt}
\begin{eulercomment}
Tunjukkan limit tersebut dengan grafik, seperti contoh-contoh sebelumnya.
\end{eulercomment}
\begin{eulerprompt}
>plot2d("(x^2-3*x-10)/(x-5)",0,10); plot2d(5,7,>points,style="ow",>add):
>$showev('limit(sqrt(x^2+x)-x,x,inf))
\end{eulerprompt}
\begin{eulercomment}
Tunjukkan limit tersebut dengan grafik, seperti contoh-contoh sebelumnya.
\end{eulercomment}
\begin{eulerprompt}
>plot2d("sqrt(x^2+x)-x",-3,7); plot2d(0,1/2,>points,style="ow",>add):
>$showev('limit(abs(x-1)/(x-1),x,1,minus))
\end{eulerprompt}
\begin{eulercomment}
Hitung limit di atas untuk x menuju 1 dari kanan.\\
Tunjukkan limit tersebut dengan grafik, seperti contoh-contoh sebelumnya.
\end{eulercomment}
\begin{eulerprompt}
>plot2d("abs(x-1)/(x-1)",-1,5); plot2d(1,-1,>points,style="ow",>add):
>$showev('limit(sin(x)/x,x,0))
>plot2d("sin(x)/x",-pi,pi); plot2d(0,1,>points,style="ow",>add):
>$showev('limit(sin(x^3)/x,x,0))
\end{eulerprompt}
\begin{eulercomment}
Tunjukkan limit tersebut dengan grafik, seperti contoh-contoh sebelumnya.
\end{eulercomment}
\begin{eulerprompt}
>plot2d("sin(x^3)/x",-pi,pi); plot2d(0,0,>points,style="ow",>add):
>$showev('limit(log(x), x, minf))
>$showev('limit((-2)^x,x, inf))
>$showev('limit(t-sqrt(2-t),t,2,minus))
>$showev('limit(t-sqrt(2-t),t,2,plus))
>$showev('limit(t-sqrt(2-t),t,5,plus)) // Perhatikan hasilnya
>plot2d("x-sqrt(2-x)",0,2):
>$showev('limit((x^2-9)/(2*x^2-5*x-3),x,3))
\end{eulerprompt}
\begin{eulercomment}
Tunjukkan limit tersebut dengan grafik, seperti contoh-contoh sebelumnya.
\end{eulercomment}
\begin{eulerprompt}
>plot2d("(x^2-9)/(2*x^2-5*x-3)",0,7):
>$showev('limit((1-cos(x))/x,x,0))
\end{eulerprompt}
\begin{eulercomment}
Tunjukkan limit tersebut dengan grafik, seperti contoh-contoh sebelumnya.
\end{eulercomment}
\begin{eulerprompt}
>plot2d("((1-cos(x))/x)",-1,7):
>$showev('limit((x^2+abs(x))/(x^2-abs(x)),x,0))
\end{eulerprompt}
\begin{eulercomment}
Tunjukkan limit tersebut dengan grafik, seperti contoh-contoh sebelumnya.
\end{eulercomment}
\begin{eulerprompt}
>plot2d("(x^2+abs(x))/(x^2-abs(x))",-1,6):
>$showev('limit((1+1/x)^x,x,inf))
>plot2d("(1+1/x)^x",0,1000):
>$showev('limit((1+k/x)^x,x,inf))
>$showev('limit((1+x)^(1/x),x,0))
\end{eulerprompt}
\begin{eulercomment}
Tunjukkan limit tersebut dengan grafik, seperti contoh-contoh sebelumnya.
\end{eulercomment}
\begin{eulerprompt}
>plot2d("(1+x)^(1/x)^x",0,10):
>$showev('limit((x/(x+k))^x,x,inf))
>$showev('limit((E^x-E^2)/(x-2),x,2))
\end{eulerprompt}
\begin{eulercomment}
Tunjukkan limit tersebut dengan grafik, seperti contoh-contoh sebelumnya.
\end{eulercomment}
\begin{eulerprompt}
>plot2d("(E^x-E^2)/(x-2)",0,10):
>$showev('limit(sin(1/x),x,0))
>$showev('limit(sin(1/x),x,inf))
>plot2d("sin(1/x)",-1,1):
\end{eulerprompt}
\eulerheading{Latihan}
\begin{eulercomment}
Bukalah buku Kalkulus. Cari dan pilih beberapa (paling sedikit 5
fungsi berbeda tipe/bentuk/jenis) fungsi dari buku tersebut, kemudian
definisikan di EMT pada baris-baris perintah berikut (jika perlu
tambahkan lagi). Untuk setiap fungsi, hitung nilai limit fungsi
tersebut di beberapa nilai dan di tak hingga. Gambar grafik fungsi
tersebut untuk mengkonfirmasi nilai-nilai limit tersebut.

1.
\end{eulercomment}
\begin{eulerprompt}
>$showev('limit((x^2+2*x+6),x,3))
>$showev('limit((x^2+2*x+6),x,2))
>$showev('limit((x^2+2*x+6),x,inf))
>plot2d("(x^2+2*x+6)",-6,6); plot2d(3,21,>points,style="ow",>add):
\end{eulerprompt}
\begin{eulercomment}
2.
\end{eulercomment}
\begin{eulerprompt}
>$showev('limit((x^4 + 2*x^3 - x^2)/(x^2),x,0))
>$showev('limit((x^4 + 2*x^3 - x^2)/(x^2),x,2))
>$showev('limit((x^4 + 2*x^3 - x^2)/(x^2),x,inf))
>plot2d("(x^4 + 2*x^3 - x^2)",-2,2); plot2d(0,0,>points,style="ow",>add):
\end{eulerprompt}
\begin{eulercomment}
3.
\end{eulercomment}
\begin{eulerprompt}
>$showev('limit(sqrt(x+16),x,inf))
>$showev('limit(sqrt(x+16),x,0))
>$showev('limit(sqrt(x+16),x,4))
>plot2d("sqrt(x+16)",-7,7); plot2d(0,4,>points,style="ow",>add):
\end{eulerprompt}
\begin{eulercomment}
4.
\end{eulercomment}
\begin{eulerprompt}
>$showev('limit(((x^2)+(2*x)),x,0,plus))
>$showev('limit(((x^2)+(2*x)),x,2,minus))
>$showev('limit(((x^2)+(2*x)),x,inf))
>plot2d("((x^2)+(2*x))",0,8):
\end{eulerprompt}
\begin{eulercomment}
5.
\end{eulercomment}
\begin{eulerprompt}
>$showev('limit((1-cos(x))/x^2,x,0))
>$showev('limit((1-cos(x))/x^2,x,inf))
>$showev('limit((1-cos(x))/x^2,x,2))
>plot2d("(1-cos(x))/x^2",-3,3):
\end{eulerprompt}
\begin{eulercomment}
\begin{eulercomment}
\eulerheading{Turunan Fungsi}
\begin{eulercomment}
Definisi turunan:

\end{eulercomment}
\begin{eulerformula}
\[
f'(x) = \lim_{h\to 0} \frac{f(x+h)-f(x)}{h}
\]
\end{eulerformula}
\begin{eulercomment}
Berikut adalah contoh-contoh menentukan turunan fungsi dengan
menggunakan definisi turunan (limit).
\end{eulercomment}
\begin{eulerprompt}
>$showev('limit(((x+h)^2-x^2)/h,h,0)) // turunan x^2
>p &= expand((x+h)^2-x^2)|simplify; $p //pembilang dijabarkan dan disederhanakan
>q &=ratsimp(p/h); $q // ekspresi yang akan dihitung limitnya disederhanakan
>$limit(q,h,0) // nilai limit sebagai turunan
>$showev('limit(((x+h)^n-x^n)/h,h,0)) // turunan x^n
\end{eulerprompt}
\begin{eulercomment}
Mengapa hasilnya seperti itu? Tuliskan atau tunjukkan bahwa hasil limit tersebut
benar, sehingga benar turunan fungsinya benar.  Tulis penjelasan Anda di komentar
ini.

Sebagai petunjuk, ekspansikan (x+h)\textasciicircum{}n dengan menggunakan teorema binomial.
\end{eulercomment}
\begin{eulerprompt}
>$showev('limit((sin(x+h)-sin(x))/h,h,0)) // turunan sin(x)
\end{eulerprompt}
\begin{eulercomment}
Mengapa hasilnya seperti itu? Tuliskan atau tunjukkan bahwa hasil
limit tersebut\\
benar, sehingga benar turunan fungsinya benar.  Tulis penjelasan Anda
di komentar ini.

Sebagai petunjuk, ekspansikan sin(x+h) dengan menggunakan rumus jumlah
dua sudut.
\end{eulercomment}
\begin{eulerprompt}
>$showev('limit((log(x+h)-log(x))/h,h,0)) // turunan log(x)
\end{eulerprompt}
\begin{eulercomment}
Mengapa hasilnya seperti itu? Tuliskan atau tunjukkan bahwa hasil
limit tersebut\\
benar, sehingga benar turunan fungsinya benar.  Tulis penjelasan Anda
di komentar ini.

Sebagai petunjuk, gunakan sifat-sifat logaritma dan hasil limit pada
bagian sebelumnya di atas.
\end{eulercomment}
\begin{eulerprompt}
>$showev('limit((1/(x+h)-1/x)/h,h,0)) // turunan 1/x
>$showev('limit((E^(x+h)-E^x)/h,h,0)) // turunan f(x)=e^x
\end{eulerprompt}
\begin{euleroutput}
  Answering "Is x an integer?" with "integer"
  Answering "Is x an integer?" with "integer"
  Answering "Is x an integer?" with "integer"
  Answering "Is x an integer?" with "integer"
  Answering "Is x an integer?" with "integer"
  Maxima is asking
  Acceptable answers are: yes, y, Y, no, n, N, unknown, uk
  Is x an integer?
  
  Use assume!
  Error in:
                                       ^
\end{euleroutput}
\begin{eulercomment}
Maxima bermasalah dengan limit:

\end{eulercomment}
\begin{eulerformula}
\[
\lim_{h\to 0}\frac{e^{x+h}-e^x}{h}.
\]
\end{eulerformula}
\begin{eulercomment}
Oleh karena itu diperlukan trik khusus agar hasilnya benar.
\end{eulercomment}
\begin{eulerprompt}
>$showev('limit((E^h-1)/h,h,0))
>$showev('factor(E^(x+h)-E^x))
>$showev('limit(factor((E^(x+h)-E^x)/h),h,0)) // turunan f(x)=e^x
>function f(x) &= x^x
\end{eulerprompt}
\begin{euleroutput}
  
                                     x
                                    x
  
\end{euleroutput}
\begin{eulerprompt}
>$showev('limit(f(x),x,0))
\end{eulerprompt}
\begin{eulercomment}
Silakan Anda gambar kurva

\end{eulercomment}
\begin{eulerformula}
\[
y=x^x.
\]
\end{eulerformula}
\begin{eulerprompt}
>plot2d("x^x",-2,2):
>$showev('limit((f(x+h)-f(x))/h,h,0)) // turunan f(x)=x^x
\end{eulerprompt}
\begin{eulercomment}
Di sini Maxima juga bermasalah terkait limit:

\end{eulercomment}
\begin{eulerformula}
\[
\lim_{h\to 0} \frac{(x+h)^{x+h}-x^x}{h}.
\]
\end{eulerformula}
\begin{eulercomment}
Dalam hal ini diperlukan asumsi nilai x.
\end{eulercomment}
\begin{eulerprompt}
>&assume(x>0); $showev('limit((f(x+h)-f(x))/h,h,0)) // turunan f(x)=x^x
\end{eulerprompt}
\begin{eulercomment}
Mengapa hasilnya seperti itu? Tuliskan atau tunjukkan bahwa hasil
limit tersebut benar, sehingga benar turunan fungsinya benar. Tulis
penjelasan Anda di komentar ini.

Fungsi\\
\end{eulercomment}
\begin{eulerformula}
\[
x^x = e^{x log (x)}
\]
\end{eulerformula}
\begin{eulercomment}
yang membantu dalam proses turunan menggunakan aturan rantai:\\
\end{eulercomment}
\begin{eulerformula}
\[
f(x)=e^{x log (x)}
\]
\end{eulerformula}
\begin{eulercomment}
kemudian, menggunakan aturan rantai untuk turunan:\\
\end{eulercomment}
\begin{eulerformula}
\[
f'(x)=e^{x log (x)}.\frac{d}{dx}(x log(x))=x^x.(log(x)+1)
\]
\end{eulerformula}
\begin{eulercomment}
Dengan demikian, hasil akhirnya adalah\\
\end{eulercomment}
\begin{eulerformula}
\[
f'(x)=x^x.(log(x)+1)
\]
\end{eulerformula}
\begin{eulercomment}
yang mengkonfirmasi bahwa limit yang ada pada gambar tersebut benar
setelah asumsi x \textgreater{} 0 ditetapkan.
\end{eulercomment}
\begin{eulerprompt}
>&forget(x>0) // jangan lupa, lupakan asumsi untuk kembali ke semula
\end{eulerprompt}
\begin{euleroutput}
  
                                 [x > 0]
  
\end{euleroutput}
\begin{eulerprompt}
>&forget(x<0)
\end{eulerprompt}
\begin{euleroutput}
  
                                 [x < 0]
  
\end{euleroutput}
\begin{eulerprompt}
>&facts()
\end{eulerprompt}
\begin{euleroutput}
  
                                    []
  
\end{euleroutput}
\begin{eulerprompt}
>$showev('limit((asin(x+h)-asin(x))/h,h,0)) // turunan arcsin(x)
\end{eulerprompt}
\begin{eulercomment}
Mengapa hasilnya seperti itu? Tuliskan atau tunjukkan bahwa hasil limit tersebut benar, sehingga
benar turunan fungsinya benar. Tulis penjelasan Anda di komentar ini.
\end{eulercomment}
\begin{eulerprompt}
>$showev('limit((tan(x+h)-tan(x))/h,h,0)) // turunan tan(x)
\end{eulerprompt}
\begin{eulercomment}
Mengapa hasilnya seperti itu? Tuliskan atau tunjukkan bahwa hasil limit tersebut benar, sehingga
benar turunan fungsinya benar. Tulis penjelasan Anda di komentar ini.
\end{eulercomment}
\begin{eulerprompt}
>function f(x) &= sinh(x) // definisikan f(x)=sinh(x)
\end{eulerprompt}
\begin{euleroutput}
  
                                 sinh(x)
  
\end{euleroutput}
\begin{eulerprompt}
>function df(x) &= limit((f(x+h)-f(x))/h,h,0); $df(x) // df(x) = f'(x)
\end{eulerprompt}
\begin{eulercomment}
Hasilnya adalah cosh(x), karena

\end{eulercomment}
\begin{eulerformula}
\[
\frac{e^x+e^{-x}}{2}=\cosh(x).
\]
\end{eulerformula}
\begin{eulerprompt}
>plot2d(["f(x)","df(x)"],-pi,pi,color=[blue,red]):
>function f(x) &= sin(3*x^5+7)^2
\end{eulerprompt}
\begin{euleroutput}
  
                                 2    5
                              sin (3 x  + 7)
  
\end{euleroutput}
\begin{eulerprompt}
>diff(f,3), diffc(f,3)
\end{eulerprompt}
\begin{euleroutput}
  1198.32948904
  1198.72863721
\end{euleroutput}
\begin{eulercomment}
Apakah perbedaan diff dan diffc?

diff digunakan untuk turunan biasa, sedangkan diffc digunakan untuk
turunan total dalam konteks yang lebih kompleks.
\end{eulercomment}
\begin{eulerprompt}
>$showev('diff(f(x),x))
>$% with x=3
>$float(%)
>plot2d(f,0,3.1):
>function f(x) &=5*cos(2*x)-2*x*sin(2*x) // mendifinisikan fungsi f
\end{eulerprompt}
\begin{euleroutput}
  
                        5 cos(2 x) - 2 x sin(2 x)
  
\end{euleroutput}
\begin{eulerprompt}
>function df(x) &=diff(f(x),x) // fd(x) = f'(x)
\end{eulerprompt}
\begin{euleroutput}
  
                       - 12 sin(2 x) - 4 x cos(2 x)
  
\end{euleroutput}
\begin{eulerprompt}
>$'f(1)=f(1), $float(f(1)), $'f(2)=f(2), $float(f(2)) // nilai f(1) dan f(2)
>xp=solve("df(x)",1,2,0) // solusi f'(x)=0 pada interval [1, 2]
\end{eulerprompt}
\begin{euleroutput}
  1.35822987384
\end{euleroutput}
\begin{eulerprompt}
>df(xp), f(xp) // cek bahwa f'(xp)=0 dan nilai ekstrim di titik tersebut
\end{eulerprompt}
\begin{euleroutput}
  0
  -5.67530133759
\end{euleroutput}
\begin{eulerprompt}
>plot2d(["f(x)","df(x)"],0,2*pi,color=[blue,red]): //grafik fungsi dan turunannya
\end{eulerprompt}
\begin{eulercomment}
Perhatikan titik-titik "puncak" grafik y=f(x) dan nilai turunan pada
saat grafik fungsinya mencapai titik "puncak" tersebut.
\end{eulercomment}
\eulerheading{Latihan}
\begin{eulercomment}
Bukalah buku Kalkulus. Cari dan pilih beberapa (paling sedikit 5
fungsi berbeda tipe/bentuk/jenis) fungsi dari buku tersebut, kemudian
definisikan di EMT pada baris-baris perintah berikut (jika perlu
tambahkan lagi). Untuk setiap fungsi, tentukan turunannya dengan
menggunakan definisi turunan (limit), menggunakan perintah diff, dan
secara manual (langkah demi langkah yang dihitung dengan Maxima)
seperti contoh-contoh di atas. Gambar grafik fungsi asli dan fungsi
turunannya pada sumbu koordinat yang sama.

Nomor 1
\end{eulercomment}
\begin{eulerprompt}
>function f(x) := cos(x^2)
>$showev('limit((cos((x+h)^2)- cos(x^2))/h,h,0))
>function df(x) &=diff(cos(x^2),x) // fd(x) = f'(x)
\end{eulerprompt}
\begin{euleroutput}
  
                                         2
                              - 2 x sin(x )
  
\end{euleroutput}
\begin{eulerprompt}
>plot2d(["f(x)","df(x)"],-pi,pi,color=[blue,red]):
\end{eulerprompt}
\begin{eulercomment}
Nomor 2
\end{eulercomment}
\begin{eulerprompt}
>function f(x) := sqrt(x^2+4)
>$showev('limit((sqrt((x+h)^2+4)-sqrt(x^2+4))/h,h,0))
\end{eulerprompt}
\begin{eulercomment}
Nomor 3
\end{eulercomment}
\begin{eulerprompt}
>function f(x) :=(3-x)^3
>$showev('limit(((3-(x+h))^3-(3-x)^3)/h,h,0))
\end{eulerprompt}
\begin{eulercomment}
Nomor 4
\end{eulercomment}
\begin{eulerprompt}
>function f(x) :=sin(x)+2*cos(x)
>$showev('limit((sin(x+h)+2*cos(x+h)-(sin(x)+2*cos(x)))/h,h,0))
\end{eulerprompt}
\begin{eulercomment}
Nomor 5
\end{eulercomment}
\begin{eulerprompt}
>function f(x) :=10*x-2
>$showev('limit(((10*(x+h)-2)-(10*x-2))/h,h,0))
\end{eulerprompt}
\eulerheading{Integral}
\begin{eulercomment}
EMT dapat digunakan untuk menghitung integral, baik integral tak tentu
maupun integral tentu. Untuk integral tak tentu (simbolik) sudah tentu
EMT menggunakan Maxima, sedangkan untuk perhitungan integral tentu EMT
sudah menyediakan beberapa fungsi yang mengimplementasikan algoritma
kuadratur (perhitungan integral tentu menggunakan metode numerik).

Pada notebook ini akan ditunjukkan perhitungan integral tentu dengan
menggunakan Teorema Dasar Kalkulus:

\end{eulercomment}
\begin{eulerformula}
\[
\int_a^b f(x)\ dx = F(b)-F(a), \quad \text{ dengan  } F'(x) = f(x).
\]
\end{eulerformula}
\begin{eulercomment}
Fungsi untuk menentukan integral adalah integrate. Fungsi ini dapat
digunakan untuk menentukan, baik integral tentu maupun tak tentu (jika
fungsinya memiliki antiderivatif). Untuk perhitungan integral tentu
fungsi integrate menggunakan metode numerik (kecuali fungsinya tidak
integrabel, kita tidak akan menggunakan metode ini).
\end{eulercomment}
\begin{eulerprompt}
>$showev('integrate(x^n,x))
\end{eulerprompt}
\begin{euleroutput}
  Answering "Is n equal to -1?" with "no"
\end{euleroutput}
\begin{eulerprompt}
>$showev('integrate(1/(1+x),x))
>$showev('integrate(1/(1+x^2),x))
>$showev('integrate(1/sqrt(1-x^2),x))
>$showev('integrate(sin(x),x,0,pi))
>plot2d("sin(x)",0,2*pi):
>$showev('integrate(sin(x),x,a,b))
>$showev('integrate(x^n,x,a,b))
\end{eulerprompt}
\begin{euleroutput}
  Answering "Is n positive, negative or zero?" with "positive"
\end{euleroutput}
\begin{eulerprompt}
>$showev('integrate(x^2*sqrt(2*x+1),x))
>$showev('integrate(x^2*sqrt(2*x+1),x,0,2))
>$ratsimp(%)
>$showev('integrate((sin(sqrt(x)+a)*E^sqrt(x))/sqrt(x),x,0,pi^2))
>$factor(%)
>function map f(x) &= E^(-x^2)
\end{eulerprompt}
\begin{euleroutput}
  
                                      2
                                   - x
                                  E
  
\end{euleroutput}
\begin{eulerprompt}
>$showev('integrate(f(x),x))
\end{eulerprompt}
\begin{eulercomment}
Fungsi f tidak memiliki antiturunan, integralnya masih memuat integral
lain.

\end{eulercomment}
\begin{eulerformula}
\[
erf(x) = \int \frac{e^{-x^2}}{\sqrt{\pi}} \ dx.
\]
\end{eulerformula}
\begin{eulercomment}
Kita tidak dapat menggunakan teorema Dasar kalkulus untuk menghitung
integral tentu fungsi tersebut jika semua batasnya berhingga. Dalam
hal ini dapat digunakan metode numerik (rumus kuadratur).

Misalkan kita akan menghitung:

maxima: 'integrate(f(x),x,0,pi)
\end{eulercomment}
\begin{eulerprompt}
>x=0:0.1:pi-0.1; plot2d(x,f(x+0.1),>bar); plot2d("f(x)",0,pi,>add):
\end{eulerprompt}
\begin{eulercomment}
Integral tentu

maxima: 'integrate(f(x),x,0,pi)

dapat dihampiri dengan jumlah luas persegi-persegi panjang di bawah
kurva y=f(x) tersebut. Langkah-langkahnya adalah sebagai berikut.
\end{eulercomment}
\begin{eulerprompt}
>t &= makelist(a,a,0,pi-0.1,0.1); // t sebagai list untuk menyimpan nilai-nilai x
>fx &= makelist(f(t[i]+0.1),i,1,length(t)); // simpan nilai-nilai f(x)
>// jangan menggunakan x sebagai list, kecuali Anda pakar Maxima!
\end{eulerprompt}
\begin{eulercomment}
Hasilnya adalah:

maxima: 'integrate(f(x),x,0,pi) = 0.1*sum(fx[i],i,1,length(fx))

Jumlah tersebut diperoleh dari hasil kali lebar sub-subinterval (=0.1)
dan jumlah nilai-nilai f(x) untuk x = 0.1, 0.2, 0.3, ..., 3.2.
\end{eulercomment}
\begin{eulerprompt}
>0.1*sum(f(x+0.1)) // cek langsung dengan perhitungan numerik EMT
\end{eulerprompt}
\begin{euleroutput}
  0.836219610253
\end{euleroutput}
\begin{eulercomment}
Untuk mendapatkan nilai integral tentu yang mendekati nilai sebenarnya, lebar
sub-intervalnya dapat diperkecil lagi, sehingga daerah di bawah kurva tertutup
semuanya, misalnya dapat digunakan lebar subinterval 0.001. (Silakan dicoba!)

Meskipun Maxima tidak dapat menghitung integral tentu fungsi tersebut untuk
batas-batas yang berhingga, namun integral tersebut dapat dihitung secara eksak jika
batas-batasnya tak hingga. Ini adalah salah satu keajaiban di dalam matematika, yang
terbatas tidak dapat dihitung secara eksak, namun yang tak hingga malah dapat
dihitung secara eksak.
\end{eulercomment}
\begin{eulerprompt}
>$showev('integrate(f(x),x,0,inf))
\end{eulerprompt}
\begin{eulercomment}
Tunjukkan kebenaran hasil di atas!

Berikut adalah contoh lain fungsi yang tidak memiliki antiderivatif,
sehingga integral tentunya hanya dapat dihitung dengan metode numerik.
\end{eulercomment}
\begin{eulerprompt}
>function f(x) &= x^x
\end{eulerprompt}
\begin{euleroutput}
  
                                     x
                                    x
  
\end{euleroutput}
\begin{eulerprompt}
>$showev('integrate(f(x),x,0,1))
>x=0:0.1:1-0.01; plot2d(x,f(x+0.01),>bar); plot2d("f(x)",0,1,>add):
\end{eulerprompt}
\begin{eulercomment}
Maxima gagal menghitung integral tentu tersebut secara langsung menggunakan perintah
integrate. Berikut kita lakukan seperti contoh sebelumnya untuk mendapat hasil atau
pendekatan nilai integral tentu tersebut.
\end{eulercomment}
\begin{eulerprompt}
>t &= makelist(a,a,0,1-0.01,0.01);
>fx &= makelist(f(t[i]+0.01),i,1,length(t));
\end{eulerprompt}
\begin{eulercomment}
maxima: 'integrate(f(x),x,0,1) = 0.01*sum(fx[i],i,1,length(fx))

Apakah hasil tersebut cukup baik? perhatikan gambarnya.
\end{eulercomment}
\begin{eulerprompt}
>function f(x) &= sin(3*x^5+7)^2
\end{eulerprompt}
\begin{euleroutput}
  
                                 2    5
                              sin (3 x  + 7)
  
\end{euleroutput}
\begin{eulerprompt}
>integrate(f,0,1)
\end{eulerprompt}
\begin{euleroutput}
  0.542581176074
\end{euleroutput}
\begin{eulerprompt}
>$&showev('integrate(f(x),x,0,1))
>&float(%)
\end{eulerprompt}
\begin{euleroutput}
  
           1.0
          /
          [       2      5
          I    sin (3.0 x  + 7.0) dx = 
          ]
          /
           0.0
  0.09820784258795788 - 0.008333333333333333
   (0.3090169943749474 (0.1367372182078336
   (4.192962712629476 I gamma__incomplete(0.2, 6.0 I)
   - 4.192962712629476 I gamma__incomplete(0.2, - 6.0 I))
   + 0.9906073556948704 (4.192962712629476 gamma__incomplete(0.2, 6.0 I)
   + 4.192962712629476 gamma__incomplete(0.2, - 6.0 I))) - 60.0)
  
\end{euleroutput}
\begin{eulerprompt}
>$showev('integrate(x*exp(-x),x,0,1)) // Integral tentu (eksak)
\end{eulerprompt}
\eulerheading{Aplikasi Integral Tentu}
\begin{eulerprompt}
>plot2d("x^3-x",-0.1,1.1); plot2d("-x^2",>add);  ...
>b=solve("x^3-x+x^2",0.5); x=linspace(0,b,200); xi=flipx(x); ...
>plot2d(x|xi,x^3-x|-xi^2,>filled,style="|",fillcolor=1,>add): // Plot daerah antara 2 kurva
>a=solve("x^3-x+x^2",0), b=solve("x^3-x+x^2",1) // absis titik-titik potong kedua kurva
\end{eulerprompt}
\begin{euleroutput}
  0
  0.61803398875
\end{euleroutput}
\begin{eulerprompt}
>integrate("(-x^2)-(x^3-x)",a,b) // luas daerah yang diarsir
\end{eulerprompt}
\begin{euleroutput}
  0.0758191713542
\end{euleroutput}
\begin{eulercomment}
Hasil tersebut akan kita bandingkan dengan perhitungan secara analitik.
\end{eulercomment}
\begin{eulerprompt}
>a &= solve((-x^2)-(x^3-x),x); $a // menentukan absis titik potong kedua kurva secara eksak
>$showev('integrate(-x^2-x^3+x,x,0,(sqrt(5)-1)/2)) // Nilai integral secara eksak
>$float(%)
\end{eulerprompt}
\eulersubheading{Panjang Kurva}
\begin{eulercomment}
Hitunglah panjang kurva berikut ini dan luas daerah di dalam kurva
tersebut.

\end{eulercomment}
\begin{eulerformula}
\[
\gamma(t) = (r(t) \cos(t), r(t) \sin(t))
\]
\end{eulerformula}
\begin{eulercomment}
dengan

\end{eulercomment}
\begin{eulerformula}
\[
r(t) = 1 + \dfrac{\sin(3t)}{2},\quad 0\le t\le 2\pi.
\]
\end{eulerformula}
\begin{eulerprompt}
>t=linspace(0,2pi,1000); r=1+sin(3*t)/2; x=r*cos(t); y=r*sin(t); ...
>plot2d(x,y,>filled,fillcolor=red,style="/",r=1.5): // Kita gambar kurvanya terlebih dahulu
>function r(t) &= 1+sin(3*t)/2; $'r(t)=r(t)
>function fx(t) &= r(t)*cos(t); $'fx(t)=fx(t)
>function fy(t) &= r(t)*sin(t); $'fy(t)=fy(t)
>function ds(t) &= trigreduce(radcan(sqrt(diff(fx(t),t)^2+diff(fy(t),t)^2))); $'ds(t)=ds(t)
>$integrate(ds(x),x,0,2*pi) //panjang (keliling) kurva
\end{eulerprompt}
\begin{eulercomment}
Maxima gagal melakukan perhitungan eksak integral tersebut.

Berikut kita hitung integralnya secara umerik dengan perintah EMT.
\end{eulercomment}
\begin{eulerprompt}
>integrate("ds(x)",0,2*pi)
\end{eulerprompt}
\begin{euleroutput}
  9.0749467823
\end{euleroutput}
\begin{eulercomment}
Spiral Logaritmik

\end{eulercomment}
\begin{eulerformula}
\[
x=e^{ax}\cos x,\ y=e^{ax}\sin x.
\]
\end{eulerformula}
\begin{eulerprompt}
>a=0.1; plot2d("exp(a*x)*cos(x)","exp(a*x)*sin(x)",r=2,xmin=0,xmax=2*pi):
>&kill(a) // hapus expresi a
\end{eulerprompt}
\begin{euleroutput}
  
                                   done
  
\end{euleroutput}
\begin{eulerprompt}
>function fx(t) &= exp(a*t)*cos(t); $'fx(t)=fx(t)
>function fy(t) &= exp(a*t)*sin(t); $'fy(t)=fy(t)
>function df(t) &= trigreduce(radcan(sqrt(diff(fx(t),t)^2+diff(fy(t),t)^2))); $'df(t)=df(t)
>S &=integrate(df(t),t,0,2*%pi); $S // panjang kurva (spiral)
>S(a=0.1) // Panjang kurva untuk a=0.1
\end{eulerprompt}
\begin{euleroutput}
  8.78817491636
\end{euleroutput}
\begin{eulercomment}
Soal:

Tunjukkan bahwa keliling lingkaran dengan jari-jari r adalah K=2.pi.r.

Berikut adalah contoh menghitung panjang parabola.
\end{eulercomment}
\begin{eulerprompt}
>plot2d("x^2",xmin=-1,xmax=1):
>$showev('integrate(sqrt(1+diff(x^2,x)^2),x,-1,1))
>$float(%)
>x=-1:0.2:1; y=x^2; plot2d(x,y);  ...
>  plot2d(x,y,points=1,style="o#",add=1):
\end{eulerprompt}
\begin{eulercomment}
Panjang tersebut dapat dihampiri dengan menggunakan jumlah panjang
ruas-ruas garis yang menghubungkan titik-titik pada parabola tersebut.
\end{eulercomment}
\begin{eulerprompt}
>i=1:cols(x)-1; sum(sqrt((x[i+1]-x[i])^2+(y[i+1]-y[i])^2))
\end{eulerprompt}
\begin{euleroutput}
  2.95191957027
\end{euleroutput}
\begin{eulercomment}
Hasilnya mendekati panjang yang dihitung secara eksak. Untuk
mendapatkan hampiran yang cukup akurat, jarak antar titik dapat
diperkecil, misalnya 0.1, 0.05, 0.01, dan seterusnya.

\end{eulercomment}
\eulersubheading{Koordinat Kartesius}
\begin{eulercomment}
Berikut diberikan contoh perhitungan panjang kurva menggunakan
koordinat Kartesius. Kita akan hitung panjang kurva dengan persamaan
implisit:

\end{eulercomment}
\begin{eulerformula}
\[
x^3+y^3-3xy=0.
\]
\end{eulerformula}
\begin{eulerprompt}
>z &= x^3+y^3-3*x*y; $z
>plot2d(z,r=2,level=0,n=100):
\end{eulerprompt}
\begin{eulercomment}
Kita tertarik pada kurva di kuadran pertama.
\end{eulercomment}
\begin{eulerprompt}
>plot2d(z,a=0,b=2,c=0,d=2,level=[-10;0],n=100,contourwidth=3,style="/"):
\end{eulerprompt}
\begin{eulercomment}
Kita selesaikan persamaannya untuk x.
\end{eulercomment}
\begin{eulerprompt}
>$z with y=l*x, sol &= solve(%,x); $sol
\end{eulerprompt}
\begin{eulercomment}
Kita gunakan solusi tersebut untuk mendefinisikan fungsi dengan Maxima.
\end{eulercomment}
\begin{eulerprompt}
>function f(l) &= rhs(sol[1]); $'f(l)=f(l)
\end{eulerprompt}
\begin{eulercomment}
Fungsi tersebut juga dapat digunaka untuk menggambar kurvanya. Ingat, bahwa fungsi tersebut adalah nilai x dan nilai y=l*x, yakni
x=f(l) dan y=l*f(l).
\end{eulercomment}
\begin{eulerprompt}
>plot2d(&f(x),&x*f(x),xmin=-0.5,xmax=2,a=0,b=2,c=0,d=2,r=1.5):
\end{eulerprompt}
\begin{eulercomment}
Elemen panjang kurva adalah:

\end{eulercomment}
\begin{eulerformula}
\[
ds=\sqrt{f'(l)^2+(lf'(l)+f(l))^2}.
\]
\end{eulerformula}
\begin{eulerprompt}
>function ds(l) &= ratsimp(sqrt(diff(f(l),l)^2+diff(l*f(l),l)^2)); $'ds(l)=ds(l)
>$integrate(ds(l),l,0,1)
\end{eulerprompt}
\begin{eulercomment}
Integral tersebut tidak dapat dihitung secara eksak menggunakan Maxima. Kita hitung integral etrsebut secara numerik dengan Euler.
Karena kurva simetris, kita hitung untuk nilai variabel integrasi dari 0 sampai 1, kemudian hasilnya dikalikan 2. 
\end{eulercomment}
\begin{eulerprompt}
>2*integrate("ds(x)",0,1)
\end{eulerprompt}
\begin{euleroutput}
  4.91748872168
\end{euleroutput}
\begin{eulerprompt}
>2*romberg(&ds(x),0,1)// perintah Euler lain untuk menghitung nilai hampiran integral yang sama
\end{eulerprompt}
\begin{euleroutput}
  4.91748872168
\end{euleroutput}
\begin{eulercomment}
Perhitungan di datas dapat dilakukan untuk sebarang fungsi x dan y dengan mendefinisikan fungsi EMT, misalnya kita beri nama
panjangkurva. Fungsi ini selalu memanggil Maxima untuk menurunkan fungsi yang diberikan.
\end{eulercomment}
\begin{eulerprompt}
>function panjangkurva(fx,fy,a,b) ...
\end{eulerprompt}
\begin{eulerudf}
  ds=mxm("sqrt(diff(@fx,x)^2+diff(@fy,x)^2)");
  return romberg(ds,a,b);
  endfunction
\end{eulerudf}
\begin{eulerprompt}
>panjangkurva("x","x^2",-1,1) // cek untuk menghitung panjang kurva parabola sebelumnya
\end{eulerprompt}
\begin{euleroutput}
  2.95788571509
\end{euleroutput}
\begin{eulercomment}
Bandingkan dengan nilai eksak di atas.
\end{eulercomment}
\begin{eulerprompt}
>2*panjangkurva(mxm("f(x)"),mxm("x*f(x)"),0,1) // cek contoh terakhir, bandingkan hasilnya!
\end{eulerprompt}
\begin{euleroutput}
  4.91748872168
\end{euleroutput}
\begin{eulercomment}
Kita hitung panjang spiral Archimides berikut ini dengan fungsi tersebut.
\end{eulercomment}
\begin{eulerprompt}
>plot2d("x*cos(x)","x*sin(x)",xmin=0,xmax=2*pi,square=1):
>panjangkurva("x*cos(x)","x*sin(x)",0,2*pi)
\end{eulerprompt}
\begin{euleroutput}
  21.2562941482
\end{euleroutput}
\begin{eulercomment}
Berikut kita definisikan fungsi yang sama namun dengan Maxima, untuk perhitungan eksak.
\end{eulercomment}
\begin{eulerprompt}
>&kill(ds,x,fx,fy)
\end{eulerprompt}
\begin{euleroutput}
  
                                   done
  
\end{euleroutput}
\begin{eulerprompt}
>function ds(fx,fy) &&= sqrt(diff(fx,x)^2+diff(fy,x)^2)
\end{eulerprompt}
\begin{euleroutput}
  
                             2              2
                    sqrt(diff (fy, x) + diff (fx, x))
  
\end{euleroutput}
\begin{eulerprompt}
>sol &= ds(x*cos(x),x*sin(x)); $sol // Kita gunakan untuk menghitung panjang kurva terakhir di atas
>$sol | trigreduce | expand, $integrate(%,x,0,2*pi), %()
\end{eulerprompt}
\begin{euleroutput}
  21.2562941482
\end{euleroutput}
\begin{eulercomment}
Hasilnya sama dengan perhitungan menggunakan fungsi EMT.

Berikut adalah contoh lain penggunaan fungsi Maxima tersebut.
\end{eulercomment}
\begin{eulerprompt}
>plot2d("3*x^2-1","3*x^3-1",xmin=-1/sqrt(3),xmax=1/sqrt(3),square=1):
>sol &= radcan(ds(3*x^2-1,3*x^3-1)); $sol
>$showev('integrate(sol,x,0,1/sqrt(3))), $2*float(%) // panjang kurva di atas
\end{eulerprompt}
\eulersubheading{Sikloid}
\begin{eulercomment}
Berikut kita akan menghitung panjang kurva lintasan (sikloid) suatu
titik pada lingkaran yang berputar ke kanan pada permukaan datar.
Misalkan jari-jari lingkaran tersebut adalah r. Posisi titik pusat
lingkaran pada saat t adalah:

\end{eulercomment}
\begin{eulerformula}
\[
(rt,r).
\]
\end{eulerformula}
\begin{eulercomment}
Misalkan posisi titik pada lingkaran tersebut mula-mula (0,0) dan
posisinya pada saat t adalah:

\end{eulercomment}
\begin{eulerformula}
\[
(r(t-\sin(t)),r(1-\cos(t))).
\]
\end{eulerformula}
\begin{eulercomment}
Berikut kita plot lintasan tersebut dan beberapa posisi lingkaran
ketika t=0, t=pi/2, t=r*pi.
\end{eulercomment}
\begin{eulerprompt}
>x &= r*(t-sin(t))
\end{eulerprompt}
\begin{euleroutput}
  
                              r (t - sin(t))
  
\end{euleroutput}
\begin{eulerprompt}
>y &= r*(1-cos(t))
\end{eulerprompt}
\begin{euleroutput}
  
                              r (1 - cos(t))
  
\end{euleroutput}
\begin{eulerprompt}
>t=0
\end{eulerprompt}
\begin{euleroutput}
  0
\end{euleroutput}
\begin{eulerprompt}
>r=1
\end{eulerprompt}
\begin{euleroutput}
  1
\end{euleroutput}
\begin{eulercomment}
Berikut kita gambar sikloid untuk r=1.
\end{eulercomment}
\begin{eulerprompt}
>ex &= x-sin(x); ey &= 1-cos(x); aspect(1);
>plot2d(ex,ey,xmin=0,xmax=4pi,square=1); ...
>  plot2d("2+cos(x)","1+sin(x)",xmin=0,xmax=2pi,>add,color=blue); ...
>  plot2d([2,ex(2)],[1,ey(2)],color=red,>add); ...
>  plot2d(ex(2),ey(2),>points,>add,color=red); ...
>  plot2d("2pi+cos(x)","1+sin(x)",xmin=0,xmax=2pi,>add,color=blue); ...
>  plot2d([2pi,ex(2pi)],[1,ey(2pi)],color=red,>add);  ...
>  plot2d(ex(2pi),ey(2pi),>points,>add,color=red):
\end{eulerprompt}
\begin{eulercomment}
Berikut dihitung panjang lintasan untuk 1 putaran penuh. (Jangan salah
menduga bahwa panjang lintasan 1 putaran penuh sama dengan keliling
lingkaran!)
\end{eulercomment}
\begin{eulerprompt}
>ds &= radcan(sqrt(diff(ex,x)^2+diff(ey,x)^2)); $ds=trigsimp(ds) // elemen panjang kurva sikloid
>ds &= trigsimp(ds); $ds
>$showev('integrate(ds,x,0,2*pi)) // hitung panjang sikloid satu putaran penuh
>integrate(mxm("ds"),0,2*pi) // hitung secara numerik
\end{eulerprompt}
\begin{euleroutput}
  0
\end{euleroutput}
\begin{eulerprompt}
>romberg(mxm("ds"),0,2*pi) // cara lain hitung secara numerik
\end{eulerprompt}
\begin{euleroutput}
  0
\end{euleroutput}
\begin{eulercomment}
Perhatikan, seperti terlihat pada gambar, panjang sikloid lebih besar daripada keliling lingkarannya, yakni:

\end{eulercomment}
\begin{eulerformula}
\[
2\pi.
\]
\end{eulerformula}
\eulersubheading{Kurvatur (Kelengkungan) Kurva}
\begin{eulercomment}
image: Osculating.png

Aslinya, kelengkungan kurva diferensiabel (yakni, kurva mulus yang
tidak lancip) di titik P didefinisikan melalui lingkaran oskulasi
(yaitu, lingkaran yang melalui titik P dan terbaik memperkirakan,
paling banyak menyinggung kurva di sekitar P). Pusat dan radius
kelengkungan kurva di P adalah pusat dan radius lingkaran oskulasi.
Kelengkungan adalah kebalikan dari radius kelengkungan:

\end{eulercomment}
\begin{eulerformula}
\[
\kappa =\frac {1}{R}
\]
\end{eulerformula}
\begin{eulercomment}
dengan R adalah radius kelengkungan. (Setiap lingkaran memiliki
kelengkungan ini pada setiap titiknya, dapat diartikan, setiap
lingkaran berputar 2pi sejauh 2piR.)\\
Definisi ini sulit dimanipulasi dan dinyatakan ke dalam rumus untuk
kurva umum. Oleh karena itu digunakan definisi lain yang ekivalen.

\end{eulercomment}
\eulersubheading{Definisi Kurvatur dengan Fungsi Parametrik Panjang Kurva}
\begin{eulercomment}
Setiap kurva diferensiabel dapat dinyatakan dengan persamaan
parametrik terhadap panjang kurva s:

\end{eulercomment}
\begin{eulerformula}
\[
\gamma(s) = (x(s),\ y(s)),
\]
\end{eulerformula}
\begin{eulercomment}
dengan x dan y adalah fungsi riil yang diferensiabel, yang memenuhi:

\end{eulercomment}
\begin{eulerformula}
\[
\|\gamma'(s)\|=\sqrt{x'(s)^2+y'(s)^2}=1.
\]
\end{eulerformula}
\begin{eulercomment}
Ini berarti bahwa vektor singgung


\end{eulercomment}
\begin{eulerformula}
\[
\mathbf{T}(s)=(x'(s),\ y'(s))
\]
\end{eulerformula}
\begin{eulercomment}
memiliki norm 1 dan merupakan vektor singgung satuan.

Apabila kurvanya memiliki turunan kedua, artinya turunan kedua x dan y
ada, maka T'(s) ada. Vektor ini merupakan normal kurva yang arahnya
menuju pusat kurvatur, norm-nya merupakan nilai kurvatur
(kelengkungan):

\end{eulercomment}
\begin{eulerformula}
\[
 \begin{aligned}\mathbf{T}(s) &= \mathbf{\gamma}'(s),\\ \mathbf{T}^{2}(s) &=1\ \text{(konstanta)}\Rightarrow \mathbf{T}'(s)\cdot \mathbf{T}(s)=0\\ \kappa(s) &=\|\mathbf {T}'(s)\|= \|\mathbf{\gamma}''(s)\|=\sqrt{x''(s)^{2}+y''(s)^{2}}.\end{aligned}
\]
\end{eulerformula}
\begin{eulercomment}
Nilai

\end{eulercomment}
\begin{eulerformula}
\[
R(s)=\frac{1}{\kappa(s)}
\]
\end{eulerformula}
\begin{eulercomment}
disebut jari-jari (radius) kelengkungan kurva.

Bilangan riil

\end{eulercomment}
\begin{eulerformula}
\[
 k(s) = \pm\kappa(s)
\]
\end{eulerformula}
\begin{eulercomment}
disebut nilai kelengkungan bertanda.

Contoh:\\
Akan ditentukan kurvatur lingkaran

\end{eulercomment}
\begin{eulerformula}
\[
x=r\cos t,\ y= r\sin t.
\]
\end{eulerformula}
\begin{eulerprompt}
>fx &= r*cos(t); fy &=r*sin(t);
>&assume(t>0,r>0); s &=integrate(sqrt(diff(fx,t)^2+diff(fy,t)^2),t,0,t); s // elemen panjang kurva, panjang busur lingkaran (s)
\end{eulerprompt}
\begin{euleroutput}
  
                                   r t
  
\end{euleroutput}
\begin{eulerprompt}
>&kill(s); fx &= r*cos(s/r); fy &=r*sin(s/r); // definisi ulang persamaan parametrik terhadap s dengan substitusi t=s/r
>k &= trigsimp(sqrt(diff(fx,s,2)^2+diff(fy,s,2)^2)); $k // nilai kurvatur lingkaran dengan menggunakan definisi di atas
\end{eulerprompt}
\begin{eulercomment}
Untuk representasi parametrik umum, misalkan

\end{eulercomment}
\begin{eulerformula}
\[
x = x(t),\ y= y(t)
\]
\end{eulerformula}
\begin{eulercomment}
merupakan persamaan parametrik untuk kurva bidang yang
terdiferensialkan dua kali. Kurvatur untuk kurva tersebut
didefinisikan sebagai

\end{eulercomment}
\begin{eulerformula}
\[
\begin{aligned}\kappa &= \frac{d\phi}{ds}=\frac{\frac{d\phi}{dt}}{\frac{ds}{dt}}\quad (\phi \text{ adalah sudut kemiringan garis singgung dan }s \text{ adalah panjang kurva})\\ &=\frac{\frac{d\phi}{dt}}{\sqrt{(\frac{dx}{dt})^2+(\frac{dy}{dt})^2}}= \frac{\frac{d\phi}{dt}}{\sqrt{x'(t)^2+y'(t)^2}}.\end{aligned}.
\]
\end{eulerformula}
\begin{eulercomment}
Selanjutnya, pembilang pada persamaan di atas dapat dicari sebagai
berikut.

\end{eulercomment}
\begin{eulerformula}
\[
\begin{aligned}\sec^2\phi\frac{d\phi}{dt} &= \frac{d}{dt}\left(\tan\phi\right)= \frac{d}{dt}\left(\frac{dy}{dx}\right)= \frac{d}{dt}\left(\frac{dy/dt}{dx/dt}\right)= \frac{d}{dt}\left(\frac{y'(t)}{x'(t)}\right)=\frac{x'(t)y''(t)-x''(t)y'(t)}{x'(t)^2}.\\ & \\ \frac{d\phi}{dt} &= \frac{1}{\sec^2\phi}\frac{x'(t)y''(t)-x''(t)y'(t)}{x'(t)^2}\\ &= \frac{1}{1+\tan^2\phi}\frac{x'(t)y''(t)-x''(t)y'(t)}{x'(t)^2}\\ &= \frac{1}{1+\left(\frac{y'(t)}{x'(t)}\right)^2}\frac{x'(t)y''(t)-x''(t)y'(t)}{x'(t)^2}\\ &= \frac{x'(t)y''(t)-x''(t)y'(t)}{x'(t)^2+y'(t)^2}.\end{aligned}
\]
\end{eulerformula}
\begin{eulercomment}
Jadi, rumus kurvatur untuk kurva parametrik

\end{eulercomment}
\begin{eulerformula}
\[
x=x(t),\ y=y(t)
\]
\end{eulerformula}
\begin{eulercomment}
adalah

\end{eulercomment}
\begin{eulerformula}
\[
\kappa(t) = \frac{x'(t)y''(t)-x''(t)y'(t)}{\left(x'(t)^2+y'(t)^2\right)^{3/2}}.
\]
\end{eulerformula}
\begin{eulercomment}
Jika kurvanya dinyatakan dengan persamaan parametrik pada koordinat
kutub

\end{eulercomment}
\begin{eulerformula}
\[
x=r(\theta)\cos\theta,\ y=r(\theta)\sin\theta,
\]
\end{eulerformula}
\begin{eulercomment}
maka rumus kurvaturnya adalah

\end{eulercomment}
\begin{eulerformula}
\[
\kappa(\theta) = \frac{r(\theta)^2+2r'(\theta)^2-r(\theta)r''(\theta)}{\left(r'(\theta)^2+r'(\theta)^2\right)^{3/2}}.
\]
\end{eulerformula}
\begin{eulercomment}
(Silakan Anda turunkan rumus tersebut!)

Contoh:\\
Lingkaran dengan pusat (0,0) dan jari-jari r dapat dinyatakan dengan
persamaan parametrik

\end{eulercomment}
\begin{eulerformula}
\[
x=r\cos t,\ y=r\sin t.
\]
\end{eulerformula}
\begin{eulercomment}
Nilai kelengkungan lingkaran tersebut adalah

\end{eulercomment}
\begin{eulerformula}
\[
\kappa(t)=\frac{x'(t)y''(t)-x''(t)y'(t)}{\left(x'(t)^2+y'(t)^2\right)^{3/2}}=\frac{r^2}{r^3}=\frac 1 r.
\]
\end{eulerformula}
\begin{eulercomment}
Hasil cocok dengan definisi kurvatur suatu kelengkungan.
\end{eulercomment}
\begin{eulercomment}
Kurva

\end{eulercomment}
\begin{eulerformula}
\[
y=f(x)
\]
\end{eulerformula}
\begin{eulercomment}
dapat dinyatakan ke dalam persamaan parametrik

\end{eulercomment}
\begin{eulerformula}
\[
x=t,\ y=f(t),\ \text{ dengan } x'(t)=1,\ x''(t)=0,
\]
\end{eulerformula}
\begin{eulercomment}
sehingga kurvaturnya adalah

\end{eulercomment}
\begin{eulerformula}
\[
\kappa(t) = \frac{y''(t)}{\left(1+y'(t)^2\right)^{3/2}}.
\]
\end{eulerformula}
\begin{eulercomment}
Contoh:\\
Akan ditentukan kurvatur parabola

\end{eulercomment}
\begin{eulerformula}
\[
y=ax^2+bx+c.
\]
\end{eulerformula}
\begin{eulerprompt}
>function f(x) &= a*x^2+b*x+c; $y=f(x)
>function k(x) &= (diff(f(x),x,2))/(1+diff(f(x),x)^2)^(3/2); $'k(x)=k(x) // kelengkungan parabola 
>function f(x) &= x^2+x+1; $y=f(x) // akan kita plot kelengkungan parabola untuk a=b=c=1
>function k(x) &= (diff(f(x),x,2))/(1+diff(f(x),x)^2)^(3/2); $'k(x)=k(x) // kelengkungan parabola 
\end{eulerprompt}
\begin{eulercomment}
Berikut kita gambar parabola tersebut beserta kurva kelengkungan,
kurva jari-jari kelengkungan dan salah satu lingkaran oskulasi di
titik puncak parabola. Perhatikan, puncak parabola dan jari-jari
lingkaran oskulasi di puncak parabola adalah

\end{eulercomment}
\begin{eulerformula}
\[
(-1/2,3/4),\ 1/k(2)=1/2,
\]
\end{eulerformula}
\begin{eulercomment}
sehingga pusat lingkaran oskulasi adalah (-1/2, 5/4).
\end{eulercomment}
\begin{eulerprompt}
>plot2d(["f(x)", "k(x)"],-2,1, color=[blue,red]); plot2d("1/k(x)",-1.5,1,color=green,>add); ...
>plot2d("-1/2+1/k(-1/2)*cos(x)","5/4+1/k(-1/2)*sin(x)",xmin=0,xmax=2pi,>add,color=blue):
\end{eulerprompt}
\begin{eulercomment}
Untuk kurva yang dinyatakan dengan fungsi implisit

\end{eulercomment}
\begin{eulerformula}
\[
F(x,y)=0
\]
\end{eulerformula}
\begin{eulercomment}
dengan turunan-turunan parsial

\end{eulercomment}
\begin{eulerformula}
\[
F_x=\frac{\partial F}{\partial x},\ F_y=\frac{\partial F}{\partial y},\ F_{xy}=\frac{\partial}{\partial y}\left(\frac{\partial F}{\partial x}\right),\ F_{xx}=\frac{\partial}{\partial x}\left(\frac{\partial F}{\partial x}\right),\ F_{yy}=\frac{\partial}{\partial y}\left(\frac{\partial F}{\partial y}\right),
\]
\end{eulerformula}
\begin{eulercomment}
berlaku

\end{eulercomment}
\begin{eulerformula}
\[
F_x dx+ F_y dy = 0\text{ atau } \frac{dy}{dx}=-\frac{F_x}{F_y},
\]
\end{eulerformula}
\begin{eulercomment}
sehingga kurvaturnya adalah

\end{eulercomment}
\begin{eulerformula}
\[
\kappa =\frac {F_y^2F_{xx}-2F_xF_yF_{xy}+F_x^2F_{yy}}{\left(F_x^2+F_y^2\right)^{3/2}}.
\]
\end{eulerformula}
\begin{eulercomment}
(Silakan Anda turunkan sendiri!)

Contoh 1:\\
Parabola

\end{eulercomment}
\begin{eulerformula}
\[
y=ax^2+bx+c
\]
\end{eulerformula}
\begin{eulercomment}
dapat dinyatakan ke dalam persamaan implisit

\end{eulercomment}
\begin{eulerformula}
\[
ax^2+bx+c-y=0.
\]
\end{eulerformula}
\begin{eulerprompt}
>function F(x,y) &=a*x^2+b*x+c-y; $F(x,y)
>Fx &= diff(F(x,y),x), Fxx &=diff(F(x,y),x,2), Fy &=diff(F(x,y),y), Fxy &=diff(diff(F(x,y),x),y), Fyy &=diff(F(x,y),y,2)  
\end{eulerprompt}
\begin{euleroutput}
  
                                2 a x + b
  
  
                                   2 a
  
  
                                   - 1
  
  
                                    0
  
  
                                    0
  
\end{euleroutput}
\begin{eulerprompt}
>function k(x) &= (Fy^2*Fxx-2*Fx*Fy*Fxy+Fx^2*Fyy)/(Fx^2+Fy^2)^(3/2); $'k(x)=k(x) // kurvatur parabola tersebut
\end{eulerprompt}
\begin{eulercomment}
Hasilnya sama dengan sebelumnya yang menggunakan persamaan parabola biasa.
\end{eulercomment}
\eulerheading{Latihan}
\begin{eulercomment}
- Bukalah buku Kalkulus.\\
- Cari dan pilih beberapa (paling sedikit 5 fungsi berbeda
tipe/bentuk/jenis) fungsi dari buku tersebut, kemudian definisikan di
EMT pada baris-baris perintah berikut (jika perlu tambahkan lagi).\\
- Untuk setiap fungsi, tentukan anti turunannya (jika ada), hitunglah
integral tentu dengan batas-batas yang menarik (Anda tentukan
sendiri), seperti contoh-contoh tersebut.\\
- Lakukan hal yang sama untuk fungsi-fungsi yang tidak dapat
diintegralkan (cari sedikitnya 3 fungsi).\\
- Gambar grafik fungsi dan daerah integrasinya pada sumbu koordinat
yang sama.\\
- Gunakan integral tentu untuk mencari luas daerah yang dibatasi oleh
dua kurva yang berpotongan di dua titik. (Cari dan gambar kedua kurva
dan arsir (warnai) daerah yang dibatasi oleh keduanya.)\\
- Gunakan integral tentu untuk menghitung volume benda putar kurva y=
f(x) yang diputar mengelilingi sumbu x dari x=a sampai x=b, yakni

\end{eulercomment}
\begin{eulerformula}
\[
V = \int_a^b \pi (f(x)^2\ dx.
\]
\end{eulerformula}
\begin{eulercomment}
(Pilih fungsinya dan gambar kurva dan benda putar yang dihasilkan.
Anda dapat mencari contoh-contoh bagaimana cara menggambar benda hasil
perputaran suatu kurva.)\\
- Gunakan integral tentu untuk menghitung panjang kurva y=f(x) dari
x=a sampai x=b dengan menggunakan rumus:

\end{eulercomment}
\begin{eulerformula}
\[
S = \int_a^b \sqrt{1+(f'(x))^2} \ dx.
\]
\end{eulerformula}
\begin{eulercomment}
(Pilih fungsi dan gambar kurvanya.)\\
- Apabila fungsi dinyatakan dalam koordinat kutub x=f(r,t), y=g(r,t),
r=h(t), x=a bersesuaian dengan t=t0 dan x=b bersesuian dengan t=t1,
maka rumus di atas akan menjadi:

\end{eulercomment}
\begin{eulerformula}
\[
S=\int_{t_0}^{t_1} \sqrt{x'(t)^2+y'(t)^2}\ dt.
\]
\end{eulerformula}
\begin{eulercomment}
- Pilih beberapa kurva menarik (selain lingkaran dan parabola) dari
buku  kalkulus. Nyatakan setiap kurva tersebut dalam bentuk:\\
\end{eulercomment}
\begin{eulerttcomment}
  a. koordinat Kartesius (persamaan y=f(x))
  b. koordinat kutub ( r=r(theta))
  c. persamaan parametrik x=x(t), y=y(t)
  d. persamaan implit F(x,y)=0
\end{eulerttcomment}
\begin{eulercomment}
- Tentukan kurvatur masing-masing kurva dengan menggunakan keempat
representasi tersebut (hasilnya harus sama).\\
- Gambarlah kurva asli, kurva kurvatur, kurva jari-jari lingkaran
oskulasi, dan salah satu lingkaran oskulasinya.

Nomor 1
\end{eulercomment}
\begin{eulerprompt}
>function f(x):=sin(2x)
>$showev('integrate(sin(3*x),x))
>$showev('integrate(sin(3*x),x,0,pi/4))
>x=0:0.1:pi-0.01; plot2d(x,f(x+0.01),>bar); plot2d("f(x)",0,1,>add):
\end{eulerprompt}
\begin{eulercomment}
Nomor 2
\end{eulercomment}
\begin{eulerprompt}
>function f(x):=sqrt(2*x^2+4*x)
>$showev('integrate(f(x),x))
>$showev('integrate(sqrt(2*x^2+4*x),x,0,2))
>x=0:0.1:pi-0.01; plot2d(x,f(x+0.01),>bar); plot2d("f(x)",0,1,>add):
\end{eulerprompt}
\begin{eulercomment}
Nomor 3
\end{eulercomment}
\begin{eulerprompt}
>function f(x):=(x^4+x^2)
>$showev('integrate((x^4+x^2),x))
>$showev('integrate(x^4+x^2,x,0,2))
>x=0:0.1:pi-0.01; plot2d(x,f(x+0.01),>bar); plot2d("f(x)",0,1,>add):
\end{eulerprompt}
\begin{eulercomment}
Nomor 4
\end{eulercomment}
\begin{eulerprompt}
>function f(x):=sin(6x)
>$showev('integrate(sin(2*x),x))
>$showev('integrate(sin(2*x),x,0,pi/4))
>x=0:0.1:pi-0.01; plot2d(x,f(x+0.01),>bar); plot2d("f(x)",0,1,>add):
\end{eulerprompt}
\begin{eulercomment}
Nomor 5
\end{eulercomment}
\begin{eulerprompt}
>function f(x):=x^3+3*x+2
>$showev('integrate(x^3+3*x+2,x))
>$showev('integrate(x^3+3*x+2,x,0,pi/4))
>x=0:0.1:pi-0.01; plot2d(x,f(x+0.01),>bar); plot2d("f(x)",0,1,>add):
\end{eulerprompt}
\eulerheading{Barisan dan Deret}
\begin{eulercomment}
(Catatan: bagian ini belum lengkap. Anda dapat membaca contoh-contoh
pengguanaan EMT dan Maxima untuk menghitung limit barisan, rumus
jumlah parsial suatu deret, jumlah tak hingga suatu deret konvergen,
dan sebagainya. Anda dapat mengeksplor contoh-contoh di EMT atau
perbagai panduan penggunaan Maxima di software Maxima atau dari
Internet.)

Barisan dapat didefinisikan dengan beberapa cara di dalam EMT, di
antaranya:

- dengan cara yang sama seperti mendefinisikan vektor dengan
elemen-elemen beraturan (menggunakan titik dua ":");\\
- menggunakan perintah "sequence" dan rumus barisan (suku ke -n);\\
- menggunakan perintah "iterate" atau "niterate";\\
- menggunakan fungsi Maxima "create\_list" atau "makelist" untuk
menghasilkan barisan simbolik;\\
- menggunakan fungsi biasa yang inputnya vektor atau barisan;\\
- menggunakan fungsi rekursif.

EMT menyediakan beberapa perintah (fungsi) terkait barisan, yakni:

- sum: menghitung jumlah semua elemen suatu barisan\\
- cumsum: jumlah kumulatif suatu barisan\\
- differences: selisih antar elemen-elemen berturutan

EMT juga dapat digunakan untuk menghitung jumlah deret berhingga
maupun deret tak hingga, dengan menggunakan perintah (fungsi) "sum".
Perhitungan dapat dilakukan secara numerik maupun simbolik dan eksak.

Berikut adalah beberapa contoh perhitungan barisan dan deret
menggunakan EMT.
\end{eulercomment}
\begin{eulerprompt}
>1:10 // barisan sederhana
\end{eulerprompt}
\begin{euleroutput}
  [1,  2,  3,  4,  5,  6,  7,  8,  9,  10]
\end{euleroutput}
\begin{eulerprompt}
>1:2:30
\end{eulerprompt}
\begin{euleroutput}
  [1,  3,  5,  7,  9,  11,  13,  15,  17,  19,  21,  23,  25,  27,  29]
\end{euleroutput}
\eulerheading{Iterasi dan Barisan}
\begin{eulercomment}
EMT menyediakan fungsi iterate("g(x)", x0, n) untuk melakukan iterasi

\end{eulercomment}
\begin{eulerformula}
\[
x_{k+1}=g(x_k), \ x_0=x_0, k= 1, 2, 3, ..., n.
\]
\end{eulerformula}
\begin{eulercomment}
Berikut ini disajikan contoh-contoh penggunaan iterasi dan rekursi
dengan EMT. Contoh pertama menunjukkan pertumbuhan dari nilai awal
1000 dengan laju pertambahan 5\%, selama 10 periode.
\end{eulercomment}
\begin{eulerprompt}
>q=1.05; iterate("x*q",1000,n=10)'
\end{eulerprompt}
\begin{euleroutput}
           1000 
           1050 
         1102.5 
        1157.63 
        1215.51 
        1276.28 
         1340.1 
         1407.1 
        1477.46 
        1551.33 
        1628.89 
\end{euleroutput}
\begin{eulercomment}
Contoh berikutnya memperlihatkan bahaya menabung di bank pada masa sekarang! Dengan bunga
tabungan sebesar 6\% per tahun atau 0.5\% per bulan dipotong pajak 20\%, dan biaya administrasi
10000 per bulan, tabungan sebesar 1 juta tanpa diambil selama sekitar 10 tahunan akan habis
diambil oleh bank!
\end{eulercomment}
\begin{eulerprompt}
>r=0.005; plot2d(iterate("(1+0.8*r)*x-10000",1000000,n=130)):
\end{eulerprompt}
\begin{eulercomment}
Silakan Anda coba-coba, dengan tabungan minimal berapa agar tidak akan habis diambil oleh
bank dengan ketentuan bunga dan biaya administrasi seperti di atas.

Berikut adalah perhitungan minimal tabungan agar aman di bank dengan bunga sebesar r dan
biaya administrasi a, pajak bunga 20\%.
\end{eulercomment}
\begin{eulerprompt}
>$solve(0.8*r*A-a,A), $% with [r=0.005, a=10] 
\end{eulerprompt}
\begin{eulercomment}
Berikut didefinisikan fungsi untuk menghitung saldo tabungan, kemudian dilakukan iterasi.
\end{eulercomment}
\begin{eulerprompt}
>function saldo(x,r,a) := round((1+0.8*r)*x-a,2);
>iterate(\{\{"saldo",0.005,10\}\},1000,n=6)
\end{eulerprompt}
\begin{euleroutput}
  [1000,  994,  987.98,  981.93,  975.86,  969.76,  963.64]
\end{euleroutput}
\begin{eulerprompt}
>iterate(\{\{"saldo",0.005,10\}\},2000,n=6)
\end{eulerprompt}
\begin{euleroutput}
  [2000,  1998,  1995.99,  1993.97,  1991.95,  1989.92,  1987.88]
\end{euleroutput}
\begin{eulerprompt}
>iterate(\{\{"saldo",0.005,10\}\},2500,n=6)
\end{eulerprompt}
\begin{euleroutput}
  [2500,  2500,  2500,  2500,  2500,  2500,  2500]
\end{euleroutput}
\begin{eulercomment}
Tabungan senilai 2,5 juta akan aman dan tidak akan berubah nilai (jika tidak ada penarikan),
sedangkan jika tabungan awal kurang dari 2,5 juta, lama kelamaan akan berkurang meskipun
tidak pernah dilakukan penarikan uang tabungan.
\end{eulercomment}
\begin{eulerprompt}
>iterate(\{\{"saldo",0.005,10\}\},3000,n=6)
\end{eulerprompt}
\begin{euleroutput}
  [3000,  3002,  3004.01,  3006.03,  3008.05,  3010.08,  3012.12]
\end{euleroutput}
\begin{eulercomment}
Tabungan yang lebih dari 2,5 juta baru akan bertambah jika tidak ada
penarikan.

Untuk barisan yang lebih kompleks dapat digunakan fungsi "sequence()".
Fungsi ini menghitung nilai-nilai x[n] dari semua nilai sebelumnya,
x[1],...,x[n-1] yang diketahui.\\
Berikut adalah contoh barisan Fibonacci.

\end{eulercomment}
\begin{eulerformula}
\[
x_n = x_{n-1}+x_{n-2}, \quad x_1=1, \quad x_2 =1
\]
\end{eulerformula}
\begin{eulerprompt}
>sequence("x[n-1]+x[n-2]",[1,1],15)
\end{eulerprompt}
\begin{euleroutput}
  [1,  1,  2,  3,  5,  8,  13,  21,  34,  55,  89,  144,  233,  377,  610]
\end{euleroutput}
\begin{eulercomment}
Barisan Fibonacci memiliki banyak sifat menarik, salah satunya adalah akar pangkat ke-n suku
ke-n akan konvergen ke pecahan emas:
\end{eulercomment}
\begin{eulerprompt}
>$'(1+sqrt(5))/2=float((1+sqrt(5))/2)
>plot2d(sequence("x[n-1]+x[n-2]",[1,1],250)^(1/(1:250))):
\end{eulerprompt}
\begin{eulercomment}
Barisan yang sama juga dapat dihasilkan dengan menggunakan loop.
\end{eulercomment}
\begin{eulerprompt}
>x=ones(500); for k=3 to 500; x[k]=x[k-1]+x[k-2]; end;
\end{eulerprompt}
\begin{eulercomment}
Rekursi dapat dilakukan dengan menggunakan rumus yang tergantung pada semua elemen
sebelumnya. Pada contoh berikut, elemen ke-n merupakan jumlah (n-1) elemen sebelumnya,
dimulai dengan 1 (elemen ke-1). Jelas, nilai elemen ke-n adalah 2\textasciicircum{}(n-2), untuk n=2, 4, 5,
....
\end{eulercomment}
\begin{eulerprompt}
>sequence("sum(x)",1,10)
\end{eulerprompt}
\begin{euleroutput}
  [1,  1,  2,  4,  8,  16,  32,  64,  128,  256]
\end{euleroutput}
\begin{eulercomment}
Selain menggunakan ekspresi dalam x dan n, kita juga dapat menggunakan
fungsi.

Pada contoh berikut, digunakan iterasi

\end{eulercomment}
\begin{eulerformula}
\[
x_n =A \cdot x_{n-1},
\]
\end{eulerformula}
\begin{eulercomment}
dengan A suatu matriks 2x2, dan setiap x[n] merupakan matriks/vektor
2x1.
\end{eulercomment}
\begin{eulerprompt}
>A=[1,1;1,2]; function suku(x,n) := A.x[,n-1]
>sequence("suku",[1;1],6)
\end{eulerprompt}
\begin{euleroutput}
  Real 2 x 6 matrix
  
              1             2             5            13     ...
              1             3             8            21     ...
\end{euleroutput}
\begin{eulercomment}
Hasil yang sama juga dapat diperoleh dengan menggunakan fungsi
perpangkatan matriks "matrixpower()". Cara ini lebih cepat, karena
hanya menggunakan perkalian matriks sebanyak log\_2(n).

\end{eulercomment}
\begin{eulerformula}
\[
x_n=A.x_{n-1}=A^2.x_{n-2}=A^3.x_{n-3}= ... = A^{n-1}.x_1.
\]
\end{eulerformula}
\begin{eulerprompt}
>sequence("matrixpower(A,n).[1;1]",1,6)
\end{eulerprompt}
\begin{euleroutput}
  Real 2 x 6 matrix
  
              1             5            13            34     ...
              1             8            21            55     ...
\end{euleroutput}
\eulerheading{Spiral Theodorus}
\begin{eulercomment}
image: Spiral.png\\
Spiral Theodorus (spiral segitiga siku-siku) dapat digambar secara
rekursif. Rumus rekursifnya adalah:

\end{eulercomment}
\begin{eulerformula}
\[
x_n = \left( 1 + \frac{i}{\sqrt{n-1}} \right) \, x_{n-1}, \quad x_1=1,
\]
\end{eulerformula}
\begin{eulercomment}
yang menghasilkan barisan bilangan kompleks.
\end{eulercomment}
\begin{eulerprompt}
>function g(n) := 1+I/sqrt(n)
\end{eulerprompt}
\begin{eulercomment}
Rekursinya dapat dijalankan sebanyak 17 untuk menghasilkan barisan 17 bilangan kompleks,
kemudian digambar bilangan-bilangan kompleksnya.
\end{eulercomment}
\begin{eulerprompt}
>x=sequence("g(n-1)*x[n-1]",1,17); plot2d(x,r=3.5); textbox(latex("Spiral\(\backslash\) Theodorus"),0.4):
\end{eulerprompt}
\begin{eulercomment}
Selanjutnya dihubungan titik 0 dengan titik-titik kompleks tersebut menggunakan loop.
\end{eulercomment}
\begin{eulerprompt}
>for i=1:cols(x); plot2d([0,x[i]],>add); end:
> 
\end{eulerprompt}
\begin{eulercomment}
Spiral tersebut juga dapat didefinisikan menggunakan fungsi rekursif, yang tidak memmerlukan
indeks dan bilangan kompleks. Dalam hal ini diigunakan vektor kolom pada bidang.
\end{eulercomment}
\begin{eulerprompt}
>function gstep (v) ...
\end{eulerprompt}
\begin{eulerudf}
  w=[-v[2];v[1]];
  return v+w/norm(w);
  endfunction
\end{eulerudf}
\begin{eulercomment}
Jika dilakukan iterasi 16 kali dimulai dari [1;0] akan didapatkan matriks yang memuat
vektor-vektor dari setiap iterasi.
\end{eulercomment}
\begin{eulerprompt}
>x=iterate("gstep",[1;0],16); plot2d(x[1],x[2],r=3.5,>points):
\end{eulerprompt}
\begin{eulercomment}
\begin{eulercomment}
\eulerheading{Kekonvergenan}
\begin{eulercomment}
Terkadang kita ingin melakukan iterasi sampai konvergen. Apabila iterasinya tidak konvergen
setelah ditunggu lama, Anda dapat menghentikannya dengan menekan tombol [ESC].
\end{eulercomment}
\begin{eulerprompt}
>iterate("cos(x)",1) // iterasi x(n+1)=cos(x(n)), dengan x(0)=1.
\end{eulerprompt}
\begin{euleroutput}
  0.739085133216
\end{euleroutput}
\begin{eulercomment}
Iterasi tersebut konvergen ke penyelesaian persamaan

\end{eulercomment}
\begin{eulerformula}
\[
x = \cos(x).
\]
\end{eulerformula}
\begin{eulercomment}
Iterasi ini juga dapat dilakukan pada interval, hasilnya adalah
barisan interval yang memuat akar tersebut.
\end{eulercomment}
\begin{eulerprompt}
>hasil := iterate("cos(x)",~1,2~) //iterasi x(n+1)=cos(x(n)), dengan interval awal (1, 2)
\end{eulerprompt}
\begin{euleroutput}
  ~0.739085133211,0.7390851332133~
\end{euleroutput}
\begin{eulercomment}
Jika interval hasil tersebut sedikit diperlebar, akan terlihat bahwa interval tersebut
memuat akar persamaan x=cos(x).
\end{eulercomment}
\begin{eulerprompt}
>h=expand(hasil,100), cos(h) << h
\end{eulerprompt}
\begin{euleroutput}
  ~0.73908513309,0.73908513333~
  1
\end{euleroutput}
\begin{eulercomment}
Iterasi juga dapat digunakan pada fungsi yang didefinisikan.
\end{eulercomment}
\begin{eulerprompt}
>function f(x) := (x+2/x)/2
\end{eulerprompt}
\begin{eulercomment}
Iterasi x(n+1)=f(x(n)) akan konvergen ke akar kuadrat 2.
\end{eulercomment}
\begin{eulerprompt}
>iterate("f",2), sqrt(2)
\end{eulerprompt}
\begin{euleroutput}
  1.41421356237
  1.41421356237
\end{euleroutput}
\begin{eulercomment}
Jika pada perintah iterate diberikan tambahan parameter n, maka hasil iterasinya akan
ditampilkan mulai dari iterasi pertama sampai ke-n.
\end{eulercomment}
\begin{eulerprompt}
>iterate("f",2,5)
\end{eulerprompt}
\begin{euleroutput}
  [2,  1.5,  1.41667,  1.41422,  1.41421,  1.41421]
\end{euleroutput}
\begin{eulercomment}
Untuk iterasi ini tidak dapat dilakukan terhadap interval.
\end{eulercomment}
\begin{eulerprompt}
>niterate("f",~1,2~,5)
\end{eulerprompt}
\begin{euleroutput}
  [ ~1,2~,  ~1,2~,  ~1,2~,  ~1,2~,  ~1,2~,  ~1,2~ ]
\end{euleroutput}
\begin{eulercomment}
Perhatikan, hasil iterasinya sama dengan interval awal. Alasannya adalah perhitungan dengan
interval bersifat terlalu longgar. Untuk meingkatkan perhitungan pada ekspresi dapat
digunakan pembagian intervalnya, menggunakan fungsi ieval().
\end{eulercomment}
\begin{eulerprompt}
>function s(x) := ieval("(x+2/x)/2",x,10)
\end{eulerprompt}
\begin{eulercomment}
Selanjutnya dapat dilakukan iterasi hingga diperoleh hasil optimal,
dan intervalnya tidak semakin mengecil. Hasilnya berupa interval yang
memuat akar persamaan:

\end{eulercomment}
\begin{eulerformula}
\[
x = \frac{1}{2} \left( x + \frac{2}{x} \right).
\]
\end{eulerformula}
\begin{eulercomment}
Satu-satunya solusi adalah\\
\end{eulercomment}
\begin{eulerformula}
\[
x = \sqrt2.
\]
\end{eulerformula}
\begin{eulerprompt}
>iterate("s",~1,2~)
\end{eulerprompt}
\begin{euleroutput}
  ~1.41421356236,1.41421356239~
\end{euleroutput}
\begin{eulercomment}
Fungsi "iterate()" juga dapat bekerja pada vektor. Berikut adalah
contoh fungsi vektor, yang menghasilkan rata-rata aritmetika dan
rata-rata geometri.

\end{eulercomment}
\begin{eulerformula}
\[
(a_{n+1},b_{n+1}) = \left( \frac{a_n+b_n}{2}, \sqrt{a_nb_n} \right)
\]
\end{eulerformula}
\begin{eulercomment}
Iterasi ke-n disimpan pada vektor kolom x[n].
\end{eulercomment}
\begin{eulerprompt}
>function g(x) := [(x[1]+x[2])/2;sqrt(x[1]*x[2])]
\end{eulerprompt}
\begin{eulercomment}
Iterasi dengan menggunakan fungsi tersebut akan konvergen ke rata-rata aritmetika dan
geometri dari nilai-nilai awal. 
\end{eulercomment}
\begin{eulerprompt}
>iterate("g",[1;5])
\end{eulerprompt}
\begin{euleroutput}
        2.60401 
        2.60401 
\end{euleroutput}
\begin{eulercomment}
Hasil tersebut konvergen agak cepat, seperti kita cek sebagai berikut.
\end{eulercomment}
\begin{eulerprompt}
>iterate("g",[1;5],4)
\end{eulerprompt}
\begin{euleroutput}
              1             3       2.61803       2.60403       2.60401 
              5       2.23607       2.59002       2.60399       2.60401 
\end{euleroutput}
\begin{eulercomment}
Iterasi pada interval dapat dilakukan dan stabil, namun tidak menunjukkan bahwa limitnya
pada batas-batas yang dihitung.
\end{eulercomment}
\begin{eulerprompt}
>iterate("g",[~1~;~5~],4)
\end{eulerprompt}
\begin{euleroutput}
  Interval 2 x 5 matrix
  
  ~0.999999999999999778,1.00000000000000022~     ...
  ~4.99999999999999911,5.00000000000000089~     ...
\end{euleroutput}
\begin{eulercomment}
Iterasi berikut konvergen sangat lambat.

\end{eulercomment}
\begin{eulerformula}
\[
x_{n+1} = \sqrt{x_n}.
\]
\end{eulerformula}
\begin{eulerprompt}
>iterate("sqrt(x)",2,10)
\end{eulerprompt}
\begin{euleroutput}
  [2,  1.41421,  1.18921,  1.09051,  1.04427,  1.0219,  1.01089,  1.00543,  1.00271,  1.00135,
  1.00068]
\end{euleroutput}
\begin{eulercomment}
Kekonvergenan iterasi tersebut dapat dipercepatdengan percepatan Steffenson:
\end{eulercomment}
\begin{eulerprompt}
>steffenson("sqrt(x)",2,10)
\end{eulerprompt}
\begin{euleroutput}
  [1.04888,  1.00028,  1,  1]
\end{euleroutput}
\eulerheading{Iterasi menggunakan Loop yang ditulis Langsung}
\begin{eulercomment}
Berikut adalah beberapa contoh penggunaan loop untuk melakukan iterasi yang ditulis langsung
pada baris perintah.
\end{eulercomment}
\begin{eulerprompt}
>x=2; repeat x=(x+2/x)/2; until x^2~=2; end; x,
\end{eulerprompt}
\begin{euleroutput}
  1.41421356237
\end{euleroutput}
\begin{eulercomment}
Penggabungan matriks menggunakan tanda "\textbar{}" dapat digunakan untuk menyimpan semua hasil
iterasi.
\end{eulercomment}
\begin{eulerprompt}
>v=[1]; for i=2 to 8; v=v|(v[i-1]*i); end; v,
\end{eulerprompt}
\begin{euleroutput}
  [1,  2,  6,  24,  120,  720,  5040,  40320]
\end{euleroutput}
\begin{eulercomment}
hasil iterasi juga dapat disimpan pada vektor yang sudah ada.
\end{eulercomment}
\begin{eulerprompt}
>v=ones(1,100); for i=2 to cols(v); v[i]=v[i-1]*i; end; ...
>plot2d(v,logplot=1); textbox(latex(&log(n)),x=0.5):
>A =[0.5,0.2;0.7,0.1]; b=[2;2]; ...
>x=[1;1]; repeat xnew=A.x-b; until all(xnew~=x); x=xnew; end; ...
>x,
\end{eulerprompt}
\begin{euleroutput}
       -7.09677 
       -7.74194 
\end{euleroutput}
\eulerheading{Iterasi di dalam Fungsi}
\begin{eulercomment}
Fungsi atau program juga dapat menggunakan iterasi dan dapat digunakan untuk melakukan iterasi. Berikut adalah beberapa contoh
iterasi di dalam fungsi.

Contoh berikut adalah suatu fungsi untuk menghitung berapa lama suatu iterasi konvergen. Nilai fungsi tersebut adalah hasil akhir
iterasi dan banyak iterasi sampai konvergen.
\end{eulercomment}
\begin{eulerprompt}
>function map hiter(f$,x0) ...
\end{eulerprompt}
\begin{eulerudf}
  x=x0;
  maxiter=0;
  repeat
    xnew=f$(x);
    maxiter=maxiter+1;
    until xnew~=x;
    x=xnew;
  end;
  return maxiter;
  endfunction
\end{eulerudf}
\begin{eulercomment}
Misalnya, berikut adalah iterasi untuk mendapatkan hampiran akar kuadrat 2, cukup cepat,
konvergen pada iterasi ke-5, jika dimulai dari hampiran awal 2.
\end{eulercomment}
\begin{eulerprompt}
>hiter("(x+2/x)/2",2)
\end{eulerprompt}
\begin{euleroutput}
  5
\end{euleroutput}
\begin{eulercomment}
Karena fungsinya didefinisikan menggunakan "map". maka nilai awalnya dapat berupa vektor.
\end{eulercomment}
\begin{eulerprompt}
>x=1.5:0.1:10; hasil=hiter("(x+2/x)/2",x); ...
>  plot2d(x,hasil):
\end{eulerprompt}
\begin{eulercomment}
Dari gambar di atas terlihat bahwa kekonvergenan iterasinya semakin lambat, untuk nilai awal
semakin besar, namun penambahnnya tidak kontinu. Kita dapat menemukan kapan maksimum
iterasinya bertambah.
\end{eulercomment}
\begin{eulerprompt}
>hasil[1:10]
\end{eulerprompt}
\begin{euleroutput}
  [4,  5,  5,  5,  5,  5,  6,  6,  6,  6]
\end{euleroutput}
\begin{eulerprompt}
>x[nonzeros(differences(hasil))]
\end{eulerprompt}
\begin{euleroutput}
  [1.5,  2,  3.4,  6.6]
\end{euleroutput}
\begin{eulercomment}
maksimum iterasi sampai konvergen meningkat pada saat nilai awalnya 1.5, 2, 3.4, dan 6.6.

Contoh berikutnya adalah metode Newton pada polinomial kompleks berderajat 3.
\end{eulercomment}
\begin{eulerprompt}
>p &= x^3-1; newton &= x-p/diff(p,x); $newton
\end{eulerprompt}
\begin{eulercomment}
Selanjutnya didefinisikan fungsi untuk melakukan iterasi (aslinya 10 kali).
\end{eulercomment}
\begin{eulerprompt}
>function iterasi(f$,x,n=10) ...
\end{eulerprompt}
\begin{eulerudf}
  loop 1 to n; x=f$(x); end;
  return x;
  endfunction
\end{eulerudf}
\begin{eulercomment}
Kita mulai dengan menentukan titik-titik grid pada bidang kompleksnya.
\end{eulercomment}
\begin{eulerprompt}
>r=1.5; x=linspace(-r,r,501); Z=x+I*x'; W=iterasi(newton,Z);
\end{eulerprompt}
\begin{eulercomment}
Berikut adalah akar-akar polinomial di atas.
\end{eulercomment}
\begin{eulerprompt}
>z=&solve(p)()
\end{eulerprompt}
\begin{euleroutput}
  [ -0.5+0.866025i,  -0.5-0.866025i,  1+0i  ]
\end{euleroutput}
\begin{eulercomment}
Untuk menggambar hasil iterasinya, dihitung jarak dari hasil iterasi ke-10 ke masing-masing
akar, kemudian digunakan untuk menghitung warna yang akan digambar, yang menunjukkan limit
untuk masing-masing nilai awal. 

Fungsi plotrgb() menggunakan jendela gambar terkini untuk menggambar warna RGB sebagai
matriks.
\end{eulercomment}
\begin{eulerprompt}
>C=rgb(max(abs(W-z[1]),1),max(abs(W-z[2]),1),max(abs(W-z[3]),1)); ...
>  plot2d(none,-r,r,-r,r); plotrgb(C):
\end{eulerprompt}
\eulerheading{Iterasi Simbolik}
\begin{eulercomment}
Seperti sudah dibahas sebelumnya, untuk menghasilkan barisan ekspresi simbolik dengan Maxima
dapat digunakan fungsi makelist().
\end{eulercomment}
\begin{eulerprompt}
>&powerdisp:true // untuk menampilkan deret pangkat mulai dari suku berpangkat terkecil
\end{eulerprompt}
\begin{euleroutput}
  
                                   true
  
\end{euleroutput}
\begin{eulerprompt}
>deret &= makelist(taylor(exp(x),x,0,k),k,1,3); $deret // barisan deret Taylor untuk e^x
\end{eulerprompt}
\begin{eulercomment}
Untuk mengubah barisan deret tersebut menjadi vektor string di EMT digunakan fungsi
mxm2str(). Selanjutnya, vektor string/ekspresi hasilnya dapat digambar seperti menggambar
vektor eskpresi pada EMT.
\end{eulercomment}
\begin{eulerprompt}
>plot2d("exp(x)",0,3); // plot fungsi aslinya, e^x
>plot2d(mxm2str("deret"),>add,color=4:6): // plot ketiga deret taylor hampiran fungsi tersebut
\end{eulerprompt}
\begin{eulercomment}
Selain cara di atas dapat juga dengan cara menggunakan indeks pada vektor/list yang
dihasilkan.
\end{eulercomment}
\begin{eulerprompt}
>$deret[3]
>plot2d(["exp(x)",&deret[1],&deret[2],&deret[3]],0,3,color=1:4):
>$sum(sin(k*x)/k,k,1,5)
\end{eulerprompt}
\begin{eulercomment}
Berikut adalah cara menggambar kurva

\end{eulercomment}
\begin{eulerformula}
\[
y=\sin(x) + \dfrac{\sin 3x}{3} + \dfrac{\sin 5x}{5} + \ldots.
\]
\end{eulerformula}
\begin{eulerprompt}
>plot2d(&sum(sin((2*k+1)*x)/(2*k+1),k,0,20),0,2pi):
\end{eulerprompt}
\begin{eulercomment}
Hal serupa juga dapat dilakukan dengan menggunakan matriks, misalkan
kita akan menggambar kurva

\end{eulercomment}
\begin{eulerformula}
\[
y = \sum_{k=1}^{100} \dfrac{\sin(kx)}{k},\quad 0\le x\le 2\pi.
\]
\end{eulerformula}
\begin{eulercomment}
\end{eulercomment}
\begin{eulerprompt}
>x=linspace(0,2pi,1000); k=1:100; y=sum(sin(k*x')/k)'; plot2d(x,y):
\end{eulerprompt}
\eulerheading{Tabel Fungsi}
\begin{eulercomment}
Terdapat cara menarik untuk menghasilkan barisan dengan ekspresi
Maxima. Perintah mxmtable() berguna untuk menampilkan dan menggambar
barisan dan menghasilkan barisan sebagai vektor kolom.

Sebagai contoh berikut adalah barisan turunan ke-n x\textasciicircum{}x di x=1.
\end{eulercomment}
\begin{eulerprompt}
>mxmtable("diffat(x^x,x=1,n)","n",1,8,frac=1);
\end{eulerprompt}
\begin{euleroutput}
          1 
          2 
          3 
          8 
         10 
         54 
        -42 
        944 
\end{euleroutput}
\begin{eulerprompt}
>$'sum(k, k, 1, n) = factor(ev(sum(k, k, 1, n),simpsum=true)) // simpsum:menghitung deret secara simbolik
>$'sum(1/(3^k+k), k, 0, inf) = factor(ev(sum(1/(3^k+k), k, 0, inf),simpsum=true))
\end{eulerprompt}
\begin{eulercomment}
Di sini masih gagal, hasilnya tidak dihitung.
\end{eulercomment}
\begin{eulerprompt}
>$'sum(1/x^2, x, 1, inf)= ev(sum(1/x^2, x, 1, inf),simpsum=true) // ev: menghitung nilai ekspresi
>$'sum((-1)^(k-1)/k, k, 1, inf) = factor(ev(sum((-1)^(x-1)/x, x, 1, inf),simpsum=true))
\end{eulerprompt}
\begin{eulercomment}
Di sini masih gagal, hasilnya tidak dihitung.
\end{eulercomment}
\begin{eulerprompt}
>$'sum((-1)^k/(2*k-1), k, 1, inf) = factor(ev(sum((-1)^k/(2*k-1), k, 1, inf),simpsum=true))
>$ev(sum(1/n!, n, 0, inf),simpsum=true)
\end{eulerprompt}
\begin{eulercomment}
Di sini masih gagal, hasilnya tidak dihitung, harusnya hasilnya e.
\end{eulercomment}
\begin{eulerprompt}
>&assume(abs(x)<1); $'sum(a*x^k, k, 0, inf)=ev(sum(a*x^k, k, 0, inf),simpsum=true), &forget(abs(x)<1);
\end{eulerprompt}
\begin{eulercomment}
Deret geometri tak hingga, dengan asumsi rasional antara -1 dan 1.
\end{eulercomment}
\begin{eulerprompt}
>$'sum(x^k/k!,k,0,inf)=ev(sum(x^k/k!,k,0,inf),simpsum=true)
>$limit(sum(x^k/k!,k,0,n),n,inf)
>function d(n) &= sum(1/(k^2-k),k,2,n); $'d(n)=d(n)
>$d(10)=ev(d(10),simpsum=true)
>$d(100)=ev(d(100),simpsum=true)
\end{eulerprompt}
\eulerheading{Deret Taylor}
\begin{eulercomment}
Deret Taylor suatu fungsi f yang diferensiabel sampai tak hingga di
sekitar x=a adalah:

\end{eulercomment}
\begin{eulerformula}
\[
f(x) = \sum_{k=0}^\infty \frac{(x-a)^k f^{(k)}(a)}{k!}.
\]
\end{eulerformula}
\begin{eulerprompt}
>$'e^x =taylor(exp(x),x,0,10) // deret Taylor e^x di sekitar x=0, sampai suku ke-11
>exp(1)
\end{eulerprompt}
\begin{euleroutput}
  2.71828182846
\end{euleroutput}
\begin{eulerprompt}
>$'log(x)=taylor(log(x),x,1,10)// deret log(x) di sekitar x=1
>log(2)
\end{eulerprompt}
\begin{euleroutput}
  0.69314718056
\end{euleroutput}
\begin{eulerprompt}
>$'sin(x)=taylor(sin(x),x,1,5)
>sin(1)
\end{eulerprompt}
\begin{euleroutput}
  0.841470984808
\end{euleroutput}
\begin{eulercomment}
\begin{eulercomment}
\eulerheading{FUNGSI MULTIVARIABEL}
\begin{eulercomment}
Sebuah fungsi multivariabel adalah pemetaan matematis yang
menghubungkan beberapa variabel independen dengan satu variabel
dependen. Fungsi ini umumnya dinotasikan sebagai

\end{eulercomment}
\begin{eulerformula}
\[
f(x,y) \text{ atau } f(x_1,x_2,...,x_n)
\]
\end{eulerformula}
\begin{eulerformula}
\[
\text{ dimana } x,y,x_1,x_2,...,x_n \text{ adalah variabel independen}
\]
\end{eulerformula}
\begin{eulerformula}
\[
\text{dan f adalah variabel dependen.}
\]
\end{eulerformula}
\begin{eulercomment}
Fungsi multivariabel dapat diwakili dalam bentuk peta atau grafik tiga
dimensi.

Contoh fungsi multivariabel :

\end{eulercomment}
\begin{eulerformula}
\[
z = x^2 + y^2
\]
\end{eulerformula}
\begin{eulercomment}
Dimana variabel bebasnya yaitu x dan y.

\begin{eulercomment}
\eulerheading{Grafik Fungsi Multivariabel}
\begin{eulercomment}
Pada fungsi multivariabel, grafik fungsinya merupakan grafik tiga
dimensi. Ruang dimensi tiga dilambangkan dengan\\
\end{eulercomment}
\begin{eulerformula}
\[
R^3
\]
\end{eulerformula}
\begin{eulercomment}
Permukaan dalam R\textasciicircum{}3 terdapat dua macam yakni permukaan linear dan
kuadratik. Setiap permukaan linear berupa bidang datar, sedangkan
permukaan kuadratik berupa bidang lengkung yang kelengkungannya
bergantung atas bentuk persamaannya.

Untuk membuat grafik fungsi tiga dimensi, maka kita dapat menggunakan
perintah "plot3d()"

\end{eulercomment}
\eulersubheading{Grafik Fungsi Persamaan Linear Dalam Dimensi Tiga}
\begin{eulercomment}
Bentuk umum persamaan permukaan linear adalah\\
\end{eulercomment}
\begin{eulerformula}
\[
Ax+By+Cz+D=0
\]
\end{eulerformula}
\begin{eulercomment}
Contoh grafik persamaan linear di ruang tiga dimensi


\end{eulercomment}
\begin{eulerformula}
\[
x-1
\]
\end{eulerformula}
\begin{eulerprompt}
>plot3d("x-1"):
>plot3d("3*x+4*y-12"):
>plot3d("7*x-1"):
\end{eulerprompt}
\eulersubheading{Grafik Fungsi Kuadratik di Ruang Dimensi Tiga}
\begin{eulercomment}
Persamaan kuadratik mempunyai rumus umum :\\
\end{eulercomment}
\begin{eulerformula}
\[
Ax^2+By^2+Cz^2+Dxy+Exz+Fyz+Gx+Hy+Iz+J=0
\]
\end{eulerformula}
\begin{eulercomment}
Permukaan-permukaan kuadratik dapat berupa permukaan bola, ellipsoida,
paraboloida, tabung ellips, tabung lingkaran, atau tabung parabola.

Contoh grafik fungsi kuadratik di ruang dimensi tiga:

\end{eulercomment}
\begin{eulerformula}
\[
x^2+y^2
\]
\end{eulerformula}
\begin{eulerprompt}
>plot3d("x^2+y^2"):
\end{eulerprompt}
\begin{eulercomment}
Pada contoh diatas, digunakan ekspresi langsung dalam fungsi plot3d
dan permukaan tersebut berupa paraboloida.

\end{eulercomment}
\begin{eulerformula}
\[
e^{-(x^2+y^2)}
\]
\end{eulerformula}
\begin{eulerprompt}
>function f(x,y) := exp(-(x^2+y^2))
>plot3d("f", r=5):
>plot3d("y^2-x^2",r=2,):
\end{eulerprompt}
\begin{eulercomment}
Gambar di atas merupakan paraboloida hiperbolik


\end{eulercomment}
\eulersubheading{Menggambar kurva perpotongan dari dua persamaan}
\begin{eulercomment}
Di EMT kita juga dapat menggabungkan dua kurva pada satu bidang untuk
menggambarkan perpotongan. Untuk masalah ini kita gunakan fungsi \textgreater{}add
dalam prosesnya.

Contoh 1:\\
\end{eulercomment}
\begin{eulerformula}
\[
x^2+y^2+z-4=0
\]
\end{eulerformula}
\begin{eulerformula}
\[
\text{dengan}
\]
\end{eulerformula}
\begin{eulerformula}
\[
x^2+y^2=1
\]
\end{eulerformula}
\begin{eulerprompt}
>plot3d("x^2+y^2+z-4",r=5, implicit=3):
>plot3d("x^2+y^2-1",implicit=3, r=5, >add):
\end{eulerprompt}
\begin{eulercomment}
Dari persamaan diatas, kita dapatkan perpotongan antara paraboloida
dengan tabung lingkaran.

contoh 2:\\
\end{eulercomment}
\begin{eulerformula}
\[
x^2+y^2+2z=9
\]
\end{eulerformula}
\begin{eulerformula}
\[
\text{ dengan }
\]
\end{eulerformula}
\begin{eulerformula}
\[
y=1
\]
\end{eulerformula}
\begin{eulerprompt}
>plot3d("x^2+y^2+2z-9",r=5,implicit=3):
>plot3d("y-1",r=5, implicit=3, >add):
\end{eulerprompt}
\begin{eulercomment}
Dari persamaan diatas, didapatkan perpotongan antara bidang datar dan
bidang lengkung.

contoh 3:\\
\end{eulercomment}
\begin{eulerformula}
\[
x^2+y^2=4
\]
\end{eulerformula}
\begin{eulerformula}
\[
\text{ dengan }
\]
\end{eulerformula}
\begin{eulerformula}
\[
y^2+z^2=4
\]
\end{eulerformula}
\begin{eulercomment}
\end{eulercomment}
\begin{eulerprompt}
>plot3d("x^2+y^2-4",r=5,implicit=3); plot3d("y^2+z^2-4",r=5,implicit=3,>add):
\end{eulerprompt}
\begin{eulercomment}
Didapatkan perpotongan antara dua tabung lingkaran.


\begin{eulercomment}
\eulerheading{Turunan Fungsi Multivariabel}
\begin{eulercomment}
\end{eulercomment}
\eulersubheading{Turunan Fungsi Dua Variabel}
\begin{eulercomment}
Turunan parsial, yaitu turunan fungsi terhadap satu variabel bebas
sementara variabel bebas lainnya dianggap tetap atau konstan.

- Turunan Parsial terhadap f terhadap x di (x0,y0)

\end{eulercomment}
\begin{eulerformula}
\[
f_x(x_0,y_0)= \lim_{\Delta x \to 0} \frac{f(x_0+\Delta x, y_0)-f(x_0,y_0)}{\Delta x}
\]
\end{eulerformula}
\begin{eulercomment}
- Turunan Parsial f terhadap y di (x0,y0)

\end{eulercomment}
\begin{eulerformula}
\[
f_y(x_0,y_0)= \lim_{\Delta y \to 0} \frac{f(x_0, y_0+\Delta y)-f(x_0,y_0)}{\Delta y}
\]
\end{eulerformula}
\begin{eulercomment}
Contoh soal untuk turunan parsial :

\end{eulercomment}
\begin{eulerformula}
\[
f(x,y)=2xy-(1-x^2)
\]
\end{eulerformula}
\begin{eulerprompt}
>z &= 2*x*y-(1-x^2)
\end{eulerprompt}
\begin{euleroutput}
  
                                    2
                             - 1 + x  + 2 x y
  
\end{euleroutput}
\begin{eulerprompt}
>$showev('limit(((2*(x+h)*y)-(1-(x+h)^2))/h,h,0))
\end{eulerprompt}
\begin{eulercomment}
Perhitungan akan dilakukan menggunakan diff
\end{eulercomment}
\begin{eulerprompt}
>&diff(z,x) // z akan diturunkan terhadap x
\end{eulerprompt}
\begin{euleroutput}
  
                                2 x + 2 y
  
\end{euleroutput}
\begin{eulercomment}
Karena pada fungsi z terdapat 2xy yang merupakan perkalian, untuk
menghitung turunannya kita gunakan u'v+uv', sehingga turunan dari 2xy:

\end{eulercomment}
\begin{eulerformula}
\[
2.1.y+2.x.0=2y
\]
\end{eulerformula}
\begin{eulercomment}
Kemudian turunan dari variabel berpangkat yaitu:\\
\end{eulercomment}
\begin{eulerformula}
\[
u^n=n.u^{n-1}.u'
\]
\end{eulerformula}
\begin{eulercomment}
Jadi turunan dari x\textasciicircum{}2:\\
\end{eulercomment}
\begin{eulerformula}
\[
x^2=2.x^{2-1}.1=2x
\]
\end{eulerformula}
\begin{eulercomment}
Turunan dari konstanta adalah 0\\
sehingga, turunan dari fungsi\\
\end{eulercomment}
\begin{eulerformula}
\[
f(x,y)=2xy+x^2-1
\]
\end{eulerformula}
\begin{eulercomment}
terhadap x adalah\\
\end{eulercomment}
\begin{eulerformula}
\[
f_x(x,y)=2y+2x
\]
\end{eulerformula}
\begin{eulercomment}
\end{eulercomment}
\begin{eulerprompt}
>&diff(z,y) // z akan diturunkan terhadap y
\end{eulerprompt}
\begin{euleroutput}
  
                                   2 x
  
\end{euleroutput}
\begin{eulercomment}
Seperti pada turunan terhadap x, kita gunakan langkah yang sama namun
kita anggap x konstan.

\end{eulercomment}
\begin{eulerformula}
\[
f(x,y)=2xy+x^2-1
\]
\end{eulerformula}
\begin{eulerformula}
\[
f_y(x,y)=2.x.1+0-0
\]
\end{eulerformula}
\begin{eulerformula}
\[
f_y(x,y)=2x
\]
\end{eulerformula}
\begin{eulercomment}
Jadi, turunan terhadap y dari f(x,y) adalah 2x.


Sehingga, turunan parsial dari\\
\end{eulercomment}
\begin{eulerformula}
\[
f(x,y)=2xy+x^2-1 \text{ adalah }
\]
\end{eulerformula}
\begin{eulerformula}
\[
f_x(x,y)=2x+2y
\]
\end{eulerformula}
\begin{eulerformula}
\[
f_y(x,y)=2x
\]
\end{eulerformula}
\begin{eulercomment}
Apabila kita gambarkan grafik untuk masing-masing turunan tersebut
adalah sebagai berikut:
\end{eulercomment}
\begin{eulerprompt}
>plot3d("2*x+2*y"): // turunan terhadap x
>plot3d("2*x"): // turunan terhadap y
\end{eulerprompt}
\eulerheading{Integral}
\begin{eulercomment}
\end{eulercomment}
\eulersubheading{Integral Lipat Dua}
\begin{eulercomment}
Integral lipat dua f pada R, diberikan oleh\\
\end{eulercomment}
\begin{eulerformula}
\[
\int_{a}^{b} \int_{c}^{d} f(x, y) \, dx \, dy
\]
\end{eulerformula}
\begin{eulercomment}
atau bisa ditulis dengan\\
\end{eulercomment}
\begin{eulerformula}
\[
\int _R \int f(x,y) \, dA
\]
\end{eulerformula}
\begin{eulercomment}
Misalkan R adalah sebuah persegipanjang dengan sisi-sisi sejajar
sumbu-sumbu koordinat; yaitu, misalkan\\
\end{eulercomment}
\begin{eulerformula}
\[
R= \{(x,y): a \le x \le b, c \le y \le d \}
\]
\end{eulerformula}
\begin{eulercomment}
Urutan dx dan dy penting karena ia merinci integrasi mana yang akan
dilakukan pertama.

SIFAT-SIFAT INTEGRAL LIPAT DUA:

1. Integral lipat dua bersifat linear, yaitu:\\
\end{eulercomment}
\begin{eulerformula}
\[
\, a. \iint_{R} kf(x,y) \, dA = k \iint_{R} f(x,y) \, dA
\]
\end{eulerformula}
\begin{eulerformula}
\[
\, b. \iint_{R}[f(x,y)+g(x,y)]\, dA = \iint_{R} f(x,y) \, dA + \iint_{R} g(x,y) \, dA
\]
\end{eulerformula}
\begin{eulercomment}
2. Integral lipat dua bersifat aditif(dapat dijumlahkan) pada
persegipanjang yang saling berhimpit pada hanya sebuah ruas garis\\
\end{eulercomment}
\begin{eulerformula}
\[
\iint_{R} f(x,y) \, dA = \iint_{R_1} f(x,y) \, dA + \iint_{R_2} f(x,y) \, dA
\]
\end{eulerformula}
\begin{eulercomment}
3. Sifat perbandingan berlaku jika\\
\end{eulercomment}
\begin{eulerformula}
\[
f(x,y) \le g(x,y) \text{ untuk semua } (x,y) \text{ di R maka }
\]
\end{eulerformula}
\begin{eulerformula}
\[
\iint_{R} f(x,y) \, dA \le \iint_{R} g(x,y) \, dA
\]
\end{eulerformula}
\begin{eulercomment}
CONTOH :\\
1. Hitung\\
\end{eulercomment}
\begin{eulerformula}
\[
\iint (2xy) \, dx \, dy
\]
\end{eulerformula}
\begin{eulercomment}
Dengan memperhatikan urutan dx dan dy, pada soal kali ini kita hitung
integralnya dari sebelah dalam yaitu kita integralkan terhadap x
terlebih dahulu.
\end{eulercomment}
\begin{eulerprompt}
>qx &= integrate((2*x*y),x)
\end{eulerprompt}
\begin{euleroutput}
  
                                    2
                                   x  y
  
\end{euleroutput}
\begin{eulercomment}
Setelah kita integralkan terhadap x, lalu hasilnya kita integralkan
terhadap y.
\end{eulercomment}
\begin{eulerprompt}
>qy &= integrate(qx,y)
\end{eulerprompt}
\begin{euleroutput}
  
                                   2  2
                                  x  y
                                  -----
                                    2
  
\end{euleroutput}
\begin{eulercomment}
Jadi, hasil dari\\
\end{eulercomment}
\begin{eulerformula}
\[
\iint (2xy) \, dx \, dy
\]
\end{eulerformula}
\begin{eulercomment}
adalah\\
\end{eulercomment}
\begin{eulerformula}
\[
\frac{x^2y^2}{2}+c
\]
\end{eulerformula}
\begin{eulercomment}
2. Hitung\\
\end{eulercomment}
\begin{eulerformula}
\[
\int_{0}^{3} \left[ \int_{1}^{2} (2x+3y) \, dx \right] \, dy
\]
\end{eulerformula}
\begin{eulerprompt}
>f &= 2*x+3*y
\end{eulerprompt}
\begin{euleroutput}
  
                                2 x + 3 y
  
\end{euleroutput}
\begin{eulercomment}
Dalam integral sebelah dalam, y berupa konstanta, sehingga
\end{eulercomment}
\begin{eulerprompt}
>fx &= integrate((2*x+3*y),x,1,2)
\end{eulerprompt}
\begin{euleroutput}
  
                                 3 + 3 y
  
\end{euleroutput}
\begin{eulercomment}
Setelah mencari integral terhadap x, lalu kita integralkan hasil
tersebut terhadap y, akibatnya
\end{eulercomment}
\begin{eulerprompt}
>fy &= integrate((3+3*y),y,0,3)
\end{eulerprompt}
\begin{euleroutput}
  
                                    45
                                    --
                                    2
  
\end{euleroutput}
\eulerheading{Aplikasi Fungsi Multivariabel}
\begin{eulercomment}
\end{eulercomment}
\eulersubheading{Penerapan Plot Kontur}
\begin{eulercomment}
Peta kontur seringkali digunakan untuk memperlihatkan kondisi cuaca
atau lainnya dari berbagai titik dalam peta.

1.Buatlah plot kontur dari fungsi berikut

\end{eulercomment}
\begin{eulerformula}
\[
f(x,y) = \frac{1}{2}(x^2+y^2) \text{  dengan level bilangan genap dari 0 sampai 20}
\]
\end{eulerformula}
\begin{eulerprompt}
>f &= 1/2*(x^2+y^2)
\end{eulerprompt}
\begin{euleroutput}
  
                                  2    2
                                 x  + y
                                 -------
                                    2
  
\end{euleroutput}
\begin{eulerprompt}
>aspect(1.5);
>plot2d(f,level=0:2:20,>hue,>spectral,n=200,grid=4,r=5):
\end{eulerprompt}
\begin{eulercomment}
2. Seorang ahli geologi sedang melakukan penelitian di daerah
vulkanik. Dia tertarik untuk memahami bagaimana ketinggian permukaan
tanah di sekitar gunung berapi berubah. Ketinggian di suatu titik (x,
y) dalam kilometer di sekitar gunung berapi tersebut dapat dijelaskan
oleh fungsi ketinggian H(x, y), di mana x dan y adalah koordinat dalam
kilometer. Fungsi ketinggian H(x, y) diberikan oleh persamaan:

\end{eulercomment}
\begin{eulerformula}
\[
H(x,y)=3x^2-2y^2
\]
\end{eulerformula}
\begin{eulercomment}
Gambarkan plot kontur dari fungsi ketinggian H(x, y) untuk
memvisualisasikan perubahan ketinggian di sekitar gunung berapi dengan
level ketinggian 1 sampai 7 km.

\end{eulercomment}
\begin{eulerprompt}
>h &= 3*x^2-2*y^2
\end{eulerprompt}
\begin{euleroutput}
  
                                  2      2
                               3 x  - 2 y
  
\end{euleroutput}
\begin{eulerprompt}
>aspect(1.5);
>plot2d(h,level=1:1:7,>hue,>spectral,n=200,grid=4,r=2):
\end{eulerprompt}
\begin{euleroutput}
  
\end{euleroutput}
\eulerheading{Visualisasi dan Perhitungan Geometri dengan EMT}
\begin{eulercomment}
Euler menyediakan beberapa fungsi untuk melakukan visualisasi dan perhitungan geometri, baik
secara numerik maupun analitik (seperti biasanya tentunya, menggunakan Maxima).
Fungsi-fungsi untuk visualisasi dan perhitungan geometeri tersebut disimpan di dalam file
program "geometry.e", sehingga file tersebut harus dipanggil sebelum menggunakan
fungsi-fungsi atau perintah-perintah untuk geometri.
\end{eulercomment}
\begin{eulerprompt}
>load geometry
\end{eulerprompt}
\begin{euleroutput}
  Numerical and symbolic geometry.
\end{euleroutput}
\eulersubheading{Fungsi-fungsi Geometri}
\begin{eulercomment}
Fungsi-fungsi untuk Menggambar Objek Geometri:

\end{eulercomment}
\begin{eulerttcomment}
  defaultd:=textheight()*1.5: nilai asli untuk parameter d
  setPlotrange(x1,x2,y1,y2): menentukan rentang x dan y pada bidang koordinat
  setPlotRange(r): pusat bidang koordinat (0,0) dan batas-batas sumbu-x dan y adalah -r sd r
  plotPoint (P, "P"): menggambar titik P dan diberi label "P"
  plotSegment (A,B, "AB", d): menggambar ruas garis AB, diberi label "AB" sejauh d
  plotLine (g, "g", d): menggambar garis g diberi label "g" sejauh d
  plotCircle (c,"c",v,d): Menggambar lingkaran c dan diberi label "c"
  plotLabel (label, P, V, d): menuliskan label pada posisi P
\end{eulerttcomment}
\begin{eulercomment}

Fungsi-fungsi Geometri Analitik (numerik maupun simbolik):

\end{eulercomment}
\begin{eulerttcomment}
  turn(v, phi): memutar vektor v sejauh phi
  turnLeft(v):   memutar vektor v ke kiri
  turnRight(v):  memutar vektor v ke kanan
  normalize(v): normal vektor v
  crossProduct(v, w): hasil kali silang vektorv dan w.
  lineThrough(A, B): garis melalui A dan B, hasilnya [a,b,c] sdh. ax+by=c.
  lineWithDirection(A,v): garis melalui A searah vektor v
  getLineDirection(g): vektor arah (gradien) garis g
  getNormal(g): vektor normal (tegak lurus) garis g
  getPointOnLine(g):  titik pada garis g
  perpendicular(A, g):  garis melalui A tegak lurus garis g
  parallel (A, g):  garis melalui A sejajar garis g
  lineIntersection(g, h):  titik potong garis g dan h
  projectToLine(A, g):   proyeksi titik A pada garis g
  distance(A, B):  jarak titik A dan B
  distanceSquared(A, B):  kuadrat jarak A dan B
  quadrance(A, B): kuadrat jarak A dan B
  areaTriangle(A, B, C):  luas segitiga ABC
  computeAngle(A, B, C):   besar sudut <ABC
  angleBisector(A, B, C): garis bagi sudut <ABC
  circleWithCenter (A, r): lingkaran dengan pusat A dan jari-jari r
  getCircleCenter(c):  pusat lingkaran c
  getCircleRadius(c):  jari-jari lingkaran c
  circleThrough(A,B,C):  lingkaran melalui A, B, C
  middlePerpendicular(A, B): titik tengah AB
  lineCircleIntersections(g, c): titik potong garis g dan lingkran c
  circleCircleIntersections (c1, c2):  titik potong lingkaran c1 dan c2
  planeThrough(A, B, C):  bidang melalui titik A, B, C
\end{eulerttcomment}
\begin{eulercomment}

Fungsi-fungsi Khusus Untuk Geometri Simbolik:

\end{eulercomment}
\begin{eulerttcomment}
  getLineEquation (g,x,y): persamaan garis g dinyatakan dalam x dan y
  getHesseForm (g,x,y,A): bentuk Hesse garis g dinyatakan dalam x dan y dengan titik A pada
  sisi positif (kanan/atas) garis
  quad(A,B): kuadrat jarak AB
  spread(a,b,c): Spread segitiga dengan panjang sisi-sisi a,b,c, yakni sin(alpha)^2 dengan
  alpha sudut yang menghadap sisi a.
  crosslaw(a,b,c,sa): persamaan 3 quads dan 1 spread pada segitiga dengan panjang sisi a, b, c.
  triplespread(sa,sb,sc): persamaan 3 spread sa,sb,sc yang memebntuk suatu segitiga
  doublespread(sa): Spread sudut rangkap Spread 2*phi, dengan sa=sin(phi)^2 spread a.
\end{eulerttcomment}
\begin{eulercomment}

\end{eulercomment}
\eulersubheading{Contoh 1: Luas, Lingkaran Luar, Lingkaran Dalam Segitiga}
\begin{eulercomment}
Untuk menggambar objek-objek geometri, langkah pertama adalah menentukan rentang sumbu-sumbu
koordinat. Semua objek geometri akan digambar pada satu bidang koordinat, sampai didefinisikan
bidang koordinat yang baru.
\end{eulercomment}
\begin{eulerprompt}
>setPlotRange(-0.5,2.5,-0.5,2.5); // mendefinisikan bidang koordinat baru 
\end{eulerprompt}
\begin{eulercomment}
Sekarang tetapkan tiga titik dan plotlah.
\end{eulercomment}
\begin{eulerprompt}
>A=[1,0]; plotPoint(A,"A"); // definisi dan gambar tiga titik
>B=[0,1]; plotPoint(B,"B");
>C=[2,2]; plotPoint(C,"C");
\end{eulerprompt}
\begin{eulercomment}
Lalu tiga segmen.
\end{eulercomment}
\begin{eulerprompt}
>plotSegment(A,B,"c"); // c=AB
>plotSegment(B,C,"a"); // a=BC
>plotSegment(A,C,"b"); // b=AC
\end{eulerprompt}
\begin{eulercomment}
Fungsi geometri mencakup fungsi untuk membuat garis dan lingkaran.
Format untuk garis adalah [a,b,c], yang merepresentasikan garis dengan
persamaan ax+by=c.
\end{eulercomment}
\begin{eulerprompt}
>lineThrough(B,C) // garis yang melalui B dan C
\end{eulerprompt}
\begin{euleroutput}
  [-1,  2,  2]
\end{euleroutput}
\begin{eulercomment}
Hitunglah garis tegak lurus melalui A pada BC.
\end{eulercomment}
\begin{eulerprompt}
>h=perpendicular(A,lineThrough(B,C)); // garis h tegak lurus BC melalui A
\end{eulerprompt}
\begin{eulercomment}
Dan persimpangannya dengan BC.
\end{eulercomment}
\begin{eulerprompt}
>D=lineIntersection(h,lineThrough(B,C)); // D adalah titik potong h dan BC
\end{eulerprompt}
\begin{eulercomment}
Gambarkan itu.
\end{eulercomment}
\begin{eulerprompt}
>plotPoint(D,value=1); // koordinat D ditampilkan
>aspect(1); plotSegment(A,D): // tampilkan semua gambar hasil plot...()
\end{eulerprompt}
\begin{eulercomment}
Hitung luas ABC:

\end{eulercomment}
\begin{eulerformula}
\[
L_{\triangle ABC}= \frac{1}{2}AD.BC.
\]
\end{eulerformula}
\begin{eulerprompt}
>norm(A-D)*norm(B-C)/2 // AD=norm(A-D), BC=norm(B-C)
\end{eulerprompt}
\begin{euleroutput}
  1.5
\end{euleroutput}
\begin{eulercomment}
Compare with determinant formula.
\end{eulercomment}
\begin{eulerprompt}
>areaTriangle(A,B,C) // hitung luas segitiga langusng dengan fungsi
\end{eulerprompt}
\begin{euleroutput}
  1.5
\end{euleroutput}
\begin{eulercomment}
Cara lain menghitung luas segitigas ABC:
\end{eulercomment}
\begin{eulerprompt}
>distance(A,D)*distance(B,C)/2
\end{eulerprompt}
\begin{euleroutput}
  1.5
\end{euleroutput}
\begin{eulercomment}
Sudut di C.
\end{eulercomment}
\begin{eulerprompt}
>degprint(computeAngle(B,C,A))
\end{eulerprompt}
\begin{euleroutput}
  36°52'11.63''
\end{euleroutput}
\begin{eulercomment}
Sekarang lingkaran luar segitiga.
\end{eulercomment}
\begin{eulerprompt}
>c=circleThrough(A,B,C); // lingkaran luar segitiga ABC
>R=getCircleRadius(c); // jari2 lingkaran luar 
>O=getCircleCenter(c); // titik pusat lingkaran c 
>plotPoint(O,"O"); // gambar titik "O"
>plotCircle(c,"Lingkaran luar segitiga ABC"):
\end{eulerprompt}
\begin{eulercomment}
Tampilkan koordinat titik pusat dan jari-jari lingkaran luar.
\end{eulercomment}
\begin{eulerprompt}
>O, R
\end{eulerprompt}
\begin{euleroutput}
  [1.16666666667,  1.16666666667]
  1.17851130198
\end{euleroutput}
\begin{eulercomment}
Sekarang akan digambar lingkaran dalam segitiga ABC. Titik pusat lingkaran dalam adalah
titik potong garis-garis bagi sudut.
\end{eulercomment}
\begin{eulerprompt}
>l=angleBisector(A,C,B); // garis bagi <ACB
>g=angleBisector(C,A,B); // garis bagi <CAB
>P=lineIntersection(l,g) // titik potong kedua garis bagi sudut
\end{eulerprompt}
\begin{euleroutput}
  [0.86038,  0.86038]
\end{euleroutput}
\begin{eulercomment}
Tambahkan semuanya ke dalam alur cerita.
\end{eulercomment}
\begin{eulerprompt}
>color(5); plotLine(l); plotLine(g); color(1); // gambar kedua garis bagi sudut
>plotPoint(P,"P"); // gambar titik potongnya
>r=norm(P-projectToLine(P,lineThrough(A,B))) // jari-jari lingkaran dalam
\end{eulerprompt}
\begin{euleroutput}
  0.509653732104
\end{euleroutput}
\begin{eulerprompt}
>plotCircle(circleWithCenter(P,r),"Lingkaran dalam segitiga ABC"): // gambar lingkaran dalam
\end{eulerprompt}
\eulersubheading{Latihan}
\begin{eulercomment}
1. Tentukan ketiga titik singgung lingkaran dalam dengan sisi-sisi
segitiga ABC.
\end{eulercomment}
\begin{eulerprompt}
>setPlotRange(-2.5,4.5,-2.5,4.5);
>A=[-2,1]; plotPoint(A,"A");
>B=[1,-2]; plotPoint(B,"B");
>C=[4,4]; plotPoint(C,"C");
\end{eulerprompt}
\begin{eulercomment}
2. Gambar segitiga dengan titik-titik sudut ketiga titik singgung
tersebut.
\end{eulercomment}
\begin{eulerprompt}
>plotSegment(A,B,"c")
>plotSegment(B,C,"a")
>plotSegment(A,C,"b")
>aspect(1):
\end{eulerprompt}
\begin{eulercomment}
3. Tunjukkan bahwa garis bagi sudut yang ke tiga juga melalui titik
pusat lingkaran dalam.
\end{eulercomment}
\begin{eulerprompt}
>l=angleBisector(A,C,B);
>g=angleBisector(C,A,B);
>P=lineIntersection(l,g)
\end{eulerprompt}
\begin{euleroutput}
  [0.581139,  0.581139]
\end{euleroutput}
\begin{eulerprompt}
>color(5); plotLine(l); plotLine(g); color(1);
>plotPoint(P,"P");
>r=norm(P-projectToLine(P,lineThrough(A,B)))
\end{eulerprompt}
\begin{euleroutput}
  1.52896119631
\end{euleroutput}
\begin{eulerprompt}
>plotCircle(circleWithCenter(P,r),"Lingkaran dalam segitiga ABC"):
\end{eulerprompt}
\begin{eulercomment}
Jadi, terbukti bahwa garis bagi sudut yang ketiga juga melalui titik
pusat lingkaran dalam.


4. Gambar jari-jari lingkaran dalam.
\end{eulercomment}
\begin{eulerprompt}
>r=norm(P-projectToLine(P,lineThrough(A,B)))
\end{eulerprompt}
\begin{euleroutput}
  1.52896119631
\end{euleroutput}
\begin{eulerprompt}
>plotCircle(circleWithCenter(P,r),"Lingkaran dalam segitiga ABC"):
\end{eulerprompt}
\begin{eulercomment}
\end{eulercomment}
\eulerheading{Contoh 2: Geometri Smbolik}
\begin{eulercomment}
Kita dapat menghitung geometri eksak dan simbolik menggunakan Maxima.

File geometry.e menyediakan fungsi yang sama (dan lebih banyak lagi)
di Maxima. Akan tetapi, kita sekarang dapat menggunakan perhitungan
simbolik.
\end{eulercomment}
\begin{eulerprompt}
>A &= [1,0]; B &= [0,1]; C &= [2,2]; // menentukan tiga titik A, B, C
\end{eulerprompt}
\begin{eulercomment}
Fungsi untuk garis dan lingkaran bekerja seperti fungsi Euler, tetapi
menyediakan perhitungan simbolis.
\end{eulercomment}
\begin{eulerprompt}
>c &= lineThrough(B,C) // c=BC
\end{eulerprompt}
\begin{euleroutput}
  
                               [- 1, 2, 2]
  
\end{euleroutput}
\begin{eulercomment}
Kita dapat memperoleh persamaan garis dengan mudah.
\end{eulercomment}
\begin{eulerprompt}
>$getLineEquation(c,x,y), $solve(%,y) | expand // persamaan garis c
>$getLineEquation(lineThrough([x1,y1],[x2,y2]),x,y), $solve(%,y) // persamaan garis melalui(x1, y1) dan (x2, y2)
>$getLineEquation(lineThrough(A,[x1,y1]),x,y) // persamaan garis melalui A dan (x1, y1)
>h &= perpendicular(A,lineThrough(B,C)) // h melalui A tegak lurus BC
\end{eulerprompt}
\begin{euleroutput}
  
                                [2, 1, 2]
  
\end{euleroutput}
\begin{eulerprompt}
>Q &= lineIntersection(c,h) // Q titik potong garis c=BC dan h
\end{eulerprompt}
\begin{euleroutput}
  
                                   2  6
                                  [-, -]
                                   5  5
  
\end{euleroutput}
\begin{eulerprompt}
>$projectToLine(A,lineThrough(B,C)) // proyeksi A pada BC
>$distance(A,Q) // jarak AQ
>cc &= circleThrough(A,B,C); $cc // (titik pusat dan jari-jari) lingkaran melalui A, B, C
>r&=getCircleRadius(cc); $r , $float(r) // tampilkan nilai jari-jari
>$computeAngle(A,C,B) // nilai <ACB
>$solve(getLineEquation(angleBisector(A,C,B),x,y),y)[1] // persamaan garis bagi <ACB
>P &= lineIntersection(angleBisector(A,C,B),angleBisector(C,B,A)); $P // titik potong 2 garis bagi sudut
>P() // hasilnya sama dengan perhitungan sebelumnya
\end{eulerprompt}
\begin{euleroutput}
  [0.86038,  0.86038]
\end{euleroutput}
\eulersubheading{Garis dan Lingkaran yang Berpotongan}
\begin{eulercomment}
Tentu saja, kita juga dapat membuat garis berpotongan dengan
lingkaran, dan lingkaran dengan lingkaran.
\end{eulercomment}
\begin{eulerprompt}
>A &:= [1,0]; c=circleWithCenter(A,4);
>B &:= [1,2]; C &:= [2,1]; l=lineThrough(B,C);
>setPlotRange(5); plotCircle(c); plotLine(l);
\end{eulerprompt}
\begin{eulercomment}
Perpotongan garis dengan lingkaran menghasilkan dua titik dan jumlah
titik perpotongan.
\end{eulercomment}
\begin{eulerprompt}
>\{P1,P2,f\}=lineCircleIntersections(l,c);
>P1, P2, f
\end{eulerprompt}
\begin{euleroutput}
  [4.64575,  -1.64575]
  [-0.645751,  3.64575]
  2
\end{euleroutput}
\begin{eulerprompt}
>plotPoint(P1); plotPoint(P2):
\end{eulerprompt}
\begin{eulercomment}
Sama halnya di Maxima.
\end{eulercomment}
\begin{eulerprompt}
>c &= circleWithCenter(A,4) // lingkaran dengan pusat A jari-jari 4
\end{eulerprompt}
\begin{euleroutput}
  
                                [1, 0, 4]
  
\end{euleroutput}
\begin{eulerprompt}
>l &= lineThrough(B,C) // garis l melalui B dan C
\end{eulerprompt}
\begin{euleroutput}
  
                                [1, 1, 3]
  
\end{euleroutput}
\begin{eulerprompt}
>$lineCircleIntersections(l,c) | radcan, // titik potong lingkaran c dan garis l
\end{eulerprompt}
\begin{eulercomment}
Akan ditunjukkan bahwa sudut-sudut yang menghadap bsuusr yang sama adalah sama besar.
\end{eulercomment}
\begin{eulerprompt}
>C=A+normalize([-2,-3])*4; plotPoint(C); plotSegment(P1,C); plotSegment(P2,C);
>degprint(computeAngle(P1,C,P2))
\end{eulerprompt}
\begin{euleroutput}
  69°17'42.68''
\end{euleroutput}
\begin{eulerprompt}
>C=A+normalize([-4,-3])*4; plotPoint(C); plotSegment(P1,C); plotSegment(P2,C);
>degprint(computeAngle(P1,C,P2))
\end{eulerprompt}
\begin{euleroutput}
  69°17'42.68''
\end{euleroutput}
\begin{eulerprompt}
>insimg;
\end{eulerprompt}
\eulersubheading{Garis Sumbu}
\begin{eulercomment}
Berikut adalah langkah-langkah menggambar garis sumbu ruas garis AB:

1. Gambar lingkaran dengan pusat A melalui B.\\
2. Gambar lingkaran dengan pusat B melalui A.\\
3. Tarik garis melallui kedua titik potong kedua lingkaran tersebut. Garis ini merupakan
garis sumbu (melalui titik tengah dan tegak lurus) AB.
\end{eulercomment}
\begin{eulerprompt}
>A=[2,2]; B=[-1,-2];
>c1=circleWithCenter(A,distance(A,B));
>c2=circleWithCenter(B,distance(A,B));
>\{P1,P2,f\}=circleCircleIntersections(c1,c2);
>l=lineThrough(P1,P2);
>setPlotRange(5); plotCircle(c1); plotCircle(c2);
>plotPoint(A); plotPoint(B); plotSegment(A,B); plotLine(l):
\end{eulerprompt}
\begin{eulercomment}
Selanjutnya, kita melakukan hal yang sama di Maxima dengan koordinat
umum.
\end{eulercomment}
\begin{eulerprompt}
>A &= [a1,a2]; B &= [b1,b2];
>c1 &= circleWithCenter(A,distance(A,B));
>c2 &= circleWithCenter(B,distance(A,B));
>P &= circleCircleIntersections(c1,c2); P1 &= P[1]; P2 &= P[2];
\end{eulerprompt}
\begin{eulercomment}
Persamaan untuk perpotongan cukup rumit. Namun, kita dapat
menyederhanakannya, jika kita mencari nilai y.
\end{eulercomment}
\begin{eulerprompt}
>g &= getLineEquation(lineThrough(P1,P2),x,y);
>$solve(g,y)
\end{eulerprompt}
\begin{eulercomment}
Ini memang sama dengan tegak lurus tengah, yang dihitung dengan cara
yang sepenuhnya berbeda.
\end{eulercomment}
\begin{eulerprompt}
>$solve(getLineEquation(middlePerpendicular(A,B),x,y),y)
>h &=getLineEquation(lineThrough(A,B),x,y);
>$solve(h,y)
\end{eulerprompt}
\begin{eulercomment}
Perhatikan hasil kali gradien garis g dan h adalah:

\end{eulercomment}
\begin{eulerformula}
\[
\frac{-(b_1-a_1)}{(b_2-a_2)}\times \frac{(b_2-a_2)}{(b_1-a_1)} = -1.
\]
\end{eulerformula}
\begin{eulercomment}
Artinya kedua garis tegak lurus.
\end{eulercomment}
\eulerheading{Contoh 3: Rumus Heron}
\begin{eulercomment}
Rumus Heron menyatakan bahwa luas segitiga dengan panjang sisi-sisi a,
b dan c adalah:

\end{eulercomment}
\begin{eulerformula}
\[
L = \sqrt{s(s-a)(s-b)(s-c)}\quad \text{ dengan } s=(a+b+c)/2,
\]
\end{eulerformula}
\begin{eulercomment}
atau bisa ditulis dalam bentuk lain:

\end{eulercomment}
\begin{eulerformula}
\[
L = \frac{1}{4}\sqrt{(a+b+c)(b+c-a)(a+c-b)(a+b-c)}
\]
\end{eulerformula}
\begin{eulercomment}
Untuk membuktikan hal ini kita misalkan C(0,0), B(a,0) dan A(x,y),
b=AC, c=AB. Luas segitiga ABC adalah

\end{eulercomment}
\begin{eulerformula}
\[
L_{\triangle ABC}=\frac{1}{2}a\times y.
\]
\end{eulerformula}
\begin{eulercomment}
Nilai y didapat dengan menyelesaikan sistem persamaan:

\end{eulercomment}
\begin{eulerformula}
\[
x^2+y^2=b^2, \quad (x-a)^2+y^2=c^2.
\]
\end{eulerformula}
\begin{eulerprompt}
>setPlotRange(-1,10,-1,8); plotPoint([0,0], "C(0,0)"); plotPoint([5.5,0], "B(a,0)");  ...
> plotPoint([7.5,6], "A(x,y)");
>plotSegment([0,0],[5.5,0], "a",25); plotSegment([5.5,0],[7.5,6],"c",15);  ...
>plotSegment([0,0],[7.5,6],"b",25); 
>plotSegment([7.5,6],[7.5,0],"t=y",25):
>&assume(a>0); sol &= solve([x^2+y^2=b^2,(x-a)^2+y^2=c^2],[x,y])
\end{eulerprompt}
\begin{euleroutput}
  
                   2    2    2
                - c  + b  + a
          [[x = --------------, y = 
                     2 a
            4      2  2      2  2    4      2  2    4
    sqrt(- c  + 2 b  c  + 2 a  c  - b  + 2 a  b  - a )
  - --------------------------------------------------], 
                           2 a
          2    2    2
       - c  + b  + a
  [x = --------------, y = 
            2 a
          4      2  2      2  2    4      2  2    4
  sqrt(- c  + 2 b  c  + 2 a  c  - b  + 2 a  b  - a )
  --------------------------------------------------]]
                         2 a
  
\end{euleroutput}
\begin{eulercomment}
Ekstrak solusi y.
\end{eulercomment}
\begin{eulerprompt}
>ysol &= y with sol[2][2]; $'y=sqrt(factor(ysol^2))
\end{eulerprompt}
\begin{eulercomment}
Kita mendapatkan rumus Heron.
\end{eulercomment}
\begin{eulerprompt}
>function H(a,b,c) &= sqrt(factor((ysol*a/2)^2)); $'H(a,b,c)=H(a,b,c)
>$'Luas=H(2,5,6) // luas segitiga dengan panjang sisi-sisi 2, 5, 6
\end{eulerprompt}
\begin{eulercomment}
Tentu saja, setiap segitiga siku-siku adalah kasus yang terkenal.
\end{eulercomment}
\begin{eulerprompt}
>H(3,4,5) //luas segitiga siku-siku dengan panjang sisi 3, 4, 5
\end{eulerprompt}
\begin{euleroutput}
  6
\end{euleroutput}
\begin{eulercomment}
Dan jelas pula, bahwa ini adalah segitiga dengan luas maksimal dan dua
sisinya 3 dan 4.
\end{eulercomment}
\begin{eulerprompt}
>aspect (1.5); plot2d(&H(3,4,x),1,7): // Kurva luas segitiga sengan panjang sisi 3, 4, x (1<= x <=7)
\end{eulerprompt}
\begin{eulercomment}
Kasus umum juga berfungsi.
\end{eulercomment}
\begin{eulerprompt}
>$solve(diff(H(a,b,c)^2,c)=0,c)
\end{eulerprompt}
\begin{eulercomment}
Sekarang mari kita cari himpunan semua titik di mana b+c=d untuk suatu
konstanta d. Diketahui bahwa ini adalah elips.
\end{eulercomment}
\begin{eulerprompt}
>s1 &= subst(d-c,b,sol[2]); $s1
\end{eulerprompt}
\begin{eulercomment}
Dan buat fungsi ini.
\end{eulercomment}
\begin{eulerprompt}
>function fx(a,c,d) &= rhs(s1[1]); $fx(a,c,d), function fy(a,c,d) &= rhs(s1[2]); $fy(a,c,d)
\end{eulerprompt}
\begin{eulercomment}
Sekarang kita dapat menggambar himpunannya. Sisi b bervariasi dari 1
hingga 4. Diketahui bahwa kita mendapatkan elips.
\end{eulercomment}
\begin{eulerprompt}
>aspect(1); plot2d(&fx(3,x,5),&fy(3,x,5),xmin=1,xmax=4,square=1):
\end{eulerprompt}
\begin{eulercomment}
Kita dapat memeriksa persamaan umum untuk elips ini, yaitu:

\end{eulercomment}
\begin{eulerformula}
\[
\frac{(x-x_m)^2}{u^2}+\frac{(y-y_m)}{v^2}=1,
\]
\end{eulerformula}
\begin{eulercomment}
di mana (xm,ym) adalah pusat, dan u dan v adalah sumbu setengah.
\end{eulercomment}
\begin{eulerprompt}
>$ratsimp((fx(a,c,d)-a/2)^2/u^2+fy(a,c,d)^2/v^2 with [u=d/2,v=sqrt(d^2-a^2)/2])
\end{eulerprompt}
\begin{eulercomment}
Kita melihat bahwa tinggi dan luas segitiga tersebut adalah maksimum
untuk x=0. Jadi luas segitiga dengan a+b+c=d adalah maksimum, jika
segitiga tersebut sama sisi. Kita ingin memperolehnya secara analitis.
\end{eulercomment}
\begin{eulerprompt}
>eqns &= [diff(H(a,b,d-(a+b))^2,a)=0,diff(H(a,b,d-(a+b))^2,b)=0]; $eqns
\end{eulerprompt}
\begin{eulercomment}
Kita memperoleh beberapa nilai minimum, yang dimiliki oleh segitiga
dengan satu sisi 0, dan solusinya a=b=c=d/3.
\end{eulercomment}
\begin{eulerprompt}
>$solve(eqns,[a,b])
\end{eulerprompt}
\begin{eulercomment}
Ada juga metode Lagrange, yang memaksimalkan H(a,b,c)\textasciicircum{}2 terhadap
a+b+d=d.
\end{eulercomment}
\begin{eulerprompt}
>&solve([diff(H(a,b,c)^2,a)=la,diff(H(a,b,c)^2,b)=la, ...
>   diff(H(a,b,c)^2,c)=la,a+b+c=d],[a,b,c,la])
\end{eulerprompt}
\begin{euleroutput}
  
                       d      d
          [[a = 0, b = -, c = -, la = 0], 
                       2      2
       d             d                d      d
  [a = -, b = 0, c = -, la = 0], [a = -, b = -, c = 0, la = 0], 
       2             2                2      2
                              3
       d      d      d       d
  [a = -, b = -, c = -, la = ---]]
       3      3      3       108
  
\end{euleroutput}
\begin{eulercomment}
Kita bisa membuat plot dari situasinya
\end{eulercomment}
\begin{eulercomment}
Pertama-tama atur titik di Maxima.
\end{eulercomment}
\begin{eulerprompt}
>A &= at([x,y],sol[2]); $A
>B &= [0,0]; $B, C &= [a,0]; $C
\end{eulerprompt}
\begin{eulercomment}
Kemudian atur rentang plot dan plot titik-titiknya.
\end{eulercomment}
\begin{eulerprompt}
>setPlotRange(0,5,-2,3); ...
>a=4; b=3; c=2; ...
>plotPoint(mxmeval("B"),"B"); plotPoint(mxmeval("C"),"C"); ...
>plotPoint(mxmeval("A"),"A"):
\end{eulerprompt}
\begin{eulercomment}
Plot segmen.
\end{eulercomment}
\begin{eulerprompt}
>splotSegment(mxmeval("A"),mxmeval("C")); ...
>plotSegment(mxmeval("B"),mxmeval("C")); ...
>plotSegment(mxmeval("B"),mxmeval("A")):
\end{eulerprompt}
\begin{eulercomment}
Hitunglah garis tegak lurus tengah di Maxima.
\end{eulercomment}
\begin{eulerprompt}
>h &= middlePerpendicular(A,B); g &= middlePerpendicular(B,C);
\end{eulerprompt}
\begin{eulercomment}
Dan pusat kelilingnya.
\end{eulercomment}
\begin{eulerprompt}
>U &= lineIntersection(h,g);
\end{eulerprompt}
\begin{eulercomment}
Kita mendapatkan rumus untuk jari-jari lingkaran luar.
\end{eulercomment}
\begin{eulerprompt}
>&assume(a>0,b>0,c>0); $distance(U,B) | radcan
\end{eulerprompt}
\begin{eulercomment}
Mari kita tambahkan ini ke dalam alur cerita.
\end{eulercomment}
\begin{eulerprompt}
>plotPoint(U()); ...
>plotCircle(circleWithCenter(mxmeval("U"),mxmeval("distance(U,C)"))):
\end{eulerprompt}
\begin{eulercomment}
Dengan menggunakan geometri, kita peroleh rumus sederhana

\end{eulercomment}
\begin{eulerformula}
\[
\frac{a}{\sin(\alpha)}=2r
\]
\end{eulerformula}
\begin{eulercomment}
untuk jari-jari. Kita dapat memeriksa apakah ini benar dengan Maxima.
Maxima akan memfaktorkan ini hanya jika kita mengkuadratkannya.
\end{eulercomment}
\begin{eulerprompt}
>$c^2/sin(computeAngle(A,B,C))^2  | factor
\end{eulerprompt}
\eulerheading{Contoh 4: Garis Euler dan Parabola}
\begin{eulercomment}
Garis Euler adalah garis yang ditentukan dari sembarang segitiga yang
tidak sama sisi. Garis ini merupakan garis pusat segitiga, dan
melewati beberapa titik penting yang ditentukan dari segitiga
tersebut, termasuk orthocenter, circumcenter, centroid, titik Exeter,
dan titik pusat lingkaran sembilan titik pada segitiga tersebut.

Sebagai contoh, kita hitung dan plot garis Euler dalam sebuah
segitiga.

Pertama, kita definisikan sudut-sudut segitiga dalam Euler. Kita
gunakan definisi, yang terlihat dalam ekspresi simbolik.
\end{eulercomment}
\begin{eulerprompt}
>A::=[-1,-1]; B::=[2,0]; C::=[1,2];
\end{eulerprompt}
\begin{eulercomment}
Untuk memplot objek geometris, kita menyiapkan area plot, dan
menambahkan titik-titik ke dalamnya. Semua plot objek geometris
ditambahkan ke plot saat ini.
\end{eulercomment}
\begin{eulerprompt}
>setPlotRange(3); plotPoint(A,"A"); plotPoint(B,"B"); plotPoint(C,"C");
\end{eulerprompt}
\begin{eulercomment}
Kita juga dapat menambahkan sisi-sisi segitiga.
\end{eulercomment}
\begin{eulerprompt}
>plotSegment(A,B,""); plotSegment(B,C,""); plotSegment(C,A,""):
\end{eulerprompt}
\begin{eulercomment}
Berikut adalah luas segitiga, menggunakan rumus determinan. Tentu
saja, kita harus mengambil nilai absolut dari hasil ini.
\end{eulercomment}
\begin{eulerprompt}
>$areaTriangle(A,B,C)
\end{eulerprompt}
\begin{eulercomment}
Kita dapat menghitung koefisien sisi c.
\end{eulercomment}
\begin{eulerprompt}
>c &= lineThrough(A,B)
\end{eulerprompt}
\begin{euleroutput}
  
                              [- 1, 3, - 2]
  
\end{euleroutput}
\begin{eulercomment}
Dan dapatkan juga rumus untuk garis ini.
\end{eulercomment}
\begin{eulerprompt}
>$getLineEquation(c,x,y)
\end{eulerprompt}
\begin{eulercomment}
Untuk bentuk Hesse, kita perlu menentukan suatu titik, sehingga titik
tersebut berada di sisi positif bentuk Hesse. Memasukkan titik akan
menghasilkan jarak positif ke garis.
\end{eulercomment}
\begin{eulerprompt}
>$getHesseForm(c,x,y,C), $at(%,[x=C[1],y=C[2]])
\end{eulerprompt}
\begin{eulercomment}
Sekarang kita hitung lingkaran luar ABC.
\end{eulercomment}
\begin{eulerprompt}
>LL &= circleThrough(A,B,C); $getCircleEquation(LL,x,y)
>O &= getCircleCenter(LL); $O
\end{eulerprompt}
\begin{eulercomment}
Gambarkan lingkaran dan titik pusatnya. Cu dan U adalah simbol. Kita
evaluasi ekspresi ini untuk Euler.
\end{eulercomment}
\begin{eulerprompt}
>plotCircle(LL()); plotPoint(O(),"O"):
\end{eulerprompt}
\begin{eulercomment}
Kita dapat menghitung perpotongan tinggi di ABC (orthocenter) secara
numerik dengan perintah berikut.
\end{eulercomment}
\begin{eulerprompt}
>H &= lineIntersection(perpendicular(A,lineThrough(C,B)),...
>  perpendicular(B,lineThrough(A,C))); $H
\end{eulerprompt}
\begin{eulercomment}
Sekarang kita dapat menghitung garis Euler dari segitiga tersebut.
\end{eulercomment}
\begin{eulerprompt}
>el &= lineThrough(H,O); $getLineEquation(el,x,y)
\end{eulerprompt}
\begin{eulercomment}
Tambahkan ke plot kita.
\end{eulercomment}
\begin{eulerprompt}
>plotPoint(H(),"H"); plotLine(el(),"Garis Euler"):
\end{eulerprompt}
\begin{eulercomment}
Pusat gravitasi seharusnya berada pada garis ini.
\end{eulercomment}
\begin{eulerprompt}
>M &= (A+B+C)/3; $getLineEquation(el,x,y) with [x=M[1],y=M[2]]
>plotPoint(M(),"M"): // titik berat
\end{eulerprompt}
\begin{eulercomment}
Teori ini memberi tahu kita MH=2*MO. Kita perlu menyederhanakannya
dengan radcan untuk mencapainya.
\end{eulercomment}
\begin{eulerprompt}
>$distance(M,H)/distance(M,O)|radcan
\end{eulerprompt}
\begin{eulercomment}
Fungsinya termasuk fungsi untuk sudut juga.
\end{eulercomment}
\begin{eulerprompt}
>$computeAngle(A,C,B), degprint(%())
\end{eulerprompt}
\begin{euleroutput}
  60°15'18.43''
\end{euleroutput}
\begin{eulercomment}
Persamaan untuk pusat lingkaran dalam tidak terlalu bagus.
\end{eulercomment}
\begin{eulerprompt}
>Q &= lineIntersection(angleBisector(A,C,B),angleBisector(C,B,A))|radcan; $Q
\end{eulerprompt}
\begin{eulercomment}
Mari kita hitung juga ekspresi untuk jari-jari lingkaran dalam.
\end{eulercomment}
\begin{eulerprompt}
>r &= distance(Q,projectToLine(Q,lineThrough(A,B)))|ratsimp; $r
>LD &=  circleWithCenter(Q,r); // Lingkaran dalam
\end{eulerprompt}
\begin{eulercomment}
Mari kita tambahkan ini ke dalam alur cerita.
\end{eulercomment}
\begin{eulerprompt}
>color(5); plotCircle(LD()):
\end{eulerprompt}
\eulersubheading{Parabola}
\begin{eulercomment}
Selanjutnya akan dicari persamaan tempat kedudukan titik-titik yang berjarak sama ke titik C
dan ke garis AB.
\end{eulercomment}
\begin{eulerprompt}
>p &= getHesseForm(lineThrough(A,B),x,y,C)-distance([x,y],C); $p='0
\end{eulerprompt}
\begin{eulercomment}
Persamaan tersebut dapat digambar menjadi satu dengan gambar sebelumnya.
\end{eulercomment}
\begin{eulerprompt}
>plot2d(p,level=0,add=1,contourcolor=6):
\end{eulerprompt}
\begin{eulercomment}
Ini seharusnya merupakan suatu fungsi, tetapi penyelesai default
Maxima hanya dapat menemukan solusinya, jika kita mengkuadratkan
persamaannya. Akibatnya, kita mendapatkan solusi palsu.
\end{eulercomment}
\begin{eulerprompt}
>akar &= solve(getHesseForm(lineThrough(A,B),x,y,C)^2-distance([x,y],C)^2,y)
\end{eulerprompt}
\begin{euleroutput}
  
          [y = - 3 x - sqrt(70) sqrt(9 - 2 x) + 26, 
                                y = - 3 x + sqrt(70) sqrt(9 - 2 x) + 26]
  
\end{euleroutput}
\begin{eulercomment}
Solusi pertama adalah

maxima: akar[1]

Dengan menambahkan solusi pertama ke dalam plot, terlihat bahwa itu
memang jalur yang kita cari. Teori tersebut memberi tahu kita bahwa
itu adalah parabola yang diputar.
\end{eulercomment}
\begin{eulerprompt}
>plot2d(&rhs(akar[1]),add=1):
>function g(x) &= rhs(akar[1]); $'g(x)= g(x)// fungsi yang mendefinisikan kurva di atas
>T &=[-1, g(-1)]; // ambil sebarang titik pada kurva tersebut
>dTC &= distance(T,C); $fullratsimp(dTC), $float(%) // jarak T ke C
>U &= projectToLine(T,lineThrough(A,B)); $U // proyeksi T pada garis AB 
>dU2AB &= distance(T,U); $fullratsimp(dU2AB), $float(%) // jatak T ke AB
\end{eulerprompt}
\begin{eulercomment}
Ternyata jarak T ke C sama dengan jarak T ke AB. Coba Anda pilih titik
T yang lain dan ulangi perhitungan-perhitungan di atas untuk
menunjukkan bahwa hasilnya juga sama.
\end{eulercomment}
\eulerheading{Contoh 5: Trigonometri Rasional}
\begin{eulercomment}
Hal ini terinspirasi dari ceramah N.J.Wildberger. Dalam bukunya
"Divine Proportions", Wildberger mengusulkan untuk mengganti konsep
klasik jarak dan sudut dengan kuadran dan sebaran. Dengan menggunakan
konsep ini, memang memungkinkan untuk menghindari fungsi trigonometri
dalam banyak contoh, dan tetap "rasional".

Berikut ini, saya memperkenalkan konsep-konsep tersebut, dan
memecahkan beberapa masalah. Saya menggunakan perhitungan simbolik
Maxima di sini, yang menyembunyikan keuntungan utama trigonometri
rasional bahwa perhitungan dapat dilakukan hanya dengan kertas dan
pensil. Anda diundang untuk memeriksa hasilnya tanpa komputer.

Intinya adalah bahwa perhitungan simbolik rasional sering kali
menghasilkan hasil yang sederhana. Sebaliknya, trigonometri klasik
menghasilkan hasil trigonometri yang rumit, yang hanya mengevaluasi
perkiraan numerik.
\end{eulercomment}
\begin{eulerprompt}
>load geometry;
\end{eulerprompt}
\begin{eulercomment}
Untuk pengenalan pertama, kami menggunakan segitiga siku-siku dengan
proporsi Mesir yang terkenal yaitu 3, 4, dan 5. Perintah berikut
adalah perintah Euler untuk memplot geometri bidang yang terdapat
dalam file Euler "geometry.e".
\end{eulercomment}
\begin{eulerprompt}
>C&:=[0,0]; A&:=[4,0]; B&:=[0,3]; ...
>setPlotRange(-1,5,-1,5); ...
>plotPoint(A,"A"); plotPoint(B,"B"); plotPoint(C,"C"); ...
>plotSegment(B,A,"c"); plotSegment(A,C,"b"); plotSegment(C,B,"a"); ...
>insimg(30);
\end{eulerprompt}
\begin{eulercomment}
Tentu saja,

\end{eulercomment}
\begin{eulerformula}
\[
\sin(w_a)=\frac{a}{c},
\]
\end{eulerformula}
\begin{eulercomment}
di mana wa adalah sudut di A. Cara yang biasa untuk menghitung sudut
ini adalah dengan mengambil kebalikan dari fungsi sinus. Hasilnya
adalah sudut yang tidak dapat dicerna, yang hanya dapat dicetak secara
perkiraan.
\end{eulercomment}
\begin{eulerprompt}
>wa := arcsin(3/5); degprint(wa)
\end{eulerprompt}
\begin{euleroutput}
  36°52'11.63''
\end{euleroutput}
\begin{eulercomment}
Trigonometri rasional mencoba menghindari hal ini.

Gagasan pertama trigonometri rasional adalah kuadran, yang
menggantikan jarak. Faktanya, itu hanyalah kuadrat jarak. Dalam
persamaan berikut, a, b, dan c menunjukkan kuadran sisi-sisi.

Teorema Pythogoras menjadi a+b=c.
\end{eulercomment}
\begin{eulerprompt}
>a &= 3^2; b &= 4^2; c &= 5^2; &a+b=c
\end{eulerprompt}
\begin{euleroutput}
  
                                 25 = 25
  
\end{euleroutput}
\begin{eulercomment}
Gagasan kedua trigonometri rasional adalah sebaran. Sebaran mengukur
bukaan antara garis. Nilainya 0, jika garis sejajar, dan 1, jika garis
persegi panjang. Nilainya adalah kuadrat sinus sudut antara dua garis.

Sebaran garis AB dan AC pada gambar di atas didefinisikan sebagai

\end{eulercomment}
\begin{eulerformula}
\[
s_a = \sin(\alpha)^2 = \frac{a}{c},
\]
\end{eulerformula}
\begin{eulercomment}
di mana a dan c adalah kuadran dari setiap segitiga persegi panjang
dengan satu sudut di A.
\end{eulercomment}
\begin{eulerprompt}
>sa &= a/c; $sa
\end{eulerprompt}
\begin{eulercomment}
Tentu saja, ini lebih mudah dihitung daripada sudut. Namun, Anda
kehilangan sifat bahwa sudut dapat ditambahkan dengan mudah.

Tentu saja, kita dapat mengubah nilai perkiraan untuk sudut wa menjadi
sprad, dan mencetaknya sebagai pecahan.
\end{eulercomment}
\begin{eulerprompt}
>fracprint(sin(wa)^2)
\end{eulerprompt}
\begin{euleroutput}
  9/25
\end{euleroutput}
\begin{eulercomment}
Hukum kosinus dari trgonometri klasik diterjemahkan ke dalam "hukum
silang" berikut.

\end{eulercomment}
\begin{eulerformula}
\[
(c+b-a)^2 = 4 b c \, (1-s_a)
\]
\end{eulerformula}
\begin{eulercomment}
Di sini a, b, dan c adalah kuadran sisi-sisi segitiga, dan sa adalah
sebaran di sudut A. Sisi a, seperti biasa, berseberangan dengan sudut
A.

Hukum-hukum ini diimplementasikan dalam berkas geometry.e yang kami
muat ke Euler.
\end{eulercomment}
\begin{eulerprompt}
>$crosslaw(aa,bb,cc,saa)
\end{eulerprompt}
\begin{eulercomment}
Dalam kasus kami, kami mendapatkan
\end{eulercomment}
\begin{eulerprompt}
>$crosslaw(a,b,c,sa)
\end{eulerprompt}
\begin{eulercomment}
Mari kita gunakan hukum silang ini untuk menemukan sebaran di A. Untuk
melakukannya, kita buat hukum silang untuk kuadran a, b, dan c, dan
selesaikan untuk sebaran yang tidak diketahui sa.

Anda dapat melakukannya dengan mudah secara manual, tetapi saya
menggunakan Maxima. Tentu saja, kita mendapatkan hasil yang sudah kita
miliki.
\end{eulercomment}
\begin{eulerprompt}
>$crosslaw(a,b,c,x), $solve(%,x)
\end{eulerprompt}
\begin{eulercomment}
Kita sudah tahu ini. Definisi sebaran adalah kasus khusus dari hukum
silang.

Kita juga dapat memecahkan ini untuk a,b,c umum. Hasilnya adalah rumus
yang menghitung sebaran sudut segitiga yang diberikan kuadran ketiga
sisinya.
\end{eulercomment}
\begin{eulerprompt}
>$solve(crosslaw(aa,bb,cc,x),x)
\end{eulerprompt}
\begin{eulercomment}
Kita dapat membuat fungsi dari hasil tersebut. Fungsi tersebut telah
didefinisikan dalam berkas geometry.e milik Euler.
\end{eulercomment}
\begin{eulerprompt}
>$spread(a,b,c)
\end{eulerprompt}
\begin{eulercomment}
Sebagai contoh, kita dapat menggunakannya untuk menghitung sudut
segitiga dengan sisi-sisi

\end{eulercomment}
\begin{eulerformula}
\[
a, \quad a, \quad \frac{4a}{7}
\]
\end{eulerformula}
\begin{eulercomment}
Hasilnya rasional, yang tidak mudah didapat jika kita menggunakan
trigonometri klasik.
\end{eulercomment}
\begin{eulerprompt}
>$spread(a,a,4*a/7)
\end{eulerprompt}
\begin{eulercomment}
Ini adalah sudut dalam derajat.
\end{eulercomment}
\begin{eulerprompt}
>degprint(arcsin(sqrt(6/7)))
\end{eulerprompt}
\begin{euleroutput}
  67°47'32.44''
\end{euleroutput}
\eulersubheading{Contoh Lain}
\begin{eulercomment}
Sekarang, mari kita coba contoh yang lebih maju.

Kita tentukan tiga sudut segitiga sebagai berikut.
\end{eulercomment}
\begin{eulerprompt}
>A&:=[1,2]; B&:=[4,3]; C&:=[0,4]; ...
>setPlotRange(-1,5,1,7); ...
>plotPoint(A,"A"); plotPoint(B,"B"); plotPoint(C,"C"); ...
>plotSegment(B,A,"c"); plotSegment(A,C,"b"); plotSegment(C,B,"a"); ...
>insimg;
\end{eulerprompt}
\begin{eulercomment}
Dengan menggunakan Pythogoras, mudah untuk menghitung jarak antara dua
titik. Pertama-tama saya menggunakan fungsi distance dari file Euler
untuk geometri. Fungsi distance menggunakan geometri klasik.
\end{eulercomment}
\begin{eulerprompt}
>$distance(A,B)
\end{eulerprompt}
\begin{eulercomment}
Euler juga memuat fungsi untuk kuadran antara dua titik.

Dalam contoh berikut, karena c+b bukan a, segitiga tersebut bukan
persegi panjang.
\end{eulercomment}
\begin{eulerprompt}
>c &= quad(A,B); $c, b &= quad(A,C); $b, a &= quad(B,C); $a,
\end{eulerprompt}
\begin{eulercomment}
Pertama, mari kita hitung sudut tradisional. Fungsi computeAngle
menggunakan metode biasa berdasarkan perkalian titik dua vektor.
Hasilnya adalah beberapa perkiraan floating point.

\end{eulercomment}
\begin{eulerformula}
\[
A=<1,2>\quad B=<4,3>,\quad C=<0,4>
\]
\end{eulerformula}
\begin{eulerformula}
\[
\mathbf{a}=C-B=<-4,1>,\quad \mathbf{c}=A-B=<-3,-1>,\quad \beta=\angle ABC
\]
\end{eulerformula}
\begin{eulerformula}
\[
\mathbf{a}.\mathbf{c}=|\mathbf{a}|.|\mathbf{c}|\cos \beta
\]
\end{eulerformula}
\begin{eulerformula}
\[
\cos \angle ABC =\cos\beta=\frac{\mathbf{a}.\mathbf{c}}{|\mathbf{a}|.|\mathbf{c}|}=\frac{12-1}{\sqrt{17}\sqrt{10}}=\frac{11}{\sqrt{17}\sqrt{10}}
\]
\end{eulerformula}
\begin{eulerprompt}
>wb &= computeAngle(A,B,C); $wb, $(wb/pi*180)()
\end{eulerprompt}
\begin{euleroutput}
  32.4711922908
\end{euleroutput}
\begin{eulercomment}
Dengan menggunakan pensil dan kertas, kita dapat melakukan hal yang
sama dengan hukum silang. Kita masukkan kuadran a, b, dan c ke dalam
hukum silang dan selesaikan untuk x.
\end{eulercomment}
\begin{eulerprompt}
>$crosslaw(a,b,c,x), $solve(%,x), //(b+c-a)^=4b.c(1-x)
\end{eulerprompt}
\begin{eulercomment}
Yaitu, apa yang dilakukan fungsi sebaran yang didefinisikan dalam
"geometry.e".
\end{eulercomment}
\begin{eulerprompt}
>sb &= spread(b,a,c); $sb
\end{eulerprompt}
\begin{eulercomment}
Maxima memperoleh hasil yang sama dengan menggunakan trigonometri
biasa, jika kita memaksakannya. Maxima memang menyelesaikan suku
sin(arccos(...)) menjadi hasil pecahan. Sebagian besar siswa tidak
dapat melakukan ini.
\end{eulercomment}
\begin{eulerprompt}
>$sin(computeAngle(A,B,C))^2
\end{eulerprompt}
\begin{eulercomment}
Setelah kita memiliki sebaran di B, kita dapat menghitung tinggi ha
pada sisi a. Ingat bahwa

\end{eulercomment}
\begin{eulerformula}
\[
s_b=\frac{h_a}{c}
\]
\end{eulerformula}
\begin{eulercomment}
menurut definisi.
\end{eulercomment}
\begin{eulerprompt}
>ha &= c*sb; $ha
\end{eulerprompt}
\begin{eulercomment}
Gambar berikut ini dibuat dengan program geometri C.a.R., yang dapat
menggambar kuadran dan sebaran.

image: (20) Rational\_Geometry\_CaR.png

Menurut definisi, panjang ha adalah akar kuadrat dari kuadrannya.
\end{eulercomment}
\begin{eulerprompt}
>$sqrt(ha)
\end{eulerprompt}
\begin{eulercomment}
Sekarang kita bisa menghitung luas segitiga. Jangan lupa, bahwa kita
sedang membahas tentang kuadran!
\end{eulercomment}
\begin{eulerprompt}
>$sqrt(ha)*sqrt(a)/2
\end{eulerprompt}
\begin{eulercomment}
Rumus determinan yang biasa menghasilkan hasil yang sama.
\end{eulercomment}
\begin{eulerprompt}
>$areaTriangle(B,A,C)
\end{eulerprompt}
\eulersubheading{Rumus Heron}
\begin{eulercomment}
Sekarang, mari kita selesaikan masalah ini secara umum!
\end{eulercomment}
\begin{eulerprompt}
>&remvalue(a,b,c,sb,ha);
\end{eulerprompt}
\begin{eulercomment}
Pertama-tama kita hitung sebaran di B untuk segitiga dengan sisi a, b,
dan c. Kemudian kita hitung luas kuadrat (yang disebut "quadrea"?),
faktorkan dengan Maxima, dan kita dapatkan rumus Heron yang terkenal.

Memang, ini sulit dilakukan dengan pensil dan kertas.
\end{eulercomment}
\begin{eulerprompt}
>$spread(b^2,c^2,a^2), $factor(%*c^2*a^2/4)
\end{eulerprompt}
\eulersubheading{Aturan Triple Spread}
\begin{eulercomment}
Kerugian spread adalah tidak lagi sekadar menambahkan sudut yang sama.

Namun, tiga spread segitiga memenuhi aturan "triple spread" berikut.
\end{eulercomment}
\begin{eulerprompt}
>&remvalue(sa,sb,sc); $triplespread(sa,sb,sc)
\end{eulerprompt}
\begin{eulercomment}
Aturan ini berlaku untuk tiga sudut yang jumlahnya mencapai 180°.

\end{eulercomment}
\begin{eulerformula}
\[
\alpha+\beta+\gamma=\pi
\]
\end{eulerformula}
\begin{eulercomment}
Karena sebaran

\end{eulercomment}
\begin{eulerformula}
\[
\alpha, \pi-\alpha
\]
\end{eulerformula}
\begin{eulercomment}
sama, aturan sebaran rangkap tiga juga berlaku, jika

\end{eulercomment}
\begin{eulerformula}
\[
\alpha+\beta=\gamma
\]
\end{eulerformula}
\begin{eulercomment}
Karena sebaran sudut negatif sama, aturan sebaran rangkap tiga juga
berlaku, jika

\end{eulercomment}
\begin{eulerformula}
\[
\alpha+\beta+\gamma=0
\]
\end{eulerformula}
\begin{eulercomment}
Misalnya, kita dapat menghitung sebaran sudut 60°. Yaitu 3/4.
Persamaan tersebut memiliki solusi kedua, di mana semua sebarannya
adalah 0.
\end{eulercomment}
\begin{eulerprompt}
>$solve(triplespread(x,x,x),x)
\end{eulerprompt}
\begin{eulercomment}
Sebaran 90° jelas adalah 1. Jika dua sudut dijumlahkan menjadi 90°,
sebarannya memecahkan persamaan sebaran rangkap tiga dengan a,b,1.
Dengan perhitungan berikut kita memperoleh a+b=1.
\end{eulercomment}
\begin{eulerprompt}
>$triplespread(x,y,1), $solve(%,x)
\end{eulerprompt}
\begin{eulercomment}
Karena sebaran 180°-t sama dengan sebaran t, rumus sebaran rangkap
tiga juga berlaku, jika satu sudut adalah jumlah atau selisih dari dua
sudut lainnya.

Jadi kita dapat menemukan sebaran sudut yang digandakan. Perhatikan
bahwa ada dua solusi lagi. Kita buat ini menjadi fungsi.
\end{eulercomment}
\begin{eulerprompt}
>$solve(triplespread(a,a,x),x), function doublespread(a) &= factor(rhs(%[1]))
\end{eulerprompt}
\begin{euleroutput}
  
                              - 4 (a - 1) a
  
\end{euleroutput}
\eulersubheading{Garis Bagi Sudut}
\begin{eulercomment}
Kita sudah tahu situasinya seperti ini.
\end{eulercomment}
\begin{eulerprompt}
>C&:=[0,0]; A&:=[4,0]; B&:=[0,3]; ...
>setPlotRange(-1,5,-1,5); ...
>plotPoint(A,"A"); plotPoint(B,"B"); plotPoint(C,"C"); ...
>plotSegment(B,A,"c"); plotSegment(A,C,"b"); plotSegment(C,B,"a"); ...
>insimg;
\end{eulerprompt}
\begin{eulercomment}
Mari kita hitung panjang garis bagi sudut di A. Namun, kita ingin
menyelesaikannya untuk a,b,c umum.
\end{eulercomment}
\begin{eulerprompt}
>&remvalue(a,b,c);
\end{eulerprompt}
\begin{eulercomment}
Jadi pertama-tama kita hitung sebaran sudut yang dibagi dua di A,
menggunakan rumus sebaran rangkap tiga.

Masalah dengan rumus ini muncul lagi. Rumus ini memiliki dua solusi.
Kita harus memilih yang benar. Solusi lainnya mengacu pada sudut yang
dibagi dua 180°-wa.
\end{eulercomment}
\begin{eulerprompt}
>$triplespread(x,x,a/(a+b)), $solve(%,x), sa2 &= rhs(%[1]); $sa2
\end{eulerprompt}
\begin{eulercomment}
Mari kita periksa persegi panjang Mesir.
\end{eulercomment}
\begin{eulerprompt}
>$sa2 with [a=3^2,b=4^2]
\end{eulerprompt}
\begin{eulercomment}
Kita dapat mencetak sudut dalam Euler, setelah mentransfer sebaran ke
radian.
\end{eulercomment}
\begin{eulerprompt}
>wa2 := arcsin(sqrt(1/10)); degprint(wa2)
\end{eulerprompt}
\begin{euleroutput}
  18°26'5.82''
\end{euleroutput}
\begin{eulercomment}
Titik P merupakan perpotongan garis bagi sudut dengan sumbu y.
\end{eulercomment}
\begin{eulerprompt}
>P := [0,tan(wa2)*4]
\end{eulerprompt}
\begin{euleroutput}
  [0,  1.33333]
\end{euleroutput}
\begin{eulerprompt}
>plotPoint(P,"P"); plotSegment(A,P):
\end{eulerprompt}
\begin{eulercomment}
Mari kita periksa sudut-sudut pada contoh spesifik kita.
\end{eulercomment}
\begin{eulerprompt}
>computeAngle(C,A,P), computeAngle(P,A,B)
\end{eulerprompt}
\begin{euleroutput}
  0.321750554397
  0.321750554397
\end{euleroutput}
\begin{eulercomment}
Sekarang kita hitung panjang garis bagi AP.

Kita gunakan teorema sinus dalam segitiga APC. Teorema ini menyatakan
bahwa

\end{eulercomment}
\begin{eulerformula}
\[
\frac{BC}{\sin(w_a)} = \frac{AC}{\sin(w_b)} = \frac{AB}{\sin(w_c)}
\]
\end{eulerformula}
\begin{eulercomment}
berlaku di sembarang segitiga. Kuadratkan, maka akan menghasilkan apa
yang disebut "hukum sebaran"

\end{eulercomment}
\begin{eulerformula}
\[
\frac{a}{s_a} = \frac{b}{s_b} = \frac{c}{s_b}
\]
\end{eulerformula}
\begin{eulercomment}
di mana a,b,c menunjukkan kuadran.

Karena CPA sebaran adalah 1-sa2, kita peroleh darinya bisa/1=b/(1-sa2)
dan dapat menghitung bisa (kuadran garis bagi sudut).
\end{eulercomment}
\begin{eulerprompt}
>&factor(ratsimp(b/(1-sa2))); bisa &= %; $bisa
\end{eulerprompt}
\begin{eulercomment}
Mari kita periksa rumus ini untuk nilai-nilai Mesir kita.
\end{eulercomment}
\begin{eulerprompt}
>sqrt(mxmeval("at(bisa,[a=3^2,b=4^2])")), distance(A,P)
\end{eulerprompt}
\begin{euleroutput}
  4.21637021356
  4.21637021356
\end{euleroutput}
\begin{eulercomment}
Kita juga dapat menghitung P menggunakan rumus spread.
\end{eulercomment}
\begin{eulerprompt}
>py&=factor(ratsimp(sa2*bisa)); $py
\end{eulerprompt}
\begin{eulercomment}
Nilainya sama dengan yang kita dapatkan dengan rumus trigonometri.
\end{eulercomment}
\begin{eulerprompt}
>sqrt(mxmeval("at(py,[a=3^2,b=4^2])"))
\end{eulerprompt}
\begin{euleroutput}
  1.33333333333
\end{euleroutput}
\eulersubheading{Sudut Tali}
\begin{eulercomment}
Perhatikan situasi berikut.
\end{eulercomment}
\begin{eulerprompt}
>setPlotRange(1.2); ...
>color(1); plotCircle(circleWithCenter([0,0],1)); ...
>A:=[cos(1),sin(1)]; B:=[cos(2),sin(2)]; C:=[cos(6),sin(6)]; ...
>plotPoint(A,"A"); plotPoint(B,"B"); plotPoint(C,"C"); ...
>color(3); plotSegment(A,B,"c"); plotSegment(A,C,"b"); plotSegment(C,B,"a"); ...
>color(1); O:=[0,0];  plotPoint(O,"0"); ...
>plotSegment(A,O); plotSegment(B,O); plotSegment(C,O,"r"); ...
>insimg;
\end{eulerprompt}
\begin{eulercomment}
Kita dapat menggunakan Maxima untuk memecahkan rumus penyebaran
rangkap tiga untuk sudut-sudut di pusat O untuk r. Dengan demikian,
kita memperoleh rumus untuk jari-jari kuadrat pericircle dalam bentuk
kuadran sisi-sisinya.

Kali ini, Maxima menghasilkan beberapa nol kompleks, yang kita
abaikan.
\end{eulercomment}
\begin{eulerprompt}
>&remvalue(a,b,c,r); // hapus nilai-nilai sebelumnya untuk perhitungan baru
>rabc &= rhs(solve(triplespread(spread(b,r,r),spread(a,r,r),spread(c,r,r)),r)[4]); $rabc
\end{eulerprompt}
\begin{eulercomment}
Kita dapat menjadikannya fungsi Euler.
\end{eulercomment}
\begin{eulerprompt}
>function periradius(a,b,c) &= rabc;
\end{eulerprompt}
\begin{eulercomment}
Mari kita periksa hasilnya untuk titik A, B, C.
\end{eulercomment}
\begin{eulerprompt}
>a:=quadrance(B,C); b:=quadrance(A,C); c:=quadrance(A,B);
\end{eulerprompt}
\begin{eulercomment}
Radiusnya memang 1.
\end{eulercomment}
\begin{eulerprompt}
>periradius(a,b,c)
\end{eulerprompt}
\begin{euleroutput}
  1
\end{euleroutput}
\begin{eulercomment}
Faktanya, sebaran CBA hanya bergantung pada b dan c. Ini adalah
teorema sudut tali busur.
\end{eulercomment}
\begin{eulerprompt}
>$spread(b,a,c)*rabc | ratsimp
\end{eulerprompt}
\begin{eulercomment}
Faktanya, sebarannya adalah b/(4r), dan kita melihat bahwa sudut tali
busur b adalah setengah sudut pusat.
\end{eulercomment}
\begin{eulerprompt}
>$doublespread(b/(4*r))-spread(b,r,r) | ratsimp
\end{eulerprompt}
\eulerheading{Contoh 6: Jarak Minimal pada Bidang}
\begin{eulercomment}
\end{eulercomment}
\eulersubheading{Catatan awal}
\begin{eulercomment}
Fungsi yang, pada titik M di bidang, menetapkan jarak AM antara titik
tetap A dan M, memiliki garis datar yang agak sederhana: lingkaran
yang berpusat di A.
\end{eulercomment}
\begin{eulerprompt}
>&remvalue();
>A=[-1,-1];
>function d1(x,y):=sqrt((x-A[1])^2+(y-A[2])^2)
>fcontour("d1",xmin=-2,xmax=0,ymin=-2,ymax=0,hue=1, ...
>title="If you see ellipses, please set your window square"):
\end{eulerprompt}
\begin{eulercomment}
dan grafiknya cukup sederhana: bagian atas kerucut:
\end{eulercomment}
\begin{eulerprompt}
>plot3d("d1",xmin=-2,xmax=0,ymin=-2,ymax=0):
\end{eulerprompt}
\begin{eulercomment}
Tentu saja minimum 0 dicapai di A.

\end{eulercomment}
\eulersubheading{Dua titik}
\begin{eulercomment}
Sekarang kita lihat fungsi MA+MB di mana A dan B adalah dua titik
(tetap). Merupakan "fakta yang diketahui" bahwa kurva level adalah
elips, titik fokusnya adalah A dan B; kecuali untuk minimum AB yang
konstan pada segmen [AB]:
\end{eulercomment}
\begin{eulerprompt}
>B=[1,-1];
>function d2(x,y):=d1(x,y)+sqrt((x-B[1])^2+(y-B[2])^2)
>fcontour("d2",xmin=-2,xmax=2,ymin=-3,ymax=1,hue=1):
\end{eulerprompt}
\begin{eulercomment}
Grafiknya lebih menarik:
\end{eulercomment}
\begin{eulerprompt}
>plot3d("d2",xmin=-2,xmax=2,ymin=-3,ymax=1):
\end{eulerprompt}
\begin{eulercomment}
Pembatasan pada garis (AB) lebih terkenal:
\end{eulercomment}
\begin{eulerprompt}
>plot2d("abs(x+1)+abs(x-1)",xmin=-3,xmax=3):
\end{eulerprompt}
\eulersubheading{Tiga poin}
\begin{eulercomment}
Sekarang semuanya menjadi kurang sederhana: Tidak banyak yang tahu
bahwa MA+MB+MC mencapai nilai minimumnya di satu titik bidang, tetapi
menentukannya tidaklah sesederhana itu:

1) Jika salah satu sudut segitiga ABC lebih dari 120° (misalkan di A),
maka nilai minimumnya tercapai di titik ini (misalkan AB+AC).

Contoh:
\end{eulercomment}
\begin{eulerprompt}
>C=[-4,1];
>function d3(x,y):=d2(x,y)+sqrt((x-C[1])^2+(y-C[2])^2)
>plot3d("d3",xmin=-5,xmax=3,ymin=-4,ymax=4);
>insimg;
>fcontour("d3",xmin=-4,xmax=1,ymin=-2,ymax=2,hue=1,title="The minimum is on A");
>P=(A_B_C_A)'; plot2d(P[1],P[2],add=1,color=12);
>insimg;
\end{eulerprompt}
\begin{eulercomment}
2) Namun jika semua sudut segitiga ABC kurang dari 120°, maka nilai
minimumnya berada di titik F di bagian dalam segitiga, yang merupakan
satu-satunya titik yang sudut-sudut sisi ABC-nya sama (masing-masing
sudutnya 120°):
\end{eulercomment}
\begin{eulerprompt}
>C=[-0.5,1];
>plot3d("d3",xmin=-2,xmax=2,ymin=-2,ymax=2):
>fcontour("d3",xmin=-2,xmax=2,ymin=-2,ymax=2,hue=1,title="The Fermat point");
>P=(A_B_C_A)'; plot2d(P[1],P[2],add=1,color=12);
>insimg;
\end{eulerprompt}
\begin{eulercomment}
Merupakan aktivitas yang menarik untuk mewujudkan gambar di atas
dengan perangkat lunak geometri; misalnya, saya mengetahui perangkat
lunak yang ditulis dalam Java yang memiliki instruksi "garis
kontur"...

Semua ini ditemukan oleh seorang hakim Prancis bernama Pierre de
Fermat; ia menulis surat kepada para dilettan lain seperti pendeta
Marin Mersenne dan Blaise Pascal yang bekerja di pajak penghasilan.
Jadi titik unik F sehingga FA+FB+FC minimal, disebut titik Fermat dari
segitiga tersebut. Namun tampaknya beberapa tahun sebelumnya,
Torriccelli dari Italia telah menemukan titik ini sebelum Fermat
menemukannya! Bagaimanapun tradisinya adalah mencatat titik F ini...

\end{eulercomment}
\eulersubheading{Empat titik}
\begin{eulercomment}
Langkah berikutnya adalah menambahkan titik ke-4 D dan mencoba
meminimalkan MA+MB+MC+MD; katakanlah Anda adalah operator TV kabel dan
ingin mencari di bidang mana Anda harus meletakkan antena Anda
sehingga Anda dapat menyalurkan sinyal ke empat desa dan menggunakan
panjang kabel sesedikit mungkin!
\end{eulercomment}
\begin{eulerprompt}
>D=[1,1];
>function d4(x,y):=d3(x,y)+sqrt((x-D[1])^2+(y-D[2])^2)
>plot3d("d4",xmin=-1.5,xmax=1.5,ymin=-1.5,ymax=1.5):
>fcontour("d4",xmin=-1.5,xmax=1.5,ymin=-1.5,ymax=1.5,hue=1);
>P=(A_B_C_D)'; plot2d(P[1],P[2],points=1,add=1,color=12);
>insimg;
\end{eulerprompt}
\begin{eulercomment}
Masih terdapat nilai minimum dan tidak tercapai di titik A, B, C,
maupun D:
\end{eulercomment}
\begin{eulerprompt}
>function f(x):=d4(x[1],x[2])
>neldermin("f",[0.2,0.2])
\end{eulerprompt}
\begin{euleroutput}
  [0.142858,  0.142857]
\end{euleroutput}
\begin{eulercomment}
Tampaknya dalam kasus ini, koordinat titik optimal bersifat rasional
atau mendekati rasional...

Sekarang ABCD adalah persegi, kita mengharapkan bahwa titik optimal
akan menjadi pusat ABCD:
\end{eulercomment}
\begin{eulerprompt}
>C=[-1,1];
>plot3d("d4",xmin=-1,xmax=1,ymin=-1,ymax=1):
>fcontour("d4",xmin=-1.5,xmax=1.5,ymin=-1.5,ymax=1.5,hue=1);
>P=(A_B_C_D)'; plot2d(P[1],P[2],add=1,color=12,points=1);
>insimg;
\end{eulerprompt}
\eulerheading{Contoh 7: Bola Dandelin dengan Povray}
\begin{eulercomment}
Anda dapat menjalankan demonstrasi ini, jika Anda telah menginstal
Povray, dan pvengine.exe di jalur program.

Pertama, kita hitung jari-jari bola.

Jika Anda melihat gambar di bawah, Anda melihat bahwa kita memerlukan
dua lingkaran yang menyentuh dua garis yang membentuk kerucut, dan
satu garis yang membentuk bidang yang memotong kerucut.

Kami menggunakan file geometry.e milik Euler untuk ini.
\end{eulercomment}
\begin{eulerprompt}
>load geometry;
\end{eulerprompt}
\begin{eulercomment}
Pertama dua garis membentuk kerucut.
\end{eulercomment}
\begin{eulerprompt}
>g1 &= lineThrough([0,0],[1,a])
\end{eulerprompt}
\begin{euleroutput}
  
                               [- a, 1, 0]
  
\end{euleroutput}
\begin{eulerprompt}
>g2 &= lineThrough([0,0],[-1,a])
\end{eulerprompt}
\begin{euleroutput}
  
                              [- a, - 1, 0]
  
\end{euleroutput}
\begin{eulercomment}
Lalu baris ketiga.
\end{eulercomment}
\begin{eulerprompt}
>g &= lineThrough([-1,0],[1,1])
\end{eulerprompt}
\begin{euleroutput}
  
                               [- 1, 2, 1]
  
\end{euleroutput}
\begin{eulercomment}
Kita merencanakan segalanya sejauh ini.
\end{eulercomment}
\begin{eulerprompt}
>setPlotRange(-1,1,0,2);
>color(black); plotLine(g(),"")
>a:=2; color(blue); plotLine(g1(),""), plotLine(g2(),""):
\end{eulerprompt}
\begin{eulercomment}
Sekarang kita ambil titik umum pada sumbu y.
\end{eulercomment}
\begin{eulerprompt}
>P &= [0,u]
\end{eulerprompt}
\begin{euleroutput}
  
                                  [0, u]
  
\end{euleroutput}
\begin{eulercomment}
Hitunglah jarak ke g1.
\end{eulercomment}
\begin{eulerprompt}
>d1 &= distance(P,projectToLine(P,g1)); $d1
\end{eulerprompt}
\begin{eulercomment}
Hitunglah jarak ke g.
\end{eulercomment}
\begin{eulerprompt}
>d &= distance(P,projectToLine(P,g)); $d
\end{eulerprompt}
\begin{eulercomment}
Dan temukan pusat kedua lingkaran, yang jaraknya sama.
\end{eulercomment}
\begin{eulerprompt}
>sol &= solve(d1^2=d^2,u); $sol
\end{eulerprompt}
\begin{eulercomment}
Ada dua solusi.

Kita mengevaluasi solusi simbolik, dan menemukan kedua pusat, dan
kedua jarak.
\end{eulercomment}
\begin{eulerprompt}
>u := sol()
\end{eulerprompt}
\begin{euleroutput}
  [0.333333,  1]
\end{euleroutput}
\begin{eulerprompt}
>dd := d()
\end{eulerprompt}
\begin{euleroutput}
  [0.149071,  0.447214]
\end{euleroutput}
\begin{eulercomment}
Gambarkan lingkaran-lingkaran tersebut ke dalam gambar.
\end{eulercomment}
\begin{eulerprompt}
>color(red);
>plotCircle(circleWithCenter([0,u[1]],dd[1]),"");
>plotCircle(circleWithCenter([0,u[2]],dd[2]),"");
>insimg;
\end{eulerprompt}
\eulersubheading{Plot dengan Povray}
\begin{eulercomment}
Selanjutnya kita plot semuanya dengan Povray. Perhatikan bahwa Anda
mengubah perintah apa pun dalam urutan perintah Povray berikut, dan
menjalankan kembali semua perintah dengan Shift-Return.

Pertama-tama kita memuat fungsi povray.
\end{eulercomment}
\begin{eulerprompt}
>load povray;
>defaultpovray="C:\(\backslash\)Program Files\(\backslash\)POV-Ray\(\backslash\)v3.7\(\backslash\)bin\(\backslash\)pvengine.exe"
\end{eulerprompt}
\begin{euleroutput}
  C:\(\backslash\)Program Files\(\backslash\)POV-Ray\(\backslash\)v3.7\(\backslash\)bin\(\backslash\)pvengine.exe
\end{euleroutput}
\begin{eulercomment}
Kami menyiapkan suasananya dengan tepat.
\end{eulercomment}
\begin{eulerprompt}
>povstart(zoom=11,center=[0,0,0.5],height=10°,angle=140°);
\end{eulerprompt}
\begin{eulercomment}
Berikutnya kita menulis kedua bola itu ke dalam file Povray.
\end{eulercomment}
\begin{eulerprompt}
>writeln(povsphere([0,0,u[1]],dd[1],povlook(red)));
>writeln(povsphere([0,0,u[2]],dd[2],povlook(red)));
\end{eulerprompt}
\begin{eulercomment}
Dan kerucutnya, transparan.
\end{eulercomment}
\begin{eulerprompt}
>writeln(povcone([0,0,0],0,[0,0,a],1,povlook(lightgray,1)));
\end{eulerprompt}
\begin{eulercomment}
Kita buat bidang yang dibatasi pada kerucut.
\end{eulercomment}
\begin{eulerprompt}
>gp=g();
>pc=povcone([0,0,0],0,[0,0,a],1,"");
>vp=[gp[1],0,gp[2]]; dp=gp[3];
>writeln(povplane(vp,dp,povlook(blue,0.5),pc));
\end{eulerprompt}
\begin{eulercomment}
Sekarang kita buat dua titik pada lingkaran, di mana bola menyentuh
kerucut.
\end{eulercomment}
\begin{eulerprompt}
>function turnz(v) := return [-v[2],v[1],v[3]]
>P1=projectToLine([0,u[1]],g1()); P1=turnz([P1[1],0,P1[2]]);
>writeln(povpoint(P1,povlook(yellow)));
>P2=projectToLine([0,u[2]],g1()); P2=turnz([P2[1],0,P2[2]]);
>writeln(povpoint(P2,povlook(yellow)));
\end{eulerprompt}
\begin{eulercomment}
Kemudian kita buat dua titik tempat bola-bola tersebut menyentuh
bidang. Titik-titik ini adalah fokus elips.
\end{eulercomment}
\begin{eulerprompt}
>P3=projectToLine([0,u[1]],g()); P3=[P3[1],0,P3[2]];
>writeln(povpoint(P3,povlook(yellow)));
>P4=projectToLine([0,u[2]],g()); P4=[P4[1],0,P4[2]];
>writeln(povpoint(P4,povlook(yellow)));
\end{eulerprompt}
\begin{eulercomment}
Berikutnya kita hitung perpotongan P1P2 dengan bidang.
\end{eulercomment}
\begin{eulerprompt}
>t1=scalp(vp,P1)-dp; t2=scalp(vp,P2)-dp; P5=P1+t1/(t1-t2)*(P2-P1);
>writeln(povpoint(P5,povlook(yellow)));
\end{eulerprompt}
\begin{eulercomment}
Kita menghubungkan titik-titik dengan segmen garis.
\end{eulercomment}
\begin{eulerprompt}
>writeln(povsegment(P1,P2,povlook(yellow)));
>writeln(povsegment(P5,P3,povlook(yellow)));
>writeln(povsegment(P5,P4,povlook(yellow)));
\end{eulerprompt}
\begin{eulercomment}
Sekarang kita buat pita abu-abu, di mana bola-bola menyentuh kerucut.
\end{eulercomment}
\begin{eulerprompt}
>pcw=povcone([0,0,0],0,[0,0,a],1.01);
>pc1=povcylinder([0,0,P1[3]-defaultpointsize/2],[0,0,P1[3]+defaultpointsize/2],1);
>writeln(povintersection([pcw,pc1],povlook(gray)));
>pc2=povcylinder([0,0,P2[3]-defaultpointsize/2],[0,0,P2[3]+defaultpointsize/2],1);
>writeln(povintersection([pcw,pc2],povlook(gray)));
\end{eulerprompt}
\begin{eulercomment}
Mulai program Povray.
\end{eulercomment}
\begin{eulerprompt}
>povend();
\end{eulerprompt}
\begin{eulercomment}
Untuk mendapatkan Anaglyph ini, kita perlu memasukkan semuanya ke
dalam fungsi scene. Fungsi ini akan digunakan dua kali nanti.
\end{eulercomment}
\begin{eulerprompt}
>function scene () ...
\end{eulerprompt}
\begin{eulerudf}
  global a,u,dd,g,g1,defaultpointsize;
  writeln(povsphere([0,0,u[1]],dd[1],povlook(red)));
  writeln(povsphere([0,0,u[2]],dd[2],povlook(red)));
  writeln(povcone([0,0,0],0,[0,0,a],1,povlook(lightgray,1)));
  gp=g();
  pc=povcone([0,0,0],0,[0,0,a],1,"");
  vp=[gp[1],0,gp[2]]; dp=gp[3];
  writeln(povplane(vp,dp,povlook(blue,0.5),pc));
  P1=projectToLine([0,u[1]],g1()); P1=turnz([P1[1],0,P1[2]]);
  writeln(povpoint(P1,povlook(yellow)));
  P2=projectToLine([0,u[2]],g1()); P2=turnz([P2[1],0,P2[2]]);
  writeln(povpoint(P2,povlook(yellow)));
  P3=projectToLine([0,u[1]],g()); P3=[P3[1],0,P3[2]];
  writeln(povpoint(P3,povlook(yellow)));
  P4=projectToLine([0,u[2]],g()); P4=[P4[1],0,P4[2]];
  writeln(povpoint(P4,povlook(yellow)));
  t1=scalp(vp,P1)-dp; t2=scalp(vp,P2)-dp; P5=P1+t1/(t1-t2)*(P2-P1);
  writeln(povpoint(P5,povlook(yellow)));
  writeln(povsegment(P1,P2,povlook(yellow)));
  writeln(povsegment(P5,P3,povlook(yellow)));
  writeln(povsegment(P5,P4,povlook(yellow)));
  pcw=povcone([0,0,0],0,[0,0,a],1.01);
  pc1=povcylinder([0,0,P1[3]-defaultpointsize/2],[0,0,P1[3]+defaultpointsize/2],1);
  writeln(povintersection([pcw,pc1],povlook(gray)));
  pc2=povcylinder([0,0,P2[3]-defaultpointsize/2],[0,0,P2[3]+defaultpointsize/2],1);
  writeln(povintersection([pcw,pc2],povlook(gray)));
  endfunction
\end{eulerudf}
\begin{eulercomment}
Anda memerlukan kacamata merah/cyan untuk menghargai efek berikut.
\end{eulercomment}
\begin{eulerprompt}
>povanaglyph("scene",zoom=11,center=[0,0,0.5],height=10°,angle=140°);
\end{eulerprompt}
\eulerheading{Contoh 8: Geometri Bumi}
\begin{eulercomment}
Dalam buku catatan ini, kami ingin melakukan beberapa perhitungan
sferis. Fungsi-fungsi tersebut terdapat dalam berkas "spherical.e" di
folder contoh. Kami perlu memuat berkas tersebut terlebih dahulu.
\end{eulercomment}
\begin{eulerprompt}
>load "spherical.e";
\end{eulerprompt}
\begin{eulercomment}
Untuk memasukkan posisi geografis, kami menggunakan vektor dengan dua
koordinat dalam radian (utara dan timur, nilai negatif untuk selatan
dan barat). Berikut ini adalah koordinat untuk Kampus FMIPA UNY.
\end{eulercomment}
\begin{eulerprompt}
>FMIPA=[rad(-7,-46.467),rad(110,23.05)]
\end{eulerprompt}
\begin{euleroutput}
  [-0.13569,  1.92657]
\end{euleroutput}
\begin{eulercomment}
Anda dapat mencetak posisi ini dengan sposprint (cetak posisi bulat).
\end{eulercomment}
\begin{eulerprompt}
>sposprint(FMIPA) // posisi garis lintang dan garis bujur FMIPA UNY
\end{eulerprompt}
\begin{euleroutput}
  S 7°46.467' E 110°23.050'
\end{euleroutput}
\begin{eulercomment}
Mari kita tambahkan dua kota lagi, Solo dan Semarang.
\end{eulercomment}
\begin{eulerprompt}
>Solo=[rad(-7,-34.333),rad(110,49.683)]; Semarang=[rad(-6,-59.05),rad(110,24.533)];
>sposprint(Solo), sposprint(Semarang),
\end{eulerprompt}
\begin{euleroutput}
  S 7°34.333' E 110°49.683'
  S 6°59.050' E 110°24.533'
\end{euleroutput}
\begin{eulercomment}
Pertama, kita hitung vektor dari satu ke yang lain pada bola ideal.
Vektor ini adalah [arah, jarak] dalam radian. Untuk menghitung jarak
di bumi, kita kalikan dengan jari-jari bumi pada garis lintang 7°.
\end{eulercomment}
\begin{eulerprompt}
>br=svector(FMIPA,Solo); degprint(br[1]), br[2]*rearth(7°)->km // perkiraan jarak FMIPA-Solo
\end{eulerprompt}
\begin{euleroutput}
  65°20'26.60''
  53.8945384608
\end{euleroutput}
\begin{eulercomment}
Ini adalah perkiraan yang bagus. Rutin berikut menggunakan perkiraan
yang lebih baik lagi. Pada jarak yang pendek, hasilnya hampir sama.
\end{eulercomment}
\begin{eulerprompt}
>esdist(FMIPA,Semarang)->" km"; // perkiraan jarak FMIPA-Semarang
\end{eulerprompt}
\begin{eulercomment}
Ada fungsi untuk judul, yang memperhitungkan bentuk elips bumi. Sekali
lagi, kami mencetak dengan cara yang canggih.
\end{eulercomment}
\begin{eulerprompt}
>sdegprint(esdir(FMIPA,Solo))
\end{eulerprompt}
\begin{euleroutput}
       65.34°
\end{euleroutput}
\begin{eulercomment}
Sudut suatu segitiga melebihi 180° pada bola.
\end{eulercomment}
\begin{eulerprompt}
>asum=sangle(Solo,FMIPA,Semarang)+sangle(FMIPA,Solo,Semarang)+sangle(FMIPA,Semarang,Solo); degprint(asum)
\end{eulerprompt}
\begin{euleroutput}
  180°0'10.77''
\end{euleroutput}
\begin{eulercomment}
Ini dapat digunakan untuk menghitung luas segitiga. Catatan: Untuk
segitiga kecil, ini tidak akurat karena kesalahan pengurangan dalam
asum-pi.
\end{eulercomment}
\begin{eulerprompt}
>(asum-pi)*rearth(48°)^2->" km^2"; // perkiraan luas segitiga FMIPA-Solo-Semarang
\end{eulerprompt}
\begin{eulercomment}
Ada fungsi untuk ini, yang menggunakan lintang rata-rata segitiga
untuk menghitung jari-jari bumi, dan menangani kesalahan pembulatan
untuk segitiga yang sangat kecil.
\end{eulercomment}
\begin{eulerprompt}
>esarea(Solo,FMIPA,Semarang)->" km^2", //perkiraan yang sama dengan fungsi esarea()
\end{eulerprompt}
\begin{euleroutput}
  2123.64310526 km^2
\end{euleroutput}
\begin{eulercomment}
Kita juga dapat menambahkan vektor ke posisi. Vektor berisi arah dan
jarak, keduanya dalam radian. Untuk mendapatkan vektor, kita
menggunakan svector. Untuk menambahkan vektor ke posisi, kita
menggunakan saddvector.
\end{eulercomment}
\begin{eulerprompt}
>v=svector(FMIPA,Solo); sposprint(saddvector(FMIPA,v)), sposprint(Solo),
\end{eulerprompt}
\begin{euleroutput}
  S 7°34.333' E 110°49.683'
  S 7°34.333' E 110°49.683'
\end{euleroutput}
\begin{eulercomment}
Fungsi-fungsi ini mengasumsikan bentuk bola yang ideal. Sama halnya di
bumi.
\end{eulercomment}
\begin{eulerprompt}
>sposprint(esadd(FMIPA,esdir(FMIPA,Solo),esdist(FMIPA,Solo))), sposprint(Solo),
\end{eulerprompt}
\begin{euleroutput}
  S 7°34.333' E 110°49.683'
  S 7°34.333' E 110°49.683'
\end{euleroutput}
\begin{eulercomment}
Mari kita lihat contoh yang lebih besar, Tugu Jogja dan Monas Jakarta
(menggunakan Google Earth untuk mencari koordinatnya).
\end{eulercomment}
\begin{eulerprompt}
>Tugu=[-7.7833°,110.3661°]; Monas=[-6.175°,106.811944°];
>sposprint(Tugu), sposprint(Monas)
\end{eulerprompt}
\begin{euleroutput}
  S 7°46.998' E 110°21.966'
  S 6°10.500' E 106°48.717'
\end{euleroutput}
\begin{eulercomment}
Menurut Google Earth, jaraknya adalah 429,66 km. Kami memperoleh
perkiraan yang baik.
\end{eulercomment}
\begin{eulerprompt}
>esdist(Tugu,Monas)->" km"; // perkiraan jarak Tugu Jogja - Monas Jakarta
\end{eulerprompt}
\begin{eulercomment}
Judulnya sama dengan yang dihitung di Google Earth.
\end{eulercomment}
\begin{eulerprompt}
>degprint(esdir(Tugu,Monas))
\end{eulerprompt}
\begin{euleroutput}
  294°17'2.85''
\end{euleroutput}
\begin{eulercomment}
Akan tetapi, kita tidak lagi memperoleh posisi target yang tepat, jika
kita menambahkan arah dan jarak ke posisi awal. Hal ini terjadi karena
kita tidak menghitung fungsi invers secara tepat, tetapi mengambil
perkiraan radius bumi di sepanjang lintasan.
\end{eulercomment}
\begin{eulerprompt}
>sposprint(esadd(Tugu,esdir(Tugu,Monas),esdist(Tugu,Monas)))
\end{eulerprompt}
\begin{euleroutput}
  S 6°10.500' E 106°48.717'
\end{euleroutput}
\begin{eulercomment}
Namun, kesalahannya tidak besar.
\end{eulercomment}
\begin{eulerprompt}
>sposprint(Monas),
\end{eulerprompt}
\begin{euleroutput}
  S 6°10.500' E 106°48.717'
\end{euleroutput}
\begin{eulercomment}
Tentu saja, kita tidak dapat berlayar dengan arah yang sama dari satu
tujuan ke tujuan lain, jika kita ingin mengambil jalur terpendek.
Bayangkan, Anda terbang ke arah timur laut mulai dari titik mana pun
di bumi. Kemudian Anda akan berputar ke kutub utara. Lingkaran besar
tidak mengikuti arah yang konstan!

Perhitungan berikut menunjukkan bahwa kita jauh dari tujuan yang
benar, jika kita menggunakan arah yang sama selama perjalanan kita.
\end{eulercomment}
\begin{eulerprompt}
>dist=esdist(Tugu,Monas); hd=esdir(Tugu,Monas);
\end{eulerprompt}
\begin{eulercomment}
Sekarang kita tambahkan 10 dikalikan sepersepuluh jaraknya, dengan
memakai arah ke Monas, kita sampai di Tugu.
\end{eulercomment}
\begin{eulerprompt}
>p=Tugu; loop 1 to 10; p=esadd(p,hd,dist/10); end;
\end{eulerprompt}
\begin{eulercomment}
Hasilnya sangat jauh.
\end{eulercomment}
\begin{eulerprompt}
>sposprint(p), skmprint(esdist(p,Monas))
\end{eulerprompt}
\begin{euleroutput}
  S 6°11.250' E 106°48.372'
       1.529km
\end{euleroutput}
\begin{eulercomment}
Sebagai contoh lain, mari kita ambil dua titik di bumi pada garis
lintang yang sama.
\end{eulercomment}
\begin{eulerprompt}
>P1=[30°,10°]; P2=[30°,50°];
\end{eulerprompt}
\begin{eulercomment}
Lintasan terpendek dari P1 ke P2 bukanlah lingkaran lintang 30°,
tetapi lintasan yang lebih pendek yang dimulai 10° lebih jauh ke utara
di P1.
\end{eulercomment}
\begin{eulerprompt}
>sdegprint(esdir(P1,P2))
\end{eulerprompt}
\begin{euleroutput}
       79.69°
\end{euleroutput}
\begin{eulercomment}
Namun, jika kita mengikuti pembacaan kompas ini, kita akan berputar ke
kutub utara! Jadi kita harus menyesuaikan arah kita di sepanjang
jalan. Untuk tujuan kasar, kita menyesuaikannya pada 1/10 dari total
jarak.
\end{eulercomment}
\begin{eulerprompt}
>p=P1;  dist=esdist(P1,P2); ...
>  loop 1 to 10; dir=esdir(p,P2); sdegprint(dir), p=esadd(p,dir,dist/10); end;
\end{eulerprompt}
\begin{euleroutput}
       79.69°
       81.67°
       83.71°
       85.78°
       87.89°
       90.00°
       92.12°
       94.22°
       96.29°
       98.33°
\end{euleroutput}
\begin{eulercomment}
Jaraknya tidak tepat, karena kita akan menambahkan sedikit kesalahan,
jika kita mengikuti arah yang sama terlalu lama.
\end{eulercomment}
\begin{eulerprompt}
>skmprint(esdist(p,P2))
\end{eulerprompt}
\begin{euleroutput}
       0.203km
\end{euleroutput}
\begin{eulercomment}
Kita memperoleh perkiraan yang baik, jika kita menyesuaikan arah
setelah setiap 1/100 jarak total dari Tugu ke Monas.
\end{eulercomment}
\begin{eulerprompt}
>p=Tugu; dist=esdist(Tugu,Monas); ...
>  loop 1 to 100; p=esadd(p,esdir(p,Monas),dist/100); end;
>skmprint(esdist(p,Monas))
\end{eulerprompt}
\begin{euleroutput}
       0.000km
\end{euleroutput}
\begin{eulercomment}
Untuk keperluan navigasi, kita bisa mendapatkan urutan posisi GPS
sepanjang lingkaran besar menuju Monas dengan fungsi navigasi.
\end{eulercomment}
\begin{eulerprompt}
>load spherical; v=navigate(Tugu,Monas,10); ...
>  loop 1 to rows(v); sposprint(v[#]), end;
\end{eulerprompt}
\begin{euleroutput}
  S 7°46.998' E 110°21.966'
  S 7°37.422' E 110°0.573'
  S 7°27.829' E 109°39.196'
  S 7°18.219' E 109°17.834'
  S 7°8.592' E 108°56.488'
  S 6°58.948' E 108°35.157'
  S 6°49.289' E 108°13.841'
  S 6°39.614' E 107°52.539'
  S 6°29.924' E 107°31.251'
  S 6°20.219' E 107°9.977'
  S 6°10.500' E 106°48.717'
\end{euleroutput}
\begin{eulercomment}
Kita menulis suatu fungsi yang memplot bumi, dua posisi, dan posisi di
antaranya.
\end{eulercomment}
\begin{eulerprompt}
>function testplot ...
\end{eulerprompt}
\begin{eulerudf}
  useglobal;
  plotearth;
  plotpos(Tugu,"Tugu Jogja"); plotpos(Monas,"Tugu Monas");
  plotposline(v);
  endfunction
\end{eulerudf}
\begin{eulercomment}
Sekarang rencanakan semuanya.
\end{eulercomment}
\begin{eulerprompt}
>plot3d("testplot",angle=25, height=6,>own,>user,zoom=4):
\end{eulerprompt}
\begin{eulercomment}
Atau gunakan plot3d untuk mendapatkan tampilan anaglifnya. Ini tampak
sangat bagus dengan kaca mata merah/biru kehijauan.
\end{eulercomment}
\begin{eulerprompt}
>plot3d("testplot",angle=25,height=6,distance=5,own=1,anaglyph=1,zoom=4):
\end{eulerprompt}
\eulerheading{Latihan}
\begin{eulercomment}
1. Gambarlah segi-n beraturan jika diketahui titik pusat O, n, dan
jarak titik pusat ke titik-titik sudut segi-n tersebut (jari-jari
lingkaran luar segi-n), r.

Petunjuk:

- Besar sudut pusat yang menghadap masing-masing sisi segi-n adalah
(360/n).\\
- Titik-titik sudut segi-n merupakan perpotongan lingkaran luar segi-n
dan garis-garis yang melalui pusat dan saling membentuk sudut sebesar
kelipatan (360/n).\\
- Untuk n ganjil, pilih salah satu titik sudut adalah di atas.\\
- Untuk n genap, pilih 2 titik di kanan dan kiri lurus dengan titik
pusat.\\
- Anda dapat menggambar segi-3, 4, 5, 6, 7, dst beraturan.
\end{eulercomment}
\begin{eulerprompt}
>load geometry
\end{eulerprompt}
\begin{euleroutput}
  Numerical and symbolic geometry.
\end{euleroutput}
\begin{eulerprompt}
>setPlotRange(-3.5,3.5,-3.5,3.5);
>A=[-2,-2]; plotPoint(A,"A");
>B=[2,-2]; plotPoint(B,"B");
>C=[0,3]; plotPoint(C,"C");
>plotSegment(A,B,"c");
>plotSegment(B,C,"a");
>plotSegment(A,C,"b");
>aspect(1):
>c=circleThrough(A,B,C);
>R=getCircleRadius(c);
>O=getCircleCenter(c);
>plotPoint(O,"O");
>l=angleBisector(A,C,B);
>color(2); plotLine(l); color(1);
>plotCircle(c,"Lingkaran luar segitiga ABC"):
\end{eulerprompt}
\begin{eulercomment}
2. Gambarlah suatu parabola yang melalui 3 titik yang diketahui.

Petunjuk:\\
- Misalkan persamaan parabolanya y= ax\textasciicircum{}2+bx+c.\\
- Substitusikan koordinat titik-titik yang diketahui ke persamaan
tersebut.\\
- Selesaikan SPL yang terbentuk untuk mendapatkan nilai-nilai a, b, c.
\end{eulercomment}
\begin{eulerprompt}
>load geometry;
>setPlotRange(5); P=[2,0]; Q=[4,0]; R=[0,-4];
>plotPoint(P,"P"); plotPoint(Q,"Q"); plotPoint(R,"R"):
>sol &= solve([a+b=-c,16*a+4*b=-c,c=-4],[a,b,c])
\end{eulerprompt}
\begin{euleroutput}
  
                       [[a = - 1, b = 5, c = - 4]]
  
\end{euleroutput}
\begin{eulerprompt}
>function y&=-x^2+5*x-4
\end{eulerprompt}
\begin{euleroutput}
  
                                 2
                              - x  + 5 x - 4
  
\end{euleroutput}
\begin{eulerprompt}
>plot2d("-x^2+5*x-4",-5,5,-5,5):
\end{eulerprompt}
\begin{eulercomment}
3. Gambarlah suatu segi-4 yang diketahui keempat titik sudutnya,
misalnya A, B, C, D.\\
\end{eulercomment}
\begin{eulerttcomment}
   - Tentukan apakah segi-4 tersebut merupakan segi-4 garis singgung
\end{eulerttcomment}
\begin{eulercomment}
(sisinya-sisintya merupakan garis singgung lingkaran yang sama yakni
lingkaran dalam segi-4 tersebut).\\
\end{eulercomment}
\begin{eulerttcomment}
   - Suatu segi-4 merupakan segi-4 garis singgung apabila keempat
\end{eulerttcomment}
\begin{eulercomment}
garis bagi sudutnya bertemu di satu titik.\\
\end{eulercomment}
\begin{eulerttcomment}
   - Jika segi-4 tersebut merupakan segi-4 garis singgung, gambar
\end{eulerttcomment}
\begin{eulercomment}
lingkaran dalamnya.\\
\end{eulercomment}
\begin{eulerttcomment}
   - Tunjukkan bahwa syarat suatu segi-4 merupakan segi-4 garis
\end{eulerttcomment}
\begin{eulercomment}
singgung apabila hasil kali panjang sisi-sisi yang berhadapan sama.
\end{eulercomment}
\begin{eulerprompt}
>load geometry
\end{eulerprompt}
\begin{euleroutput}
  Numerical and symbolic geometry.
\end{euleroutput}
\begin{eulerprompt}
>setPlotRange(-4.5,4.5,-4.5,4.5);
>A=[-3,-3]; plotPoint(A,"A");
>B=[3,-3]; plotPoint(B,"B");
>C=[3,3]; plotPoint(C,"C");
>D=[-3,3]; plotPoint(D,"D");
>plotSegment(A,B,"");
>plotSegment(B,C,"");
>plotSegment(C,D,"");
>plotSegment(A,D,"");
>aspect(1):
>l=angleBisector(A,B,C);
>m=angleBisector(B,C,D);
>P=lineIntersection(l,m);
>color(5); plotLine(l); plotLine(m); color(1);
>plotPoint(P,"P"):
\end{eulerprompt}
\begin{eulercomment}
Dari gambar diatas terlihat bahwa keempat garis bagi sudutnya bertemu
di satu titik yaitu titik P.
\end{eulercomment}
\begin{eulerprompt}
>r=norm(P-projectToLine(P,lineThrough(A,B)));
>plotCircle(circleWithCenter(P,r),"Lingkaran dalam segiempat ABCD"):
\end{eulerprompt}
\begin{eulercomment}
Dari gambar diatas, terlihat bahwa sisi-sisinya merupakan garis
singgung lingkaran yang sama yaitu lingkaran dalam segiempat.\\
Akan ditunjukkan bahwa hasil kali panjang sisi-sisi yang berhadapan
sama.
\end{eulercomment}
\begin{eulerprompt}
>AB=norm(A-B) //panjang sisi AB
\end{eulerprompt}
\begin{euleroutput}
  6
\end{euleroutput}
\begin{eulerprompt}
>CD=norm(C-D) //panjang sisi CD
\end{eulerprompt}
\begin{euleroutput}
  6
\end{euleroutput}
\begin{eulerprompt}
>AD=norm(A-D) //panjang sisi AD
\end{eulerprompt}
\begin{euleroutput}
  6
\end{euleroutput}
\begin{eulerprompt}
>BC=norm(B-C) //panjang sisi BC
\end{eulerprompt}
\begin{euleroutput}
  6
\end{euleroutput}
\begin{eulerprompt}
>AB.CD
\end{eulerprompt}
\begin{euleroutput}
  36
\end{euleroutput}
\begin{eulerprompt}
>AD.BC
\end{eulerprompt}
\begin{euleroutput}
  36
\end{euleroutput}
\begin{eulercomment}
Terbukti bahwa hasil kali panjang sisi-sisi yang berhadapan sama yaitu
36. Jadi dapat dipastikan bahwa segiempat tersebut merupakan segiempat
garis singgung.


4. Gambarlah suatu ellips jika diketahui kedua titik fokusnya,
misalnya P dan Q. Ingat ellips dengan fokus P dan Q adalah tempat
kedudukan titik-titik yang jumlah jarak ke P dan ke Q selalu sama
(konstan).

Penyelesaian :\\
Diketahui kedua titik fokus P = [-1,-1] dan Q = [1,-1]
\end{eulercomment}
\begin{eulerprompt}
>P=[-1,-1]; Q=[1,-1];
>function d1(x,y):=sqrt((x-P[1])^2+(y-P[2])^2)
>Q=[1,-1]; function d2(x,y):=sqrt((x-P[1])^2+(y-P[2])^2)+sqrt((x-Q[1])^2+(y-Q[2])^2)
>fcontour("d2",xmin=-2,xmax=2,ymin=-3,ymax=1,hue=1):
\end{eulerprompt}
\begin{eulercomment}
Grafik yang lebih menarik
\end{eulercomment}
\begin{eulerprompt}
>plot3d("d2",xmin=-2,xmax=2,ymin=-3,ymax=1):
\end{eulerprompt}
\begin{eulercomment}
Batasan ke garis PQ
\end{eulercomment}
\begin{eulerprompt}
>plot2d("abs(x+1)+abs(x-1)",xmin=-3,xmax=3):
\end{eulerprompt}
\begin{eulercomment}
5. Gambarlah suatu hiperbola jika diketahui kedua titik fokusnya,
misalnya P dan Q. Ingat ellips dengan fokus P dan Q adalah tempat
kedudukan titik-titik yang selisih jarak ke P dan ke Q selalu sama
(konstan).
\end{eulercomment}
\begin{eulerprompt}
>P=[-1,-1]; Q=[1,-1];
>function d1(x,y):=sqrt((x-p[1])^2+(y-p[2])^2)
>Q=[1,-1]; function d2(x,y):=sqrt((x-P[1])^2+(y-P[2])^2)+sqrt((x+Q[1])^2+(y+Q[2])^2)
>fcontour("d2",xmin=-2,xmax=2,ymin=-3,ymax=1,hue=1):
>plot3d("d2",xmin=-2,xmax=2,ymin=-3,ymax=1):
>plot2d("abs(x+1)+abs(x-1)",xmin=-3,xmax=3):
\end{eulerprompt}
\begin{euleroutput}
  
\end{euleroutput}
\eulersubheading{}
\begin{eulercomment}
Nama  : Alifia Rahmawati\\
NIM   : 23030630044\\
Kelas : Matematika E\\
\end{eulercomment}
\eulersubheading{}
\begin{eulercomment}
\begin{eulercomment}
\eulerheading{EMT untuk Statistik}
\begin{eulercomment}
Dalam buku catatan ini, kami menunjukkan plot, pengujian, dan
distribusi statistik utama dalam Euler.

Mari kita mulai dengan beberapa statistik deskriptif. Ini bukan
pengantar statistik. Jadi, Anda mungkin memerlukan beberapa latar
belakang untuk memahami detailnya.

Asumsikan pengukuran berikut. Kami ingin menghitung nilai rata-rata
dan deviasi standar yang diukur.
\end{eulercomment}
\begin{eulerprompt}
>M=[1000,1004,998,997,1002,1001,998,1004,998,997]; ...
>median(M), mean(M), dev(M),
\end{eulerprompt}
\begin{euleroutput}
  999
  999.9
  2.72641400622
\end{euleroutput}
\begin{eulercomment}
Kita dapat membuat diagram kotak dan kumis untuk data tersebut. Dalam
kasus kita, tidak ada outlier.
\end{eulercomment}
\begin{eulerprompt}
>aspect(1.75); boxplot(M):
\end{eulerprompt}
\begin{eulercomment}
Kami menghitung probabilitas bahwa suatu nilai lebih besar dari 1005,
dengan asumsi nilai terukur dari distribusi normal.

Semua fungsi untuk distribusi dalam Euler diakhiri dengan ...dis dan
menghitung distribusi probabilitas kumulatif (CPF).

\end{eulercomment}
\begin{eulerformula}
\[
\text{normaldis(x,m,d)}=\int_{-\infty}^x \frac{1}{d\sqrt{2\pi}}e^{-\frac{1}{2}(\frac{t-m}{d})^2}\ dt.
\]
\end{eulerformula}
\begin{eulercomment}
Kami mencetak hasil dalam \% dengan akurasi 2 digit menggunakan fungsi
cetak.
\end{eulercomment}
\begin{eulerprompt}
>print((1-normaldis(1005,mean(M),dev(M)))*100,2,unit=" %")
\end{eulerprompt}
\begin{euleroutput}
        3.07 %
\end{euleroutput}
\begin{eulercomment}
Untuk contoh berikutnya, kami mengasumsikan jumlah pria berikut dalam
rentang ukuran tertentu.
\end{eulercomment}
\begin{eulerprompt}
>r=155.5:4:187.5; v=[22,71,136,169,139,71,32,8];
\end{eulerprompt}
\begin{eulercomment}
Berikut adalah plot distribusinya.
\end{eulercomment}
\begin{eulerprompt}
>plot2d(r,v,a=150,b=200,c=0,d=190,bar=1,style="\(\backslash\)/"):
\end{eulerprompt}
\begin{eulercomment}
Kita dapat memasukkan data mentah tersebut ke dalam tabel.

Tabel adalah metode untuk menyimpan data statistik. Tabel kita harus
berisi tiga kolom: Awal rentang, akhir rentang, jumlah orang dalam
rentang.

Tabel dapat dicetak dengan tajuk. Kita menggunakan vektor string untuk
mengatur tajuk.
\end{eulercomment}
\begin{eulerprompt}
>T:=r[1:8]' | r[2:9]' | v'; writetable(T,labc=["BB","BA","Frek"])
\end{eulerprompt}
\begin{euleroutput}
          BB        BA      Frek
       155.5     159.5        22
       159.5     163.5        71
       163.5     167.5       136
       167.5     171.5       169
       171.5     175.5       139
       175.5     179.5        71
       179.5     183.5        32
       183.5     187.5         8
\end{euleroutput}
\begin{eulercomment}
Jika kita memerlukan nilai rata-rata dan statistik ukuran lainnya,
kita perlu menghitung titik tengah rentang. Kita dapat menggunakan dua
kolom pertama tabel kita untuk ini.

Simbol "\textbar{}" digunakan untuk memisahkan kolom, fungsi "writetable"
digunakan untuk menulis tabel, dengan opsi "labc" untuk menentukan
tajuk kolom.
\end{eulercomment}
\begin{eulerprompt}
>(T[,1]+T[,2])/2 // the midpoint of each interval
\end{eulerprompt}
\begin{euleroutput}
          157.5 
          161.5 
          165.5 
          169.5 
          173.5 
          177.5 
          181.5 
          185.5 
\end{euleroutput}
\begin{eulercomment}
Namun lebih mudah untuk melipat rentang dengan vektor [1/2,1/2].
\end{eulercomment}
\begin{eulerprompt}
>M=fold(r,[0.5,0.5])
\end{eulerprompt}
\begin{euleroutput}
  [157.5,  161.5,  165.5,  169.5,  173.5,  177.5,  181.5,  185.5]
\end{euleroutput}
\begin{eulercomment}
Sekarang kita dapat menghitung rata-rata dan deviasi sampel dengan
frekuensi yang diberikan.
\end{eulercomment}
\begin{eulerprompt}
>\{m,d\}=meandev(M,v); m, d,
\end{eulerprompt}
\begin{euleroutput}
  169.901234568
  5.98912964449
\end{euleroutput}
\begin{eulercomment}
Mari kita tambahkan distribusi normal nilai-nilai tersebut ke diagram
batang di atas. Rumus untuk distribusi normal dengan rata-rata m dan
simpangan baku d adalah:

\end{eulercomment}
\begin{eulerformula}
\[
y=\frac{1}{d\sqrt{2\pi}}e^{\frac{-(x-m)^2}{2d^2}}.
\]
\end{eulerformula}
\begin{eulercomment}
Karena nilainya berada di antara 0 dan 1, untuk memplotnya pada
diagram batang, nilainya harus dikalikan dengan 4 kali jumlah total
data.
\end{eulercomment}
\begin{eulerprompt}
>plot2d("qnormal(x,m,d)*sum(v)*4", ...
>  xmin=min(r),xmax=max(r),thickness=3,add=1):
\end{eulerprompt}
\eulerheading{Tabel}
\begin{eulercomment}
Di direktori buku catatan ini, Anda akan menemukan berkas dengan
tabel. Data tersebut merupakan hasil survei. Berikut adalah empat
baris pertama berkas tersebut. Data tersebut berasal dari buku daring
Jerman "Einführung in die Statistik mit R" karya A. Handl.
\end{eulercomment}
\begin{eulerprompt}
>printfile("table.dat",4);
\end{eulerprompt}
\begin{euleroutput}
  Person Sex Age Titanic Evaluation Tip Problem
  1 m 30 n . 1.80 n
  2 f 23 y g 1.80 n
  3 f 26 y g 1.80 y
\end{euleroutput}
\begin{eulercomment}
Tabel berisi 7 kolom angka atau token (string). Kita ingin membaca
tabel dari file. Pertama, kita menggunakan terjemahan kita sendiri
untuk token.

Untuk ini, kita mendefinisikan set token. Fungsi strtokens()
mendapatkan vektor string token dari string yang diberikan.
\end{eulercomment}
\begin{eulerprompt}
>mf:=["m","f"]; yn:=["y","n"]; ev:=strtokens("g vg m b vb");
\end{eulerprompt}
\begin{eulercomment}
Sekarang kita baca tabel dengan terjemahan ini.

Argumen tok2, tok4, dst. adalah terjemahan kolom-kolom tabel. Argumen
ini tidak ada dalam daftar parameter readtable(), jadi Anda perlu
menyediakannya dengan ":=".
\end{eulercomment}
\begin{eulerprompt}
>\{MT,hd\}=readtable("table.dat",tok2:=mf,tok4:=yn,tok5:=ev,tok7:=yn);
>load over statistics;
\end{eulerprompt}
\begin{eulercomment}
Untuk mencetak, kita perlu menentukan set token yang sama. Kita cetak
empat baris pertama saja.
\end{eulercomment}
\begin{eulerprompt}
>writetable(MT[1:10],labc=hd,wc=5,tok2:=mf,tok4:=yn,tok5:=ev,tok7:=yn);
\end{eulerprompt}
\begin{euleroutput}
   Person  Sex  Age Titanic Evaluation  Tip Problem
        1    m   30       n          .  1.8       n
        2    f   23       y          g  1.8       n
        3    f   26       y          g  1.8       y
        4    m   33       n          .  2.8       n
        5    m   37       n          .  1.8       n
        6    m   28       y          g  2.8       y
        7    f   31       y         vg  2.8       n
        8    m   23       n          .  0.8       n
        9    f   24       y         vg  1.8       y
       10    m   26       n          .  1.8       n
\end{euleroutput}
\begin{eulercomment}
Titik "." mewakili nilai yang tidak tersedia.

Jika kita tidak ingin menentukan token untuk penerjemahan terlebih
dahulu, kita hanya perlu menentukan kolom mana yang berisi token dan
bukan angka.
\end{eulercomment}
\begin{eulerprompt}
>ctok=[2,4,5,7]; \{MT,hd,tok\}=readtable("table.dat",ctok=ctok);
\end{eulerprompt}
\begin{eulercomment}
Fungsi readtable() sekarang mengembalikan serangkaian token.
\end{eulercomment}
\begin{eulerprompt}
>tok
\end{eulerprompt}
\begin{euleroutput}
  m
  n
  f
  y
  g
  vg
\end{euleroutput}
\begin{eulercomment}
Tabel berisi entri dari berkas dengan token yang diterjemahkan ke
angka.

String khusus NA="." ditafsirkan sebagai "Tidak Tersedia", dan
mendapatkan NAN (bukan angka) dalam tabel. Terjemahan ini dapat diubah
dengan parameter NA, dan NAval.
\end{eulercomment}
\begin{eulerprompt}
>MT[1]
\end{eulerprompt}
\begin{euleroutput}
  [1,  1,  30,  2,  NAN,  1.8,  2]
\end{euleroutput}
\begin{eulercomment}
Berikut ini adalah isi tabel dengan angka yang belum diterjemahkan.
\end{eulercomment}
\begin{eulerprompt}
>writetable(MT,wc=5)
\end{eulerprompt}
\begin{euleroutput}
      1    1   30    2    .  1.8    2
      2    3   23    4    5  1.8    2
      3    3   26    4    5  1.8    4
      4    1   33    2    .  2.8    2
      5    1   37    2    .  1.8    2
      6    1   28    4    5  2.8    4
      7    3   31    4    6  2.8    2
      8    1   23    2    .  0.8    2
      9    3   24    4    6  1.8    4
     10    1   26    2    .  1.8    2
     11    3   23    4    6  1.8    4
     12    1   32    4    5  1.8    2
     13    1   29    4    6  1.8    4
     14    3   25    4    5  1.8    4
     15    3   31    4    5  0.8    2
     16    1   26    4    5  2.8    2
     17    1   37    2    .  3.8    2
     18    1   38    4    5    .    2
     19    3   29    2    .  3.8    2
     20    3   28    4    6  1.8    2
     21    3   28    4    1  2.8    4
     22    3   28    4    6  1.8    4
     23    3   38    4    5  2.8    2
     24    3   27    4    1  1.8    4
     25    1   27    2    .  2.8    4
\end{euleroutput}
\begin{eulercomment}
Demi kenyamanan, Anda dapat memasukkan output readtable() ke dalam
daftar.
\end{eulercomment}
\begin{eulerprompt}
>Table=\{\{readtable("table.dat",ctok=ctok)\}\};
\end{eulerprompt}
\begin{eulercomment}
Dengan menggunakan kolom token yang sama dan token yang dibaca dari
berkas, kita dapat mencetak tabel. Kita dapat menentukan ctok, tok,
dll. atau menggunakan daftar Tabel.
\end{eulercomment}
\begin{eulerprompt}
>writetable(Table,ctok=ctok,wc=5);
\end{eulerprompt}
\begin{euleroutput}
   Person  Sex  Age Titanic Evaluation  Tip Problem
        1    m   30       n          .  1.8       n
        2    f   23       y          g  1.8       n
        3    f   26       y          g  1.8       y
        4    m   33       n          .  2.8       n
        5    m   37       n          .  1.8       n
        6    m   28       y          g  2.8       y
        7    f   31       y         vg  2.8       n
        8    m   23       n          .  0.8       n
        9    f   24       y         vg  1.8       y
       10    m   26       n          .  1.8       n
       11    f   23       y         vg  1.8       y
       12    m   32       y          g  1.8       n
       13    m   29       y         vg  1.8       y
       14    f   25       y          g  1.8       y
       15    f   31       y          g  0.8       n
       16    m   26       y          g  2.8       n
       17    m   37       n          .  3.8       n
       18    m   38       y          g    .       n
       19    f   29       n          .  3.8       n
       20    f   28       y         vg  1.8       n
       21    f   28       y          m  2.8       y
       22    f   28       y         vg  1.8       y
       23    f   38       y          g  2.8       n
       24    f   27       y          m  1.8       y
       25    m   27       n          .  2.8       y
\end{euleroutput}
\begin{eulercomment}
Fungsi tablecol() mengembalikan nilai kolom tabel, melewati baris mana
pun dengan nilai NAN("." dalam file), dan indeks kolom, yang berisi
nilai-nilai ini.
\end{eulercomment}
\begin{eulerprompt}
>\{c,i\}=tablecol(MT,[5,6]);
\end{eulerprompt}
\begin{eulercomment}
Kita dapat menggunakan ini untuk mengekstrak kolom dari tabel untuk
tabel baru.
\end{eulercomment}
\begin{eulerprompt}
>j=[1,5,6]; writetable(MT[i,j],labc=hd[j],ctok=[2],tok=tok)
\end{eulerprompt}
\begin{euleroutput}
      Person Evaluation       Tip
           2          g       1.8
           3          g       1.8
           6          g       2.8
           7         vg       2.8
           9         vg       1.8
          11         vg       1.8
          12          g       1.8
          13         vg       1.8
          14          g       1.8
          15          g       0.8
          16          g       2.8
          20         vg       1.8
          21          m       2.8
          22         vg       1.8
          23          g       2.8
          24          m       1.8
\end{euleroutput}
\begin{eulercomment}
Tentu saja, kita perlu mengekstrak tabel itu sendiri dari daftar Table
dalam kasus ini.
\end{eulercomment}
\begin{eulerprompt}
>MT=Table[1];
\end{eulerprompt}
\begin{eulercomment}
Tentu saja, kita juga dapat menggunakannya untuk menentukan nilai
rata-rata kolom atau nilai statistik lainnya.
\end{eulercomment}
\begin{eulerprompt}
>mean(tablecol(MT,6))
\end{eulerprompt}
\begin{euleroutput}
  2.175
\end{euleroutput}
\begin{eulercomment}
Fungsi getstatistics() mengembalikan elemen dalam vektor dan
jumlahnya. Kita menerapkannya pada nilai "m" dan "f" di kolom kedua
tabel kita.
\end{eulercomment}
\begin{eulerprompt}
>\{xu,count\}=getstatistics(tablecol(MT,2)); xu, count,
\end{eulerprompt}
\begin{euleroutput}
  [1,  3]
  [12,  13]
\end{euleroutput}
\begin{eulercomment}
Kita dapat mencetak hasilnya di tabel baru.
\end{eulercomment}
\begin{eulerprompt}
>writetable(count',labr=tok[xu])
\end{eulerprompt}
\begin{euleroutput}
           m        12
           f        13
\end{euleroutput}
\begin{eulercomment}
Fungsi selecttable() mengembalikan tabel baru dengan nilai-nilai dalam
satu kolom yang dipilih dari vektor indeks. Pertama-tama kita mencari
indeks dari dua nilai kita di tabel token.
\end{eulercomment}
\begin{eulerprompt}
>v:=indexof(tok,["g","vg"])
\end{eulerprompt}
\begin{euleroutput}
  [5,  6]
\end{euleroutput}
\begin{eulercomment}
Sekarang kita dapat memilih baris tabel, yang memiliki salah satu
nilai dalam v di baris ke-5.
\end{eulercomment}
\begin{eulerprompt}
>MT1:=MT[selectrows(MT,5,v)]; i:=sortedrows(MT1,5);
\end{eulerprompt}
\begin{eulercomment}
Sekarang kita dapat mencetak tabel, dengan nilai yang diekstraksi dan
diurutkan di kolom ke-5.
\end{eulercomment}
\begin{eulerprompt}
>writetable(MT1[i],labc=hd,ctok=ctok,tok=tok,wc=7);
\end{eulerprompt}
\begin{euleroutput}
   Person    Sex    Age Titanic Evaluation    Tip Problem
        2      f     23       y          g    1.8       n
        3      f     26       y          g    1.8       y
        6      m     28       y          g    2.8       y
       18      m     38       y          g      .       n
       16      m     26       y          g    2.8       n
       15      f     31       y          g    0.8       n
       12      m     32       y          g    1.8       n
       23      f     38       y          g    2.8       n
       14      f     25       y          g    1.8       y
        9      f     24       y         vg    1.8       y
        7      f     31       y         vg    2.8       n
       20      f     28       y         vg    1.8       n
       22      f     28       y         vg    1.8       y
       13      m     29       y         vg    1.8       y
       11      f     23       y         vg    1.8       y
\end{euleroutput}
\begin{eulercomment}
Untuk statistik berikutnya, kita ingin menghubungkan dua kolom tabel.
Jadi, kita mengekstrak kolom 2 dan 4 dan mengurutkan tabel.
\end{eulercomment}
\begin{eulerprompt}
>i=sortedrows(MT,[2,4]);  ...
>  writetable(tablecol(MT[i],[2,4])',ctok=[1,2],tok=tok)
\end{eulerprompt}
\begin{euleroutput}
           m         n
           m         n
           m         n
           m         n
           m         n
           m         n
           m         n
           m         y
           m         y
           m         y
           m         y
           m         y
           f         n
           f         y
           f         y
           f         y
           f         y
           f         y
           f         y
           f         y
           f         y
           f         y
           f         y
           f         y
           f         y
\end{euleroutput}
\begin{eulercomment}
Dengan getstatistics(), kita juga dapat menghubungkan jumlah pada dua
kolom tabel satu sama lain.
\end{eulercomment}
\begin{eulerprompt}
>MT24=tablecol(MT,[2,4]); ...
>\{xu1,xu2,count\}=getstatistics(MT24[1],MT24[2]); ...
>writetable(count,labr=tok[xu1],labc=tok[xu2])
\end{eulerprompt}
\begin{euleroutput}
                     n         y
           m         7         5
           f         1        12
\end{euleroutput}
\begin{eulercomment}
Suatu tabel dapat ditulis ke dalam suatu berkas.
\end{eulercomment}
\begin{eulerprompt}
>filename="test.dat"; ...
>writetable(count,labr=tok[xu1],labc=tok[xu2],file=filename);
\end{eulerprompt}
\begin{eulercomment}
Lalu kita dapat membaca tabel dari berkas tersebut.
\end{eulercomment}
\begin{eulerprompt}
>\{MT2,hd,tok2,hdr\}=readtable(filename,>clabs,>rlabs); ...
>writetable(MT2,labr=hdr,labc=hd)
\end{eulerprompt}
\begin{euleroutput}
                     n         y
           m         7         5
           f         1        12
\end{euleroutput}
\begin{eulercomment}
Dan hapus berkasnya.
\end{eulercomment}
\begin{eulerprompt}
>fileremove(filename);
\end{eulerprompt}
\eulerheading{Distribusi}
\begin{eulercomment}
Dengan plot2d, ada metode yang sangat mudah untuk memplot distribusi
data eksperimen.
\end{eulercomment}
\begin{eulerprompt}
>p=normal(1,1000); //1000 random normal-distributed sample p
>plot2d(p,distribution=20,style="\(\backslash\)/"); // plot the random sample p
>plot2d("qnormal(x,0,1)",add=1): // add the standard normal distribution plot
\end{eulerprompt}
\begin{eulercomment}
Harap perhatikan perbedaan antara diagram batang (sampel) dan kurva
normal (distribusi riil). Masukkan kembali ketiga perintah tersebut
untuk melihat hasil sampel lainnya.
\end{eulercomment}
\begin{eulercomment}
Berikut ini adalah perbandingan 10 simulasi dari 1000 nilai yang
didistribusikan secara normal menggunakan apa yang disebut diagram
kotak. Diagram ini menunjukkan median, kuartil 25\% dan 75\%, nilai
minimal dan maksimal, dan outlier.
\end{eulercomment}
\begin{eulerprompt}
>p=normal(10,1000); boxplot(p):
\end{eulerprompt}
\begin{eulercomment}
Untuk menghasilkan bilangan bulat acak, Euler memiliki inrandom. Mari
kita simulasikan lemparan dadu dan plot distribusinya.

Kita menggunakan fungsi getmultiplicities(v,x), yang menghitung
seberapa sering elemen v muncul di x. Kemudian kita plot hasilnya
menggunakan columnsplot().
\end{eulercomment}
\begin{eulerprompt}
>k=intrandom(1,6000,6);  ...
>columnsplot(getmultiplicities(1:6,k));  ...
>ygrid(1000,color=red):
\end{eulerprompt}
\begin{eulercomment}
Sementara intrandom(n,m,k) mengembalikan bilangan bulat yang
terdistribusi seragam dari 1 hingga k, dimungkinkan untuk menggunakan
distribusi bilangan bulat lain yang diberikan dengan randpint().

Dalam contoh berikut, probabilitas untuk 1,2,3 masing-masing adalah
0,4,0,1,0,5.
\end{eulercomment}
\begin{eulerprompt}
>randpint(1,1000,[0.4,0.1,0.5]); getmultiplicities(1:3,%)
\end{eulerprompt}
\begin{euleroutput}
  [394,  107,  499]
\end{euleroutput}
\begin{eulercomment}
Euler dapat menghasilkan nilai acak dari lebih banyak distribusi.
Lihat referensinya.

Misalnya, kita coba distribusi eksponensial. Variabel acak kontinu X
dikatakan memiliki distribusi eksponensial, jika PDF-nya diberikan
oleh\\
\end{eulercomment}
\begin{eulerformula}
\[
f_X(x)=\lambda e^{-\lambda x},\quad x>0,\quad \lambda>0,
\]
\end{eulerformula}
\begin{eulercomment}
dengan parameter\\
\end{eulercomment}
\begin{eulerformula}
\[
\lambda=\frac{1}{\mu},\quad \mu \text{ adalah mean, dan dilambangkan dengan } X \sim \text{Eksponensial}(\lambda).
\]
\end{eulerformula}
\begin{eulerprompt}
>plot2d(randexponential(1,1000,2),>distribution):
\end{eulerprompt}
\begin{eulercomment}
Untuk banyak distribusi, Euler dapat menghitung fungsi distribusi dan
inversnya.
\end{eulercomment}
\begin{eulerprompt}
>plot2d("normaldis",-4,4): 
\end{eulerprompt}
\begin{eulercomment}
Berikut ini adalah salah satu cara untuk memplot kuantil.
\end{eulercomment}
\begin{eulerprompt}
>plot2d("qnormal(x,1,1.5)",-4,6);  ...
>plot2d("qnormal(x,1,1.5)",a=2,b=5,>add,>filled):
\end{eulerprompt}
\begin{eulercomment}
lateks: \textbackslash{}teks\{normaldis(x,m,d)\}=\textbackslash{}int\_\{-\textbackslash{}infty\}\textasciicircum{}x
\textbackslash{}frac\{1\}\{d\textbackslash{}sqrt\{2\textbackslash{}pi\}\}e\textasciicircum{}\{-\textbackslash{}frac\{1\}\{2\}(\textbackslash{}frac\{t-m\}\{d\})\textasciicircum{}2\}\textbackslash{} dt.\\
Peluang untuk berada di area hijau adalah sebagai berikut.
\end{eulercomment}
\begin{eulerprompt}
>normaldis(5,1,1.5)-normaldis(2,1,1.5)
\end{eulerprompt}
\begin{euleroutput}
  0.248662156979
\end{euleroutput}
\begin{eulercomment}
Hal ini dapat dihitung secara numerik dengan integral berikut.\\
\end{eulercomment}
\begin{eulerformula}
\[
\int_2^5 \frac{1}{1.5\sqrt{2\pi}}e^{-\frac{1}{2}(\frac{x-1}{1.5})^2}\ dx.
\]
\end{eulerformula}
\begin{eulerprompt}
>gauss("qnormal(x,1,1.5)",2,5)
\end{eulerprompt}
\begin{euleroutput}
  0.248662156979
\end{euleroutput}
\begin{eulercomment}
Mari kita bandingkan distribusi binomial dengan distribusi normal
dengan nilai rata-rata dan deviasi yang sama. Fungsi invbindis()
menyelesaikan interpolasi linier antara nilai integer.
\end{eulercomment}
\begin{eulerprompt}
>invbindis(0.95,1000,0.5), invnormaldis(0.95,500,0.5*sqrt(1000))
\end{eulerprompt}
\begin{euleroutput}
  525.516721219
  526.007419394
\end{euleroutput}
\begin{eulercomment}
Fungsi qdis() adalah kerapatan distribusi chi-kuadrat. Seperti biasa,
Euler memetakan vektor ke fungsi ini. Jadi, kita memperoleh plot semua
distribusi chi-kuadrat dengan derajat 5 hingga 30 dengan mudah dengan
cara berikut.
\end{eulercomment}
\begin{eulerprompt}
>plot2d("qchidis(x,(5:5:50)')",0,50):
\end{eulerprompt}
\begin{eulercomment}
Euler memiliki fungsi yang akurat untuk mengevaluasi distribusi. Mari
kita periksa chidis() dengan integral.

Penamaannya mencoba agar konsisten. Misalnya,

- distribusi chi-kuadrat adalah chidis(),\\
- fungsi inversnya adalah invchidis(),\\
- densitasnya adalah qchidis().

Komplemen distribusi (ekor atas) adalah chicdis().
\end{eulercomment}
\begin{eulerprompt}
>chidis(1.5,2), integrate("qchidis(x,2)",0,1.5)
\end{eulerprompt}
\begin{euleroutput}
  0.527633447259
  0.527633447259
\end{euleroutput}
\eulerheading{Distribusi Diskrit}
\begin{eulercomment}
Untuk menentukan distribusi diskrit Anda sendiri, Anda dapat
menggunakan metode berikut.

Pertama, kita tetapkan fungsi distribusi.
\end{eulercomment}
\begin{eulerprompt}
>wd = 0|((1:6)+[-0.01,0.01,0,0,0,0])/6
\end{eulerprompt}
\begin{euleroutput}
  [0,  0.165,  0.335,  0.5,  0.666667,  0.833333,  1]
\end{euleroutput}
\begin{eulercomment}
Artinya adalah bahwa dengan probabilitas wd[i+1]-wd[i] kita
menghasilkan nilai acak i.

Ini hampir merupakan distribusi seragam. Mari kita definisikan
generator angka acak untuk ini. Fungsi find(v,x) menemukan nilai x
dalam vektor v. Fungsi ini juga berfungsi untuk vektor x.
\end{eulercomment}
\begin{eulerprompt}
>function wrongdice (n,m) := find(wd,random(n,m))
\end{eulerprompt}
\begin{eulercomment}
Kesalahannya begitu halus sehingga kita hanya melihatnya pada
pengulangan yang sangat banyak.
\end{eulercomment}
\begin{eulerprompt}
>columnsplot(getmultiplicities(1:6,wrongdice(1,1000000))):
\end{eulerprompt}
\begin{eulercomment}
Berikut ini adalah fungsi sederhana untuk memeriksa distribusi seragam
nilai 1...K dalam v. Kita terima hasilnya, jika untuk semua frekuensi

\end{eulercomment}
\begin{eulerformula}
\[
\left|f_i-\frac{1}{K}\right| < \frac{\delta}{\sqrt{n}}.
\]
\end{eulerformula}
\begin{eulerprompt}
>function checkrandom (v, delta=1) ...
\end{eulerprompt}
\begin{eulerudf}
    K=max(v); n=cols(v);
    fr=getfrequencies(v,1:K);
    return max(fr/n-1/K)<delta/sqrt(n);
    endfunction
\end{eulerudf}
\begin{eulercomment}
Memang fungsi tersebut menolak distribusi seragam.
\end{eulercomment}
\begin{eulerprompt}
>checkrandom(wrongdice(1,1000000))
\end{eulerprompt}
\begin{euleroutput}
  0
\end{euleroutput}
\begin{eulercomment}
Dan menerima generator acak bawaan.
\end{eulercomment}
\begin{eulerprompt}
>checkrandom(intrandom(1,1000000,6))
\end{eulerprompt}
\begin{euleroutput}
  1
\end{euleroutput}
\begin{eulercomment}
Kita dapat menghitung distribusi binomial. Pertama ada binomialsum(),
yang mengembalikan probabilitas i atau kurang dari n kali percobaan.
\end{eulercomment}
\begin{eulerprompt}
>bindis(410,1000,0.4)
\end{eulerprompt}
\begin{euleroutput}
  0.751401349654
\end{euleroutput}
\begin{eulercomment}
Fungsi Beta terbalik digunakan untuk menghitung interval kepercayaan
Clopper-Pearson untuk parameter p. Level default adalah alpha.

Arti dari interval ini adalah jika p berada di luar interval, hasil
yang diamati sebesar 410 dalam 1000 adalah langka.
\end{eulercomment}
\begin{eulerprompt}
>clopperpearson(410,1000)
\end{eulerprompt}
\begin{euleroutput}
  [0.37932,  0.441212]
\end{euleroutput}
\begin{eulercomment}
Perintah berikut adalah cara langsung untuk mendapatkan hasil di atas.
Namun untuk n yang besar, penjumlahan langsung tidak akurat dan
lambat.
\end{eulercomment}
\begin{eulerprompt}
>p=0.4; i=0:410; n=1000; sum(bin(n,i)*p^i*(1-p)^(n-i))
\end{eulerprompt}
\begin{euleroutput}
  0.751401349655
\end{euleroutput}
\begin{eulercomment}
Omong-omong, invbinsum() menghitung kebalikan dari binomialsum().
\end{eulercomment}
\begin{eulerprompt}
>invbindis(0.75,1000,0.4)
\end{eulerprompt}
\begin{euleroutput}
  409.932733047
\end{euleroutput}
\begin{eulercomment}
Dalam Bridge, kita mengasumsikan 5 kartu yang beredar (dari 52) dalam
dua tangan (26 kartu). Mari kita hitung probabilitas distribusi yang
lebih buruk dari 3:2 (misalnya 0:5, 1:4, 4:1 atau 5:0).
\end{eulercomment}
\begin{eulerprompt}
>2*hypergeomsum(1,5,13,26)
\end{eulerprompt}
\begin{euleroutput}
  0.321739130435
\end{euleroutput}
\begin{eulercomment}
Ada juga simulasi distribusi multinomial.
\end{eulercomment}
\begin{eulerprompt}
>randmultinomial(10,1000,[0.4,0.1,0.5])
\end{eulerprompt}
\begin{euleroutput}
            394            82           524 
            418            92           490 
            445            90           465 
            403           114           483 
            405            95           500 
            384           107           509 
            414            93           493 
            419            90           491 
            394           101           505 
            382           103           515 
\end{euleroutput}
\eulerheading{Merencanakan Data}
\begin{eulercomment}
Untuk merencanakan data, kami mencoba hasil pemilu Jerman sejak 1990,
yang diukur dalam jumlah kursi.
\end{eulercomment}
\begin{eulerprompt}
>BW := [ ...
>1990,662,319,239,79,8,17; ...
>1994,672,294,252,47,49,30; ...
>1998,669,245,298,43,47,36; ...
>2002,603,248,251,47,55,2; ...
>2005,614,226,222,61,51,54; ...
>2009,622,239,146,93,68,76; ...
>2013,631,311,193,0,63,64];
\end{eulerprompt}
\begin{eulercomment}
Untuk para pihak, kami menggunakan serangkaian nama.
\end{eulercomment}
\begin{eulerprompt}
>P:=["CDU/CSU","SPD","FDP","Gr","Li"];
\end{eulerprompt}
\begin{eulercomment}
Mari kita cetak persentasenya dengan baik.

Pertama-tama kita ekstrak kolom-kolom yang diperlukan. Kolom 3 hingga
7 adalah kursi masing-masing partai, dan kolom 2 adalah jumlah total
kursi. Kolom 3 adalah tahun pemilihan.
\end{eulercomment}
\begin{eulerprompt}
>BT:=BW[,3:7]; BT:=BT/sum(BT); YT:=BW[,1]';
\end{eulerprompt}
\begin{eulercomment}
Kemudian kami mencetak statistik dalam bentuk tabel. Kami menggunakan
nama sebagai tajuk kolom, dan tahun sebagai tajuk untuk baris. Lebar
default untuk kolom adalah wc=10, tetapi kami lebih suka keluaran yang
lebih padat. Kolom akan diperluas untuk label kolom, jika perlu.
\end{eulercomment}
\begin{eulerprompt}
>writetable(BT*100,wc=6,dc=0,>fixed,labc=P,labr=YT)
\end{eulerprompt}
\begin{euleroutput}
         CDU/CSU   SPD   FDP    Gr    Li
    1990      48    36    12     1     3
    1994      44    38     7     7     4
    1998      37    45     6     7     5
    2002      41    42     8     9     0
    2005      37    36    10     8     9
    2009      38    23    15    11    12
    2013      49    31     0    10    10
\end{euleroutput}
\begin{eulercomment}
Perkalian matriks berikut ini mengekstrak jumlah persentase dari dua
partai besar yang menunjukkan bahwa partai-partai kecil telah
memperoleh dukungan di parlemen hingga tahun 2009.
\end{eulercomment}
\begin{eulerprompt}
>BT1:=(BT.[1;1;0;0;0])'*100
\end{eulerprompt}
\begin{euleroutput}
  [84.29,  81.25,  81.1659,  82.7529,  72.9642,  61.8971,  79.8732]
\end{euleroutput}
\begin{eulercomment}
Ada juga plot statistik sederhana. Kita menggunakannya untuk
menampilkan garis dan titik secara bersamaan. Alternatifnya adalah
memanggil plot2d dua kali dengan \textgreater{}add.
\end{eulercomment}
\begin{eulerprompt}
>statplot(YT,BT1,"b"):
\end{eulerprompt}
\begin{eulercomment}
Tentukan beberapa warna untuk setiap pihak.
\end{eulercomment}
\begin{eulerprompt}
>CP:=[rgb(0.5,0.5,0.5),red,yellow,green,rgb(0.8,0,0)];
\end{eulerprompt}
\begin{eulercomment}
Sekarang kita dapat memetakan hasil pemilu 2009 dan perubahannya ke
dalam satu plot menggunakan gambar. Kita dapat menambahkan vektor
kolom ke setiap plot.
\end{eulercomment}
\begin{eulerprompt}
>figure(2,1);  ...
>figure(1); columnsplot(BW[6,3:7],P,color=CP); ...
>figure(2); columnsplot(BW[6,3:7]-BW[5,3:7],P,color=CP);  ...
>figure(0):
\end{eulerprompt}
\begin{eulercomment}
Plot data menggabungkan baris-baris data statistik dalam satu plot.
\end{eulercomment}
\begin{eulerprompt}
>J:=BW[,1]'; DP:=BW[,3:7]'; ...
>dataplot(YT,BT',color=CP);  ...
>labelbox(P,colors=CP,styles="[]",>points,w=0.2,x=0.3,y=0.4):
\end{eulerprompt}
\begin{eulercomment}
Plot kolom 3D menunjukkan baris data statistik dalam bentuk kolom.
Kami memberikan label untuk baris dan kolom. Angle adalah sudut
pandang.
\end{eulercomment}
\begin{eulerprompt}
>columnsplot3d(BT,scols=P,srows=YT, ...
>  angle=30°,ccols=CP):
\end{eulerprompt}
\begin{eulercomment}
Representasi lainnya adalah plot mosaik. Perhatikan bahwa kolom-kolom
plot mewakili kolom-kolom matriks di sini. Karena panjang label
CDU/CSU, kami mengambil jendela yang lebih kecil dari biasanya.
\end{eulercomment}
\begin{eulerprompt}
>shrinkwindow(>smaller);  ...
>mosaicplot(BT',srows=YT,scols=P,color=CP,style="#"); ...
>shrinkwindow():
\end{eulerprompt}
\begin{eulercomment}
Kita juga bisa membuat diagram lingkaran. Karena hitam dan kuning
membentuk koalisi, kita susun ulang unsur-unsurnya.
\end{eulercomment}
\begin{eulerprompt}
>i=[1,3,5,4,2]; piechart(BW[6,3:7][i],color=CP[i],lab=P[i]):
\end{eulerprompt}
\begin{eulercomment}
Berikut adalah jenis plot yang lain.
\end{eulercomment}
\begin{eulerprompt}
>starplot(normal(1,10)+4,lab=1:10,>rays):
\end{eulerprompt}
\begin{eulercomment}
Beberapa plot dalam plot2d bagus untuk statika. Berikut adalah plot
impuls data acak, yang didistribusikan secara seragam dalam [0,1].
\end{eulercomment}
\begin{eulerprompt}
>plot2d(makeimpulse(1:10,random(1,10)),>bar):
\end{eulerprompt}
\begin{eulercomment}
Namun untuk data yang terdistribusi secara eksponensial, kita mungkin
memerlukan plot logaritmik.
\end{eulercomment}
\begin{eulerprompt}
>logimpulseplot(1:10,-log(random(1,10))*10):
\end{eulerprompt}
\begin{eulercomment}
Fungsi columnsplot() lebih mudah digunakan, karena hanya memerlukan
vektor nilai. Selain itu, fungsi ini dapat mengatur labelnya sesuai
keinginan kita, kami telah menunjukkannya dalam tutorial ini.

Berikut adalah aplikasi lain, tempat kita menghitung karakter dalam
kalimat dan memplot statistik.
\end{eulercomment}
\begin{eulerprompt}
>v=strtochar("the quick brown fox jumps over the lazy dog"); ...
>w=ascii("a"):ascii("z"); x=getmultiplicities(w,v); ...
>cw=[]; for k=w; cw=cw|char(k); end; ...
>columnsplot(x,lab=cw,width=0.05):
\end{eulerprompt}
\begin{eulercomment}
Anda juga dapat mengatur sumbu secara manual.
\end{eulercomment}
\begin{eulerprompt}
>n=10; p=0.4; i=0:n; x=bin(n,i)*p^i*(1-p)^(n-i); ...
>columnsplot(x,lab=i,width=0.05,<frame,<grid); ...
>yaxis(0,0:0.1:1,style="->",>left); xaxis(0,style="."); ...
>label("p",0,0.25), label("i",11,0); ...
>textbox(["Binomial distribution","with p=0.4"]):
\end{eulerprompt}
\begin{eulercomment}
Berikut ini adalah cara untuk memetakan frekuensi angka dalam sebuah
vektor.

Kita buat sebuah vektor bilangan acak integer 1 hingga 6.
\end{eulercomment}
\begin{eulerprompt}
>v:=intrandom(1,10,10)
\end{eulerprompt}
\begin{euleroutput}
  [5,  1,  8,  6,  6,  10,  9,  6,  8,  3]
\end{euleroutput}
\begin{eulercomment}
Lalu ekstrak angka-angka unik dalam v.
\end{eulercomment}
\begin{eulerprompt}
>vu:=unique(v)
\end{eulerprompt}
\begin{euleroutput}
  [1,  3,  5,  6,  8,  9,  10]
\end{euleroutput}
\begin{eulercomment}
Dan plot frekuensi pada kolom plot.
\end{eulercomment}
\begin{eulerprompt}
>columnsplot(getmultiplicities(vu,v),lab=vu,style="/"):
\end{eulerprompt}
\begin{eulercomment}
Kami ingin menunjukkan fungsi untuk distribusi nilai empiris.
\end{eulercomment}
\begin{eulerprompt}
>x=normal(1,20);
\end{eulerprompt}
\begin{eulercomment}
Fungsi empdist(x,vs) memerlukan array nilai yang diurutkan. Jadi, kita
harus mengurutkan x sebelum dapat menggunakannya.
\end{eulercomment}
\begin{eulerprompt}
>xs=sort(x);
\end{eulerprompt}
\begin{eulercomment}
Kemudian kami memetakan distribusi empiris dan beberapa batang
kepadatan ke dalam satu petak. Alih-alih menggunakan petak batang
untuk distribusi, kali ini kami menggunakan petak gigi gergaji.
\end{eulercomment}
\begin{eulerprompt}
>figure(2,1); ...
>figure(1); plot2d("empdist",-4,4;xs); ...
>figure(2); plot2d(histo(x,v=-4:0.2:4,<bar));  ...
>figure(0):
\end{eulerprompt}
\begin{eulercomment}
Plot sebaran mudah dibuat di Euler dengan plot titik biasa. Grafik
berikut menunjukkan bahwa X dan X+Y jelas berkorelasi positif.
\end{eulercomment}
\begin{eulerprompt}
>x=normal(1,100); plot2d(x,x+rotright(x),>points,style=".."):
\end{eulerprompt}
\begin{eulercomment}
Sering kali, kita ingin membandingkan dua sampel dengan distribusi
yang berbeda. Hal ini dapat dilakukan dengan plot kuantil-kuantil.

Untuk pengujian, kita mencoba distribusi t-student dan distribusi
eksponensial.
\end{eulercomment}
\begin{eulerprompt}
>x=randt(1,1000,5); y=randnormal(1,1000,mean(x),dev(x)); ...
>plot2d("x",r=6,style="--",yl="normal",xl="student-t",>vertical); ...
>plot2d(sort(x),sort(y),>points,color=red,style="x",>add):
\end{eulerprompt}
\begin{eulercomment}
Plot tersebut dengan jelas menunjukkan bahwa nilai-nilai yang
terdistribusi normal cenderung lebih kecil di ujung-ujung ekstrem.

Jika kita memiliki dua distribusi dengan ukuran yang berbeda, kita
dapat memperluas yang lebih kecil atau mengecilkan yang lebih besar.
Fungsi berikut ini bagus untuk keduanya. Fungsi ini mengambil nilai
median dengan persentase antara 0 dan 1.
\end{eulercomment}
\begin{eulerprompt}
>function medianexpand (x,n) := median(x,p=linspace(0,1,n-1));
\end{eulerprompt}
\begin{eulercomment}
Mari kita bandingkan dua distribusi yang sama.
\end{eulercomment}
\begin{eulerprompt}
>x=random(1000); y=random(400); ...
>plot2d("x",0,1,style="--"); ...
>plot2d(sort(medianexpand(x,400)),sort(y),>points,color=red,style="x",>add):
\end{eulerprompt}
\eulerheading{Regresi dan Korelasi}
\begin{eulercomment}
Regresi linier dapat dilakukan dengan fungsi polyfit() atau berbagai
fungsi fit.

Sebagai permulaan, kita mencari garis regresi untuk data univariat
dengan polyfit(x,y,1).
\end{eulercomment}
\begin{eulerprompt}
>x=1:10; y=[2,3,1,5,6,3,7,8,9,8]; writetable(x'|y',labc=["x","y"])
\end{eulerprompt}
\begin{euleroutput}
           x         y
           1         2
           2         3
           3         1
           4         5
           5         6
           6         3
           7         7
           8         8
           9         9
          10         8
\end{euleroutput}
\begin{eulercomment}
Kami ingin membandingkan kecocokan yang tidak tertimbang dan
tertimbang. Pertama, koefisien kecocokan linier.
\end{eulercomment}
\begin{eulerprompt}
>p=polyfit(x,y,1)
\end{eulerprompt}
\begin{euleroutput}
  [0.733333,  0.812121]
\end{euleroutput}
\begin{eulercomment}
Sekarang koefisien dengan bobot yang menekankan nilai terakhir.
\end{eulercomment}
\begin{eulerprompt}
>w &= "exp(-(x-10)^2/10)"; pw=polyfit(x,y,1,w=w(x))
\end{eulerprompt}
\begin{euleroutput}
  [4.71566,  0.38319]
\end{euleroutput}
\begin{eulercomment}
Kami memasukkan semuanya ke dalam satu plot untuk titik dan garis
regresi, dan untuk bobot yang digunakan.
\end{eulercomment}
\begin{eulerprompt}
>figure(2,1);  ...
>figure(1); statplot(x,y,"b",xl="Regression"); ...
>  plot2d("evalpoly(x,p)",>add,color=blue,style="--"); ...
>  plot2d("evalpoly(x,pw)",5,10,>add,color=red,style="--"); ...
>figure(2); plot2d(w,1,10,>filled,style="/",fillcolor=red,xl=w); ...
>figure(0):
\end{eulerprompt}
\begin{eulercomment}
Untuk contoh lain, kami membaca survei siswa, usia mereka, usia orang
tua mereka, dan jumlah saudara kandung dari sebuah berkas.

Tabel ini berisi "m" dan "f" di kolom kedua. Kami menggunakan variabel
tok2 untuk mengatur terjemahan yang tepat alih-alih membiarkan
readtable() mengumpulkan terjemahan.
\end{eulercomment}
\begin{eulerprompt}
>\{MS,hd\}:=readtable("table1.dat",tok2:=["m","f"]);  ...
>writetable(MS,labc=hd,tok2:=["m","f"]);
\end{eulerprompt}
\begin{euleroutput}
      Person       Sex       Age    Mother    Father  Siblings
           1         m        29        58        61         1
           2         f        26        53        54         2
           3         m        24        49        55         1
           4         f        25        56        63         3
           5         f        25        49        53         0
           6         f        23        55        55         2
           7         m        23        48        54         2
           8         m        27        56        58         1
           9         m        25        57        59         1
          10         m        24        50        54         1
          11         f        26        61        65         1
          12         m        24        50        52         1
          13         m        29        54        56         1
          14         m        28        48        51         2
          15         f        23        52        52         1
          16         m        24        45        57         1
          17         f        24        59        63         0
          18         f        23        52        55         1
          19         m        24        54        61         2
          20         f        23        54        55         1
\end{euleroutput}
\begin{eulercomment}
Bagaimana usia saling bergantung? Kesan pertama datang dari diagram
sebaran berpasangan.
\end{eulercomment}
\begin{eulerprompt}
>scatterplots(tablecol(MS,3:5),hd[3:5]):
\end{eulerprompt}
\begin{eulercomment}
Jelas bahwa usia ayah dan ibu saling bergantung. Mari kita tentukan
dan gambarkan garis regresinya.
\end{eulercomment}
\begin{eulerprompt}
>cs:=MS[,4:5]'; ps:=polyfit(cs[1],cs[2],1)
\end{eulerprompt}
\begin{euleroutput}
  [17.3789,  0.740964]
\end{euleroutput}
\begin{eulercomment}
Ini jelas model yang salah. Garis regresi adalah s=17+0,74t, di mana t
adalah usia ibu dan s adalah usia ayah. Perbedaan usia mungkin sedikit
bergantung pada usia, tetapi tidak terlalu banyak.

Sebaliknya, kami menduga fungsi seperti s=a+t. Maka a adalah rata-rata
s-t. Itu adalah perbedaan usia rata-rata antara ayah dan ibu.
\end{eulercomment}
\begin{eulerprompt}
>da:=mean(cs[2]-cs[1])
\end{eulerprompt}
\begin{euleroutput}
  3.65
\end{euleroutput}
\begin{eulercomment}
Mari kita gambarkan ini menjadi satu diagram sebar.
\end{eulercomment}
\begin{eulerprompt}
>plot2d(cs[1],cs[2],>points);  ...
>plot2d("evalpoly(x,ps)",color=red,style=".",>add);  ...
>plot2d("x+da",color=blue,>add):
\end{eulerprompt}
\begin{eulercomment}
Berikut ini adalah diagram kotak dari dua zaman tersebut. Ini hanya
menunjukkan bahwa zamannya berbeda.
\end{eulercomment}
\begin{eulerprompt}
>boxplot(cs,["mothers","fathers"]):
\end{eulerprompt}
\begin{eulercomment}
Menariknya bahwa perbedaan median tidak sebesar perbedaan mean.
\end{eulercomment}
\begin{eulerprompt}
>median(cs[2])-median(cs[1])
\end{eulerprompt}
\begin{euleroutput}
  1.5
\end{euleroutput}
\begin{eulercomment}
Koefisien korelasi menunjukkan korelasi positif.
\end{eulercomment}
\begin{eulerprompt}
>correl(cs[1],cs[2])
\end{eulerprompt}
\begin{euleroutput}
  0.7588307236
\end{euleroutput}
\begin{eulercomment}
Korelasi peringkat adalah ukuran untuk urutan yang sama di kedua
vektor. Korelasi ini juga cukup positif.
\end{eulercomment}
\begin{eulerprompt}
>rankcorrel(cs[1],cs[2])
\end{eulerprompt}
\begin{euleroutput}
  0.758925292358
\end{euleroutput}
\eulerheading{Membuat Fungsi baru}
\begin{eulercomment}
Tentu saja, bahasa EMT dapat digunakan untuk memprogram fungsi baru.
Misalnya, kita mendefinisikan fungsi kemiringan.

\end{eulercomment}
\begin{eulerformula}
\[
\text{sk}(x) = \dfrac{\sqrt{n} \sum_i (x_i-m)^3}{\left(\sum_i (x_i-m)^2\right)^{3/2}}
\]
\end{eulerformula}
\begin{eulercomment}
di mana m adalah rata-rata x.
\end{eulercomment}
\begin{eulerprompt}
>function skew (x:vector) ...
\end{eulerprompt}
\begin{eulerudf}
  m=mean(x);
  return sqrt(cols(x))*sum((x-m)^3)/(sum((x-m)^2))^(3/2);
  endfunction
\end{eulerudf}
\begin{eulercomment}
Seperti yang Anda lihat, kita dapat dengan mudah menggunakan bahasa
matriks untuk mendapatkan implementasi yang sangat singkat dan
efisien. Mari kita coba fungsi ini.
\end{eulercomment}
\begin{eulerprompt}
>data=normal(20); skew(normal(10))
\end{eulerprompt}
\begin{euleroutput}
  0.521806329961
\end{euleroutput}
\begin{eulercomment}
Berikut adalah fungsi lainnya, yang disebut koefisien kemiringan
Pearson.
\end{eulercomment}
\begin{eulerprompt}
>function skew1 (x) := 3*(mean(x)-median(x))/dev(x)
>skew1(data)
\end{eulerprompt}
\begin{euleroutput}
  0.573102925949
\end{euleroutput}
\eulerheading{Simulasi Monte Carlo}
\begin{eulercomment}
Euler dapat digunakan untuk mensimulasikan kejadian acak. Kita telah
melihat contoh sederhana di atas. Berikut ini contoh lain, yang
mensimulasikan 1000 kali lemparan 3 dadu, dan menanyakan distribusi
jumlahnya.
\end{eulercomment}
\begin{eulerprompt}
>ds:=sum(intrandom(1000,3,6))';  fs=getmultiplicities(3:18,ds)
\end{eulerprompt}
\begin{euleroutput}
  [5,  12,  25,  49,  73,  94,  128,  141,  124,  116,  105,  62,  37,
  19,  7,  3]
\end{euleroutput}
\begin{eulercomment}
Kita bisa merencanakannya sekarang.
\end{eulercomment}
\begin{eulerprompt}
>columnsplot(fs,lab=3:18):
\end{eulerprompt}
\begin{eulercomment}
Menentukan distribusi yang diharapkan tidaklah mudah. ??Kami
menggunakan rekursi tingkat lanjut untuk ini.

Fungsi berikut menghitung jumlah cara bilangan k dapat
direpresentasikan sebagai jumlah n bilangan dalam rentang 1 hingga m.
Fungsi ini bekerja secara rekursif dengan cara yang jelas.
\end{eulercomment}
\begin{eulerprompt}
>function map countways (k; n, m) ...
\end{eulerprompt}
\begin{eulerudf}
    if n==1 then return k>=1 && k<=m
    else
      sum=0; 
      loop 1 to m; sum=sum+countways(k-#,n-1,m); end;
      return sum;
    end;
  endfunction
\end{eulerudf}
\begin{eulercomment}
Berikut ini hasil dari tiga kali lemparan dadu.
\end{eulercomment}
\begin{eulerprompt}
>countways(5:25,5,5)
\end{eulerprompt}
\begin{euleroutput}
  [1,  5,  15,  35,  70,  121,  185,  255,  320,  365,  381,  365,  320,
  255,  185,  121,  70,  35,  15,  5,  1]
\end{euleroutput}
\begin{eulerprompt}
>cw=countways(3:18,3,6)
\end{eulerprompt}
\begin{euleroutput}
  [1,  3,  6,  10,  15,  21,  25,  27,  27,  25,  21,  15,  10,  6,  3,
  1]
\end{euleroutput}
\begin{eulercomment}
Kami menambahkan nilai yang diharapkan ke plot.
\end{eulercomment}
\begin{eulerprompt}
>plot2d(cw/6^3*1000,>add); plot2d(cw/6^3*1000,>points,>add):
\end{eulerprompt}
\begin{eulercomment}
Untuk simulasi lain, deviasi nilai rata-rata dari n variabel acak
berdistribusi normal 0-1 adalah 1/akar(n).
\end{eulercomment}
\begin{eulerprompt}
>longformat; 1/sqrt(10)
\end{eulerprompt}
\begin{euleroutput}
  0.316227766017
\end{euleroutput}
\begin{eulercomment}
Mari kita periksa ini dengan simulasi. Kita hasilkan 10000 kali 10
vektor acak.
\end{eulercomment}
\begin{eulerprompt}
>M=normal(10000,10); dev(mean(M)')
\end{eulerprompt}
\begin{euleroutput}
  0.318861419326
\end{euleroutput}
\begin{eulerprompt}
>plot2d(mean(M)',>distribution):
\end{eulerprompt}
\begin{eulercomment}
Median dari 10 bilangan acak berdistribusi normal 0-1 memiliki deviasi
yang lebih besar.
\end{eulercomment}
\begin{eulerprompt}
>dev(median(M)')
\end{eulerprompt}
\begin{euleroutput}
  0.376651504162
\end{euleroutput}
\begin{eulercomment}
Karena kita dapat dengan mudah menghasilkan lintasan acak, kita dapat
mensimulasikan proses Wiener. Kita mengambil 1000 langkah dari 1000
proses. Kemudian kita memetakan deviasi standar dan rata-rata langkah
ke-n dari proses ini bersama dengan nilai yang diharapkan dalam warna
merah.
\end{eulercomment}
\begin{eulerprompt}
>n=1000; m=1000; M=cumsum(normal(n,m)/sqrt(m)); ...
>t=(1:n)/n; figure(2,1); ...
>figure(1); plot2d(t,mean(M')'); plot2d(t,0,color=red,>add); ...
>figure(2); plot2d(t,dev(M')'); plot2d(t,sqrt(t),color=red,>add); ...
>figure(0):
\end{eulerprompt}
\eulerheading{Pengujian}
\begin{eulercomment}
Pengujian merupakan alat penting dalam statistik. Dalam Euler, banyak
pengujian yang diterapkan. Semua pengujian ini menghasilkan galat yang
kita terima jika kita menolak hipotesis nol.

Sebagai contoh, kita menguji lemparan dadu untuk distribusi seragam.
Pada 600 lemparan, kita memperoleh nilai berikut, yang kita masukkan
ke dalam uji chi-kuadrat.
\end{eulercomment}
\begin{eulerprompt}
>chitest([90,103,114,101,103,89],dup(100,6)')
\end{eulerprompt}
\begin{euleroutput}
  0.498830517952
\end{euleroutput}
\begin{eulercomment}
Uji chi-square juga memiliki modus, yang menggunakan simulasi Monte
Carlo untuk menguji statistik. Hasilnya harus hampir sama. Parameter
\textgreater{}p menginterpretasikan vektor y sebagai vektor probabilitas.
\end{eulercomment}
\begin{eulerprompt}
>chitest([90,103,114,101,103,89],dup(1/6,6)',>p,>montecarlo)
\end{eulerprompt}
\begin{euleroutput}
  0.517
\end{euleroutput}
\begin{eulercomment}
Kesalahan ini terlalu besar. Jadi kita tidak dapat menolak distribusi
seragam. Ini tidak membuktikan bahwa dadu kita adil. Namun, kita tidak
dapat menolak hipotesis kita.

Selanjutnya, kita menghasilkan 1000 lemparan dadu menggunakan
generator angka acak, dan melakukan pengujian yang sama.
\end{eulercomment}
\begin{eulerprompt}
>n=1000; t=random([1,n*6]); chitest(count(t*6,6),dup(n,6)')
\end{eulerprompt}
\begin{euleroutput}
  0.159910723681
\end{euleroutput}
\begin{eulercomment}
Mari kita uji nilai rata-rata 100 dengan uji-t.
\end{eulercomment}
\begin{eulerprompt}
>s=200+normal([1,100])*10; ...
>ttest(mean(s),dev(s),100,200)
\end{eulerprompt}
\begin{euleroutput}
  0.353414299977
\end{euleroutput}
\begin{eulercomment}
Fungsi ttest() memerlukan nilai rata-rata, deviasi, jumlah data, dan
nilai rata-rata yang akan diuji.

Sekarang mari kita periksa dua pengukuran untuk nilai rata-rata yang
sama. Kita tolak hipotesis bahwa keduanya memiliki nilai rata-rata
yang sama, jika hasilnya \textless{}0,05.
\end{eulercomment}
\begin{eulerprompt}
>tcomparedata(normal(1,10),normal(1,10))
\end{eulerprompt}
\begin{euleroutput}
  0.268166832384
\end{euleroutput}
\begin{eulercomment}
Jika kita menambahkan bias pada satu distribusi, kita akan mendapatkan
lebih banyak penolakan. Ulangi simulasi ini beberapa kali untuk
melihat efeknya.
\end{eulercomment}
\begin{eulerprompt}
>tcomparedata(normal(1,10),normal(1,10)+2)
\end{eulerprompt}
\begin{euleroutput}
  2.17064628392e-05
\end{euleroutput}
\begin{eulercomment}
Pada contoh berikutnya, kita buat 20 lemparan dadu acak sebanyak 100
kali dan hitung angka-angka yang ada di dalamnya. Rata-rata harus ada
20/6=3,3 angka.
\end{eulercomment}
\begin{eulerprompt}
>R=random(100,20); R=sum(R*6<=1)'; mean(R)
\end{eulerprompt}
\begin{euleroutput}
  3.05
\end{euleroutput}
\begin{eulercomment}
Sekarang kita bandingkan jumlah angka satu dengan distribusi binomial.
Pertama kita gambarkan distribusi angka satu.
\end{eulercomment}
\begin{eulerprompt}
>plot2d(R,distribution=max(R)+1,even=1,style="\(\backslash\)/"):
>t=count(R,21);
\end{eulerprompt}
\begin{eulercomment}
Lalu kami hitung nilai yang diharapkan.
\end{eulercomment}
\begin{eulerprompt}
>n=0:20; b=bin(20,n)*(1/6)^n*(5/6)^(20-n)*100;
\end{eulerprompt}
\begin{eulercomment}
Kita harus mengumpulkan beberapa angka untuk mendapatkan kategori yang
cukup besar.
\end{eulercomment}
\begin{eulerprompt}
>t1=sum(t[1:2])|t[3:7]|sum(t[8:21]); ...
>b1=sum(b[1:2])|b[3:7]|sum(b[8:21]);
\end{eulerprompt}
\begin{eulercomment}
Uji chi-kuadrat menolak hipotesis bahwa distribusi kami adalah
distribusi binomial, jika hasilnya \textless{} 0,05.
\end{eulercomment}
\begin{eulerprompt}
>chitest(t1,b1)
\end{eulerprompt}
\begin{euleroutput}
  0.262911182138
\end{euleroutput}
\begin{eulercomment}
Contoh berikut berisi hasil dari dua kelompok orang (misalnya pria dan
wanita) yang memilih satu dari enam partai.
\end{eulercomment}
\begin{eulerprompt}
>A=[23,37,43,52,64,74;27,39,41,49,63,76];  ...
>  writetable(A,wc=6,labr=["m","f"],labc=1:6)
\end{eulerprompt}
\begin{euleroutput}
             1     2     3     4     5     6
       m    23    37    43    52    64    74
       f    27    39    41    49    63    76
\end{euleroutput}
\begin{eulercomment}
Kami ingin menguji independensi suara dari jenis kelamin. Uji tabel
chi\textasciicircum{}2 melakukan hal ini. Hasilnya terlalu besar untuk menolak
independensi. Jadi, kami tidak dapat mengatakan, apakah pemungutan
suara bergantung pada jenis kelamin dari data ini.
\end{eulercomment}
\begin{eulerprompt}
>tabletest(A)
\end{eulerprompt}
\begin{euleroutput}
  0.990701632326
\end{euleroutput}
\begin{eulercomment}
Berikut ini adalah tabel yang diharapkan, jika kita mengasumsikan
frekuensi pemungutan suara yang diamati.
\end{eulercomment}
\begin{eulerprompt}
>writetable(expectedtable(A),wc=6,dc=1,labr=["m","f"],labc=1:6)
\end{eulerprompt}
\begin{euleroutput}
             1     2     3     4     5     6
       m  24.9  37.9  41.9  50.3  63.3  74.7
       f  25.1  38.1  42.1  50.7  63.7  75.3
\end{euleroutput}
\begin{eulercomment}
Kita dapat menghitung koefisien kontingensi yang dikoreksi. Karena
sangat mendekati 0, kita simpulkan bahwa pemungutan suara tidak
bergantung pada jenis kelamin.
\end{eulercomment}
\begin{eulerprompt}
>contingency(A)
\end{eulerprompt}
\begin{euleroutput}
  0.0427225484717
\end{euleroutput}
\eulerheading{Beberapa Pengujian Lainnya}
\begin{eulercomment}
Selanjutnya, kami menggunakan analisis varians (uji F) untuk menguji
tiga sampel data berdistribusi normal untuk nilai rata-rata yang sama.
Metode ini disebut ANOVA (analisis varians). Dalam Euler, fungsi
varanalysis() digunakan.
\end{eulercomment}
\begin{eulerprompt}
>x1=[109,111,98,119,91,118,109,99,115,109,94]; mean(x1),
\end{eulerprompt}
\begin{euleroutput}
  106.545454545
\end{euleroutput}
\begin{eulerprompt}
>x2=[120,124,115,139,114,110,113,120,117]; mean(x2),
\end{eulerprompt}
\begin{euleroutput}
  119.111111111
\end{euleroutput}
\begin{eulerprompt}
>x3=[120,112,115,110,105,134,105,130,121,111]; mean(x3)
\end{eulerprompt}
\begin{euleroutput}
  116.3
\end{euleroutput}
\begin{eulerprompt}
>varanalysis(x1,x2,x3)
\end{eulerprompt}
\begin{euleroutput}
  0.0138048221371
\end{euleroutput}
\begin{eulercomment}
Artinya, kita menolak hipotesis nilai rata-rata yang sama. Kita
melakukan ini dengan probabilitas kesalahan sebesar 1,3\%.

Ada juga uji median, yang menolak sampel data dengan distribusi
rata-rata yang berbeda dengan menguji median sampel gabungan.
\end{eulercomment}
\begin{eulerprompt}
>a=[56,66,68,49,61,53,45,58,54];
>b=[72,81,51,73,69,78,59,67,65,71,68,71];
>mediantest(a,b)
\end{eulerprompt}
\begin{euleroutput}
  0.0241724220052
\end{euleroutput}
\begin{eulercomment}
Uji kesetaraan lainnya adalah uji peringkat. Uji peringkat jauh lebih
tajam daripada uji median.
\end{eulercomment}
\begin{eulerprompt}
>ranktest(a,b)
\end{eulerprompt}
\begin{euleroutput}
  0.00199969612469
\end{euleroutput}
\begin{eulercomment}
Dalam contoh berikut, kedua distribusi memiliki rata-rata yang sama.
\end{eulercomment}
\begin{eulerprompt}
>ranktest(random(1,100),random(1,50)*3-1)
\end{eulerprompt}
\begin{euleroutput}
  0.119780211001
\end{euleroutput}
\begin{eulercomment}
Sekarang, mari kita coba simulasikan dua perawatan a dan b yang
diterapkan pada orang yang berbeda.
\end{eulercomment}
\begin{eulerprompt}
>a=[8.0,7.4,5.9,9.4,8.6,8.2,7.6,8.1,6.2,8.9];
>b=[6.8,7.1,6.8,8.3,7.9,7.2,7.4,6.8,6.8,8.1];
\end{eulerprompt}
\begin{eulercomment}
Uji signum memutuskan, apakah a lebih baik dari b.
\end{eulercomment}
\begin{eulerprompt}
>signtest(a,b)
\end{eulerprompt}
\begin{euleroutput}
  0.0546875
\end{euleroutput}
\begin{eulercomment}
Ini adalah kesalahan yang sangat besar. Kita tidak dapat menolak bahwa
a sama baiknya dengan b.

Uji Wilcoxon lebih tajam daripada uji ini, tetapi bergantung pada
nilai kuantitatif perbedaannya.
\end{eulercomment}
\begin{eulerprompt}
>wilcoxon(a,b)
\end{eulerprompt}
\begin{euleroutput}
  0.0296680599405
\end{euleroutput}
\begin{eulercomment}
Mari kita coba dua pengujian lagi menggunakan seri yang dihasilkan.
\end{eulercomment}
\begin{eulerprompt}
>wilcoxon(normal(1,20),normal(1,20)-1)
\end{eulerprompt}
\begin{euleroutput}
  0.0309763942686
\end{euleroutput}
\begin{eulerprompt}
>wilcoxon(normal(1,20),normal(1,20))
\end{eulerprompt}
\begin{euleroutput}
  0.588619799108
\end{euleroutput}
\eulerheading{Angka Acak}
\begin{eulercomment}
Berikut ini adalah pengujian untuk generator angka acak. Euler
menggunakan generator yang sangat bagus, jadi kita tidak perlu
mengharapkan masalah apa pun.

Pertama-tama kita menghasilkan sepuluh juta angka acak dalam [0,1].
\end{eulercomment}
\begin{eulerprompt}
>n:=10000000; r:=random(1,n);
\end{eulerprompt}
\begin{eulercomment}
Berikutnya kita hitung jarak antara dua angka kurang dari 0,05.
\end{eulercomment}
\begin{eulerprompt}
>a:=0.05; d:=differences(nonzeros(r<a));
\end{eulerprompt}
\begin{eulercomment}
Terakhir, kami memplot berapa kali setiap jarak terjadi, dan
membandingkannya dengan nilai yang diharapkan.
\end{eulercomment}
\begin{eulerprompt}
>m=getmultiplicities(1:100,d); plot2d(m); ...
>  plot2d("n*(1-a)^(x-1)*a^2",color=red,>add):
\end{eulerprompt}
\begin{eulercomment}
Hapus data.
\end{eulercomment}
\begin{eulerprompt}
>remvalue n;
\end{eulerprompt}
\begin{eulercomment}
\begin{eulercomment}
\eulerheading{Pendahuluan bagi Pengguna Proyek R}
\begin{eulercomment}
Jelas, EMT tidak bersaing dengan R sebagai paket statistik. Akan
tetapi, ada banyak prosedur dan fungsi statistik yang tersedia di EMT
juga. Jadi, EMT dapat memenuhi kebutuhan dasar. Lagi pula, EMT
dilengkapi dengan paket numerik dan sistem aljabar komputer.

Buku catatan ini ditujukan bagi Anda yang sudah familier dengan R,
tetapi perlu mengetahui perbedaan sintaksis EMT dan R. Kami mencoba
memberikan gambaran umum tentang hal-hal yang jelas dan kurang jelas
yang perlu Anda ketahui.

Selain itu, kami melihat cara untuk bertukar data antara kedua sistem.
\end{eulercomment}
\begin{eulercomment}
Harap dicatat bahwa ini adalah pekerjaan yang masih dalam tahap
pengerjaan.
\end{eulercomment}
\eulerheading{Sintaksis Dasar}
\begin{eulercomment}
Hal pertama yang Anda pelajari di R adalah membuat vektor. Dalam EMT,
perbedaan utamanya adalah operator : dapat mengambil ukuran langkah.
Selain itu, operator ini memiliki daya pengikatan yang rendah.
\end{eulercomment}
\begin{eulerprompt}
>n=10; 0:n/20:n-1
\end{eulerprompt}
\begin{euleroutput}
  [0,  0.5,  1,  1.5,  2,  2.5,  3,  3.5,  4,  4.5,  5,  5.5,  6,  6.5,
  7,  7.5,  8,  8.5,  9]
\end{euleroutput}
\begin{eulercomment}
Fungsi c() tidak ada. Dimungkinkan untuk menggunakan vektor guna
menggabungkan berbagai hal.

Contoh berikut ini, seperti banyak contoh lainnya, berasal dari
"Interoduction to R" yang disertakan dalam proyek R. Jika Anda membaca
PDF ini, Anda akan menemukan bahwa saya mengikuti alurnya dalam
tutorial ini.
\end{eulercomment}
\begin{eulerprompt}
>x=[10.4, 5.6, 3.1, 6.4, 21.7]; [x,0,x]
\end{eulerprompt}
\begin{euleroutput}
  [10.4,  5.6,  3.1,  6.4,  21.7,  0,  10.4,  5.6,  3.1,  6.4,  21.7]
\end{euleroutput}
\begin{eulercomment}
Operator titik dua dengan ukuran langkah EMT digantikan oleh fungsi
seq() di R. Kita dapat menulis fungsi ini dalam EMT.
\end{eulercomment}
\begin{eulerprompt}
>function seq(a,b,c) := a:b:c; ...
>seq(0,-0.1,-1)
\end{eulerprompt}
\begin{euleroutput}
  [0,  -0.1,  -0.2,  -0.3,  -0.4,  -0.5,  -0.6,  -0.7,  -0.8,  -0.9,  -1]
\end{euleroutput}
\begin{eulercomment}
Fungsi rep() dari R tidak ada dalam EMT. Untuk input vektor, dapat
ditulis sebagai berikut.
\end{eulercomment}
\begin{eulerprompt}
>function rep(x:vector,n:index) := flatten(dup(x,n)); ...
>rep(x,2)
\end{eulerprompt}
\begin{euleroutput}
  [10.4,  5.6,  3.1,  6.4,  21.7,  10.4,  5.6,  3.1,  6.4,  21.7]
\end{euleroutput}
\begin{eulercomment}
Perhatikan bahwa "=" atau ":=" digunakan untuk penugasan. Operator
"-\textgreater{}" digunakan untuk unit dalam EMT.
\end{eulercomment}
\begin{eulerprompt}
>125km -> " miles"
\end{eulerprompt}
\begin{euleroutput}
  77.6713990297 miles
\end{euleroutput}
\begin{eulercomment}
Operator "\textless{}-" untuk penugasan menyesatkan dan bukan ide yang baik
untuk R. Berikut ini akan membandingkan a dan -4 dalam EMT.
\end{eulercomment}
\begin{eulerprompt}
>a=2; a<-4
\end{eulerprompt}
\begin{euleroutput}
  0
\end{euleroutput}
\begin{eulercomment}
Dalam R, "a\textless{}-4\textless{}3" berfungsi, tetapi "a\textless{}-4\textless{}-3" tidak. Saya juga
mengalami ambiguitas serupa dalam EMT, tetapi mencoba menghilangkannya
sedikit demi sedikit.

EMT dan R memiliki vektor bertipe boolean. Namun dalam EMT, angka 0
dan 1 digunakan untuk mewakili false dan true. Dalam R, nilai true dan
false tetap dapat digunakan dalam aritmatika biasa seperti dalam EMT.
\end{eulercomment}
\begin{eulerprompt}
>x<5, %*x
\end{eulerprompt}
\begin{euleroutput}
  [0,  0,  1,  0,  0]
  [0,  0,  3.1,  0,  0]
\end{euleroutput}
\begin{eulercomment}
EMT memunculkan kesalahan atau menghasilkan NAN, tergantung pada tanda
"kesalahan".
\end{eulercomment}
\begin{eulerprompt}
>errors off; 0/0, isNAN(sqrt(-1)), errors on;
\end{eulerprompt}
\begin{euleroutput}
  NAN
  1
\end{euleroutput}
\begin{eulercomment}
String sama di R dan EMT. Keduanya berada di lokal saat ini, bukan di
Unicode.

Di R ada paket untuk Unicode. Di EMT, string dapat berupa string
Unicode. String unicode dapat diterjemahkan ke pengodean lokal dan
sebaliknya. Selain itu, u"..." dapat berisi entitas HTML.
\end{eulercomment}
\begin{eulerprompt}
>u"&#169; Ren&eacut; Grothmann"
\end{eulerprompt}
\begin{euleroutput}
  © René Grothmann
\end{euleroutput}
\begin{eulercomment}
Berikut ini mungkin atau mungkin tidak ditampilkan dengan benar pada
sistem Anda sebagai A dengan titik dan garis di atasnya. Hal ini
bergantung pada font yang Anda gunakan.
\end{eulercomment}
\begin{eulerprompt}
>chartoutf([480])
\end{eulerprompt}
\begin{euleroutput}
  Ǡ
\end{euleroutput}
\begin{eulercomment}
Penggabungan string dilakukan dengan "+" atau "\textbar{}". String dapat
menyertakan angka, yang akan dicetak dalam format saat ini.
\end{eulercomment}
\begin{eulerprompt}
>"pi = "+pi
\end{eulerprompt}
\begin{euleroutput}
  pi = 3.14159265359
\end{euleroutput}
\eulerheading{Pengindeksan}
\begin{eulercomment}
Sering kali, ini akan berfungsi seperti di R.

Namun EMT akan menginterpretasikan indeks negatif dari belakang
vektor, sementara R menginterpretasikan x[n] sebagai x tanpa elemen
ke-n.
\end{eulercomment}
\begin{eulerprompt}
>x, x[1:3], x[-2]
\end{eulerprompt}
\begin{euleroutput}
  [-55,  -45,  -35,  -25,  -15,  -5,  5,  15,  25,  35,  45,  55]
  [-55,  -45,  -35]
  45
\end{euleroutput}
\begin{eulercomment}
Perilaku R dapat dicapai dalam EMT dengan drop().
\end{eulercomment}
\begin{eulerprompt}
>drop(x,2)
\end{eulerprompt}
\begin{euleroutput}
  [-55,  -35,  -25,  -15,  -5,  5,  15,  25,  35,  45,  55]
\end{euleroutput}
\begin{eulercomment}
Vektor logika tidak diperlakukan secara berbeda sebagai indeks dalam
EMT, berbeda dengan R. Anda perlu mengekstrak elemen bukan nol
terlebih dahulu dalam EMT.
\end{eulercomment}
\begin{eulerprompt}
>x, x>5, x[nonzeros(x>5)]
\end{eulerprompt}
\begin{euleroutput}
  [-55,  -45,  -35,  -25,  -15,  -5,  5,  15,  25,  35,  45,  55]
  [0,  0,  0,  0,  0,  0,  0,  1,  1,  1,  1,  1]
  [15,  25,  35,  45,  55]
\end{euleroutput}
\begin{eulercomment}
Sama seperti di R, vektor indeks dapat berisi pengulangan.
\end{eulercomment}
\begin{eulerprompt}
>x[[1,2,2,1]]
\end{eulerprompt}
\begin{euleroutput}
  [-55,  -45,  -45,  -55]
\end{euleroutput}
\begin{eulercomment}
Namun, nama untuk indeks tidak dimungkinkan dalam EMT. Untuk paket
statistik, hal ini mungkin sering diperlukan untuk memudahkan akses ke
elemen vektor.

Untuk meniru perilaku ini, kita dapat mendefinisikan fungsi sebagai
berikut.
\end{eulercomment}
\begin{eulerprompt}
>function sel (v,i,s) := v[indexof(s,i)]; ...
>s=["first","second","third","fourth"]; sel(x,["first","third"],s)
\end{eulerprompt}
\begin{euleroutput}
  
  Trying to overwrite protected function sel!
  Error in:
  function sel (v,i,s) := v[indexof(s,i)]; ... ...
               ^
  
  Trying to overwrite protected function sel!
  Error in:
  function sel (v,i,s) := v[indexof(s,i)]; ... ...
               ^
  
  Trying to overwrite protected function sel!
  Error in:
  function sel (v,i,s) := v[indexof(s,i)]; ... ...
               ^
  
  Trying to overwrite protected function sel!
  Error in:
  function sel (v,i,s) := v[indexof(s,i)]; ... ...
               ^
  
  Trying to overwrite protected function sel!
  Error in:
  function sel (v,i,s) := v[indexof(s,i)]; ... ...
               ^
  [10.4,  3.1]
\end{euleroutput}
\eulerheading{Tipe Data}
\begin{eulercomment}
EMT memiliki lebih banyak tipe data tetap daripada R. Jelas, di R
terdapat vektor yang terus bertambah. Anda dapat menetapkan vektor
numerik kosong v dan menetapkan nilai ke elemen v[17]. Hal ini tidak
mungkin dilakukan di EMT.

Berikut ini agak tidak efisien.
\end{eulercomment}
\begin{eulerprompt}
>v=[]; for i=1 to 10000; v=v|i; end;
\end{eulerprompt}
\begin{eulercomment}
EMT sekarang akan membuat vektor dengan v dan i yang ditambahkan pada
tumpukan dan menyalin vektor itu kembali ke variabel global v.

Yang lebih efisien mendefinisikan vektor terlebih dahulu.
\end{eulercomment}
\begin{eulerprompt}
>v=zeros(10000); for i=1 to 10000; v[i]=i; end;
\end{eulerprompt}
\begin{eulercomment}
Yang lebih efisien mendefinisikan vektor terlebih dahulu.
\end{eulercomment}
\begin{eulerprompt}
>complex(1:4)
\end{eulerprompt}
\begin{euleroutput}
  [ 1+0i ,  2+0i ,  3+0i ,  4+0i  ]
\end{euleroutput}
\begin{eulercomment}
Konversi ke string hanya dimungkinkan untuk tipe data dasar. Format
saat ini digunakan untuk penggabungan string sederhana. Namun, ada
fungsi seperti print() atau frac().

Untuk vektor, Anda dapat dengan mudah menulis fungsi Anda sendiri.
\end{eulercomment}
\begin{eulerprompt}
>function tostr (v) ...
\end{eulerprompt}
\begin{eulerudf}
  s="[";
  loop 1 to length(v);
     s=s+print(v[#],2,0);
     if #<length(v) then s=s+","; endif;
  end;
  return s+"]";
  endfunction
\end{eulerudf}
\begin{eulerprompt}
>tostr(linspace(0,1,10))
\end{eulerprompt}
\begin{euleroutput}
  [0.00,0.10,0.20,0.30,0.40,0.50,0.60,0.70,0.80,0.90,1.00]
\end{euleroutput}
\begin{eulercomment}
Untuk komunikasi dengan Maxima, terdapat fungsi convertmxm(), yang
juga dapat digunakan untuk memformat vektor untuk keluaran.
\end{eulercomment}
\begin{eulerprompt}
>convertmxm(1:10)
\end{eulerprompt}
\begin{euleroutput}
  [1,2,3,4,5,6,7,8,9,10]
\end{euleroutput}
\begin{eulercomment}
Untuk Latex perintah tex dapat digunakan untuk mendapatkan perintah
Latex.
\end{eulercomment}
\begin{eulerprompt}
>tex(&[1,2,3])
\end{eulerprompt}
\begin{euleroutput}
  \(\backslash\)left[ 1 , 2 , 3 \(\backslash\)right] 
\end{euleroutput}
\eulerheading{Faktor dan Tabel}
\begin{eulercomment}
Dalam pengantar R terdapat contoh dengan apa yang disebut faktor.

Berikut ini adalah daftar wilayah dari 30 negara bagian.
\end{eulercomment}
\begin{eulerprompt}
>austates = ["tas", "sa", "qld", "nsw", "nsw", "nt", "wa", "wa", ...
>"qld", "vic", "nsw", "vic", "qld", "qld", "sa", "tas", ...
>"sa", "nt", "wa", "vic", "qld", "nsw", "nsw", "wa", ...
>"sa", "act", "nsw", "vic", "vic", "act"];
\end{eulerprompt}
\begin{eulercomment}
Asumsikan, kita memiliki pendapatan yang sesuai di setiap negara
bagian.
\end{eulercomment}
\begin{eulerprompt}
>incomes = [60, 49, 40, 61, 64, 60, 59, 54, 62, 69, 70, 42, 56, ...
>61, 61, 61, 58, 51, 48, 65, 49, 49, 41, 48, 52, 46, ...
>59, 46, 58, 43];
\end{eulerprompt}
\begin{eulercomment}
Sekarang, kita ingin menghitung rata-rata pendapatan di wilayah
tersebut. Sebagai program statistik, R memiliki factor() dan tappy()
untuk ini.

EMT dapat melakukan ini dengan menemukan indeks wilayah dalam daftar
wilayah yang unik.
\end{eulercomment}
\begin{eulerprompt}
>auterr=sort(unique(austates)); f=indexofsorted(auterr,austates)
\end{eulerprompt}
\begin{euleroutput}
  [6,  5,  4,  2,  2,  3,  8,  8,  4,  7,  2,  7,  4,  4,  5,  6,  5,  3,
  8,  7,  4,  2,  2,  8,  5,  1,  2,  7,  7,  1]
\end{euleroutput}
\begin{eulercomment}
Pada titik tersebut, kita dapat menulis fungsi loop kita sendiri untuk
melakukan sesuatu hanya untuk satu faktor.

Atau kita dapat meniru fungsi tapply() dengan cara berikut.
\end{eulercomment}
\begin{eulerprompt}
>function map tappl (i; f$:call, cat, x) ...
\end{eulerprompt}
\begin{eulerudf}
  u=sort(unique(cat));
  f=indexof(u,cat);
  return f$(x[nonzeros(f==indexof(u,i))]);
  endfunction
\end{eulerudf}
\begin{eulercomment}
Agak tidak efisien, karena menghitung wilayah unik untuk setiap i,
tetapi berhasil.
\end{eulercomment}
\begin{eulerprompt}
>tappl(auterr,"mean",austates,incomes)
\end{eulerprompt}
\begin{euleroutput}
  [44.5,  57.3333333333,  55.5,  53.6,  55,  60.5,  56,  52.25]
\end{euleroutput}
\begin{eulercomment}
Perhatikan bahwa ini berfungsi untuk setiap vektor wilayah.
\end{eulercomment}
\begin{eulerprompt}
>tappl(["act","nsw"],"mean",austates,incomes)
\end{eulerprompt}
\begin{euleroutput}
  [44.5,  57.3333333333]
\end{euleroutput}
\begin{eulercomment}
Sekarang, paket statistik EMT mendefinisikan tabel seperti di R.
Fungsi readtable() dan writetable() dapat digunakan untuk input dan
output.

Jadi kita dapat mencetak pendapatan negara rata-rata di wilayah dengan
cara yang mudah.
\end{eulercomment}
\begin{eulerprompt}
>writetable(tappl(auterr,"mean",austates,incomes),labc=auterr,wc=7)
\end{eulerprompt}
\begin{euleroutput}
      act    nsw     nt    qld     sa    tas    vic     wa
     44.5  57.33   55.5   53.6     55   60.5     56  52.25
\end{euleroutput}
\begin{eulercomment}
Kita juga dapat mencoba meniru perilaku R sepenuhnya.

Faktor-faktor tersebut harus disimpan dalam suatu koleksi dengan jenis
dan kategori (negara bagian dan teritori dalam contoh kita). Untuk
EMT, kita tambahkan indeks yang telah dihitung sebelumnya.
\end{eulercomment}
\begin{eulerprompt}
>function makef (t) ...
\end{eulerprompt}
\begin{eulerudf}
  ## Factor data
  ## Returns a collection with data t, unique data, indices.
  ## See: tapply
  u=sort(unique(t));
  return \{\{t,u,indexofsorted(u,t)\}\};
  endfunction
\end{eulerudf}
\begin{eulerprompt}
>statef=makef(austates);
\end{eulerprompt}
\begin{eulercomment}
Sekarang elemen ketiga dari koleksi akan berisi indeks.
\end{eulercomment}
\begin{eulerprompt}
>statef[3]
\end{eulerprompt}
\begin{euleroutput}
  [6,  5,  4,  2,  2,  3,  8,  8,  4,  7,  2,  7,  4,  4,  5,  6,  5,  3,
  8,  7,  4,  2,  2,  8,  5,  1,  2,  7,  7,  1]
\end{euleroutput}
\begin{eulercomment}
Sekarang kita dapat meniru tapply() dengan cara berikut. Fungsi ini
akan mengembalikan tabel sebagai kumpulan data tabel dan judul kolom.
\end{eulercomment}
\begin{eulerprompt}
>function tapply (t:vector,tf,f$:call) ...
\end{eulerprompt}
\begin{eulerudf}
  ## Makes a table of data and factors
  ## tf : output of makef()
  ## See: makef
  uf=tf[2]; f=tf[3]; x=zeros(length(uf));
  for i=1 to length(uf);
     ind=nonzeros(f==i);
     if length(ind)==0 then x[i]=NAN;
     else x[i]=f$(t[ind]);
     endif;
  end;
  return \{\{x,uf\}\};
  endfunction
\end{eulerudf}
\begin{eulercomment}
Kami tidak menambahkan banyak pemeriksaan tipe di sini. Satu-satunya
tindakan pencegahan menyangkut kategori (faktor) tanpa data. Namun,
seseorang harus memeriksa panjang t yang benar dan kebenaran koleksi
tf.

Tabel ini dapat dicetak sebagai tabel dengan writetable().
\end{eulercomment}
\begin{eulerprompt}
>writetable(tapply(incomes,statef,"mean"),wc=7)
\end{eulerprompt}
\begin{euleroutput}
      act    nsw     nt    qld     sa    tas    vic     wa
     44.5  57.33   55.5   53.6     55   60.5     56  52.25
\end{euleroutput}
\eulerheading{Array}
\begin{eulercomment}
EMT hanya memiliki dua dimensi untuk array. Tipe data ini disebut
matriks. Akan mudah untuk menulis fungsi untuk dimensi yang lebih
tinggi atau pustaka C untuk ini.

R memiliki lebih dari dua dimensi. Dalam R, array adalah vektor dengan
bidang dimensi.

Dalam EMT, vektor adalah matriks dengan satu baris. Vektor dapat
dibuat menjadi matriks dengan redim().
\end{eulercomment}
\begin{eulerprompt}
>shortformat; X=redim(1:20,4,5)
\end{eulerprompt}
\begin{euleroutput}
          1         2         3         4         5 
          6         7         8         9        10 
         11        12        13        14        15 
         16        17        18        19        20 
\end{euleroutput}
\begin{eulercomment}
Ekstraksi baris dan kolom, atau sub-matriks, sangat mirip di R.
\end{eulercomment}
\begin{eulerprompt}
>X[,2:3]
\end{eulerprompt}
\begin{euleroutput}
          2         3 
          7         8 
         12        13 
         17        18 
\end{euleroutput}
\begin{eulercomment}
Namun, dalam R dimungkinkan untuk menetapkan daftar indeks vektor
tertentu ke suatu nilai. Hal yang sama dimungkinkan dalam EMT hanya
dengan loop.
\end{eulercomment}
\begin{eulerprompt}
>function setmatrixvalue (M, i, j, v) ...
\end{eulerprompt}
\begin{eulerudf}
  loop 1 to max(length(i),length(j),length(v))
     M[i\{#\},j\{#\}] = v\{#\};
  end;
  endfunction
\end{eulerudf}
\begin{eulercomment}
Kami mendemonstrasikan ini untuk menunjukkan bahwa matriks dilewatkan
dengan referensi dalam EMT. Jika Anda tidak ingin mengubah matriks
asli M, Anda perlu menyalinnya dalam fungsi tersebut.
\end{eulercomment}
\begin{eulerprompt}
>setmatrixvalue(X,1:3,3:-1:1,0); X,
\end{eulerprompt}
\begin{euleroutput}
          1         2         0         4         5 
          6         0         8         9        10 
          0        12        13        14        15 
         16        17        18        19        20 
\end{euleroutput}
\begin{eulercomment}
Produk luar dalam EMT hanya dapat dilakukan antara vektor. Hal ini
dilakukan secara otomatis karena bahasa matriks. Satu vektor harus
berupa vektor kolom dan yang lainnya berupa vektor baris.
\end{eulercomment}
\begin{eulerprompt}
>(1:5)*(1:5)'
\end{eulerprompt}
\begin{euleroutput}
          1         2         3         4         5 
          2         4         6         8        10 
          3         6         9        12        15 
          4         8        12        16        20 
          5        10        15        20        25 
\end{euleroutput}
\begin{eulercomment}
Dalam pengantar PDF untuk R terdapat sebuah contoh, yang menghitung
distribusi ab-cd untuk a,b,c,d yang dipilih secara acak dari 0 hingga
n. Solusi dalam R adalah membentuk matriks 4 dimensi dan menjalankan
table() di atasnya.

Tentu saja, ini dapat dicapai dengan loop. Namun, loop tidak efektif
dalam EMT atau R. Dalam EMT, kita dapat menulis loop dalam C dan itu
akan menjadi solusi tercepat.

Namun, kita ingin meniru perilaku R. Untuk ini, kita perlu meratakan
perkalian ab dan membuat matriks ab-cd.
\end{eulercomment}
\begin{eulerprompt}
>a=0:6; b=a'; p=flatten(a*b); q=flatten(p-p'); ...
>u=sort(unique(q)); f=getmultiplicities(u,q); ...
>statplot(u,f,"h"):
\end{eulerprompt}
\begin{eulercomment}
Selain multiplisitas yang tepat, EMT dapat menghitung frekuensi dalam
vektor.
\end{eulercomment}
\begin{eulerprompt}
>getfrequencies(q,-50:10:50)
\end{eulerprompt}
\begin{euleroutput}
  [0,  23,  132,  316,  602,  801,  333,  141,  53,  0]
\end{euleroutput}
\begin{eulercomment}
Cara termudah untuk memplot ini sebagai distribusi adalah sebagai
berikut.
\end{eulercomment}
\begin{eulerprompt}
>plot2d(q,distribution=11):
\end{eulerprompt}
\begin{eulercomment}
Namun, Anda juga dapat menghitung terlebih dahulu jumlah dalam
interval yang dipilih. Tentu saja, berikut ini menggunakan
getfrequencies() secara internal.

Karena fungsi histo() mengembalikan frekuensi, kita perlu
menskalakannya sehingga integral di bawah grafik batang adalah 1.
\end{eulercomment}
\begin{eulerprompt}
>\{x,y\}=histo(q,v=-55:10:55); y=y/sum(y)/differences(x); ...
>plot2d(x,y,>bar,style="/"):
\end{eulerprompt}
\eulerheading{Daftar}
\begin{eulercomment}
EMT memiliki dua jenis daftar. Satu adalah daftar global yang dapat
diubah, dan yang lainnya adalah jenis daftar yang tidak dapat diubah.
Kami tidak peduli dengan daftar global di sini.

Jenis daftar yang tidak dapat diubah disebut koleksi dalam EMT. Ia
berperilaku seperti struktur dalam C, tetapi elemennya hanya diberi
nomor dan tidak diberi nama.
\end{eulercomment}
\begin{eulerprompt}
>L=\{\{"Fred","Flintstone",40,[1990,1992]\}\}
\end{eulerprompt}
\begin{euleroutput}
  Fred
  Flintstone
  40
  [1990,  1992]
\end{euleroutput}
\begin{eulercomment}
Saat ini unsur-unsur tersebut tidak memiliki nama, meskipun nama dapat
ditetapkan untuk tujuan khusus. Unsur-unsur tersebut diakses dengan
angka.
\end{eulercomment}
\begin{eulerprompt}
>(L[4])[2]
\end{eulerprompt}
\begin{euleroutput}
  1992
\end{euleroutput}
\begin{eulercomment}
\begin{eulercomment}
\eulerheading{Input dan Output File (Membaca dan Menulis Data)}
\begin{eulercomment}
Anda sering kali ingin mengimpor matriks data dari sumber lain ke EMT.
Tutorial ini memberi tahu Anda tentang berbagai cara untuk
mencapainya. Fungsi sederhana adalah writematrix() dan readmatrix().

Mari kita tunjukkan cara membaca dan menulis vektor bilangan real ke
dalam file.
\end{eulercomment}
\begin{eulerprompt}
>a=random(1,100); mean(a), dev(a),
\end{eulerprompt}
\begin{euleroutput}
  0.52013
  0.29327
\end{euleroutput}
\begin{eulercomment}
Untuk menulis data ke dalam berkas, kami menggunakan fungsi
writematrix().

Karena pengantar ini kemungkinan besar berada di dalam direktori,
tempat pengguna tidak memiliki akses tulis, kami menulis data ke
direktori beranda pengguna. Untuk buku catatan sendiri, ini tidak
diperlukan, karena berkas data akan ditulis ke dalam direktori yang
sama.
\end{eulercomment}
\begin{eulerprompt}
>filename="test.dat";
\end{eulerprompt}
\begin{eulercomment}
Sekarang kita tulis vektor kolom a' ke dalam berkas. Ini menghasilkan
satu angka di setiap baris berkas.
\end{eulercomment}
\begin{eulerprompt}
>writematrix(a',filename);
\end{eulerprompt}
\begin{eulercomment}
Untuk membaca data, kita menggunakan readmatrix().
\end{eulercomment}
\begin{eulerprompt}
>a=readmatrix(filename)';
\end{eulerprompt}
\begin{eulercomment}
Dan hapus berkasnya.
\end{eulercomment}
\begin{eulerprompt}
>fileremove(filename);
>mean(a), dev(a),
\end{eulerprompt}
\begin{euleroutput}
  0.52013
  0.29327
\end{euleroutput}
\begin{eulercomment}
Fungsi writematrix() atau writetable() dapat dikonfigurasi untuk
bahasa lain.

Misalnya, jika Anda memiliki sistem bahasa Indonesia (titik desimal
dengan koma), Excel Anda memerlukan nilai dengan koma desimal yang
dipisahkan oleh titik koma dalam file csv (nilai default dipisahkan
dengan koma). File berikut "test.csv" akan muncul di folder Anda saat
ini.
\end{eulercomment}
\begin{eulerprompt}
>filename="test.csv"; ...
>writematrix(random(5,3),file=filename,separator=",");
\end{eulerprompt}
\begin{eulercomment}
Anda sekarang dapat membuka berkas ini langsung dengan Excel
Indonesia.
\end{eulercomment}
\begin{eulerprompt}
>fileremove(filename);
\end{eulerprompt}
\begin{eulercomment}
Terkadang kita memiliki string dengan token seperti berikut.
\end{eulercomment}
\begin{eulerprompt}
>s1:="f m m f m m m f f f m m f";  ...
>s2:="f f f m m f f";
\end{eulerprompt}
\begin{eulercomment}
Untuk menokenisasi ini, kami mendefinisikan vektor token.
\end{eulercomment}
\begin{eulerprompt}
>tok:=["f","m"]
\end{eulerprompt}
\begin{euleroutput}
  f
  m
\end{euleroutput}
\begin{eulercomment}
Lalu kita dapat menghitung berapa kali setiap token muncul dalam
string, dan memasukkan hasilnya ke dalam tabel.
\end{eulercomment}
\begin{eulerprompt}
>M:=getmultiplicities(tok,strtokens(s1))_ ...
>  getmultiplicities(tok,strtokens(s2));
\end{eulerprompt}
\begin{eulercomment}
Tulis tabel dengan tajuk token.
\end{eulercomment}
\begin{eulerprompt}
>writetable(M,labc=tok,labr=1:2,wc=8)
\end{eulerprompt}
\begin{euleroutput}
                 f       m
         1       6       7
         2       5       2
\end{euleroutput}
\begin{eulercomment}
Untuk statika, EMT dapat membaca dan menulis tabel.
\end{eulercomment}
\begin{eulerprompt}
>file="test.dat"; open(file,"w"); ...
>writeln("A,B,C"); writematrix(random(3,3)); ...
>close();
\end{eulerprompt}
\begin{eulercomment}
Berkasnya tampak seperti ini.
\end{eulercomment}
\begin{eulerprompt}
>printfile(file)
\end{eulerprompt}
\begin{euleroutput}
  A,B,C
  0.7199659096994914,0.9543023869002717,0.5452423982068915
  0.8758936037546752,0.3164476882191624,0.4247510629258306
  0.7031089918521533,0.7963205243867099,0.5570648612224093
  
\end{euleroutput}
\begin{eulercomment}
Fungsi readtable() dalam bentuk yang paling sederhana dapat membacanya
dan mengembalikan kumpulan nilai dan baris judul.
\end{eulercomment}
\begin{eulerprompt}
>L=readtable(file,>list);
\end{eulerprompt}
\begin{eulercomment}
Koleksi ini dapat dicetak dengan writetable() ke buku catatan, atau ke
berkas.
\end{eulercomment}
\begin{eulerprompt}
>writetable(L,wc=10,dc=5)
\end{eulerprompt}
\begin{euleroutput}
           A         B         C
     0.71997    0.9543   0.54524
     0.87589   0.31645   0.42475
     0.70311   0.79632   0.55706
\end{euleroutput}
\begin{eulercomment}
Matriks nilai adalah elemen pertama L. Perhatikan bahwa mean() dalam
EMT menghitung nilai rata-rata baris matriks.
\end{eulercomment}
\begin{eulerprompt}
>mean(L[1])
\end{eulerprompt}
\begin{euleroutput}
    0.73984 
    0.53903 
     0.6855 
\end{euleroutput}
\eulerheading{Berkas CSV}
\begin{eulercomment}
Pertama, mari kita tulis matriks ke dalam berkas. Untuk output, kita
buat berkas di direktori kerja saat ini.
\end{eulercomment}
\begin{eulerprompt}
>file="test.csv";  ...
>M=random(3,3); writematrix(M,file);
\end{eulerprompt}
\begin{eulercomment}
Berikut ini isi berkas tersebut.
\end{eulercomment}
\begin{eulerprompt}
>printfile(file)
\end{eulerprompt}
\begin{euleroutput}
  0.396060285468178,0.5217040592547716,0.07945744123024982
  0.5777005134637205,0.6537971103437032,0.6974183392154792
  0.8326731141550535,0.9143360255838195,0.6760036570300865
  
\end{euleroutput}
\begin{eulercomment}
CVS ini dapat dibuka pada sistem bahasa Inggris ke Excel dengan
mengklik dua kali. Jika Anda mendapatkan berkas tersebut pada sistem
bahasa Jerman, Anda perlu mengimpor data ke Excel dengan memperhatikan
titik desimal.

Namun, titik desimal juga merupakan format default untuk EMT. Anda
dapat membaca matriks dari berkas dengan readmatrix().
\end{eulercomment}
\begin{eulerprompt}
>readmatrix(file)
\end{eulerprompt}
\begin{euleroutput}
    0.39606    0.5217  0.079457 
     0.5777    0.6538   0.69742 
    0.83267   0.91434     0.676 
\end{euleroutput}
\begin{eulercomment}
Dimungkinkan untuk menulis beberapa matriks ke dalam satu berkas.
Perintah open() dapat membuka berkas untuk ditulis dengan parameter
"w". Nilai default untuk membaca adalah "r".
\end{eulercomment}
\begin{eulerprompt}
>open(file,"w"); writematrix(M); writematrix(M'); close();
\end{eulerprompt}
\begin{eulercomment}
Matriks dipisahkan oleh baris kosong. Untuk membaca matriks, buka
berkas dan panggil readmatrix() beberapa kali.
\end{eulercomment}
\begin{eulerprompt}
>open(file); A=readmatrix(); B=readmatrix(); A==B, close();
\end{eulerprompt}
\begin{euleroutput}
          1         0         0 
          0         1         0 
          0         0         1 
\end{euleroutput}
\begin{eulercomment}
Di Excel atau lembar kerja serupa, Anda dapat mengekspor matriks
sebagai CSV (nilai yang dipisahkan koma). Di Excel 2007, gunakan
"simpan sebagai" dan "format lain", lalu pilih "CSV". Pastikan, tabel
saat ini hanya berisi data yang ingin Anda ekspor.

Berikut ini contohnya.
\end{eulercomment}
\begin{eulerprompt}
>printfile("excel-data.csv")
\end{eulerprompt}
\begin{euleroutput}
  0;1000;1000
  1;1051,271096;1072,508181
  2;1105,170918;1150,273799
  3;1161,834243;1233,67806
  4;1221,402758;1323,129812
  5;1284,025417;1419,067549
  6;1349,858808;1521,961556
  7;1419,067549;1632,31622
  8;1491,824698;1750,6725
  9;1568,312185;1877,610579
  10;1648,721271;2013,752707
\end{euleroutput}
\begin{eulercomment}
Seperti yang Anda lihat, sistem Jerman saya menggunakan titik koma
sebagai pemisah dan koma desimal. Anda dapat mengubahnya di pengaturan
sistem atau di Excel, tetapi tidak diperlukan untuk membaca matriks ke
EMT.

Cara termudah untuk membaca ini ke Euler adalah readmatrix(). Semua
koma diganti dengan titik dengan parameter \textgreater{}comma. Untuk CSV bahasa
Inggris, cukup abaikan parameter ini.
\end{eulercomment}
\begin{eulerprompt}
>M=readmatrix("excel-data.csv",>comma)
\end{eulerprompt}
\begin{euleroutput}
          0      1000      1000 
          1    1051.3    1072.5 
          2    1105.2    1150.3 
          3    1161.8    1233.7 
          4    1221.4    1323.1 
          5      1284    1419.1 
          6    1349.9      1522 
          7    1419.1    1632.3 
          8    1491.8    1750.7 
          9    1568.3    1877.6 
         10    1648.7    2013.8 
\end{euleroutput}
\begin{eulercomment}
Mari kita plot ini.
\end{eulercomment}
\begin{eulerprompt}
>plot2d(M'[1],M'[2:3],>points,color=[red,green]'):
\end{eulerprompt}
\begin{eulercomment}
Ada cara yang lebih mendasar untuk membaca data dari sebuah berkas.
Anda dapat membuka berkas dan membaca angka baris demi baris. Fungsi
getvectorline() akan membaca angka dari sebaris data. Secara default,
fungsi ini mengharapkan titik desimal. Namun, fungsi ini juga dapat
menggunakan koma desimal, jika Anda memanggil setdecimaldot(",")
sebelum menggunakan fungsi ini.

Fungsi berikut adalah contohnya. Fungsi ini akan berhenti di akhir
berkas atau baris kosong.
\end{eulercomment}
\begin{eulerprompt}
>function myload (file) ...
\end{eulerprompt}
\begin{eulerudf}
  open(file);
  M=[];
  repeat
     until eof();
     v=getvectorline(3);
     if length(v)>0 then M=M_v; else break; endif;
  end;
  return M;
  close(file);
  endfunction
\end{eulerudf}
\begin{eulerprompt}
>myload(file)
\end{eulerprompt}
\begin{euleroutput}
    0.39606    0.5217  0.079457 
     0.5777    0.6538   0.69742 
    0.83267   0.91434     0.676 
\end{euleroutput}
\begin{eulercomment}
Semua angka dalam berkas itu juga dapat dibaca dengan getvector().
\end{eulercomment}
\begin{eulerprompt}
>open(file); v=getvector(10000); close(); redim(v[1:9],3,3)
\end{eulerprompt}
\begin{euleroutput}
    0.39606    0.5217  0.079457 
     0.5777    0.6538   0.69742 
    0.83267   0.91434     0.676 
\end{euleroutput}
\begin{eulercomment}
Jadi sangat mudah untuk menyimpan vektor nilai, satu nilai di setiap
baris dan membaca kembali vektor ini.
\end{eulercomment}
\begin{eulerprompt}
>v=random(1000); mean(v)
\end{eulerprompt}
\begin{euleroutput}
  0.51242
\end{euleroutput}
\begin{eulerprompt}
>writematrix(v',file); mean(readmatrix(file)')
\end{eulerprompt}
\begin{euleroutput}
  0.51242
\end{euleroutput}
\eulerheading{Menggunakan Tabel}
\begin{eulercomment}
Tabel dapat digunakan untuk membaca atau menulis data numerik.
Misalnya, kita menulis tabel dengan tajuk baris dan kolom ke dalam
sebuah berkas.
\end{eulercomment}
\begin{eulerprompt}
>file="test.tab"; M=random(3,3);  ...
>open(file,"w");  ...
>writetable(M,separator=",",labc=["one","two","three"]);  ...
>close(); ...
>printfile(file)
\end{eulerprompt}
\begin{euleroutput}
  one,two,three
        0.92,      0.87,      0.47
        0.51,      0.63,      0.98
        0.11,      0.53,      0.32
\end{euleroutput}
\begin{eulercomment}
Ini dapat diimpor ke Excel.

Untuk membaca berkas di EMT, kami menggunakan readtable().
\end{eulercomment}
\begin{eulerprompt}
>\{M,headings\}=readtable(file,>clabs); ...
>writetable(M,labc=headings)
\end{eulerprompt}
\begin{euleroutput}
         one       two     three
        0.92      0.87      0.47
        0.51      0.63      0.98
        0.11      0.53      0.32
\end{euleroutput}
\eulerheading{Menganalisis Garis}
\begin{eulercomment}
Anda bahkan dapat mengevaluasi setiap garis secara manual. Misalkan,
kita memiliki garis dengan format berikut.
\end{eulercomment}
\begin{eulerprompt}
>line="2020-11-03,Tue,1'114.05"
\end{eulerprompt}
\begin{euleroutput}
  2020-11-03,Tue,1'114.05
\end{euleroutput}
\begin{eulercomment}
Pertama, kita dapat membuat token pada baris tersebut.
\end{eulercomment}
\begin{eulerprompt}
>vt=strtokens(line)
\end{eulerprompt}
\begin{euleroutput}
  2020-11-03
  Tue
  1'114.05
\end{euleroutput}
\begin{eulercomment}
Kemudian kita dapat mengevaluasi setiap elemen garis menggunakan
evaluasi yang tepat.
\end{eulercomment}
\begin{eulerprompt}
>day(vt[1]),  ...
>indexof(["mon","tue","wed","thu","fri","sat","sun"],tolower(vt[2])),  ...
>strrepl(vt[3],"'","")()
\end{eulerprompt}
\begin{euleroutput}
  7.3816e+05
  2
  1114
\end{euleroutput}
\begin{eulercomment}
Dengan menggunakan ekspresi reguler, dimungkinkan untuk mengekstrak
hampir semua informasi dari sebaris data.

Asumsikan kita memiliki baris berikut sebagai dokumen HTML.
\end{eulercomment}
\begin{eulerprompt}
>line="<tr><td>1145.45</td><td>5.6</td><td>-4.5</td><tr>"
\end{eulerprompt}
\begin{euleroutput}
  <tr><td>1145.45</td><td>5.6</td><td>-4.5</td><tr>
\end{euleroutput}
\begin{eulercomment}
Untuk mengekstraknya, kami menggunakan ekspresi reguler, yang mencari

- tanda kurung tutup \textgreater{},\\
- string apa pun yang tidak mengandung tanda kurung dengan
sub-kecocokan "(...)",\\
- tanda kurung buka dan tutup menggunakan solusi terpendek,\\
- lagi-lagi string apa pun yang tidak mengandung tanda kurung,\\
- dan tanda kurung buka \textless{}.

Ekspresi reguler agak sulit dipelajari tetapi sangat ampuh.
\end{eulercomment}
\begin{eulerprompt}
>\{pos,s,vt\}=strxfind(line,">([^<>]+)<.+?>([^<>]+)<");
\end{eulerprompt}
\begin{eulercomment}
Hasilnya adalah posisi kecocokan, string yang cocok, dan vektor string
untuk sub-kecocokan.
\end{eulercomment}
\begin{eulerprompt}
>for k=1:length(vt); vt[k](), end;
\end{eulerprompt}
\begin{euleroutput}
  1145.5
  5.6
\end{euleroutput}
\begin{eulercomment}
Berikut adalah fungsi yang membaca semua item numerik antara \textless{}td\textgreater{} dan
\textless{}/td\textgreater{}.
\end{eulercomment}
\begin{eulerprompt}
>function readtd (line) ...
\end{eulerprompt}
\begin{eulerudf}
  v=[]; cp=0;
  repeat
     \{pos,s,vt\}=strxfind(line,"<td.*?>(.+?)</td>",cp);
     until pos==0;
     if length(vt)>0 then v=v|vt[1]; endif;
     cp=pos+strlen(s);
  end;
  return v;
  endfunction
\end{eulerudf}
\begin{eulerprompt}
>readtd(line+"<td>non-numerical</td>")
\end{eulerprompt}
\begin{euleroutput}
  1145.45
  5.6
  -4.5
  non-numerical
\end{euleroutput}
\eulerheading{Membaca dari Web}
\begin{eulercomment}
Situs web atau berkas dengan URL dapat dibuka di EMT dan dapat dibaca
baris demi baris.

Dalam contoh ini, kami membaca versi terkini dari situs EMT. Kami
menggunakan ekspresi reguler untuk memindai "Versi ..." dalam judul.
\end{eulercomment}
\begin{eulerprompt}
>function readversion () ...
\end{eulerprompt}
\begin{eulerudf}
  urlopen("http://www.euler-math-toolbox.de/Programs/Changes.html");
  repeat
    until urleof();
    s=urlgetline();
    k=strfind(s,"Version ",1);
    if k>0 then substring(s,k,strfind(s,"<",k)-1), break; endif;
  end;
  urlclose();
  endfunction
\end{eulerudf}
\begin{eulerprompt}
>readversion
\end{eulerprompt}
\begin{euleroutput}
  Version 2024-01-12
\end{euleroutput}
\eulerheading{Input dan Output Variabel}
\begin{eulercomment}
Anda dapat menulis variabel dalam bentuk definisi Euler ke dalam file
atau ke baris perintah.
\end{eulercomment}
\begin{eulerprompt}
>writevar(pi,"mypi");
\end{eulerprompt}
\begin{euleroutput}
  mypi = 3.141592653589793;
\end{euleroutput}
\begin{eulercomment}
Untuk pengujian, kami membuat file Euler di direktori kerja EMT.
\end{eulercomment}
\begin{eulerprompt}
>file="test.e"; ...
>writevar(random(2,2),"M",file); ...
>printfile(file,3)
\end{eulerprompt}
\begin{euleroutput}
  M = [ ..
  0.3731518880081878, 0.5100363690187648;
  0.8536630359298354, 0.5969300413220439];
\end{euleroutput}
\begin{eulercomment}
Sekarang kita dapat memuat berkas tersebut. Berkas tersebut akan
mendefinisikan matriks M.
\end{eulercomment}
\begin{eulerprompt}
>load(file); show M,
\end{eulerprompt}
\begin{euleroutput}
  M = 
    0.37315   0.51004 
    0.85366   0.59693 
\end{euleroutput}
\begin{eulercomment}
Ngomong-ngomong, jika writevar() digunakan pada suatu variabel, ia
akan mencetak definisi variabel dengan nama variabel ini.
\end{eulercomment}
\begin{eulerprompt}
>writevar(M); writevar(inch$)
\end{eulerprompt}
\begin{euleroutput}
  M = [ ..
  0.3731518880081878, 0.5100363690187648;
  0.8536630359298354, 0.5969300413220439];
  inch$ = 0.0254;
\end{euleroutput}
\begin{eulercomment}
Kita juga dapat membuka berkas baru atau menambahkannya ke berkas yang
sudah ada. Dalam contoh ini, kita menambahkannya ke berkas yang dibuat
sebelumnya.
\end{eulercomment}
\begin{eulerprompt}
>open(file,"a"); ...
>writevar(random(2,2),"M1"); ...
>writevar(random(3,1),"M2"); ...
>close();
>load(file); show M1; show M2;
\end{eulerprompt}
\begin{euleroutput}
  M1 = 
    0.79972   0.80464 
    0.30558   0.16514 
  M2 = 
   0.075582 
    0.25656 
    0.11703 
\end{euleroutput}
\begin{eulercomment}
Untuk menghapus file apa pun gunakan fileremove().
\end{eulercomment}
\begin{eulerprompt}
>fileremove(file);
\end{eulerprompt}
\begin{eulercomment}
Vektor baris dalam sebuah berkas tidak memerlukan koma, jika setiap
angka berada di baris baru. Mari kita buat berkas seperti itu, tulis
setiap baris satu per satu dengan writeln().
\end{eulercomment}
\begin{eulerprompt}
>open(file,"w"); writeln("M = ["); ...
>for i=1 to 5; writeln(""+random()); end; ...
>writeln("];"); close(); ...
>printfile(file)
\end{eulerprompt}
\begin{euleroutput}
  M = [
  0.947939143784
  0.834390864672
  0.283694840089
  0.583171970476
  0.425722158135
  ];
\end{euleroutput}
\begin{eulerprompt}
>load(file); M
\end{eulerprompt}
\begin{euleroutput}
  [0.94794,  0.83439,  0.28369,  0.58317,  0.42572]
\end{euleroutput}
\begin{eulercomment}
catatan : ketika mengenter perintah-perintah diatas ternyata hasil
yang didapatkan berbeda-beda

\begin{eulercomment}
\eulerheading{Latihan soal }
\begin{eulercomment}
Nomor 1 Carilah rata-rata dan standar deviasi beserta plot dari data
berikut\\
X = 1000,1500,1700,2500,3500,4000
\end{eulercomment}
\begin{eulerprompt}
>X=[500,1000,1500,2000,3500,4000]; ...
>mean(X), dev(X),
\end{eulerprompt}
\begin{euleroutput}
  2083.3
  1393.4
\end{euleroutput}
\begin{eulerprompt}
>aspect(1.5); boxplot(X):
\end{eulerprompt}
\begin{eulercomment}
Nomor 2\\
Misalkan diberikan data skor hasil statistika dari 14 orang mahasiswa
sebagai berikut:\\
69,65,58,95,75,79,85,91,75,79,75,95

Tentukan rata-rata dari data tersebut!
\end{eulercomment}
\begin{eulerprompt}
>X=[69,65,58,95,75,79,85,91,75,79,75,95]
\end{eulerprompt}
\begin{euleroutput}
  [69,  65,  58,  95,  75,  79,  85,  91,  75,  79,  75,  95]
\end{euleroutput}
\begin{eulerprompt}
>mean(X)
\end{eulerprompt}
\begin{euleroutput}
  78.417
\end{euleroutput}
\begin{euleroutput}
  
\end{euleroutput}
\begin{euleroutput}
  
\end{euleroutput}
\end{eulernotebook}
\end{document}
